\section{Improving bounds on the Standard RSHs}
\begin{lemma}\label{OneMaxResult}
If $w_1 \ge W/2$  then the RSH reaches the optimal value in expected time $\Theta(n\log{}n)$
\end{lemma}
\begin{proof}
The optimal solution is putting $w_1$ in one bin and all other elements in the other bin. So the problem is almost identical to OneMax/ZeroMax. A flip of the first bit can only happen if the emptier bin has a weight of at most $\frac {W-w_1}{2}$. After this flip the weight of the emptier bin is at least $\frac {W-w_1}{2}$ and therefore another flip of $w_1$ can only happen before a different bit is flipped. After a different bit has been flipped the RSH wont flip the first bit again because it will never result in an improvement. So the run can be devided into three phases:
\begin{itemize}
    \item[Phase 1:] The RSH behaves exactly like OneMax/ZeroMax and flips every bit to the opposite of the first bit.
    \item[Phase 2:] The RSH flips only the first bit or bits that do not result in an improvement.
    \item[Phase 3:] The RSH behaves exactly like ZeroMax/OneMax and flips every bit to the opposite of the first bit.
\end{itemize}

The expected lenght of the first phase is $\mathcal{O}(n)$ because the probability of flipping the first bit is at least ${\frac{1}{n}} \cdot {(1 - \frac{1}{n})}^{n-1} >= \frac{1}{ne}$ and therefore the expected time for such a step is at most $\mathcal{O}(\frac{1}{ne}^{-1}) = \mathcal{O}(ne) = \mathcal{O}(n)$.\newline
The lenght of the second phase is $\mathcal{O}(n)$ because the solution is either optimal or there is at least one bit that needs to be flipped for an optimal solution. This again results in expected time $\mathcal{O}(n)$. Since the expected length of Phase 1 is $\mathcal{O}(n)$ the solution produced by the RSH won't be optimal in expectation due to the bound of $\Theta(n\log{}n)$ for OneMax/ZeroMax.\newline
The lenght of the third phase is identical to a run of the RSH on OneMax/ZeroMax where flips of the first bit are ignored as if it was already correctly flipped and therefore the expected time is $\Theta(n\log{}n)$

So the expected time is $\mathcal{O}(n) + \mathcal{O}(n) + \Theta(n\log{}n) = \Theta(n\log{}n)$ 
\end{proof}

\begin{lemma}\label{approximationLemma}
The RSH reaches an approximation ratio of at most 4/3 in expected time $\mathcal{O}(n\log{}n)$ if $w_1 \le W/2$ 
\end{lemma}
\begin{proof}
Helpful statements	
\begin{itemize}
    \item[(1)]\label{helpPoint1} If the weight of the fuller bin is at most $\frac{2}{3}$ the approximation ratio is at most \(\frac{2}{3} / opt \le \frac{2}{3} / \frac{1}{2} <= \frac{4}{3}\)
    \item[(2)]\label{helpPoint2} If \(w_1 \ge \frac{1}{3}\) and \(w_1\) is in the emptier bin, then the weight of the fuller bin at most \(1 - w_1 \le 1 - \frac{1}{3} = \frac{2}{3}\) -> approximation  \(\le \frac{4}{3}\) \ref{helpPoint1}
    \item[(3)]\label{helpPoint3} \(b_F - b_E \ge v <-> b_F \ge b_E + v\) and therefore any object of weight at most v can be moved from $b_F$ to $b_E$ if \(b_F - b_E \ge v\)
    \item[(4)]\label{helpPoint4} \(b_F \ge \frac{2}{3} -> b_E \le \frac{1}{3} -> b_F - b_E \ge \frac{2}{3} - \frac{1}{3} = \frac{1}{3}\) -> Every object \(\le \frac{1}{3}\) can be moved from $b_F$ to $b_E$ as long as \(b_F \ge \frac{2}{3}\) \ref{helpPoint3}
    \item[(5)]\label{helpPoint5} In Time $\mathcal{O}(n\log{}n)$ the weight of the fuller bin can be decreased to <= 2/3 if every item besides the biggest in the bin is $\le \frac{1}{3}$.
\end{itemize}
Proof of (5): \newline
In time $\mathcal{O}(n\log{}n)$ the RSH can move every object $\le \frac{1}{3}$ to the emptier bin as long as $b_F \ge \frac{2}{3}$ \ref{helpPoint4}. So in Time $\mathcal{O}(n\log{}n)$ the solution can be shifted to $w_1$ being in the first bin and all other objects in the other bin. The algorithm will only stop moving the elements if the condition $b_F \ge \frac{2}{3}$ is no longer satisfied \ref{helpPoint4}. If every object was moved to the emptier bin, the weight of the fuller bin is at most \(\max\{1-w_1, w_1\} \le \frac{2}{3}\). So either the RSH moves all objects to the emptier bin or stops moving objects because $b_F < \frac{2}{3}$. Either way the fuller bin is at most $\frac{2}{3}$ and with \ref{helpPoint1} the result follows.


If \(w_1+w_2 > \frac{2}{3}\) after time $\mathcal{O}(n)$ $w_1$ and $w_2$ are separated (Proof by by C.Witt) and will remain separated afterwards. From then on the following holds. If $w_1$ is in the emptier bin, the result follows by \ref{helpPoint2}. Otherwise the result follows by \ref{helpPoint5}
\end{proof}

\begin{corollary}
    The RSH reaches an approximation ratio of at most 4/3 in expected time $\mathcal{O}(n\log{}n)$
\end{corollary}
\begin{proof}
    This follows directly from  \ref{OneMaxResult} and  \ref{approximationLemma}
\end{proof}