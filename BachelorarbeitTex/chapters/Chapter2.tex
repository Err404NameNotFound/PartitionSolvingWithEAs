\section{Improving bounds on the Standard RSHs}
\begin{lemma}\label{OneMaxResult}
    If $w_1 \ge \frac W 2$  then the SRSHs reach the optimal value in expected time $\Theta(n\log{}n)$
\end{lemma}
\begin{proof}
    The optimal solution is putting $w_1$ in one bin and all other elements in the other bin. So the problem is almost identical to OneMax/ZeroMax. A flip of only the first bit can only happen, if the emptier bin has a weight of at most $\frac {W-w_1}{2}$. After this flip the weight of the emptier bin is at least $\frac {W-w_1}{2}$ and therefore another single bit flip of $w_1$ can only happen before a different bit is flipped. After a different bit has been flipped, the RSH wont flip the first bit alone again, because it will never result in an improvement. The first bit could also be flipped with multiple other bits by the (1+1) EA. TODO insert proof for (1+1) EA. So the run can be devided into three phases:
    \begin{itemize}
        \item[Phase 1:] The RSH behaves exactly like OneMax/ZeroMax and flips every bit to the opposite of the first bit (except for the first bit).
        \item[Phase 2:] The RSH flips only the first bit or bits that do not result in an improvement.
        \item[Phase 3:] The RSH behaves exactly like ZeroMax/OneMax and flips every bit to the opposite of the first bit (except for the first bit).
    \end{itemize}

    The expected lenght of the first phase is $\mathcal{O}(n)$ because the probability of flipping the first bit is at least ${\frac{1}{n}} \cdot {(1 - \frac{1}{n})}^{n-1} \ge \frac{1}{ne}$ and therefore the expected time for such a step is at most $\mathcal{O}(\frac{1}{ne}^{-1}) = \mathcal{O}(ne) = \mathcal{O}(n)$.\newline
    The lenght of the second phase is $\mathcal{O}(n)$ because the solution is either optimal or there is at least one bit that needs to be flipped for an optimal solution. Since the expected length of Phase 1 is $\mathcal{O}(n)$ the solution produced by the RSH won't be optimal in expectation due to the bound of $\Theta(n\log{}n)$ for OneMax/ZeroMax. This again results in expected time $\mathcal{O}(n)$.\newline
    The lenght of the third phase is identical to a run of the RSH on OneMax/ZeroMax where flips of the first bit are ignored as if it was already correctly flipped and therefore the expected time is $\Theta(n\log{}n)$

    So the total expected time is $\mathcal{O}(n) + \mathcal{O}(n) + \Theta(n\log{}n) = \Theta(n\log{}n)$
\end{proof}

\begin{lemma}\label{approximationLemma}
    The SRSHs reach an approximation ratio of at most 4/3 in expected time $\mathcal{O}(n\log{}n)$ if $w_1 \le W/2$
\end{lemma}
\begin{proof}
    Helpful statements
    \begin{itemize}
        \item[(1)]\label{helpPoint1} If \(b_F \le \frac{2}{3}\) the approximation ratio is at most \(\frac{b_F}{opt} = \frac{2}{3} / opt \le \frac{2}{3} / \frac{1}{2} \le \frac{4}{3}\)
        \item[(2)]\label{helpPoint2} If \(w_1 \ge \frac{1}{3}\) and \(w_1\) is in the emptier bin, then \( b_F \le 1 - w_1 \le 1 - \frac{1}{3} = \frac{2}{3} \Rightarrow\) approximation  \(\le \frac{4}{3}\) (1)
        \item[(3)]\label{helpPoint3} \(b_F - b_E \ge v \Leftrightarrow b_F \ge b_E + v\) and therefore any object of weight at most v can be moved from $b_F$ to $b_E$ if \(b_F - b_E \ge v\)
        \item[(4)]\label{helpPoint4} \(b_F \ge \frac{2}{3} \Rightarrow b_E \le \frac{1}{3} \Rightarrow b_F - b_E \ge \frac{2}{3} - \frac{1}{3} = \frac{1}{3} \Rightarrow\) Every object \(\le \frac{1}{3}\) can be moved from $b_F$ to $b_E$ as long as \(b_F \ge \frac{2}{3}\) (3)
        \item[(5)]\label{helpPoint5} In Time $\mathcal{O}(n\log{}n)$ the weight of the fuller bin can be decreased to \(\le \frac{2}{3}\) if every item besides the biggest in the fuller bin is at most $\frac{1}{3}$ and \(w_1 \le \frac{W}{2}\).
    \end{itemize}
    Proof of (5): \newline
    In time $\mathcal{O}(n\log{}n)$ the RSH can move every object $\le \frac{1}{3}$ to the emptier bin as long as $b_F \ge \frac{2}{3}$ (4). So in Time $\mathcal{O}(n\log{}n)$ the solution can be shifted to $w_1$ being in one bin and all other objects in the other bin. The RSH will only stop moving the elements if the condition $b_F \ge \frac{2}{3}$ is no longer satisfied (4). If \(w_1 \ge \frac{1}{3}\) and every object was moved to the emptier bin, then \(b_F = \max\{1-w_1, w_1\} = 1-w_1 \le \frac{2}{3}\). If \(w_1 < \frac{1}{3}\) then the RSH will stop moving objects to the emptier bin since \(1-w_1 > \frac{2}{3}\) and therefore \(b_F < \frac{2}{3}\) must hold after the RSH stops moving elements(4). So either the RSH moves all objects to the emptier bin or stops moving objects because $b_F < \frac{2}{3}$. This results in $b_F \le \frac{2}{3}$


    If \(w_1+w_2 > \frac{2}{3}\) after time $\mathcal{O}(n)$ $w_1$ and $w_2$ are separated (Proof by by C.Witt) and will remain separated afterwards. From then on the following holds. If $w_1$ is in the emptier bin, the result follows by (2). Otherwise the result follows by (5) and (1)
\end{proof}

\begin{corollary}
    The RSH reaches an approximation ratio of at most 4/3 in expected time $\mathcal{O}(n\log{}n)$
\end{corollary}
\begin{proof}
    This follows directly from Lemma \ref{OneMaxResult} and Lemma \ref{approximationLemma}
\end{proof}