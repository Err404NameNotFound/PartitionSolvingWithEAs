\section{Carsten Witt's worst case input}
C. Witt proved that the RLS and the (1+1) EA find a $(4/3+\epsilon)$ approximation in expected time $\mathcal{O}(n)$ and a $(4/3)$-approximation in expected time $\mathcal{O}(n^2)$~\cite{witt2005worst}.
He then introduced an almost worst case input to prove the bound for the approximation ratio is at least almost tight.
The input is defined as followed for any $0<\epsilon<1/3$ and even $n$:\newline
The input contains two numbers of value $1/3 - \epsilon/4$ and $n-2$ elements of value $(1/3+\epsilon/2)/(n-2)$. 
The total volume is normalised to 1.
When the two large values are in the same bin, the RSHs are tricked into a local optima, where only $w_1$ and $w_2$ are in the first bin and the remaining elements in the other bin.
This results in an almost worst case.
As all researched inputs in this paper contained only integer values this input is adjusted a bit.
To prevent the small values to be below zero they are instead normalised to 1.
The two big values are scaled by the same factor of $((1/3+\epsilon/2)/(n-2))^{-1}$.
The higher the value for $\epsilon$ the more likely the input is to get stuck in the local optima.
With increasing $\epsilon$ the local optima becomes better and better.
For the small values of $\epsilon$ there were only a few cases where some algorithms did not find an optimal solution.
To make this effect more visible the value of $\epsilon$ was set to $\epsilon=0.3$.\newline
For $n=10,000$ this evaluates to $w_1=w_2=5344$ and $W=9998 \cdot 1 + 2 \cdot 5344 = 20686$.
The input then looks like this: $[1, 1, \dots, 1, 1, 5344, 5344]$.
The fitness of the local optimum is $f(x) = 2 \cdot 5433 = 10688$.
To leave the local optimum the algorithm therefore has to flip at least  $5433+9998-10688 = 4654$ bits as well in the same step.
The best fitness is $f(x) = 5344 + 9998/2 = 10343$, which leads to a difference of $f(localOptimum)-f(opt) = 345$ and a approximation ratio of $f(localOptimum)/f(opt)=10688/10343=1.033$.
This is not really close to the worst case of 4/3 any more but with this setting at least many algorithms are stuck in the local optimum at least once for the 1000 runs.
\subsection{RLS Comparison}


\makebox[\linewidth]{
\begin{tabular}{lp{3cm}p{6cm}p{6cm}}
\begin{tabular}[h]{cccccccc}
algo type&            \RLSN&     \RLSR&     \RLSR&     \RLSN&     \RLSR&     \RLSN&       RLS\\
algo param&             b=2&       s=3&       s=4&       b=3&       s=2&       b=4&         -\\
avg mut/change&       2.000&     1.996&     2.476&     3.000&     1.502&     4.000&     1.000\\
avg mut/step&         2.000&     2.000&     2.500&     3.000&     1.500&     4.000&     1.000\\
\hline
total avg count&     83,118&   104,748&   105,513&   112,223&   114,486&   121,927& 2,443,567\\
avg eval count&      83,118&   104,748&   105,513&   112,223&   114,486&   121,927&    45,834\\
max eval count&     778,110& 1,453,252&   898,974& 1,377,471&   915,268&   816,633&   485,275\\
min eval count&         197&       126&        45&       212&       271&       155&       128\\
\hline
fail ratio&           0.000&     0.000&     0.000&     0.000&     0.000&     0.000&     0.447\\
avg fail dif&             -&         -&         -&         -&         -&         -&         1\\
\end{tabular}
\end{tabular}
}


The RLS is by far most likely to get stuck in the local optimum.
The general tendency is the more bits the algorithm flips in expectation the more unlikely the local optimum becomes.
The only case where this does not hold is the \RLSN[2] being better than the \RLSR[4], although the \RLSR[4] flips more bits in expectation.
This might be caused by the higher probability for flipping only one bit for the \RLSR[4].
This causes the algorithm to find the local optimum faster before separating $w_1$ and $w_2$.
So the ranking of the algorithms is completely inverted compared to the OneMax input.
The \RLSN[2] and \RLSN[3] both had runs where they neither found the global nor one of the two local optima.
The algorithms were most likely tricked into the direction of the local optimum and did not manage to leave it.
But they were also not fast enough to reach the local optimum because of their low probability to flip only one bit.
\subsection{(1+1) EA Comparison}


\makebox[\linewidth]{
\begin{tabular}{lp{3cm}p{6cm}p{6cm}}
\begin{tabular}[h]{ccccccccc}
algo type&          (1+1) EA&   (1+1) EA&   (1+1) EA&   (1+1) EA&      (1+1) EA&   (1+1) EA&   (1+1) EA&   (1+1) EA\\
algo param&           3/n&     4/n&     2/n&     5/n&       -&    10/n&    50/n&   100/n\\
avg mut/change&     3.101&   3.968&   2.343&   4.859&   1.698&   9.732&  49.544&  99.494\\
avg mut/step&       2.999&   4.003&   2.002&   4.999&   1.001&   9.998&  49.998&  99.997\\
\hline
total avg count&      646&     701&     706&     857&   1,123&   1,508&   8,175&  15,485\\
avg eval count&       646&     701&     706&     857&   1,123&   1,508&   8,175&  15,485\\
max eval count&     5,346&   5,692&   3,415&   5,572&   7,001&  12,112&  52,831& 145,269\\
min eval count&        23&       4&      30&       9&      23&      14&      27&      69\\
\hline
fails&                  0&       0&       0&       0&       0&       0&       0&       0\\
fail ratio&         0.000&   0.000&   0.000&   0.000&   0.000&   0.000&   0.000&   0.000\\
avg fail dif&           -&       -&       -&       -&       -&       -&       -&       -\\
\end{tabular}
\end{tabular}
}


For the (1+1) EA the result is also the inversion of the results for the OneMax equivalent.
The higher the mutation rate the faster the algorithms reach a global optimum.
This holds at least up to $p_m\le100/n$.
With mutation rate $p_m\le3/n$ the algorithm reaches the worst case at least once in 1000 runs.
If the algorithm did not manage to find an optimal solution the fitness was always the same.
So there was no run where any algorithm neither found a global nor the local optimum.
\subsection{pmut Comparison}


\makebox[\linewidth]{
\scriptsize
\begin{tabular}{lp{3cm}p{6cm}p{6cm}}
\begin{tabular}[h]{m{2.5cm}m{0,40cm}m{0,40cm}m{0,40cm}m{0,40cm}m{0,40cm}m{0,40cm}m{0,40cm}m{0,40cm}m{0,40cm}m{0,40cm}m{0,40cm}m{0,40cm}m{0,40cm}m{0,40cm}m{0,40cm}m{0,40cm}m{0,40cm}m{0,40cm}}
\multicolumn{1}{c}{algo type}&\multicolumn{2}{c}{            pmut}&\multicolumn{2}{c}{     pmut}&\multicolumn{2}{c}{     pmut}&\multicolumn{2}{c}{     pmut}&\multicolumn{2}{c}{     pmut}&\multicolumn{2}{c}{     pmut}&\multicolumn{2}{c}{     pmut}&\multicolumn{2}{c}{     pmut}&\multicolumn{2}{c}{     pmut}\\
\multicolumn{1}{c}{algo param}&\multicolumn{2}{c}{           3.25}&\multicolumn{2}{c}{     3.00}&\multicolumn{2}{c}{     2.75}&\multicolumn{2}{c}{     2.50}&\multicolumn{2}{c}{     2.25}&\multicolumn{2}{c}{     2.00}&\multicolumn{2}{c}{     1.75}&\multicolumn{2}{c}{     1.50}&\multicolumn{2}{c}{     1.25}\\
\multicolumn{1}{c}{avg mut/change}&\multicolumn{2}{c}{      1.583}&\multicolumn{2}{c}{    1.737}&\multicolumn{2}{c}{    2.002}&\multicolumn{2}{c}{    2.423}&\multicolumn{2}{c}{    3.303}&\multicolumn{2}{c}{    5.830}&\multicolumn{2}{c}{   12.519}&\multicolumn{2}{c}{   30.910}&\multicolumn{2}{c}{   73.182}\\
\multicolumn{1}{c}{avg mut/step}&\multicolumn{2}{c}{        1.729}&\multicolumn{2}{c}{    1.934}&\multicolumn{2}{c}{    2.274}&\multicolumn{2}{c}{    2.895}&\multicolumn{2}{c}{    4.360}&\multicolumn{2}{c}{    8.452}&\multicolumn{2}{c}{   22.278}&\multicolumn{2}{c}{   70.532}&\multicolumn{2}{c}{  224.421}\\
\hline
\multicolumn{1}{c}{avg eval count}&\multicolumn{2}{c}{        540}&\multicolumn{2}{c}{      569}&\multicolumn{2}{c}{      594}&\multicolumn{2}{c}{      641}&\multicolumn{2}{c}{      712}&\multicolumn{2}{c}{      808}&\multicolumn{2}{c}{      967}&\multicolumn{2}{c}{    1,285}&\multicolumn{2}{c}{    2,081}\\
\multicolumn{1}{c}{max eval count}&\multicolumn{2}{c}{      3,110}&\multicolumn{2}{c}{    2,891}&\multicolumn{2}{c}{    3,504}&\multicolumn{2}{c}{    3,896}&\multicolumn{2}{c}{    5,152}&\multicolumn{2}{c}{    4,274}&\multicolumn{2}{c}{    5,610}&\multicolumn{2}{c}{    6,190}&\multicolumn{2}{c}{   14,984}\\
\multicolumn{1}{c}{min eval count}&\multicolumn{2}{c}{         22}&\multicolumn{2}{c}{        9}&\multicolumn{2}{c}{       36}&\multicolumn{2}{c}{       25}&\multicolumn{2}{c}{       28}&\multicolumn{2}{c}{       27}&\multicolumn{2}{c}{       27}&\multicolumn{2}{c}{       13}&\multicolumn{2}{c}{       33}\\
\hline
\multicolumn{1}{c}{fail ratio}&\multicolumn{2}{c}{          0.000}&\multicolumn{2}{c}{    0.000}&\multicolumn{2}{c}{    0.000}&\multicolumn{2}{c}{    0.000}&\multicolumn{2}{c}{    0.000}&\multicolumn{2}{c}{    0.000}&\multicolumn{2}{c}{    0.000}&\multicolumn{2}{c}{    0.000}&\multicolumn{2}{c}{    0.000}\\
\hline
\multicolumn{1}{c}{p-value}&&\multicolumn{2}{c}{0.0000}&\multicolumn{2}{c}{0.0000}&\multicolumn{2}{c}{0.0000}&\multicolumn{2}{c}{0.0000}&\multicolumn{2}{c}{0.0000}&\multicolumn{2}{c}{0.0000}&\multicolumn{2}{c}{0.0000}&\multicolumn{2}{c}{0.0000}\\
&&&&&&&&&&&&&&&&&&\end{tabular}
\end{tabular}
}


For $pmut$ the result is the exact same as for the (1+1) EA.
The lower $\beta$ the better the performance as more bits are flipped in each step.
The algorithm is tricked into the local optima only for $\beta\le2.5$.
If the algorithm is on the path to the local optimum it is always fast enough to reach it within the time limit.
\subsection{Comparison of the best variants}


\makebox[\linewidth]{
\begin{tabular}{lp{3cm}p{6cm}p{6cm}}
\begin{tabular}[h]{cccc}
algo type&            RLS&    pmut&      EA\\
algo param&             -&    3.25&       -\\
avg mut/change&     1.000&   1.287&   1.272\\
avg mut/step&       1.000&   1.729&   1.000\\
\hline
avg eval count&    91,171& 143,121& 231,082\\
max eval count&   153,143& 227,737& 446,942\\
min eval count&    65,783&  93,602& 165,818\\
\hline
fail ratio&         0.000&   0.000&   0.000\\
\end{tabular}
\end{tabular}
}


The $pmut_1.25$ and the (1+1) EA with $p_m=100/n$ perform the best and always find an optimal solution within 600 iterations and even under 100 on the average case.
The \RLSN[4] performs significantly worse.
% The other algorithm flip so many bits that they are almost close to random sampling.
For the OneMax input $pmut_1.25$ flipped 224 bits on average per step, but the average of the successful steps was only 5.
Here the average of all steps is again 224, but the average of the successful steps is at 194.
The heavy tail here really increases the performance as most of the high values are accepted.
In the experiment with different input sizes the mutation rate of $p_m=100/n$ is $\ge1$ for $n\le100$.
If the algorithm flips every bit then it won't change its solution.
In these cases the mutation rate was then set to $p_m=1/2$.

\begin{tabular}[h]{ccccccccc}
fails in 1000 runs&20&50&100&500&1000&5000&10000&50000\\\hline
RLS&984&773&411&1&0&0&0&0\\
\RLSR[2]&890&241&14&0&0&0&0&0\\
(1+1) EA (1$/n$)&711&75&5&0&0&0&0&0\\
(1+1) EA (2$/n$)&541&14&0&0&0&0&0&0\\
pmut (3.0)&566&63&4&0&0&0&0&0\\
pmut (3.25)&587&63&7&0&0&0&0&0\\
\end{tabular}


Only the RLS variant had runs where it did not reach a global optimum.
This happened in less than 0.5 \% of the inputs for $n=20$ and $n=100$.
For the other input sizes it also managed to reach a global optimum for all inputs.

\begin{tabular}[h]{ccccccccc}
avg&20&50&100&500&1000&5000&10000&50000\\\hline
RLS&32&79&153&579&950&1859&1922&1797\\
\RLSR[2]&391&2124&5005&4218&3530&2362&2160&2229\\
(1+1) EA (1$/n$)&22471&18343&12834&8342&6511&3815&3458&3371\\
(1+1) EA (2$/n$)&16360&9243&6452&4503&4020&3171&3141&3133\\
pmut (3.25)&23440&15929&9658&5644&4406&2434&2162&2172\\
pmut (3.0)&21901&14696&9186&5222&4150&2510&2208&2213\\
\end{tabular}


For the lower input sizes the RLS is slower than the remaining algorithm even it manages to find a global optimum.

\begin{tabular}[h]{ccccccccc}
total avg&20&50&100&500&1000&5000&10000&50000\\\hline
RLS&99470&93459&74385&3474&447&389&421&555\\
\RLSR[2]&96305&63088&19364&815&589&565&595&699\\
(1+1) EA (1$/n$)&89444&52886&21141&1410&869&850&875&1046\\
(1+1) EA (2$/n$)&79603&37214&14107&1302&992&942&957&1055\\
pmut (3.25)&84226&48403&18280&930&542&526&546&642\\
pmut (3.0)&82474&46286&17825&917&576&548&573&668\\
\end{tabular}



The $pmut_{1.25}$ is the best variant closely followed by the (1+1) EA.
The RLS version is by far slower than the other to variants for the bigger input sizes.
Even for the smaller inputs it is still slower.
