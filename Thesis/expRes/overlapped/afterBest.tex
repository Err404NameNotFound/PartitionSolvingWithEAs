The results here are the same as for the binomial input. The $RLS-N_2$ performs better than the (1+1) EA and $pmut_\beta$ mutation for all values of $c/n$ and $\beta$ by a factor of 1.5 with a step limit of $10 \cdot n \ln(n)$. For the lower values of $n$ this does not hold to that extreme.

\begin{tabular}[h]{ccccccccc}
fails in 1000 runs&20&50&100&500&1000&5000&10000&50000\\\hline
RLS&984&773&411&1&0&0&0&0\\
\RLSR[2]&890&241&14&0&0&0&0&0\\
(1+1) EA (1$/n$)&711&75&5&0&0&0&0&0\\
(1+1) EA (2$/n$)&541&14&0&0&0&0&0&0\\
pmut (3.0)&566&63&4&0&0&0&0&0\\
pmut (3.25)&587&63&7&0&0&0&0&0\\
\end{tabular}


Here almost no algorithm manages to find an optimal solution in most cases as all algorithms fail for more than 80 \% of the inputs.
The RLS variants perform the worst again for the small input sizes.
Only for $n\ge500$ the $RLS-N_{2}$ is a good option due to the hug performance increase between $n=100$ and $n=500$.
Overlapped inputs stay hard to solve relatively long as only for $n\ge50,000$ all best variants of each algorithm find an optimal solution for every input.

\begin{tabular}[h]{ccccccccc}
avg&20&50&100&500&1000&5000&10000&50000\\\hline
RLS&32&79&153&579&950&1859&1922&1797\\
\RLSR[2]&391&2124&5005&4218&3530&2362&2160&2229\\
(1+1) EA (1$/n$)&22471&18343&12834&8342&6511&3815&3458&3371\\
(1+1) EA (2$/n$)&16360&9243&6452&4503&4020&3171&3141&3133\\
pmut (3.25)&23440&15929&9658&5644&4406&2434&2162&2172\\
pmut (3.0)&21901&14696&9186&5222&4150&2510&2208&2213\\
\end{tabular}


For $n\le1000$ the performance seems almost constant, but this is caused by the constant step limit of 100,000.
After $n=1000$ the value of $10 \cdot n \ln(n)$ was bigger than 100,000.

\begin{tabular}[h]{ccccccccc}
total avg&20&50&100&500&1000&5000&10000&50000\\\hline
RLS&99470&93459&74385&3474&447&389&421&555\\
\RLSR[2]&96305&63088&19364&815&589&565&595&699\\
(1+1) EA (1$/n$)&89444&52886&21141&1410&869&850&875&1046\\
(1+1) EA (2$/n$)&79603&37214&14107&1302&992&942&957&1055\\
pmut (3.25)&84226&48403&18280&930&542&526&546&642\\
pmut (3.0)&82474&46286&17825&917&576&548&573&668\\
\end{tabular}



No algorithm is very successful for the small values of N.
Choosing the (1+1) EA for $n<500$ should result in the best runtime in most of the cases.
Only the $RLS-N_4$ is faster for $50\le n \le 100$.
For bigger input sizes the $RLS-N_2$ is clearly the best option.