The results here are the same as for the binomial input. The $RLS-N_2$ performs better than the (1+1) EA and $pmut_\beta$ mutation for all values of $c/n$ and $\beta$ by a factor of 1.5 with a step limit of $10 \cdot n \ln(n)$. For the lower values of $n$ this does not hold to that extreme.

\begin{tabular}[h]{cccccccc}
fails&20&50&100&500&1000&5000&10000\\\hline
RLS-N (2)&998&984&973&763&541&49&1\\
RLS-N (4)&998&989&975&837&719&150&27\\
(1+1) EA (3/n)&994&990&981&862&714&194&35\\
(1+1) EA (4/n)&998&993&976&852&725&180&51\\
pmut (-2.0)&997&989&983&905&784&229&55\\
pmut (-2.25)&997&988&987&875&790&213&56\\
\end{tabular}


Here almost no algorithm manages to find an optimal solution in most cases as all algorithms fail for more than 70 \% of the inputs.
The RLS variants perform the worst again for the small input sizes.
Only for $n\ge500$ the $RLS-N_{2}$ is a good option due to the huge performance increase between $n=100$ and $n=500$.
Overlapped inputs stay hard to solve relatively long as only for $n\ge50,000$ all best variants of each algorithm find an optimal solution for every input.

\begin{tabular}[h]{ccccccccc}
avg&20&50&100&500&1000&5000&10000&50000\\\hline
RLS-N (2)&172&1247&7598&38282&36214&105152&117792&125415\\
RLS-N (4)&8626&40966&43921&41924&42289&138187&201183&214170\\
(1+1) EA (3/n)&36470&42555&42439&43801&45530&145209&212281&252713\\
(1+1) EA (4/n)&37212&43144&44340&47246&46689&143697&220208&265620\\
pmut (-2.0)&40368&42601&39603&47443&46591&148390&232398&311606\\
pmut (-2.25)&39476&42113&40325&40548&43691&157816&226129&292367\\
\end{tabular}


For $n\le1000$ the performance seems almost constant, but this is caused by the constant step limit of 100,000.
After $n=1000$ the value of $10 \cdot n \ln(n)$ was bigger than 100,000.

\begin{tabular}[h]{ccccccccc}
total avg&20&50&100&500&1000&5000&10000&50000\\\hline
RLS&98050&94574&89765&60172&39985&15639&4128&1797\\
\RLSR[2]&89401&54556&18266&4218&3530&2362&2160&2229\\
(1+1) EA (1$/n$)&54971&26688&15684&8452&6567&3815&3458&3371\\
(1+1) EA (2$/n$)&26104&9724&6508&4503&4020&3171&3141&3133\\
pmut (3.25)&40934&19972&10977&5644&4406&2434&2162&2172\\
pmut (3.0)&35272&17545&10040&5222&4150&2510&2208&2213\\
\end{tabular}



No algorithm is very successful for the small values of $n$.
Choosing the (1+1) EA for $n<500$ should result in the best runtime in most of the cases.
Only the $RLS-N_4$ is faster for $50\le n \le 100$.
For bigger input sizes the $RLS-N_2$ is clearly the best option.