This input is similar to the mixed input from the last subsection.
For this distribution the values are not chosen uniform random from one of the base distributions, but instead a value from every distribution is chosen and added together.
Hence the name overlapped distribution.
For this input the step limit was increased to $100\cdot n \ln n$.
The comparison with the step limit $10\cdot n \ln(n)$ is contained in the last subsubsection which compares the best variants.

\begin{figure}[h]
      \caption{Distribution of an overlapped input with \textasciitilde$U(1,999)$, \textasciitilde$B(1000,0.1)$, \textasciitilde$Geo(0.01)$, powerlaw dist with $\beta=-1.25$}
      \centering
      \includegraphics[width=0.7\textwidth]{figures/images/numberGenerator/overlapped.png}\label{fig:overlappedDistExample}
\end{figure}

Figure~\ref{fig:overlappedDistExample} looks completely different from figure~\ref{fig:mixedDistExample}.
No value is generated more than 20 times as opposed to the maximum amount of 350 for the mixed distribution.
In this figure no distribution is clearly visible.

The used distributions were \textasciitilde$U(1,49999)$, \textasciitilde$B(10000,0.1)$, \textasciitilde$Geo(0.001)$, powerlaw distribution with $\beta=-1.25$.