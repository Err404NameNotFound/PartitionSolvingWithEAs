The $pmut_-1.25$ and the (1+1) EA with $p_m=100/n$ perform the best and always find an optimal solution within 600 iterations and even under 100 on the average case.
The $RLS-N_4$ performs significantly worse.
The other algorithm flip so many bits that they are almost close to random sampling.
In the experiment with different input sizes the mutation rate of $p_m=100/n$ is $\ge1$ for $n\le100$.
If the algorithm flips every bit then it won't change its solution.
In these cases the mutation rate was then set to $p_m=1/2$.

\begin{tabular}[h]{cccccccc}
fails&20&50&100&500&1000&5000&10000\\\hline
RLS-N (2)&998&984&973&763&541&49&1\\
RLS-N (4)&998&989&975&837&719&150&27\\
(1+1) EA (3/n)&994&990&981&862&714&194&35\\
(1+1) EA (4/n)&998&993&976&852&725&180&51\\
pmut (-2.0)&997&989&983&905&784&229&55\\
pmut (-2.25)&997&988&987&875&790&213&56\\
\end{tabular}


Only the RLS variant had runs where it did not reach a global optimum.
This happened in less than 0.5 \% of the inputs for $n=20$ and $n=100$.
For the other input sizes it also managed to reach a global optimum for all inputs.

\begin{tabular}[h]{ccccccccc}
avg&20&50&100&500&1000&5000&10000&50000\\\hline
RLS-N (2)&172&1247&7598&38282&36214&105152&117792&125415\\
RLS-N (4)&8626&40966&43921&41924&42289&138187&201183&214170\\
(1+1) EA (3/n)&36470&42555&42439&43801&45530&145209&212281&252713\\
(1+1) EA (4/n)&37212&43144&44340&47246&46689&143697&220208&265620\\
pmut (-2.0)&40368&42601&39603&47443&46591&148390&232398&311606\\
pmut (-2.25)&39476&42113&40325&40548&43691&157816&226129&292367\\
\end{tabular}


For the lower input sizes the RLS is slower than the remaining algorithm even it manages to find a global optimum.

\begin{tabular}[h]{ccccccccc}
total avg&20&50&100&500&1000&5000&10000&50000\\\hline
RLS&98050&94574&89765&60172&39985&15639&4128&1797\\
\RLSR[2]&89401&54556&18266&4218&3530&2362&2160&2229\\
(1+1) EA (1$/n$)&54971&26688&15684&8452&6567&3815&3458&3371\\
(1+1) EA (2$/n$)&26104&9724&6508&4503&4020&3171&3141&3133\\
pmut (3.25)&40934&19972&10977&5644&4406&2434&2162&2172\\
pmut (3.0)&35272&17545&10040&5222&4150&2510&2208&2213\\
\end{tabular}



The $pmut_{-1.25}$ is the best variant closely followed by the (1+1) EA.
The RLS version is by far slower than the other to variants for the bigger input sizes.
Even for the smaller inputs it is still slower.