The $pmut_-1.25$ and the (1+1) EA with $p_m=100/n$ perform the best and always find an optimal solution within 600 iterations and even under 100 on the average case.
The $RLS-N_4$ performs significantly worse.
The other algorithm flip so many bits that they are almost close to random sampling.
In the experiment with different input sizes the mutation rate of $p_m=100/n$ is $\ge1$ for $n\le100$.
If the algorithm flips every bit then it won't change its solution.
In these cases the mutation rate was then set to $p_m=1/2$.

\begin{tabular}[h]{ccccccccc}
fails in 1000 runs&20&50&100&500&1000&5000&10000&50000\\\hline
RLS&984&773&411&1&0&0&0&0\\
\RLSR[2]&890&241&14&0&0&0&0&0\\
(1+1) EA (1$/n$)&711&75&5&0&0&0&0&0\\
(1+1) EA (2$/n$)&541&14&0&0&0&0&0&0\\
pmut (3.0)&566&63&4&0&0&0&0&0\\
pmut (3.25)&587&63&7&0&0&0&0&0\\
\end{tabular}


Only the RLS variant had runs where it did not reach a global optimum.
This happened in less than 0.5 \% of the inputs for $n=20$ and $n=100$.
For the other input sizes it also managed to reach a global optimum for all inputs.

\begin{tabular}[h]{ccccccccc}
avg&20&50&100&500&1000&5000&10000&50000\\\hline
RLS&32&79&153&579&950&1859&1922&1797\\
\RLSR[2]&391&2124&5005&4218&3530&2362&2160&2229\\
(1+1) EA (1$/n$)&22471&18343&12834&8342&6511&3815&3458&3371\\
(1+1) EA (2$/n$)&16360&9243&6452&4503&4020&3171&3141&3133\\
pmut (3.25)&23440&15929&9658&5644&4406&2434&2162&2172\\
pmut (3.0)&21901&14696&9186&5222&4150&2510&2208&2213\\
\end{tabular}


For the lower input sizes the RLS is slower than the remaining algorithm even it manages to find a global optimum.

\begin{tabular}[h]{ccccccccc}
total avg&20&50&100&500&1000&5000&10000&50000\\\hline
RLS&99470&93459&74385&3474&447&389&421&555\\
\RLSR[2]&96305&63088&19364&815&589&565&595&699\\
(1+1) EA (1$/n$)&89444&52886&21141&1410&869&850&875&1046\\
(1+1) EA (2$/n$)&79603&37214&14107&1302&992&942&957&1055\\
pmut (3.25)&84226&48403&18280&930&542&526&546&642\\
pmut (3.0)&82474&46286&17825&917&576&548&573&668\\
\end{tabular}



The $pmut_{-1.25}$ is the best variant closely followed by the (1+1) EA.
The RLS version is by far slower than the other to variants for the bigger input sizes.
Even for the smaller inputs it is still slower.