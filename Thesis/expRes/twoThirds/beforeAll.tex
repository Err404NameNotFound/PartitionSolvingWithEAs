C. Witt proved the RLS and the (1+1) EA find a $(4/3+\epsilon)$ approximation in expected time $\mathcal{O}(n)$ and a $(4/3)$-approximation in expected time $\mathcal{O}(n^2)$~\cite{diekert2005stacs}.
He then introduced an almost worst case input to prove the bound for the approximation ratio is at least almost tight.
The input is defined as followed for any $0<\epsilon<1/3$ and even $n$:
The input contains to numbers of value $1/3 - \epsilon/4$ and n-2 elements of value $(1/3+\epsilon/2)/(n-2)$. 
The total volume is normalised to 1.
When the two large values are in the same bin. The RSHs are tricked into a local optima, where only $w_1$ and $w_2$ are in the first bin and the remaining elements in the other bin.
This results in an almost worst case.
As all researched inputs in this paper contained only integer values this input is adjusted a bit.
All values for the n-2 small objects are set to 1 and the value of the two other objects is set to $n-2-Math.round((0.3\cdot(n - 2)) /4)$.
The total volume is given by $2\cdot(n-2-Math.round((\epsilon\cdot(n - 2)) /4))+n-2 = 3(n-2)-2\cdot Math.round((0.3\cdot(n - 2)))$