This distribution has mostly small values, but occasionally it also generates bigger values.
The higher (absolute lower) the parameter the higher the values get and also the amount of big values increases.
For a parameter of $\beta=-2.75$ the distribution looks like in Figure~\ref{fig:powerDistExample1}.

\begin{figure}[h]
      \caption{Distribution of a random powerlaw input with $\beta=-2.75$}
      \centering
      \includegraphics[width=0.7\textwidth]{figures/images/numberGenerator/powerlaw_-2_75.png}\label{fig:powerDistExample1}
\end{figure}

For a value of $\beta=-1.25$ the distribution looks a bit different.
There are less small values close to one and instead also big values even over 1000.
Figure~\ref{fig:powerDistExample2} is cropped to get a more clear view for the smaller values.
The higher values mostly occurred 0 to 2 times.
The highest value 8848 occurred only once.

\begin{figure}[h]
      \caption{Distribution of a random powerlaw input with $\beta=-1.25$}
      \centering
      \includegraphics[width=0.7\textwidth]{figures/images/numberGenerator/powerlaw_-1_25.png}\label{fig:powerDistExample2}
\end{figure}