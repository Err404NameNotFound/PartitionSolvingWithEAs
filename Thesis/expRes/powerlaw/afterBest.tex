The ranking is according to the amount of bits the algorithm flip on average per step.
$pmut_{-1.25}$ manages to find the solution in just 56 iterations on average.
The (1+1) EA with $p_m=50/n$ is slower than $pmut_{-1.25}$ but instead has a lower value for the maximum number of iterations.
Both options seem fine.
Even the $RLS-N_4$ is still very fast for the powerlaw distributed input with $\beta = -2.75$.
For $\beta = -1.25$ the results are a bit different.

\makebox[\linewidth]{
\begin{tabular}{lp{3cm}p{6cm}p{6cm}}
\begin{tabular}[h]{cccc}
algo type&            pmut&    RLS-R&    EA-SM\\
algo param&          -1.50&      r=4&      4/n\\
avg mut/change&     30.913&    2.438&    3.917\\
avg mut/step&       70.577&    2.501&    4.000\\
\hline
total avg count&       178&      251&      251\\
avg eval count&        178&      251&      251\\
max eval count&        629&      778&    1,018\\
min eval count&         17&       19&       13\\
\hline
fails&                   0&        0&        0\\
fail ratio&          0.000&    0.000&    0.000\\
avg fail dif&            -&        -&        -\\
\end{tabular}
\end{tabular}
}


The $RLS-R_4$ performs now better than the (1+1) EA variant with $p_m=4/n$, but is still slower than $pmut_{-1.5}$.
As the first inputs were less difficult to solve than the inputs with $\beta = -1.25$ the second was chosen for the fine evaluation.

\begin{tabular}[h]{cccccccc}
fails&20&50&100&500&1000&5000&10000\\\hline
RLS-N (2)&998&984&973&763&541&49&1\\
RLS-N (4)&998&989&975&837&719&150&27\\
(1+1) EA (3/n)&994&990&981&862&714&194&35\\
(1+1) EA (4/n)&998&993&976&852&725&180&51\\
pmut (-2.0)&997&989&983&905&784&229&55\\
pmut (-2.25)&997&988&987&875&790&213&56\\
\end{tabular}


The RLS is once again the algorithm that is the most likely to be stuck in a local optimum.
Compared to the other algorithms it is not as drastic as for the binomial input for example.
Only for $n<500$ the algorithms do not find a global optimum in every run.
The setting of the parameter almost doesn't affect the amount of runs without an optimal result.
The main differences are between the different algorithms themselves.

\begin{tabular}[h]{ccccccccc}
avg&20&50&100&500&1000&5000&10000&50000\\\hline
RLS-N (2)&172&1247&7598&38282&36214&105152&117792&125415\\
RLS-N (4)&8626&40966&43921&41924&42289&138187&201183&214170\\
(1+1) EA (3/n)&36470&42555&42439&43801&45530&145209&212281&252713\\
(1+1) EA (4/n)&37212&43144&44340&47246&46689&143697&220208&265620\\
pmut (-2.0)&40368&42601&39603&47443&46591&148390&232398&311606\\
pmut (-2.25)&39476&42113&40325&40548&43691&157816&226129&292367\\
\end{tabular}


The easiest are inputs with size $n=500$.
For smaller values of $n$ the algorithms sometimes fail and even in a good run the need more iterations to find an optimal solution.
Due to the increasing size of the input the algorithms need more time for the bigger values.

\begin{tabular}[h]{ccccccccc}
total avg&20&50&100&500&1000&5000&10000&50000\\\hline
RLS&98050&94574&89765&60172&39985&15639&4128&1797\\
\RLSR[2]&89401&54556&18266&4218&3530&2362&2160&2229\\
(1+1) EA (1$/n$)&54971&26688&15684&8452&6567&3815&3458&3371\\
(1+1) EA (2$/n$)&26104&9724&6508&4503&4020&3171&3141&3133\\
pmut (3.25)&40934&19972&10977&5644&4406&2434&2162&2172\\
pmut (3.0)&35272&17545&10040&5222&4150&2510&2208&2213\\
\end{tabular}



$pmut_-1.75$ is not only the best variant for the bigger values of n but also for smaller inputs as well.
It is the least likely to be stuck in a local optimum, and it is also the fastest if it reaches a global optimum.