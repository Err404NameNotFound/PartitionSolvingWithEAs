This input seems generally easy to solve as for every base algorithm multiple variants reach an optimal solution within 1000 steps.
The standard RLS reaches an optimal solution the fastest, but the other algorithms are almost equally fast.
All algorithms finish within 501 steps on average and always in less than 2500 steps.

\begin{tabular}[h]{ccccccccc}
fails in 1000 runs&20&50&100&500&1000&5000&10000&50000\\\hline
RLS&984&773&411&1&0&0&0&0\\
\RLSR[2]&890&241&14&0&0&0&0&0\\
(1+1) EA (1$/n$)&711&75&5&0&0&0&0&0\\
(1+1) EA (2$/n$)&541&14&0&0&0&0&0&0\\
pmut (3.0)&566&63&4&0&0&0&0&0\\
pmut (3.25)&587&63&7&0&0&0&0&0\\
\end{tabular}


This input is only hard to solve for $n<100$.
Only the RLS is stuck in a local optimum for many inputs.
The other algorithms manage to escape the local optima in most cases even for $n=100$.
For $n\ge500$ every input is solved by each of the chosen algorithms except for the standard RLS failing once for $n=500$.
This is probably caused by the many small values from the powerlaw and geometric distribution.
The longer the input the more of these small values can be used to fill the gaps.

\begin{tabular}[h]{ccccccccc}
avg&20&50&100&500&1000&5000&10000&50000\\\hline
RLS&32&79&153&579&950&1859&1922&1797\\
\RLSR[2]&391&2124&5005&4218&3530&2362&2160&2229\\
(1+1) EA (1$/n$)&22471&18343&12834&8342&6511&3815&3458&3371\\
(1+1) EA (2$/n$)&16360&9243&6452&4503&4020&3171&3141&3133\\
pmut (3.25)&23440&15929&9658&5644&4406&2434&2162&2172\\
pmut (3.0)&21901&14696&9186&5222&4150&2510&2208&2213\\
\end{tabular}


The RLS variants are only able to solve the inputs if they are lucky.
The (1+1) EAs and the $pmut$ variants also find other optimal solutions but need way more steps if they do so.
The input becomes drastically easier until $n=500$ and very slowly becomes harder with rising $n$ again.

\begin{tabular}[h]{ccccccccc}
total avg&20&50&100&500&1000&5000&10000&50000\\\hline
RLS&99470&93459&74385&3474&447&389&421&555\\
\RLSR[2]&96305&63088&19364&815&589&565&595&699\\
(1+1) EA (1$/n$)&89444&52886&21141&1410&869&850&875&1046\\
(1+1) EA (2$/n$)&79603&37214&14107&1302&992&942&957&1055\\
pmut (3.25)&84226&48403&18280&930&542&526&546&642\\
pmut (3.0)&82474&46286&17825&917&576&548&573&668\\
\end{tabular}



For $n\le100$ the (1+1) EA with $p_m=2/n$ performs the best.
It is still good for the bigger values of $n$ but after $n\ge500$ $pmut_{-3.0}$ reaches an optimal solution faster.
The standard RLS is only the fastest for a small range of values but has the disadvantage of the poor performance for small inputs.
Choosing one of the other variants is a much safer choice.