For this setting of $m=10000, p=0.1, n=10000$ the RLS-N with $k=2$ performs better than the  (1+1) EA and $pmut_\beta$ mutation for all values of $c/n$ and $\beta$ by a factor of at least 2.
This is likely from the fact that this version of the RLS flips almost only two bits which seems to be close to optimal for this kind of input.
There are many values close to the expected value which can be switched to make small adjustments to the fitness value.
The (1+1) EA with $p=3/n$ and $pmut_\beta$ algorithm with $\beta=-2.25$ perform almost the same.
The (1+1) EA has a slightly lower average value but also has a higher minimum value and a higher maximum value.
To further investigate which input performs best on all binomial distributed inputs now a comparison with different input lengths follows.
The parameters of the distribution were not changed.\newline
The following table is the amount of runs in which the algorithms did not find an optimal solution within 50,000 steps.
The time limit was set 50,000 because the algorithm normally reach the optimal solution within a few thousand steps.
If the solutions is not found after 50,000 steps, the algorithm is most likely stuck in a local optimum which could only be left by flipping more bits than possible for the algorithm.

\begin{tabular}[h]{cccccccc}
fails&20&50&100&500&1000&5000&10000\\\hline
RLS-N (2)&998&984&973&763&541&49&1\\
RLS-N (4)&998&989&975&837&719&150&27\\
(1+1) EA (3/n)&994&990&981&862&714&194&35\\
(1+1) EA (4/n)&998&993&976&852&725&180&51\\
pmut (-2.0)&997&989&983&905&784&229&55\\
pmut (-2.25)&997&988&987&875&790&213&56\\
\end{tabular}


The (1+1) EA and $pmut_\beta$ always reach an optimal solution but the RLS does not.
The RLS variants that can only flip two steps per step perform significantly worse for small inputs.
They are probably more likely to get stuck in a local optimum where a step flipping 4 bits or more would be necessary.
So the RLS-N(2) does perform better for larger inputs but is much more likely to get stuck in a local optima.
The next table contains the average number of iterations the algorithm needed to find an optimal solution for all runs where the algorithms managed to find an optimal solution.
Here it still looks like the RLS-N(2) finds the solution with the lowest amount of steps, because the cases where the algorithm is stuck in a local optima are not contained in this table.

\begin{tabular}[h]{ccccccccc}
avg&20&50&100&500&1000&5000&10000&50000\\\hline
RLS-N (2)&172&1247&7598&38282&36214&105152&117792&125415\\
RLS-N (4)&8626&40966&43921&41924&42289&138187&201183&214170\\
(1+1) EA (3/n)&36470&42555&42439&43801&45530&145209&212281&252713\\
(1+1) EA (4/n)&37212&43144&44340&47246&46689&143697&220208&265620\\
pmut (-2.0)&40368&42601&39603&47443&46591&148390&232398&311606\\
pmut (-2.25)&39476&42113&40325&40548&43691&157816&226129&292367\\
\end{tabular}


The next table contains the overall average amount of iterations for every run.
So runs where no optimal was found add 50000 thousand to the sum of all iterations.

\begin{tabular}[h]{ccccccccc}
total avg&20&50&100&500&1000&5000&10000&50000\\\hline
RLS&98050&94574&89765&60172&39985&15639&4128&1797\\
\RLSR[2]&89401&54556&18266&4218&3530&2362&2160&2229\\
(1+1) EA (1$/n$)&54971&26688&15684&8452&6567&3815&3458&3371\\
(1+1) EA (2$/n$)&26104&9724&6508&4503&4020&3171&3141&3133\\
pmut (3.25)&40934&19972&10977&5644&4406&2434&2162&2172\\
pmut (3.0)&35272&17545&10040&5222&4150&2510&2208&2213\\
\end{tabular}



Here the RLS-N(2) is only the best algorithm for values of $n \ge 500$.
Below this bound choosing the (1+1) EA with static mutation rate $3/n$ is a safer choice as the (1+1) EA reaches an optimal solution for every input (in this experiment).