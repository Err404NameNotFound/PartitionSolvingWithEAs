\documentclass{thesisclass}
% Based on thesisclass.cls of Timo Rohrberg, 2009
% ----------------------------------------------------------------
% Thesis - Main document
% ----------------------------------------------------------------


%% -------------------------------
%% |  Information for PDF file   |
%% -------------------------------
\hypersetup{
 pdfauthor={Daniel Lipp},
 pdftitle={EAs for Partition},
 pdfsubject={?},
 pdfkeywords={?}
}


%% ---------------------------------
%% | Information about the thesis  |
%% ---------------------------------

\newcommand{\myname}{Daniel Lipp}
\newcommand{\mytitle}{Theoretical and empirical runtime analysis of evolutionary algorithms for the PARTITION problem}
\newcommand{\myinstitute}{Chair of algorithms for intelligent systems}

\newcommand{\reviewerone}{?}
\newcommand{\reviewertwo}{?}
\newcommand{\advisor}{Prof.\ Dr.\ Dirk Sudholt}
\newcommand{\advisortwo}{?}

\newcommand{\timestart}{14th May 2023}
\newcommand{\timeend}{14th August 2023}


\usepackage{graphicx}
\usepackage{subcaption}
\usepackage{amsmath}
\usepackage{array}
\usepackage{wrapfig}
\usepackage{multirow}
\usepackage{array}  
\usepackage{calc}  
\usepackage{enumitem}
\newcommand{\probP}{\text{I\kern-0.15em P}}
\newcommand{\RLSR}[1][k]{RLS$^{S}_{#1}$}
\newcommand{\RLSN}[1][k]{RLS$^{B}_{#1}$}
\newcommand{\TODO}[1]{\textbf{\newline\textit{TODO:\ #1}}}
\newcommand{\etal}{\emph{et.\ al.\ }}
\newcommand{\N}{\mathbb{N}}
\newcommand{\bigO}{\mathcal{O}}

%% ---------------------------------
%% | Commands                      |
%% ---------------------------------

\newtheorem{definition}{Definition} \numberwithin{definition}{chapter}
\newtheorem{theorem}[definition]{Theorem}
\newtheorem{lemma}[definition]{Lemma}
\newtheorem{corollary}[definition]{Corollary}
\newtheorem{conjecture}[definition]{Conjecture}


%% --------------------------------
%% | Settings for word separation |
%% --------------------------------
% Help for separation:
% In german package the following hints are additionally available:
% "- = Additional separation
% "| = Suppress ligation and possible separation (e.g. Schaf"|fell)
% "~ = Hyphenation without separation (e.g. bergauf und "~ab)
% "= = Hyphenation with separation before and after
% "" = Separation without a hyphenation (e.g. und/""oder)

% Describe separation hints here:
\hyphenation{
% Pro-to-koll-in-stan-zen
% Ma-na-ge-ment  Netz-werk-ele-men-ten
% Netz-werk Netz-werk-re-ser-vie-rung
% Netz-werk-adap-ter Fein-ju-stier-ung
% Da-ten-strom-spe-zi-fi-ka-tion Pa-ket-rumpf
% Kon-troll-in-stanz
}


%% ------------------------
%% |    Including files   |
%% ------------------------
% Only files listed here will be included!
% Userful command for partially translating the document (for bug-fixing e.g.)
\includeonly{
chapters/titlepage,
chapters/introduction,
chapters/preliminaries,
chapters/proofs,
chapters/ExperimentalResults,
chapters/HeavyTailedMutations,
conclusion,
appendix
}


%%%%%%%%%%%%%%%%%%%%%%%%%%%%%%%%%
%% Here, main documents begins %%
%%%%%%%%%%%%%%%%%%%%%%%%%%%%%%%%%
\begin{document}

% Remove the following line for German text
\selectlanguage{english}

\frontmatter
\pagenumbering{roman}
%% titlepage.tex
%%

\begin{titlepage}

  \iffalse
  \begin{textblock}{10}[0,0](4,2.5)
		\includegraphics[]{logos/UPLogo.pdf}
	\end{textblock}
        \begin{textblock}{10}[0,0](14.5,2.45)
          \includegraphics[]{logos/UPLogo.pdf}
	\end{textblock}
\fi

        \changefont{phv}{m}{n}	% helvetica	
	\vspace*{3.75cm}
	\begin{center}
		\Huge{\mytitle}
		\vspace*{2.25cm}\\
		\Large{
			\iflanguage{english}{Bachelor Thesis of}			
												  {Masterarbeit\\von}
		}\\
		\vspace*{1cm}
		\huge{\myname}\\
		\vspace*{1cm}
		\Large{
			\iflanguage{english}{At the Department of Informatics and Mathematics}			
													{An der Fakult\"at f\"ur Informatik und Mathematik}
			\\
			\myinstitute\\
                      }
	\end{center}
        \begin{center}
        \includegraphics[]{logos/UPLogo.pdf}
      \end{center}

	\vspace*{1cm}
                      

        \Large{
\begin{center}
\begin{tabular}[ht]{l c l}
  % Gutachter sind die Professoren, die die Arbeit bewerten. 
%   \iflanguage{english}{Reviewers}{Erstgutachter}: & \hfill & \reviewerone\\
%   \iflanguage{english}{}{Zweitgutachter:} & \hfill & \reviewertwo\\
  \iflanguage{english}{Advisors}{Betreuende Mitarbeiter}: & \hfill & \advisor\\
%   \iflanguage{english}{}{} & \hfill & \advisortwo\\
  % Betreuende Mitarbeiter wenn nicht vorhanden ggf. weglassen. 
\end{tabular}
\end{center}
}


\vspace{2cm}
\begin{center}
\large{\iflanguage{english}{Time Period}{Bearbeitungszeit}: \ \timestart{} \ -- \ \timeend}
\end{center}

\end{titlepage}

\blankpage

%% -------------------------------
%% |   Statement of Authorship   |
%% -------------------------------

\thispagestyle{plain}

\vspace*{\fill}

\centerline{\textbf{Statement of Authorship}}

\vspace{0.25cm}

I hereby declare that this document has been composed by myself and describes my own work, unless otherwise acknowledged in the text.

\vspace{2.5cm}

\hspace{0.25cm} Passau, \today

\vspace{2cm}

\blankpage

%% -------------------
%% |   Abstract      |
%% -------------------

\thispagestyle{plain}

\begin{addmargin}{0.5cm}

      \centerline{\textbf{Abstract}}

      When dealing with $NP$-hard problems calculating an exact solution will take exponential time for some inputs unless $P=NP$.
      One possible way of dealing with the high runtime is abandoning the optimality for a better runtime which is the concept of approximation-algorithms.
      Randomised Search Heuristics such as Evolutionary Algorithms can be used for this purpose.
      In this thesis the runtime of multiple Evolutionary Algorithms such as the (1+1) EA, variants of the RLS and the $pmut$ mutation operator is analysed theoretically and empirically on PARTITION.\ 
      Some of the previously known bounds for the (1+1) EA and RLS are improved and expanded to different version of those algorithms.
      In the end there is an empirical study on the best algorithm variant depending on input type and size revealing the optimal mutation rate is dependent on the input and not always the same for every input of PARTITION.\

      \centerline{\textbf{Deutsche Zusammenfassung}}

      $NP$-schwere Probleme lassen sich nicht immer in polynomieller Zeit lösen, es sei denn $P=NP$ gilt.
      Eine Möglichkeit die exponentielle Laufzeit zu vermeiden ist es statt der optimalen Lösung auch schlechtere Lösungen zu akzeptieren, was dem Prinzip der Approximationsalgorithmen entspricht.
      Randomised Search Heuristics wie Evolutionäre Algorithmen können für diesen Zweck verwendet werden.
      In dieser Bachelorarbeit wird die Laufzeit mehrerer Evolutionärer Algorithmen wie dem (1+1) EA, Varianten des RLS sowie des $pmut$ Operators theoretisch und empirisch analysiert für PARTITION.\ 
      Bereits bestehende Ergebnisse werden verschärft für den (1+1) EA und den RLS und erweitert für Varianten der beiden Algorithmen.
      Am Ende der Arbeit ist eine empirische Analyse, die die verschiedenen Algorithmen auf verschiedenen Eingabetypen und Eingabegrößen vergleicht.
      Ein der vielen Erkenntnisse ist, dass die optimale Mutationsrate abhängig von der Art der Eingabe ist und nicht global für PARTITION gilt.


\end{addmargin}

\blankpage

%% -------------------
%% |   Directories   |
%% -------------------

\tableofcontents
\blankpage

%% -----------------
%% |   Main part   |
%% -----------------

\mainmatter
\pagenumbering{arabic}
%% introduction.tex
%%

%% ==============================
\chapter{Introduction}\label{ch:introduction}
%% ==============================

% This chapter should contain
% \begin{enumerate}
%   \item A short description of the thesis topic and its background.
%   \item An overview of related work in this field.
%   \item Contributions of the thesis.
%   \item Outline of the thesis.
% \end{enumerate}

The question of $P=NP$ is still unanswered to this day and solving $NP$-hard problems for every instance still requires exponential time.
To avoid the exponential running time on $NP$-hard optimisation problems one might use approximation algorithms.
Those algorithms do not always return the best possible solution but only a solution with a guaranteed solution quality.
For a minimisation problem a (1+0.5)-approximation algorithm will always return a solution that has at most 1.5 times the optimal value.
An example of an $NP$-hard optimisation problem is PARTITION.\ 
An instance of this problem is a multiset of $n$ positive numbers $\{w_1,\dots,w_n\}$ which has to be divided in two subsets with sums that are as close as possible.
So a solution of PARTITION is a subset of $I\subset \{1,\dots,n\}$ which splits the multiset into two subsets.
The quality of the solution then is $\max\{\sum_{i\in I}w_i, \sum_{i\notin I}w_i\}$.
PARTITION is one of the easiest $NP$-hard problems and has even been dubbed the easiest $NP$-hard problem~\cite{hayes2002computing}.
There are multiple algorithms specifically designed for PARTITION.\ 
Some of them return approximations and others always return the best solution.
The exact algorithms have a runtime exponential in the input size due to the $NP$-hardness.
Problem specific approximation algorithms are mostly deterministic such as the greedy method, the KK-algorithm or the FPTAS for the subsetsum problem which can be used for partition as well.
Another class of non-deterministic algorithms are the so-called Evolutionary Algorithms which mimic the behaviour of evolution.
Those algorithms start with a random population of solutions which are then changed with random steps.
If the solution is good enough it survives and replaces one of the worst individuals in the population.
The EA continues to generate new offspring in an endless loop.
In practice the algorithm is given stopping conditions such as reaching a number of iterations or a specific solution quality.
This is the principle of an anytime algorithm which can be terminated at anytime and output a valid solution.
The longer the waiting time the better the solution might get.
The main usage of EA lies in problems without a problem specific algorithms as these mostly outperform the EAs.
To better understand their behaviour analysing them on well researched problems might still be beneficial to learn more about this class of algorithms.
This thesis researches the runtime of basic EAs such as (1+1) EA and variations of the RLS.\ 
The first part is a theoretical analysis with a main focus of lowering bounds that were previously shown.
Additionally there are new results for other algorithm variants and also a lemma on different type of inputs.
The remainder of this thesis is an empirical analysis of multiple base algorithms with different parameter setting on different kinds of inputs.
Here not only the (1+1) EA with different mutation rate and variants of the RLS with different parameter values are researched but also a heavy tailed mutation operator.
Apart from typical distributions such as the uniform, geometric and binomial distribution there are also results for problem specific instances such as an input where one values is as large as all other values combined.
In the end the empirical results are condensed into a personal suggestion which algorithm should best be chosen to solve the problem, depending on the input but also in general.


%% preliminaries.tex
%%

%% ==============
\chapter{Preliminaries}\label{ch:preliminaries}
%% ==============

%This chapter should provide the foundations of the thesis.
\section{Notation}
A short list of vocabulary used throughout the paper.
\begin{description}[leftmargin=!,labelwidth=\widthof{\bfseries RSH}]
    \item[EA] Evolutionary Algorithm
    \item[RSH] Randomised Search Heuristic referring to all analysed Evolutionary algorithms
    \item[$n$] The input length of the problem
    \item[$w_i$] The $i$-th object of the input. If not mentioned otherwise the weights are sorted in non-increasing order so: \(w_1 \ge w_2 \ge \ldots \ge w_{n-1} \ge w_{n}\)
    \item[$W$] The sum of all objects: W = $\sum_{i=1}^{n}w_i$
    \item[bin] When solving Partition a set of numbers is divided into two distinct subsets and in this paper both subsets are referred to as bins
    \item[$b_F$] The fuller bin (the bin with more total weight)
    \item[$b_E$] The emptier bin (the bin with less total weight)
%     \item[$b_{w_i}$] The bin containing the object $w_i$
    \item[$opt$] The optimal solution for a given partition instance.
    \item[$x$] A vector $x \in {\{0, 1\}}^n$ describing a solution
    \item[$f(x)$] The fitness function for PARTITION.\ This means a solution $x$ has a solution quality of \(f(x)=\max\{\sum_{i=1}^{n}w_i\cdot x_i, \sum_{i=1}^{n}w_i\cdot(1-x_i)\}=b_F\)
\end{description}

\section{Background}
\subsection{Known algorithms for partition}
Multiple methods for generating a solution of PARTITION already exist.
Solving the problem with a greedy approach in runtime $\mathcal{O}(n)$ results in an approximation ratio of 3/2 if the elements are not sorted or a ratio of 7/6 if the numbers are sorted~\cite{graham1966bounds}.
Greedy in this case means putting each element in the currently emptier set while looking at each value exactly once.
Another approximation algorithm is the KK-algorithm or also called Largest Differencing Method.
With expected time $\mathcal{O}(n\log{}n)$ it has the same runtime as greedy with sorting and also the same worst case approximation of 7/6.
For inputs chosen uniform random from [0,1] the KK has an expected ratio of \(1+\frac{1}{n^{\Theta(\log{}n)}}\) in comparison to the greedy algorithm which only reaches an approximation ratio of \(1+\mathcal{O}(\frac{1}{n})\).
Instead of putting each element in the currently emptier set the currently largest two values are combined to one value by either subtracting or adding them depending on which results in a better solution.
Adding them corresponds to putting the elements in the same set whereas subtracting them means putting them in different sets. 
There is even a fully polynomial time approximation algorithm (FPTAS) for the subsetsum problem~\cite{KELLERER2003349} which can be used for PARTITION by setting the required sum to $\lfloor W/2\rfloor$.
FPTAS return a solution of at most (1+$\epsilon$) the optimum in a time that is polynomial both in $n$ and in $\frac{1}{\epsilon}$.
There are lots of other approximation-algorithms but also some algorithms that always return the best solution.
The Pseudopolynomial time number partitioning algorithm which uses dynamic programming always returns an optimal solution but needs time and space $\mathcal{O}(n\frac{m}{2})$ where $m$ is the largest number in the input~\cite{korf2009multi}.
The runtime is only pseudopolynomial because to encode $m$ in the input only $\log_{2}{(m)}$ bits are required which causes \(m=2^{\log_{2}{(m)}}\) to be exponential in the input size.
The Complete Greedy Algorithm (CGA)~\cite{korf1998complete} traverses a binary tree depth first and searches the complete $2^n$ search space in a greedy way.
It functions the same way as the simple greedy algorithm but instead of only looking at only the greedy option at each height it also evaluates the non-greedy option after evaluating the complete subtree of the greedy option.
The algorithm continues the depth first search until it either finds a perfect partition or has traversed the whole tree.
In the second case it will return the best value found on the way.
While the space complexity in only $\mathcal{O}(n)$ the runtime is $\mathcal{O}(2^n)$.
Another exact algorithm is the Complete Karmarkar-Karp (CKK)~\cite{korf1998complete}.
This algorithm works similar to the complete greedy approach by traversing the binary tree of all solutions.
Instead of greedily selecting the next edge here the algorithm behaves like the KK-algorithm described above.
It performs better than the CGA for the same reasons as before but also has the same worst case running time as the GCA.

\subsection{Evolutionary Algorithm}
Evolutionary Algorithms mimic the process of evolution and normally behave mostly the same.
A run typically looks like this:
\begin{enumerate}
      \item Generate initial population at random
      \item If stopping condition are met return the currently best solution
      \item Generate offspring population (e.g.\ by mutation)
      \item Evaluate fitness of the offspring
      \item Select fittest individuals and update population
      \item Go back to step 2.
\end{enumerate}
For PARTITION a solution $x\in{\{0,1\}}^{n}$ separates all numbers into two different sets with $x_i=0$ meaning $w_i$ is in set 0 whereas $x_i=1$ meaning $w_i$ is in set 1.
So every possible value of $x$ describes a feasible solution but not necessarily a good one.
To evaluate the quality of a solution the EA is given a fitness function.
The fitness function in this case is \(f(x)=\max\{\sum_{i=1}^{n}w_i\cdot x_i, \sum_{i=1}^{n}w_i\cdot(1-x_i)\}=b_F\).
The goal of the algorithm is to return a solution with minimal fitness.
A mutation step in the PARTITION problem will change an algorithm-dependent number of bits from 1 to 0 or vice versa.
In this case flipping a bit means putting the element in the other set.
A simple implementation of an EA is the so called (1+1) EA (Algorithm~\ref{alg:EA}).\
The first 1 in the brackets refers to size of the population and the second to the amount of mutants created in each iteration of the loop.
So it always has only one solution and generates just one new solution in each step.
The mutation of the current individual is performed by flipping each bit independently with probability $1/n$.
The amount of flipped bits is binomial distributed with an expected value of $n\cdot\frac{1}{n}=1$.
% By changing the mutation rate $1/n$ to $c/n$ for any constant $c$ the algorithm now flips $n\cdot c/n=c$ bits in expectation.\newline
\begin{algorithm}[bt]
      \caption{\textsc{(1+1) EA}}\label{alg:EA}

      % Some settings
      \DontPrintSemicolon %dontprintsemicolon
      \SetFuncSty{textsc}

      % The algorithm
      \BlankLine
      choose x uniform from ${\{0,1\}}^n$\;
      \While{$x$ not optimal}
      {
      $x' \leftarrow x$\;
      flip every bit of $x'$ with probability $1/n$\;
      {
      \If{$f(x') \le f(x)$}
      {
            $x \leftarrow x'$\;
      }
      }
      }
\end{algorithm}
Another simple EA is the Randomised Local Search (RLS see Algorithm~\ref{alg:RLS}).\
Instead of flipping each bit with probability $1/n$ here one bit is flipped at uniform random.
Apart from that the algorithms are the same.
\begin{algorithm}[bt]
      \caption{\textsc{RLS}}\label{alg:RLS}

      % Some settings
      \DontPrintSemicolon %dontprintsemicolon
      \SetFuncSty{textsc}

      % The algorithm
      \BlankLine
      choose x uniform from ${\{0,1\}}^n$\;
      \While{$x$ not optimal}
      {
      $x' \leftarrow x$\;
      flip one uniform random bit of $x'$\;
      {
      \If{$f(x') \le f(x)$}
      {
            $x \leftarrow x'$\;
      }
      }
      }
\end{algorithm}

\subsection{Literature on the RLS and (1+1) EA for PARTITION}
Carsten Witt proved that the RLS and the (1+1) EA find a $(4/3+\epsilon)$ approximation in expected time $\bigO(n)$ and a $(4/3)$-approximation in expected time $\bigO(n^2)$~\cite{witt2005worst}.
He then introduced an almost worst case input to prove the bound for the approximation ratio is at least almost tight.
The input is defined as followed for any $0<\epsilon<1/3$ and even $n$:\newline
The input contains two numbers of value $1/3 - \epsilon/4$ and $n-2$ elements of value $(1/3+\epsilon/2)/(n-2)$. 
The total volume is normalised to 1.
When the two large values are in the same bin, the RSHs are tricked into a local optimum, where only $w_1$ and $w_2$ are in the first bin and the remaining elements in the other bin.
This results in an almost worst case.
To leave this local optimum $\Omega(n)$ bits must be moved in a step separating the two large values.
Such a step will never happen for the RLS and only in expected exponential time for the (1+1) EA.\
This worst case happens with probability $\Omega(1)$.
He also proved both RSHs return a (1+$\epsilon$)-approximation for $\epsilon\ge4n$ in expected time \(\lceil en\ln(4/\epsilon)\rceil\) for the (1+1) EA and \(\lceil en\ln(4/\epsilon)\rceil\) for the RLS with probability at least \(2^{-(e\log{e}+e)\lceil 2/\epsilon\rceil \ln(4/\epsilon)-\lceil 2/\epsilon\rceil}\) for the (1+1) EA and at least \(2^{-(\log{e}+1)\lceil 2/\epsilon\rceil \ln(4/\epsilon)-\lceil 2/\epsilon\rceil}\) for the RLS.\
Afterwards he proved both RHSs reach a solution where the difference between the two bins is at most 1 for uniform distributed inputs on [0,1] after expected time $\bigO(n^2)$ for the (1+1) EA and $\bigO(n\log{n})$ for the RLS.\
The difference between the two bis is even bounded by $\bigO(\log{n}/n)$ after $\bigO(n^{c+4}\log{n})$ steps with probability at least $1-\bigO(1-1/n^c)$.
This leads to an expected difference of $\bigO(\log{n}/n)$ after $\bigO(n^{c+4}\log{n})$ steps.
He also analysed exponential distributed inputs with parameter 1.
With probability $1-\bigO(1/n^c)$ the difference on those inputs is bounded by $\bigO(\log{n})$ after $\bigO(n^2\log{n})$ steps and even by $\bigO(\log{n}/n)$ after $\bigO(n^{c+4}\log^2{n})$ steps.
Additionally he described a polynomial time randomised approximation scheme (PRAS) for the RLS and the (1+1) EA for values of $\epsilon=\Omega(\log{\log{}}n/\log{n})$.\newline
For MAKESPAN-SCHEDULING a list of processing times has to be distributed on a set of machines while minimising the total time of the fullest machine.
With 2 machines this problem is exactly the same as PARTITION.\
So in a sense MAKESPAN-SCHEDULING is a more general version of PARTITION.\
This lead to Christian Gunia generalising some results previously shown by C. Witt to MAKESPAN-SCHEDULING on $k$ machines~\cite{gunia2005analysis}.
Solutions for MAKESPAN-SCHEDULING are \(x\in{\{0,\dots,k-1\}}^n\) and therefore during a mutation $x_i$ is set to a uniform random value from $\{0,\dots,k-1\}\text{\textbackslash}\{x_i\}$ instead of $1-x_i$.
The adapted RSHs reach an approximation ratio of $(2k/k+1)$ in expected time $\bigO(Wn^{2k-2}/w_n)$.
On an instance where every weight is the same the expected optimisation time is bounded by $\bigO(n\log{n})$.
He also adapted the almost worse case to the more general problem and proved the RLS does not find a solution better than \((2k/(k+1)-\epsilon)\) in finite time for any $\epsilon>0$ with constant probability.
The second statement for PARTITION on the uniform distributed inputs on [0,1] has an equivalent lemma for MAKESPAN-SCHEDULING on $k$ machines as well.

Another way of dealing with $NP$-hard problems is identifying a parameter $k$ which defines how hard the problem is to solve.
One possible parametrisation of PARTITION is solving whether there is a solution of $f(x)\le k$.
A fixed-parameter tractable problem is a problem that can be solved in time polynomial in the size of the input and $g(k)$ where $g$ is any arbitrary function.
Partition falls into this category as it can be decided in time at most $\bigO(4^k)$\cite{fernau2005parameterized}.
Andrew M. Sutton and Frank Neumann made a parametrised analysis of PARTITION\cite{sutton2012parameterized}.\
They parametrised the problem in multiple ways.
One of their parametrisation was: given an integer $k$, is there a solution of at most $W/2+W/k$?
They showed that a multistart of the (1+1) EA or RLS using runs of length \(\lceil en\ln(2k)\rceil\) is a Monte Carlo-fpt algorithm for this parametrised version of PARTITION.\
They also analysed a parametrisation of the size of the critical path and also the discrepancy (the difference between the two bins).

\subsection{Problems for Evolutionary Algorithms}
There are two well analysed problems for Evolutionary Algorithms which are relevant for this thesis.
One of those is OneMax.
For OneMax the fitness function $f(x)=\sum_{i=1}^{n}{x_i}$ has to be maximised.
In the optimal solution every bit is set to one.
This problem is rather easy as it has only one global optimum and no local optimum.
So every step the algorithms makes decreases the Hamming distance to the optimum if the fitness increases.
A more general version of this problem are linear functions where every bit is given a weight.
Here the fitness function is $f(x)=\sum_{i=1}^{n}{w_i \cdot x_i}$ which either has to be maximised or minimised.
The weights $w_i$ can be any real value.
In contrast to OneMax increasing the fitness can increase the Hamming distance to the global optimum if multiple small values switch places with a big value.
The runtime for both the RLS and the (1+1) EA is $\Theta(n\log{n})$ for both problems and the optimal mutation rate for the (1+1) EA is $1/n$ which was proven multiple times (\cite{witt2013tight},~\cite{doerr2023tight}).
The optimality of $1/n$ does not hold for every problem which leads to the next section.

\section{Higher mutation rates and heavy tailed mutations}
The RLS and the standard (1+1) EA flip one bit in expectation which is optimal for some problems as seen in the last section.
This is not the case for every fitness function.
$\text{Jump}_k$ has an optimal mutation rate of $k/n$ and a small constant factor deviation from $k/n$ results in an increase of the runtime exponential in $\Omega(k)$\cite{doerr2017fast}.
The same might hold for PARTITION, because the previously discussed literature only analyses mutation rates with a 1-bit flip in expectation.\newline
One way of creating algorithms with higher mutation rates is adjusting the currently existing algorithms.
For the (1+1) EA this can be done easily.
By changing the mutation rate $1/n$ to $c/n$ for any constant $c$ the algorithm now flips $n\cdot c/n=c$ bits in expectation.
\begin{algorithm}[bt]
      \caption{\textsc{(1+1) EA with static mutation rate}}\label{alg:EA_SM}

      % Some settings
      \DontPrintSemicolon %dontprintsemicolon
      \SetFuncSty{textsc}

      % The algorithm
      \BlankLine
      choose x uniform from ${\{0,1\}}^n$\;
      \While{$x$ not optimal}
      {
      $x' \leftarrow x$\;
      flip every bit of $x'$ with probability $c/n$\;
      {
      \If{$f(x') \le f(x)$}
      {
            $x \leftarrow x'$\;
      }
      }
      }
\end{algorithm}

For the RLS it is not that simple, as the RLS chooses a random bit and flips it.
Instead of flipping c bits in every step there should be the possibility to flip different amounts of bits in every step.
The standard RLS chooses a random neighbour with Hamming distance one.
So the variant of the RLS could simply choose neighbours that have a Hamming distance larger than one.
The selection should still be uniform random to keep the idea of the RLS intact.
One possible way is to choose a random neighbour with Hamming distance $\le k$.
This algorithm will be called \RLSN[k] from now on, because it chooses a random neighbour within the Hamming ball with radius $k$.
The amount of neighbours with Hamming distance $y$ is given by $\binom{n}{y}$.
For $k=4$, this results in $n$ neighbours with Hamming distance 1, $n(n-1)/2$ neighbours with Hamming distance 2, $n(n-1)(n-2)/6$ for 3
and $n(n-1)(n-2)(n-3)/24$ for 4.
The probability to choose a random neighbour with Hamming distance $y \le k$ for $k = \mathcal{O}(1)$ is given by
\[P(\text{\RLSN}\text{ flips }y\text{ bits}) = \frac{\binom{n}{y}}{\sum_{i=1}^k \binom{n}{i}} = \frac{\Theta(n^y)}{\sum_{i=1}^k \Theta(n^i)}
      = \frac{\Theta(n^y)}{\Theta(n^k)} = \Theta(n^{y-k}) = \Theta(\frac{1}{n^{k-y}})\]
This variant of the RLS is likely to choose a neighbour with Hamming distance k as the number of neighbours with hamming
distance $k$ rises with $k$ for $k \le n/2$.
The probability of flipping only one bit is $\mathcal{O}(\frac{1}{n^{k-1}})$.
For some inputs flipping only one bit might be more optimal which is rather unlikely for this variant of the RLS.\newline
An alternative way of changing the RLS is to first choose $y \in \{1, \dots, k\}$ uniform random and then choose a neighbour with Hamming distance $y$ uniform random.
Here the probability of flipping $y \le k$ bits is given by $1/k$, so the algorithm is much more likely to choose to flip only one bit.
This variant of the RLS will be referred to as \RLSR[k] because it first choses the Hamming sphere and afterwards the neighbour within the selected Hamming sphere.

\begin{algorithm}[bt]
      \caption{\textsc{\RLSN}}\label{alg:rlsN}

      % Some settings
      \DontPrintSemicolon %dontprintsemicolon
      \SetFuncSty{textsc}

      % The algorithm
      \BlankLine
      choose x uniform from ${\{0,1\}}^n$\;
      \While{$x$ not optimal}
      {
      $x' \leftarrow \text{uniform random neighbour of x with Hamming distance} \le k$\;
      {
      \If{$f(x') \le f(x)$}
      {
            $x \leftarrow x'$\;
      }
      }
      }
\end{algorithm}

\begin{algorithm}[bt]
      \caption{\textsc{\RLSR}}\label{alg:rlsR}

      % Some settings
      \DontPrintSemicolon %dontprintsemicolon
      \SetFuncSty{textsc}

      % The algorithm
      \BlankLine
      choose x uniform from ${\{0,1\}}^n$\;
      \While{$x$ not optimal}
      {
      $y \leftarrow \text{uniform random value }\in \{1,\dots,k\}$\;
      $x' \leftarrow \text{uniform random neighbour of x with Hamming Distance } y$\;
      {
      \If{$f(x') \le f(x)$}
      {
            $x \leftarrow x'$\;
      }
      }
      }
\end{algorithm}

Both variants of the RLS change at most $k$ bits in each step and therefore only a constant amount of bits.
For the (1+1) EA the algorithm will also flip mostly $\mathcal{O}(c)$ bits which is also constant.
So neither of the new variants is likely to change up to $\Theta(n)$ bits.
Quinzan \textit{et al.} therefore introduced another mutation operator in~\cite{friedrich2018evolutionary} called $pmut_\beta$.
This operator chooses $k$ from a powerlaw distribution $D^\beta_n$ with exponent $\beta$ and maximum value $n$ and then flips $k$ uniform random bits.
This algorithm will mostly flip a small number of bits but occasionally up to n bits.
Mutation operators like this are called heavy tailed mutations because their tail is not bounded exponentially.
\chapter{Runtime Analysis of different EAs on PARTITION}\label{ch:Content1}

This chapter focuses on the theoretical analysis of the runtime of Evolutionary Algorithms for PARTITION.\
The first section is focused on improving previously shown bounds for the (1+1) EA and the RLS.\
Afterwards there is also an analysis of Evolutionary Algorithms that flip more than one bit in expectation.
The last section analyses inputs that follow a binomial distribution.

\section{Improving bounds on the RLS and the (1+1) EA}
As discussed in the Background section C. Witt already showed multiple results for the RLS and the standard (1+1) EA.\
One of the results is the approximation ratio of at most 4/3 after expected time at most $\bigO(n^2)$.
In this section this bound will be lowered to $\bigO(n\log{n})$ for both algorithms.
The special input with $w_1\ge W/2$ is also analysed for the (1+1) EA with mutation rate $c/n$.
For the analysis of mutation rates $c/n$ the work of C. Witt only has to be slightly adjusted.
The first Lemma is a modification of a Lemma in~\cite{witt2005worst}.

\begin{lemma}\label{lemma:CWittRefined}
    Let $w_1\ge W/2$, then for any $\gamma$ > 1 and 0 < $\delta$ < 1, the (1+1) EA with mutation rate $c/n$ with constant $0<c<\sqrt{n}$ (\RLSR[k]) reaches an f-value at most $w_1$ + $\delta(W-w_1)$ in at most $\lceil\frac{e^c}{c\cdot(1-o(1))}n\ln(\gamma/\delta)\rceil$ $(\lceil kn\ln(\gamma/\delta)\rceil)$ steps with probability at least $1-\gamma^{-1}$. Moreover, the expected number of steps is at most $2\lceil\frac{e^c}{c\cdot(1-o(1))}n\ln(2/\delta)\rceil$ $(2\lceil kn\ln(2/\delta)\rceil)$.
\end{lemma}
\begin{proof}
    This Lemma is very similar to C. Witt's  Lemma 2 from section `2. Definitions and Proof Methods' in~\cite{witt2005worst}.
    The proof is mostly the same.
    He first defines a potential function $p(x)=f(x)-f(opt)$.
    While $p(x)>0$ all steps moving only a small object to the emptier bin are accepted.
    The expected p-decrease is at least $p_0\cdot q$ where $q$ is a lower bound on the probability of the algorithm to flip one specific bit.
    This leads to a next $p$ value of $(1-q)p_0$.
    Since all steps of both algorithms are independent this argumentation remains valid even if the $p$ value is only an expected value.
    With $q=1/yn$ for a constant $y>0$ the expected $p$ value after $t=yn\ln(\gamma/\delta)$ steps is at most
    \[p_t\le p_0{(1-1/yn)}^t=p_0{(1-1/yn)}^{yn\ln(\gamma/\delta)}\le p_0\cdot e^{-\frac{1}{y}\cdot yn\ln(\gamma/\delta)}=p_0{(\gamma/\delta)}^{-1} = p_0(\delta/\gamma)\]
    Applying Markov's inequality to the non-negative $p$ value implies $p_t\le p_0\delta$ with probability $1-1/\gamma$.
    Repeating independent phases of length $\lceil yn\ln(2/\delta)\rceil$ the expected number of phases is at most 2.
    Up until here the proof is the same.\newline
    Instead of choosing $W/2$ as the general upper bound for $p_0$ as in the original lemma here the lower value $W-w_1\le W/2$ is chosen because it is more tight for the special case $w_1\ge W/2$ with $f(opt)=w_1$.
    The probability of the \RLSR[k] to flip one specific bit is \(\frac{1}{k}\cdot\frac{1}{n}\) and for (1+1) EA with mutation rate $c/n$ at least
    \[
        \frac{c}{n}{(1-\frac{c}{n})}^{n-1}
        \ge \frac{c}{n}{(1-\frac{c}{n})}^{n}
        \ge \frac{c}{n}e^{-c}(1-\frac{c^2}{n})
        = \frac{c}{e^c n}(1-o(1))
        % =\frac{c}{n}\cdot\frac{1}{1-\frac{c}{n}}{(1-\frac{c}{n})}^{n}
        % \ge\frac{c}{e^c n}\cdot(1+\frac{\frac{c}{n}}{1-\frac{c}{n}})
        % =\frac{c}{e^c n}\cdot(1+\frac{1}{\frac{n}{c}-1})
        % =\frac{c}{e^c n}\cdot(1+o(1))
    \]
    The inequality \({(1+x/n)}^n\ge e^x (1-{x^2}/n)\) requires $n\ge1, |c|\le n$ which both hold.
    Setting $y=\frac{e^c n}{c\cdot(1-o(1))}$ for the (1+1) EA and $y=k$ for the \RLSR~concludes the result.
\end{proof}

The next Lemma also analyses inputs with $w_1>W/2$.
Its goal is to bound the probability of the (1+1) EA with mutation rate $c/n$ and the \RLSR[k\ge2] of flipping the first bit after a certain solution quality has been reached.
Inputs with $w_1>W/2$ are similar to linear functions which strongly suggest a runtime of $\bigO(n\log{n})$ without flips of $w_1$.
If such a step is too unlikely the algorithms might find an optimal solution before $w_1$ is flipped and the Hamming distance to the optimum might increase drastically.

\begin{lemma}\label{lemma:W1FlipWontHappen}
    For instances with $w_1>W/2$ the probability of flipping $w_1$ when $b_E = c\cdot\frac{W-w_1}{2}$ for $1<c<2$ holds, is at most \(\frac{2y{(y-1)}^2}{n(n-1)(c-1)}\) in a step where any algorithm flips $2\le y\le n/2$ bits.
\end{lemma}
\begin{proof}
    For a successful flip of $w_1$ after $b_E \ge \frac{W-w_1}{2}$ holds a total volume of at least $z\ge2\cdot(b_E-\frac{W-w_1}{2})$ must be shifted from $b_E$ to $b_F$.
    Otherwise the step is rejected because
    \[b_F'=b_E+w_1-z>b_E+w_1-2\cdot(b_E-\frac{W-w_1}{2})=b_E+w_1-2b_E+W-w_1=W-b_E=b_F\]
    which results in an increase of the fitness ($b_F = f(x), b_F' = f(x')$).
    Let $I$ be the set of indices of all elements moved from $b_E$ to $b_F$ and $w_{\max}=\max{\{w_i|i\in I\}}$.
    $|I|\le y-1$ holds because at least $w_1$ is moved from $b_F$ to $b_E$.
    The sum of all elements is at most $w_{\max} \cdot |I| \le (y-1)w_{\max}$.
    If \(w_{\max}<2\cdot(b_E-\frac{W-w_1}{2})/(y-1)\) then \((y-1)w_{\max}<2\cdot(b_E-\frac{W-w_1}{2})\) and the step is rejected.
    Thus at least one of the objects moved from $b_E$ to $b_F$ must have a volume of at least $2\cdot(b_E-\frac{W-w_1}{2})/(y-1)$.
    At most \(d\le\frac{b_E}{w_{\max}}\) of these objects can be in $b_E$ if they made up the complete volume of $b_E$. Simplifying this inequality leads to at most
    \[
        d \le \frac{b_E}{w_{\max}}
        \le \frac{W-w_1}{w_{\max}}
        \le \frac{W-w_1}{2(c\frac{W-w_1}{2}-\frac{W-w_1}{2})/(y-1)}
        = \frac{(W-w_1)(y-1)}{(W-w_1)(c-1)}
        = \frac{(y-1)}{(c-1)}
    \]
    objects having at least a volume of $w_{\max}$.
    For a successful flip $w_1$ and at least one of these $d$ objects must switch bins and the probability for such a step flipping $y$ bits is therefore at most
    \begin{gather}
        \nonumber \probP(y \text{ bits are flipped})\cdot\probP(\text{the correct $y$ bits are flipped} | y \text{ bits are flipped})\\ \nonumber
        \le 1\cdot \frac{\binom{1}{1}\cdot\binom{\lceil d\rceil}{1}\cdot\binom{n-2}{y-2}}{\binom{n}{y}}
        =\frac{\lceil d\rceil\frac{(n-2)!}{(n-2-y+2)!\cdot (y-2)!}}{\frac{n!}{(n-y)!\cdot y!}}
        =\frac{\lceil d\rceil\cdot(n-2)!\cdot(n-y)!\cdot y!}{n!\cdot(n-y)!\cdot(y-2)!}
        = \frac{\lceil d\rceil y{(y-1)}}{n(n-1)}\\ \nonumber
        \le \frac{y{(y-1)}}{n(n-1)}\cdot(\frac{y-1}{c-1}+1)
        = \frac{y{(y-1)}}{n(n-1)}\cdot(\frac{y-1+c-1}{c-1})
        \le \frac{2y{(y-1)}^2}{n(n-1)(c-1)}
    \end{gather}
\end{proof}

Theorem~\ref{theo:OneMaxResult} is the last analysis on inputs with $w_1 \ge \frac W 2$.
It uses the results of the two previous lemmas and gives an asymptotic bound on the runtime.
Instead of giving a runtime for a 4/3-approximation this lemma gives the expected runtime for reaching one of the two optimal solutions.
For a non-optimal solution moving one element from the fuller to the emptier bin will always result in an improvement.
There are also no local optima which neither of the algorithms is unlikely to leave.
So this input is rather easy to solve for the RLS and (1+1) EA and comparable to a linear function or even OneMax if $w_2=\cdots=w_n$.

\begin{theorem}\label{theo:OneMaxResult}
    If $w_1 \ge \frac W 2$  then the RLS and the (1+1) EA with mutation rate $k/n$ with constant $0<k<\sqrt{n}$ reach the optimal solution in expected time $\Theta(n\log{}n)$
\end{theorem}
\begin{proof}
    The optimal solution is putting $w_1$ in one bin and all other elements in the other bin.
    So the problem is almost identical to OneMax/ZeroMax.
    A single bit flip of the first bit can only happen, if the emptier bin has a weight of at most $\frac {W-w_1}{2}$.
    After this flip the weight of the emptier bin is at least $\frac {W-w_1}{2}$ and therefore another single bit flip of $w_1$ can only happen before another bit is flipped.
    The run of the RLS can be divided into two phases:
    \begin{description}
        \item[Phase 1:] The RLS reaches a search point with $b_E > \frac {W-w_1}{2}$.
        \item[Phase 2:] The RLS reaches an optimal solution $\Rightarrow w_1$ is in one bin and all other elements are in the other bin.
    \end{description}

    The expected length of the first phase is at most $2n$ because the probability of flipping the first bit is $\frac{1}{n}$ and the expected time for such a step then is at most $n$.
    After such a step $b_E \ge \frac {W-w_1}{2}$ holds.
    If the solution is already optimal $b_E = W-w_1>\frac {W-w_1}{2}$, otherwise there is at least one bit that can be flipped.
    This bit will be flipped in expected time at most $n$ for the same reason as for $w_1$.
    This leads to a total length of first phase of at most $2n$.
    In the second phase the RLS can no longer flip $w_1$ as it does not result in an improvement ever again.
    Therefore the RLS behaves exactly as on OneMax/ZeroMax depending on the value of the first bit and reaches an optimal solution in $\Theta(n\log{}n)$ resulting in a total runtime of $\Theta(n\log{}n)$ (Theorem 3 in~\cite{witt2014fitness}).\newline
    As long as $w_1$ does not flip the (1+1) EA has to minimize a linear function of $n-1$ bits which takes $(1+o(1))\frac{e^k}{k}n\ln n$ time (Corollary 4.2 in~\cite{witt2013tight}).
    The only steps that could hinder the algorithm from optimising the linear function in $\Theta(n\log{}n)$ would be a flip of the first bit.
    Such steps invert the optimal solution which could decrease the progress of minimising the linear function.
    If such a step has an expected time of $\omega(n\log{}n)$ the linear function is likely to be optimised in expectation before such a step happens.
    % The probability of the (1+1) EA to flip more than $1+\sqrt{6\ln(n)}$ is limited by Chernoff bounds:
    % \begin{gather}
    %     \nonumber \probP(\text{(1+1) EA flips more than }1+\sqrt{6\ln(n)}\text{ bits})\\ \nonumber
    %     \le\probP(X\ge (1+\sqrt{6\ln(n)})\cdot 1)
    %     \le e^{-1\cdot{\sqrt{6\ln(n)}}^2/3}
    %     = e^{-6\ln(n)/3}
    %     = n^{-2}
    % \end{gather}
    % So the expected time for such a step is at least \(n^2=\omega(n\ln(n))\).
    % Now let's look at steps that flip at most $1+\sqrt{6\ln(n)}$ bits in a single step.
    % Such a step only successfully flips $w_1$ if both $w_1$ is flipped and enough total volume is shifted from $b_E$ to $b_F$.
    % Due to Lemma~\ref{lemma:CWittRefined} with $\delta=\frac{1}{n}$ (for $n>1$) the solution is at most $w_1+\delta(W-w_1)$ after expected time
    % \[
    %     2\lceil en\ln(2/\delta)\rceil
    %     =2\lceil en\ln(2/\frac{1}{n})\rceil
    %     =2\lceil en\ln(2n)\rceil
    %     =2\lceil en(\ln(n)+\ln(2))\rceil
    %     \le 2en\ln(n)+4
    % \]
    % The value of $b_E$ is then at least \(W-(w_1+\delta(W-w_1))=(1-\delta)(W-w_1)=(1-\frac{1}{n})(W-w_1)\).
    % Lemma~\ref{lemma:W1FlipWontHappen} states that the probability of a step flipping with $w_1$ together with $y-1$ other bits is at most $\frac{2y{(y-1)}^2}{n(n-1)(c-1)}$.
    % Applying the bound $y\le1+\sqrt{6\ln(n)}$ and the value $c=2(1-\frac{1}{n})$ this simplifies to
    % \[
    %     \frac{2y{(y-1)}^2}{n(n-1)(c-1)}
    %     \le\frac{2{(1+\sqrt{6\ln(n)})}^3}{n(n-1)(1-\frac{2}{n})}
    %     =\frac{2{(1+\sqrt{6\ln(n)})}^3}{n(n-1)\frac{n-2}{n}}
    %     =\frac{2{(1+\sqrt{6\ln(n)})}^3}{(n-1)(n-2)}
    % \]
    % The probability of one of these steps to happen for any value of $y$ is given by
    % \begin{gather}
    %     \nonumber \sum_{y=2}^{1+\sqrt{6\ln(n)}}{\probP(y \text{ bits are flipped})\cdot\probP(\text{the correct $y$ bits are flipped} | y \text{ bits are flipped})}\\
    %     \nonumber \le ({1+\sqrt{6\ln(n)}})\cdot\frac{2{(1+\sqrt{6\ln(n)})}^{3}}{(n-2)(n-1)}
    %     = \frac{2{(1+\sqrt{6\ln(n)})}^{4}}{(n-2)(n-1)}\\ \nonumber
    %     = \frac{2{(o({n}^{1/8}))}^{4}}{(n-2)(n-1)}
    %     = \frac{o(n^{0.5})}{\mathcal{O}(n^{2})}
    %     = \mathcal{O}(n^{-1.5})
    % \end{gather}
    % The expected time for such a step is then $\Omega(n^{1.5})=\omega(n\ln n)$.
    % The probability that such a step still happens before the linear function is optimised is at most
    % \[
    %     \frac{1}{\mathcal{O}(n^{1.5})}\cdot(1+o(1))en\ln n
    %     % =\frac{(1+o(1))en\ln n}{\mathcal{O}(n^{1.5})}
    %     =\frac{(1+o(1))e\ln n}{\mathcal{O}(n^{0.5})}
    %     =\frac{o(n^{0.1})}{\mathcal{O}(n^{0.5})}
    %     =o(\frac{1}{n^{0.4}})=o(1)\]
    % The probability of $w_1$ not being flipped after expected time $2en\ln n+4$ is \(1-o(\frac{1}{n^{0.4}})=1-o(1)\).
    % If such a step happens the fitness does not decrease and the bound on the probability of flipping $w_1$ still holds.
    % The algorithm will still find the solution in expected time at most $(1+o(1))en\ln n$.
    % Since even after a flip all condition are still true the expected time of optimising the linear function after expected time $2en\ln n+4$ is given by \(\frac{1}{1-o(1)}\cdot(1+o(1))en\ln n=\Theta(n\log{}n)\)

    % The total runtime for the (1+1) EA is $(2en\ln n+4) + \frac{1+o(1)}{1-o(1)}\cdot en\ln n =\Theta(n\log{}n)$.


    % \(\frac{1}{1-o(\frac{1}{n^{0.4}})}=\frac{1-o(\frac{1}{n^{0.4}})+o(\frac{1}{n^{0.4}})}{1-o(\frac{1}{n^{0.4}})}=1+\frac{o(\frac{1}{n^{0.4}})}{1-o(\frac{1}{n^{0.4}})}\)

    % \begin{gather}\nonumber
    %     E(T)\le\frac{E(T*)}{1-p_{\text{fail}}}
    %     \le\frac{E(T*)}{1-(E(T*)p)}
    %     =\frac{1}{\frac{1}{E(T*)}-p}
    %     =\frac{1}{\frac{1}{(1+o(1))en\ln n}-\mathcal{O}(n^{-1.5})}\\ \nonumber
    %     =\frac{1}{\frac{1-\mathcal{O}(n^{-1.5})\cdot(1+o(1))en\ln n}{(1+o(1))en\ln n}}
    %     =\frac{(1+o(1))en\ln n}{1-\mathcal{O}(n^{-1.5})\cdot(1+o(1))en\ln n}
    %     =\frac{(1+o(1))en\ln n}{1-o(\frac{1}{n^{0.4}})}
    % \end{gather}

    % So in conclusion a step moving the first bit after expected time $2en\ln n+4$ has passed is either unlikely due to the amount of bits shifted or due to the small amount of values needed to be flipped for such a step.
    % The expected time for such a step is $\omega(n\ln(n))$ and will therefore not happen in expectation before the linear function is optimised.

    % -----------------------\newline

    The probability of the (1+1) EA to flip more than $k+\sqrt{6k\ln(n)}=k+k\sqrt{6\ln(n)/k}$ is limited by Chernoff bounds:
    \begin{gather}
        \nonumber \probP(\text{(1+1) EA flips more than }k+k\sqrt{6\ln(n)/k}\text{ bits})\\ \nonumber
        \le\probP(X\ge (1+\sqrt{6\ln(n)/k})\cdot k)
        \le e^{-k\cdot{\sqrt{6\ln(n)/k}}^2/3}
        = e^{-k\cdot \frac{6\ln n}{3k}}
        = n^{-2}
    \end{gather}
    So the expected time for such a step is at least \(n^2=\omega(n\ln(n))\).
    Now let's look at steps that flip at most $k+\sqrt{6k\ln(n)}$ bits in a single step.
    Such a step only successfully flips $w_1$ if both $w_1$ is flipped and enough total volume is shifted from $b_E$ to $b_F$.
    Due to Lemma~\ref{lemma:CWittRefined} with $\delta=\frac{1}{n}$ (for $n>1$) the solution is at most $w_1+\delta(W-w_1)$ after expected time
    \begin{gather}\nonumber
        2\lceil\frac{e^k}{k(1-o(1))}n\ln(2/\delta)\rceil
        =2\lceil\frac{e^k}{k(1-o(1))}n\ln(2/\frac{1}{n})\rceil
        =2\lceil\frac{e^k}{k(1-o(1))}n\ln(2n)\rceil \\ \nonumber
        =2\lceil\frac{e^k}{k(1-o(1))}n(\ln(n)+\ln(2))\rceil
        \le\frac{2e^k}{k(1-o(1))}n(\ln(n)+4)
    \end{gather}
    The value of $b_E$ is then at least \(W-(w_1+\delta(W-w_1))=(1-\delta)(W-w_1)=(1-\frac{1}{n})(W-w_1)\).

    Lemma~\ref{lemma:W1FlipWontHappen} states that the probability of a step flipping with $w_1$ together with $y-1$ other bits is at most $\frac{2y{(y-1)}^2}{n(n-1)(c-1)}$.
    Applying the bound $y\le k+\sqrt{6k\ln(n)}$ and the value $c=2(1-\frac{1}{n})$ this simplifies to
    \[
        \frac{2y{(y-1)}^2}{n(n-1)(c-1)}
        \le\frac{2{(k+\sqrt{6k\ln(n)})}^3}{n(n-1)(1-\frac{2}{n})}
        =\frac{2{(k+\sqrt{6k\ln(n)})}^3}{n(n-1)\frac{n-2}{n}}
        =\frac{2{(k+\sqrt{6k\ln(n)})}^3}{(n-1)(n-2)}
    \]
    The probability of one of these steps to happen for any value of $y$ is given by
    \begin{gather}
        \nonumber \sum_{y=2}^{k+\sqrt{6k\ln(n)}}{\probP(y \text{ bits are flipped})\cdot\probP(\text{the correct $y$ bits are flipped} | y \text{ bits are flipped})}\\
        \nonumber \le ({k+\sqrt{6k\ln(n)}})\cdot\frac{2{(k+\sqrt{6k\ln(n)})}^{3}}{(n-2)(n-1)}
        = \frac{2{(k+\sqrt{6k\ln(n)})}^{4}}{(n-2)(n-1)}\\ \nonumber
        = \frac{2{(o({n}^{1/8}))}^{4}}{(n-2)(n-1)}
        = \frac{o(n^{0.5})}{\mathcal{O}(n^{2})}
        = \mathcal{O}(n^{-1.5})
    \end{gather}

    The expected time for any step successfully flipping $w_1$ is then $\Omega(n^{1.5})=\omega(n\ln n)$.
    Let $T$ be the time until the linear function is optimised and $p$ the probability of successfully flipping $w_1$.
    Then the probability that $w_1$ flips after expected time $\frac{2e^k}{k(1-o(1))}n\ln(n)+4$ before the linear function is optimised is at most
    \[
        p\cdot E(T) \le \frac{1}{\mathcal{O}(n^{1.5})}\cdot(1+o(1))\frac{e^k}{k}n\ln n
        % =\frac{(1+o(1))\frac{e^k}{k}n\ln n}{\mathcal{O}(n^{1.5})}
        =\frac{(1+o(1))\frac{e^k}{k}\ln n}{\mathcal{O}(n^{0.5})}
        =\frac{o(n^{0.1})}{\mathcal{O}(n^{0.5})}
        =o(\frac{1}{n^{0.4}})=o(1)\]
    The probability of $w_1$ not being flipped after expected time $\frac{2e^k}{k(1-o(1))}n\ln(n)+4$ is \(1-o(\frac{1}{n^{0.4}})=1-o(1)\).
    If such a step happens the fitness does not decrease and the bound on the probability of flipping $w_1$ still holds.
    The algorithm will still find the solution in expected time at most $(1+o(1))\frac{e^k}{k}n\ln n$.
    Since even after a flip all condition are still true the expected time of optimising the linear function after expected time $\frac{2e^k}{k(1-o(1))}n\ln(n)+4$ is given by \(\frac{1}{1-o(1)}\cdot(1+o(1))\frac{e^k}{k}n\ln n=\Theta(n\log{}n)\)

    The total runtime for the (1+1) EA is $\frac{2e^k}{k(1-o(1))}n\ln(n)+4 + \frac{1+o(1)}{1-o(1)}\cdot \frac{e^k}{k}n\ln n =\Theta(n\log{}n)$.

\end{proof}

Theorem~\ref{theo:OneMaxResult} only proved the asymptotic expected runtime of the (1+1) EA with different mutation rates.
It does not proof which mutation rate reaches the solution the fastest in expectation, but it suggests the runtime increases for higher mutation rates.
Later in the thesis there is an analysis for this lemma in Section~\ref{evalSec:onemax} which also suggests the optimality of $1/n$.\newline
The next few Lemmas and Corollaries help to prove the runtime of $\bigO(n\log{n})$ for the (1+1) EA and the RLS on inputs with $w_1<W/2$ for reaching an approximation ratio of 4/3.
They are mostly rather short with a proof of only a few lines.
The only Lemma with a longer proof is Lemma~\ref{lemma:movingObjects2}.
It shows the runtime bound for a more restricted type of input and is used to simplify the proof of Lemma~\ref{lemma:approximation}

\begin{lemma}\label{lemma:approximationHelp}
    If \(b_F \le \frac{2}{3} \cdot W\) the approximation ratio is at most $\frac{4}{3}$
\end{lemma}
\begin{proof}
    \(\frac{b_F}{opt} \le \frac{(2/3) \cdot W}{opt} \le \frac{(2/3) \cdot W}{(1/2) \cdot W} = \frac{4}{3}\), since \(opt \ge \frac{W}{2}\)
\end{proof}

\begin{corollary}\label{cor:approximationHelp}
    If \(w_1 \ge \frac{W}{3}\) and \(w_1\) is in the emptier bin, then the approximation ratio is at most $\frac{4}{3}$
\end{corollary}
\begin{proof}
    $w_1$ is in the emptier bin, so \( b_F \le W - w_1 \le W - \frac{W}{3} = \frac{2W}{3} \) and with Lemma~\ref{lemma:approximationHelp} the assumption follows.
\end{proof}

\begin{lemma}\label{lemma:movingObjects}
    Any object of weight $v$ can be moved from $b_F$ to $b_E$ if and only if \(b_F - b_E \ge v\)
\end{lemma}
\begin{proof}
    $''\Leftarrow''$:\newline
    \(b_F - b_E \ge v \Leftrightarrow b_F \ge b_E + v\), so after moving an object with weight $v$ from $b_F$ to $b_E$, the new weight of $b_E$ is at most the weight of $b_F$ before moving the object, thus the RSH accepts the step.\newline
    $''\Rightarrow''$:\newline
    \(b_F - b_E < v \Leftrightarrow b_F < b_E + v\), so moving an object of weight $v$ results in ${b_F}' = b_E+v > b_F$ which results in the step being rejected.
\end{proof}

\begin{corollary}\label{cor:RLSStuck}
    The RLS is stuck in a local optimum if \(b_F-b_E \le w_n\) holds and \(b_F > opt\).
\end{corollary}
\begin{proof}
    A single bit flip of weight $v$ can only happen if \(b_F - b_E \ge v\) (Corollary~\ref{cor:RLSStuck}). If \(b_F-b_E < w_n\) there is no weight which satisfies the condition and therefore no single bit flip is possible.
    If \(b_F-b_E = w_n\) then only objects with weight \(w_n\) can be flipped, but those do not change the fitness ($b_F' = b_E + w_n = b_F - w_n +w_n = b_F$).
    Since the RLS can only move one bit at a time and only if it results in an improvement, the RLS is stuck.
\end{proof}

\begin{corollary}\label{cor:movingObjects}
    Every object \(\le \frac{W}{3}\) can be moved from $b_F$ to $b_E$ if \(b_F \ge \frac{2W}{3}\)
\end{corollary}
\begin{proof}
    \(b_F \ge \frac{2W}{3} \Rightarrow b_E \le W - \frac{2W}{3} \le \frac{W}{3} \Rightarrow b_F - b_E \ge \frac{2W}{3} - \frac{W}{3} = \frac{W}{3}\) and with Lemma~\ref{lemma:movingObjects} the assumption follows.
\end{proof}

\begin{lemma}\label{lemma:movingObjects2}
    In expected time $\mathcal{O}(n\log{}n)$ the weight of the fuller bin can be decreased to \(\le \frac{2W}{3}\) by the RLS and the (1+1) EA if every object besides the biggest in the fuller bin is at most $\frac{W}{3}$ and \(w_1 \le \frac{W}{2}\).
\end{lemma}
\begin{proof}
    In expected time $\mathcal{O}(n\log{}n)$ the RLS can move every object $\le \frac{W}{3}$ to the emptier bin as long as $b_F \ge \frac{2W}{3}$ due to Corollary~\ref{cor:movingObjects} and Theorem~\ref{theo:OneMaxResult}.
    So in expected time $\mathcal{O}(n\log{}n)$ the solution can be shifted to $w_1$ being in one bin and all other objects in the other bin.
    The RLS will only stop moving the elements if the condition $b_F \ge \frac{2W}{3}$ is no longer satisfied (Corollary~\ref{cor:movingObjects}).
    If \(w_1 \ge \frac{W}{3}\) and every object was moved to the bin without $w_1$, then \(b_F = \max\{W-w_1, w_1\} = W-w_1 \le \frac{2W}{3}\), because \(w_1 \le \frac{W}{2}\).
    So either the RLS moves all objects to the emptier bin or stops moving objects because $b_F < \frac{2W}{3}$ both resulting in $b_F \le \frac{2W}{3}$.
    If $w_1$ is not in the fuller bin, then the result follows by Corollary~\ref{cor:approximationHelp}.\newline
    Now assume \(w_1 < \frac{W}{3}\).
    In this case the RLS will move one object per step to the emptier bin.
    Each object has weight $< \frac{W}{3}$ and therefore one step cannot decrease the weight of the fuller bin from $> \frac{2W}{3}$ to $\le \frac{W}{3}$.
    If all objects except one where moved to one bin, the other bin would have a weight of at least \(W-w_1 > \frac{2W}{3}\).
    Therefore the RLS will find a solution with $b_F < \frac{2W}{3}$ before moving all elements from the first to the second bin.\newline
    The proof for the (1+1) EA is mostly the same.
    The main difference is the (1+1) EA being able to flip more than one bit in a single step.
    Such a step could make the emptier bin the fuller bin or increase the number of bits that must be shifted to the emptier bin.
    But with the results of Theorem~\ref{theo:OneMaxResult} the proof works exactly the same as for the RLS.\
    The case \(w_1 \ge \frac{W}{3}\) does not change only the bin containing $w_1$ might change.
    Apart from that there is no difference for the (1+1) EA.\
    The case $w_1 < \frac{W}{3}$ is also rather similar.
    The (1+1) EA will move elements from the fuller bin to the emptier bin until $b_F < \frac{2W}{3}$ holds. The (1+1) EA can make the emptier bin the fuller bin by moving multiple objects in one step, but this does not hinder it from reaching $b_F < \frac{2W}{3}$.
    After a step making the previously fuller bin the emptier bin it will continue moving elements until the condition holds.
\end{proof}

\begin{lemma}\label{lemma:approximation}
    The RLS and the (1+1) EA reach an approximation ratio of at most $\frac{4}{3}$ in expected time $\mathcal{O}(n\log{}n)$ if $w_1 < W/2$
\end{lemma}
\begin{proof}
    If \(w_1+w_2 > \frac{2W}{3}\) after time $\mathcal{O}(n)$ $w_1$ and $w_2$ are separated and will remain separated afterwards (3. Average case analysis, Theorem 1 in~\cite{witt2005worst}).
    From then on the following holds.
    If $w_1$ is in the emptier bin, then the result follows directly by Corollary~\ref{cor:approximationHelp}.
    Otherwise all elements in the fuller bin except $w_1$ have a weight of at most $\frac{1}{3}$ and therefore the result follows by Lemma~\ref{lemma:movingObjects2} and Lemma~\ref{lemma:approximationHelp}.
    If \(w_1+w_2 \le \frac{2W}{3}\) the result follows directly by Lemma~\ref{lemma:movingObjects2} and Lemma~\ref{lemma:approximationHelp}.
\end{proof}

\begin{corollary}
    The RLS and the (1+1) EA reach an approximation ratio of at most $\frac{4}{3}$ for every input in expected time $\mathcal{O}(n\log{}n)$
\end{corollary}
\begin{proof}
    This follows directly from Theorem~\ref{theo:OneMaxResult} and Lemma~\ref{lemma:approximation}.
\end{proof}

\TODO{find out what to write here as a conclusion of this section}

\section{Runtime Analysis of higher mutation rates}

The first analysed input is again the input with $w_1\ge W/2$.

\begin{corollary}\label{corollary:W1FlipWontHappen}
    The expected time until the \RLSR[k] and \RLSN[k] flips $w_1\ge W/2$ after they reached a solution with $b_E = c\cdot\frac{W-w_1}{2}$ with $1<c<2$ is at least \(\frac{n(n-1)(c-1)}{2k^2{(k-1)}^2}\). For constant values of $c$ the expected time is $=\Omega(n^2)$.
\end{corollary}
\begin{proof}
    Using Lemma~\ref{lemma:W1FlipWontHappen}, the upper bound of $y\le k$ and the fact that both algorithms flip at most $k$ bits leads to the first part
    \[ {(k\cdot\frac{2y{(y-1)}^2}{n(n-1)(c-1)})}^{-1}=\frac{n(n-1)(c-1)}{2ky{(y-1)}^2} \le\frac{n(n-1)(c-1)}{2k^2{(k-1)}^2}\]
    $k$ is constant and if c is constant too, this leads to expected time $=\Omega(n^2)$ for a flip of the first bit if $b_E = c\cdot\frac{W-w_1}{2}$ with $1<c<2$ which concludes the second statement.
\end{proof}

\begin{lemma}\label{lemma:RLSRoneMaxInput}
    The \RLSR[k] for $2\le k=O(1)$ reaches the optimal solution on an input with $w_1\ge W/2$ in expected time $\mathcal{O}(n\log{}n)$
\end{lemma}
\begin{proof}
    This proof is similar to proof for the (1+1) EA in Theorem~\ref{theo:OneMaxResult} and is divided in the same two parts.
    The first part is proving the \RLSR~does not flip the first bit in expected time $\mathcal{O}(n\log{}n)$ after some time $T$ has passed.
    After time $T$ the algorithm then minimises the linear function in expected time $\mathcal{O}(n\log{}n)$ before the first bit is flipped and the progress of the linear function might be reset.
    After expected time $(2\lceil kn\ln(2/0.4)\rceil)=2\lceil kn\ln(5)\rceil\le2\lceil 1.61\cdot kn\rceil\le4kn$ the \RLSR[k] reaches a solution of \(f(x)\le w_1+0.4(W-w_1)\) (Lemma~\ref{lemma:CWittRefined}).
    This means \(b_E \ge 0.6(W-w_1) = 1.2\cdot\frac{W-w_1}{2}\).
    With Lemma~\ref{corollary:W1FlipWontHappen} the expected time of at least \(\frac{n(n-1)(c-1)}{k^2{(k-1)}^2}\frac{n(n-1)}{5k^2{(k-1)}^2}=\Omega(n^2)\) for a successful flip of $w_1$ follows. This concludes the first part\newline
    % So if the \RLSR~manages to find the optimum in time $\mathcal{O}(n\log{}n)$ the first bit won't be flipped and the global optimum does not change for the rest of the run in expectation.\newline
    The \RLSR~can be seen as an unbiased unary black box algorithm with a sequence $\mathcal{D}=(p_1,\dots,p_n)$ where $p_i=1/k$ for $1\le i\le k$ and $p_i=0$ otherwise.
    The mean of this sequence is \(\mathcal{X}=\sum_{i=1}^{n}{\probP(X=i)i}=\sum_{i=1}^{k}{\frac{i}{k}}=\frac{1}{k}\cdot\sum_{i=1}^{k}{i}=\frac{k+1}{2}\).
    For any constant value $k$ both the probability $p_1=\frac{1}{k}=\Theta(1)$ and mean $\mathcal{X}=\frac{k+1}{2}=O(1)$ meet the conditions of Theorem 1 in~\cite{doerr2023tight} and therefore the \RLSR~optimises the linear function in expected time \((1+o(1))\frac{1}{p_1}n\ln n=(1+o(1))kn\ln n\).
    Applying the same arguments as in Theorem~\ref{theo:OneMaxResult} this results in a probability of not flipping $w_1$ after expected time $4kn$ of at least
    \begin{gather}\nonumber
        1-\frac{(1+o(1))kn\ln n\cdot5k^2{(k-1)}^2}{n(n-1)}
        =1-\frac{(1+o(1))5k^5\ln n}{n-1}
        =1-\frac{o(k^5n^{0.5})}{n-1}
        =1-\frac{1}{o(\sqrt{n})}
    \end{gather}
    The expected time of minimising the linear function then is
    \[\frac{1}{1-\frac{1}{o(\sqrt{n})}}\cdot(1+o(1))kn\ln n=\frac{1}{1-o(1)}\cdot(1+o(1))kn\ln n=\Theta(n\log{}n)\]
    The total expected runtime therefore is at most $4kn+\frac{1+o(1)}{1-o(1)}\cdot kn\ln n=\Theta(n\log{}n)$.
\end{proof}

In expectation all \RLSR~variants have the same asymptotic runtime on inputs with $w_1\ge W/2$.
They also have the same expected asymptotic Runtime as the standard RLS and the (1+1) EA with static mutation rate $c/n$ for constant values of $c$.
Here the same applies as for the (1+1) EA.\
There proof does not contain the optimality of flipping only 1 bit in expectation, but it strongly suggests it.
The main factor deciding the runtime is the runtime on linear functions which is minimal for $k=1$.
Later in Section~\ref{evalSec:onemax} there is also an experiment for the different values of $k$ for the \RLSR[k].
Another algorithms mentioned later is the \RLSN~which has a much worse runtime as the next lemma suggests.

\begin{lemma}
    The expected optimisation time of the \RLSN[k] for $2\le k=O(1)$ on inputs with $w_1\ge W/2$ is $\Omega(n^k)$ if the solution is not already optimal after the initialisation.
\end{lemma}
\begin{proof}
    The solution has only two global optima.
    The optimum is defined by $x_1$ because every other $x_i$ must have a value of $1-x_1$.
    If the \RLSN[k] reaches a search point where the Hamming distance to the optimum is $y\le k$ there are exactly $y$ bits left that must be flipped for an optimal solution.
    The other solution requires $n-y$ bits to be flipped which needs even more time.
    The probability of reaching the optimum from a Hamming distance of $y$ is given by
    \begin{gather}\nonumber
        \probP(\text{\RLSN[k] flips $y$ bits})\cdot\probP(\text{\RLSN[k] flips the correct $y$ bits | \RLSN[k] flips $y$ bits})\\ \nonumber
        =  \Theta(\frac{1}{n^{k-y}}) \cdot \frac{\binom{y}{y}}{\binom{n}{y}}
        =  \Theta(\frac{1}{n^{k-y}}) \cdot \frac{1}{\frac{n!}{(n-y)!y!}}
        =  \Theta(\frac{1}{n^{k-y}}) \cdot \Theta(\frac{y!}{n^y})
        =  \Theta(\frac{y!}{n^{k}})
    \end{gather}
    Because $y\le k = O(1)$ holds $\Theta(\frac{y!}{n^{k}})=\Theta(\frac{1}{n^{k}})$ also holds.
    The expected time for such a step to happen is ${(\Theta(\frac{1}{n^{k}}))}^{-1}=\Theta(n^{k})$.
    If the algorithm instead successfully flips less than $y$ bits it reaches a search point where the expected optimisation time is still $\Theta(n^{k})$.
    Either way the expected optimisation time is $\Theta(n^{k})$ when the algorithm has reached a search point with Hamming distance $1\le y\le k$.
    For every value of $k$ the \RLSN[k] can only flip at most $k$ bits in each step.
    So it will reach a search point of \RLSN[k] if the initial Hamming distance is at least one.
    This will happen with probability $1-1\cdot{(1/2)}^{n-1}=1-\frac{1}{2^{n-1}}$.
\end{proof}

For smaller values of $n$ the actual runtime in might be lower due to constants such as $y!\le k!$ which were absorbed by the big-O notation.
This difference should be noticeable for the smaller values of $n$ but not for higher values.
Since the standard RLS is the same as the \RLSN[1] the optimal values for the \RLSN[k] variants again is at $k=1$.
This algorithm is also evaluated in the same Section~\ref{evalSec:onemax} as the other algorithms that were analysed for inputs similar to a linear function.

\section{Binomial distributed input}
This section discusses inputs following a binomial distribution\textasciitilde$B(m,p)$.
Since $n$ is reserved for the size of the input $m$ is used for the distribution instead.
At first they seem uninteresting as all values are rather close to the expected value of the distribution.
So it seems like the Evolutionary Algorithm only has to find a search point where both bins have an equal amount of values.
After investigating this input experimentally this is not true to that extend.
Solutions with an equal amount of bits with value 0 and value 1 are indeed close to the optimum, but they are just close and not optimal.
There is mostly still a difference of at least one to the optimum.
This might cause algorithms like the RLS to get stuck despite the input looking easy at first glance if there are no small elements.
If elements can be swapped this input should become solvable again because there are many values slightly bigger or smaller than the expected value.
If the algorithm switches those the small difference to the optimum can be closed.
The first lemma tries to prove that these input are indeed easy to solve in a sense that this input is very likely to have a perfect partition.

\begin{lemma}\label{lemma:BinomialSolvable}
    A binomial distributed input \textasciitilde$B(m,p)$ has a perfect partition ($b_F - b_E = 0$ for even $W$ and $b_F - b_E = 1$ for uneven $W$) with high probability if the input size $n$ is large enough.
\end{lemma}
\begin{proof}
    Sketch:
    \begin{itemize}
        \item The initial distribution is likely rather close to the optimum
        \item The difference between the bins is probably not more than 10 expected values
        \item the large values
    \end{itemize}
    Consider a random separation of all values into two sets with equal size if $n$ is even or one set with one value more than the other if $n$ is odd. The sum X of one set is a sum of $\frac{n}{2}\cdot m$ independent Bernoulli trials with probability $p$. With Chernoff Bounds the following inequality follows:
    \[\probP(X\ge(\frac{n}{2}+\sqrt{\frac{n}{2}})\cdot m \cdot p) = \probP(X\ge(1+2\sqrt{\frac{2}{n}})\cdot \frac{nmp}{2}) \le e^{-\frac{mnp}{2}\cdot2\sqrt{\frac{2}{n}}^2 /3} = e^{-\frac{2mp}{3}}\]
    For $mp\ge1.5$ the probability is less than $\frac{1}{e}$. Otherwise the input is rather trivial, since the numbers will be concentrated around $mp\le1.5$ and most values will be below 10.\newline
    After moving $\mathcal{O}(\sqrt{\frac{n}{2}}/2)$ objects to the emptier set, the difference between the two sets is at most half the expected value $mp$ of a single value.
    \dots
\end{proof}


\begin{lemma}
    With high probability the RLS does not find a perfect partition for an input with distribution \textasciitilde$B(m,p)$ if n is large enough, $mp\ge50$ and $m-mp\ge50$.
\end{lemma}
\begin{proof}
    % Due to Lemma~\ref{lemma:BinomialSolvable} the input has an optimal solution with high probability.
    % As long as $b_F-b_E>w_1$ holds moving any object $w_i$ from $b_F$ to $b_E$ results in a decrease of the fitness of $w_i$.
    % The RLS will continue moving elements to the emptier bin at least until $b_F-b_E$ holds which will eventually happen because there is always at least one bit left in $b_F$ that can be shifted to the emptier bin as long as $b_F-b_E>w_1$.
    % When the RLS reaches a search point of $b_F-b_E\le w_1$ there are two cases.
    % In the first case $b_F-b_E<w_n$ which means that the RLS is stuck due to Corollary~\ref{cor:RLSStuck}.
    % In the other case $b_F-b_E\ge w_n$ holds and therefore there is at least one object left in the fuller bin, that can be moved to the emptier bin.
    % If the moved object $w_i\le (b_F-b_E/2)$ then $b_F'-b_E'=(b_F-w_i)-(b_E-w_i)\le w_1-w_n$.
    % If the moved object $w_i> b_F-b_E$ then $b_F'-b_E'=(b_F-w_i)-(b_E-w_i)=b_F-b_E-2w_i\le \dots$.

    % Due to Lemma~\ref{lemma:BinomialSolvable} the input has an optimal solution with high probability
    The RLS will always move one object per step from fuller to the emptier bin.
    As long as $b_F-b_E>w_i$ holds moving any object of weight at most $w_i$ from $b_F$ to $b_E$ results in a decrease of the fitness ($b_F'=b_E+w_i<b_F-w_i+w_i=b_F$).
    If the RLS does not decrease the difference $b_F-b_E$ to at most $w_n$ in any step the solution is never optimal and the RLS therefore must be stuck.
    So now assume the RLS eventually reaches a search point with $b_F-b_E\le w_n$.
    Consider the step which decreases the difference from more than $w_n$ to at most $w_n$.
    If this step does not decrease the difference to 0 for even $n$ or 1 for odd $n$ the RLS is stuck due to Corollary~\ref{cor:RLSStuck}.
    For a step to decrease $b_F-b_E$ from more than $w_n$ to 0 the RLS must move an object of weight $y$ which is given by $b_F'-b_E'=(b_F-y)-(b_E+y)=0\Leftrightarrow b_F-b_E-2y=0\Leftrightarrow y=(b_F-b_E)/2$.
    Such an element can only exist for even $n$ because only for even n either both bins have an even sum or neither of them.
    Odd inputs will always have exactly one bin with an even sum and one with an odd sum.
    For odd $n$ a difference of 1 suffices for a perfect partition and therefore the value of $y$ must be either $\lceil(b_F-b_E)/2\rceil$ or $\lfloor(b_F-b_E)/2\rfloor$.
    For any binomial distribution the value which occurs the most in expectation is the expected value if $mp\in\N$ or $\lceil mp\rceil$ and $\lfloor mp\rfloor$ otherwise.
    Even if the behaviour of the algorithm forces values smaller or bigger than $mp$ in the emptier bin, the number of bigger/smaller values in the fuller bin will still be less in expectation than the amount of $mp$ in both bins.
    Let's assume that objects with exactly this volume must be selected and that there are two options because $mp\notin\N$.
    This will give an upper bound on the probability of flipping the right object in the crucial step.
    The probability of a number to be $\lfloor mp\rfloor$ is $p^{\lfloor mp\rfloor}{(1-p)}^{m-\lfloor mp\rfloor}\le0.5^{50}=\frac{1}{1024^5}<10^{-15}$ because both $mp\ge50$ and $m-mp\ge50$ and $p\le0.5$ or $1-p\le0.5$.
    For $\lceil mp\rceil$ the same bound applies with the same arguments.
    The probability of either of these numbers to be drawn from the binomial distribution is at most $2\cdot 0.5^{50}=0.5^{49}\le10^{-15}$.
    This leads to $n\cdot 10^{-15}$ values with the right value in expectation.
    All bits are flipped uniformly at random and therefore the probability of flipping a good bit is at most
    \(\frac{n\cdot 10^{-15}}{|b_F|}\le \frac{n\cdot 10^{-15}}{n} =10^{-15}\).


    This means the RLS is very unlikely to find a perfect partition.
    Given the solution has a perfect partition the RLS will get stuck in a local optimum.
\end{proof}

The restriction of $mp\ge50$ and $m-mp\ge50$ was not chosen to make the proof work but is also necessary.
Without those restrictions the distribution might have many small values and especially elements close to 1.
When there are many small values these can be used to fill the small gaps which makes it easy for the RLS to find a perfect partition.
On the other hand if $p$ is almost one, then almost every element will be same for smaller values of $m$.
If every element is the same then the RLS must only find a search point with equal amounts of 0s and 1s.
\chapter{Experimental Results}\label{ch:expRes}

In the following chapter the different variants of the RLS and the (1+1) EA are now analysed empirically for the best algorithm depending on the input.
Additionally for most lemmas from the previous chapters there are also tests if they actually hold in practice.

\section{Code}
The complete java code used for all empirical studies is available on \href{https://github.com/Err404NameNotFound/PartitionSolvingWithEAs}{GitHub}.
\subsection{The Algorithms}
All different variants of the RLS function more or the less the same. They start with an initial random value and then optimise this one value in the loop. The loop can be summarised like this:
\begin{enumerate}
      \item generate a number $k$ of bits to be flipped (algorithm specific)
      \item flip $k$ random bits
      \item evaluate fitness of the mutated individual
      \item replace old value with new value if new value is better
      \item repeat if not optimal
\end{enumerate}
The (1+1) EA variants behave differently at first glance as they flip each bit independently with probability $c/n$.
This can be seen as $n$ independent Bernoulli trials with probability $c/n$.
The amount of bits that are flipped is therefore binomial distributed and the algorithm can be implemented exactly as the versions of the RLS. The same holds for the $pmut$ operator which generates a number $k$ from a powerlaw distribution and then flips $k$ bits.
This leads to only one implementation of a partition solving algorithm which is not only given the input array of numbers but also a generator for the amount of bits to be flipped in each step.
The random values for the amount of bits to be flipped are generated according to this table:

\begin{tabular}[h]{c c}
      Algorithm & Returned value                                                                          \\
      \hline
      RLS       & 1                                                                                       \\
      \RLSN~    & $y \in \{1,\dots,k\}$ with probability $\frac{\binom{n}{y}}{\sum_{i=1}^k \binom{n}{i}}$ \\
      \RLSR~    & uniform random value $y \in \{1,\dots,k\}$                                              \\
      (1+1) EA  & binomial distributed value from \textasciitilde$B(n,c/n)$                               \\
      pmut      & powerlaw distributed value from \textasciitilde$D^\beta_{n/2}$                          \\
\end{tabular}



\begin{algorithm}[bt]
      \caption{\textsc{GenericPartitionSolver}}\label{alg:genericPartition}

      % Some settings
      \DontPrintSemicolon %dontprintsemicolon
      \SetFuncSty{textsc}

      % The algorithm
      \BlankLine
      choose x uniform random from ${\{0,1\}}^n$\;
      \While{$x$ not optimal}
      {
      $x' \leftarrow x$\;
      $k \leftarrow \text{kGenerator.generate()}$\;
      flip $k$ uniform random bits of $x'$\;
      {
      \If{$f(x') \le f(x)$}
      {
            $x \leftarrow x'$\;
      }
      }
      }
\end{algorithm}

\subsection{Random number generation}
Java only provides a random number generator for uniform distributed values for any integer interval or random double values $\in \left[0, 1\right)$.
The same holds for the \href{http://www.math.sci.hiroshima-u.ac.jp/m-mat/MT/emt.html}{MersenneTwister} with an implementation used from this \href{https://cs.gmu.edu/~sean/research/}{page}.
All experiment were executed with both uniform random number generators.
The results were rather similar, so only the results for the Fast Mersenne Twister are shown.
For this project uniform random numbers do not suffice as for an efficient way of implementing the (1+1) EA or simply for generating a binomial distributed input another random number generator is needed.
One of the needed distributions is a binomial distribution.
The simplest way to generate a number \textasciitilde$B(m,p)$ would be to run a loop $m$ times and add 1 to the generated number if a uniform random value $\in \left[0, 1\right)$ is less than $p$.
This works perfectly fine and generates numbers according to the distribution.
With low values for p this approach is rather inefficient and especially for values of $p=1/m$.
The expected value in this case is 1 but generating a random number takes time $\mathcal{O}(m)$.
Another more efficient way was implemented by StackOverflow user \href{https://stackoverflow.com/users/2166798/pjs}{pjs} on \href{https://stackoverflow.com/questions/23561551/a-efficient-binomial-random-number-generator-code-in-java}{stackoverflow} inspired by Devroyes method introduced in~\cite{devroye2006nonuniform}.
This method has an expected running time of $\mathcal{O}(mp)$ which is equal to the expected value of the distribution.
For the case of $p=1/m$ this runs in expected constant time in comparison to $\mathcal{O}(m)$ for the naive way.
This number generation was also used for the implementation of the (1+1) EA instead of running a loop in every step.
Algorithm~\ref{alg:binomialRNG} uses the second waiting time method which uses the fact that X\textasciitilde$B(m,p)$ if X is the smallest integer so that \(\sum_{i=1}^{X+1}{\frac{E_i}{n-i+1}}<-\log(1-p)\) for $E_i$ iid exponential random variables (Lemma 4.5 section X.4~\cite{devroye2006nonuniform}).

\begin{algorithm}[h]
      \caption{\textsc{Binomial random number generator}}\label{alg:binomialRNG}

      % Some settings
      \DontPrintSemicolon%dontprintsemicolon
      \SetFuncSty{textsc}
      $q \leftarrow \ln(1.0 - p)$\;
      $x \leftarrow 0$\;
      $sum \leftarrow 0$\;
      \While{true}
      {
      $sum \leftarrow sum +\ln(\text{random()}) / (n - x)$\; \tcp{random() generates a random value $\in \left[0, 1\right)$}
      \If{sum < q}
      {
            return $x$\;
      }
      $x \leftarrow x + 1$\;
      }
\end{algorithm}

The next generator needed is for geometric distributed values.
This generator is only necessary for the generation of geometric distributed inputs but not for the algorithms.
The easiest way to generate geometric distributed values is the naive way:
generating a uniform random value $p'$ until $p'<p$ holds.
The expected running time of this algorithm is equal to the expected value of the distribution $1/p$.
So this method is comparably effective to the approach used for binomial random number generation.

\begin{algorithm}[h]
      \caption{\textsc{Geometric random number generator}}\label{alg:geometricRNG}

      % Some settings
      \DontPrintSemicolon %dontprintsemicolon
      \SetFuncSty{textsc}
      $sum \leftarrow 0$\; \tcp{random() generates a random value $\in \left[0, 1\right)$}
      \While{\text{random()} $\ge$ q}
      {
            $sum \leftarrow sum+1$\;
      }
      return $sum$\;
\end{algorithm}

The last generator needed is for powerlaw distributed values.
This generator is in contrast to the geometric number generator needed for both the algorithm with the $pmut_\beta$ mutation operator and for generating inputs.
This implementation is also from stackoverflow. The user \href{https://stackoverflow.com/users/52738/gnovice}{gnovice} provided the following formula on \href{https://stackoverflow.com/questions/918736/random-number-generator-that-produces-a-power-law-distribution}{this} page on stackoverflow:
\[
      x = {[(b^{n+1} - a^{n+1})*y + a^{n+1}]}^{1/(n+1)}
\]
$a$ is the lower bound, $b$ the upper bound, $n$ the parameter of the distribution and $y$ the number generated uniform random $\in \left[0, 1\right)$.
The idea behind the formula and the formula itself is explained in a \href{https://mathworld.wolfram.com/RandomNumber.html}{mathworld} page.
For a powerlaw distribution \(P(x)=Cx^n\) for \(x\in[a,b]\) normalisation gives
\[\int_{a}^{b}{P(x)dx}=C\frac{{[x^{n+1}]}^{b}_{a}}{n+1}=1\Leftrightarrow C = \frac{n+1}{b^{n+1}-a^{n+1}}\]
Let Y be a uniformly random distributed variate on [0,1]. Then
\[D(x)=\int_{a}^{x}{P(x')dx'}=C\int_{a}^{x}{{x'}^{n}dx'}=\frac{C}{n+1}(x^{n+1}-a^{n+1})=\frac{(x^{n+1}-a^{n+1})}{b^{n+1}-a^{n+1}}\equiv y\]
and the variate is given by
\[X={[(b^{n+1} - a^{n+1})*y + a^{n+1}]}^{1/(n+1)}\]
The values inserted in this formula must be negative.
In the original paper for the $pmut_{\beta}$ operator and in the definition normally a powerlaw distribution is \(P(x)=Cx^{-n}\) and therefore any positive value for $n$ in this case was negated.
Apart from this negation the generator was not changed.

\section{Do binomial inputs have perfect partitions?}

Lemma~\ref{lemma:BinomialSolvable} is only valid for larger $n$.
In practice the bound is much smaller depending on the expected value of a single number.
Another factor deciding how likely an input is to have a perfect partition is whether $n$ is even or odd.
To determine the influence of all factors multiple experiments were conducted.
The goal of the first experiment was to determine the influence of the array size to the input having a perfect partition and the fact if $n$ is even or odd.
So for every possible combination of $p \in \{0.1, 0.2, \dots , 0.8, 0.9\}, m \in \{10,100,1000,10^4,10^5\} \text{ and } n \in \{2,3,4,\dots,19,20\}$ 1000 randomly generated inputs of size $n$ were tested for a perfect partition.
Due to the small values for $n$ it was possible to brute force the results in a short amount of time.
The results are visualised in figure~\ref{fig:firstBinPercentage} to figure~\ref{fig:lastBinPercentage}.

\begin{figure}[h]
      \caption{Percentage of Binomial inputs with perfect partitions for $m = 10$}
      \centering
      \includegraphics[width=0.45\textwidth]{figures/images/solvabilityOfInputs/binomial_Input_Solvable_m10.png}\label{fig:firstBinPercentage}
\end{figure}

On the $x$-axis is the size of the input and on the $y$-axis the percentage of inputs that had a perfect partition.
The different graphs in each figure resemble the different values of $p$ used for generating the inputs.
The graph for $p=0.1$ resembles the percentage of inputs that had a perfect partition with values generated from the distribution \textasciitilde$B(m,0.1)$ with $m$ being dependent on the figure.
For figure~\ref{fig:firstBinPercentage} $m$ has the value 10.\newline
Figure~\ref{fig:firstBinPercentage} is a bit overloaded with information and the zigzag makes it hard to gain any benefit from the graphs.
That's why for $m \in \{10,100,1000\}$ there is one figure for the even input sizes of $n$ and one for the odd.
Figures which show only results for either even or odd values of $n$ have dotted graphs, because the values in between the points do not exist.
The dotted lines are only in the figure for a better visualisation of the trend and not meant for interpretation apart from the marked values.
For $n\ge10,000$ all values for the odd input sizes are 0 \%, so there is no point in showing the data in a separate figure.

\begin{figure}[h]
      \centering
      \begin{minipage}[b]{0.45\textwidth}
            \caption{Percentage of Binomial inputs with perfect partitions for $m = 10$ for even $n$}
            \includegraphics[width=\textwidth]{figures/images/solvabilityOfInputs/binomial_Input_Solvable_m10_even.png}\label{fig:firstBinPercentageEven}
      \end{minipage}
      \hspace{0.75cm}
      \begin{minipage}[b]{0.45\textwidth}
            \caption{Percentage of Binomial inputs with perfect partitions for $m = 10$ for odd $n$}
            \includegraphics[width=\textwidth]{figures/images/solvabilityOfInputs/binomial_Input_Solvable_m10_uneven.png}\label{fig:firstBinPercentageUneven}
      \end{minipage}
\end{figure}

\begin{figure}[h]
      \centering
      \begin{minipage}[b]{0.45\textwidth}
            \caption{Percentage of Binomial inputs with perfect partitions for $m = 100$ for even $n$}
            \includegraphics[width=\textwidth]{figures/images/solvabilityOfInputs/binomial_Input_Solvable_m100_even.png}
      \end{minipage}
      \hspace{0.75cm}
      \begin{minipage}[b]{0.45\textwidth}
            \caption{Percentage of Binomial inputs with perfect partitions for $m = 100$ for odd $n$}
            \includegraphics[width=\textwidth]{figures/images/solvabilityOfInputs/binomial_Input_Solvable_m100_uneven.png}
      \end{minipage}
\end{figure}

\begin{figure}[h]
      \centering
      \begin{minipage}[b]{0.45\textwidth}
            \caption{Percentage of Binomial inputs with perfect partitions for $m = 1000$ for even $n$}
            \includegraphics[width=\textwidth]{figures/images/solvabilityOfInputs/binomial_Input_Solvable_m1000_even.png}
      \end{minipage}
      \hspace{0.75cm}
      \begin{minipage}[b]{0.45\textwidth}
            \caption{Percentage of Binomial inputs with perfect partitions for $m = 1000$ for odd $n$}
            \includegraphics[width=\textwidth]{figures/images/solvabilityOfInputs/binomial_Input_Solvable_m1000_uneven.png}
      \end{minipage}
\end{figure}

\begin{figure}[h]
      \centering
      \begin{minipage}[b]{0.45\textwidth}
            \caption{Percentage of Binomial inputs with perfect partitions for $m = 10,000$}
            \includegraphics[width=\textwidth]{figures/images/solvabilityOfInputs/binomial_Input_Solvable_m10000.png}
      \end{minipage}
      \hspace{0.75cm}
      \begin{minipage}[b]{0.45\textwidth}
            \caption{Percentage of Binomial inputs with perfect partitions for $m = 100,000$}
            \includegraphics[width=\textwidth]{figures/images/solvabilityOfInputs/binomial_Input_Solvable_m100000.png}\label{fig:lastBinPercentage}
      \end{minipage}
\end{figure}

It is easy to see that for small inputs sizes it is relevant if $n$ is even or odd for higher expected values as all curves in figure~\ref{fig:lastBinPercentage} oscillate between 0\% and 100\% for $n\ge14$.
For odd inputs the probability of a perfect partition decreases much more drastically with $m$ as for even inputs because the expected value of a single number increases with $m$.
If all values are much higher the small differences between the values can no longer even out the fact of one set having more elements than the other.
The oscillation therefore increases with increasing m.
For $n=20$ all 1000 inputs had a perfect partition for every combination of $p$ and $m$ but for $n=19$ only combinations where $mp\le300$ holds had at least one input with a perfect partition.
For expected values of up to $10^5$ it seems to be almost granted that an input of length 20 has a perfect partition if it is binomial distributed.
Even for only 12 binomial generated values more than 50\% of the inputs had a perfect partition (see figure~\ref{fig:lastBinPercentage}).
Another visible effect is the decreasing percentage with rising $p$.
This may be a direct result of the value chosen for $p$ but can also be an indirect result as the value for $p$ changes the expected value for a constant $m$.
The expected value may have an influence on the number of perfect partitions because it influences the highest value of the input.
For uniform distributed inputs Borgs \etal~showed that the coefficient of number of bits needed to encode the max value/$n$ has a huge impact on the number of perfect partitions~\cite{borgs2001phase}.
For a coefficient < 1 the probability of a perfect partition tends to 1 and for a coefficient > 1 it tends to 0.
This was only proven for the uniform distributed input, but it might also hold for a binomial distributed input.
This leads to the second experiment.\newline
In the second experiment the inputs were generated a bit differently.
Here the goal was to keep the expected value fixed for any combination of $p$ and $n$ and set the value of $m$ to $e/p$ for all $e \in \{10, 20, 30, 40, 50, 100, 200, 500, 1000, 2000, 5000, 10000, 50000\}$ so that $E(X)=mp=e/p\cdot p=e$.
With this setup the influence of the expected value is almost isolated from the other parameters.
The probability is still linked to $p$ as $p$ also influences the variance $mp(1-p)$.\newline
Figure~\ref{fig:firstBinPercentage2} to figure~\ref{fig:lastBinPercentage2} again show the percentage of perfect inputs with different settings of $m,p,n$.
The $x$-axis is the expected value $mp$ of a single number of the input. The different graphs show the percentage for different input sizes.
It seems as if the value of $p$ has a much smaller influence than the expected value.
For a fixed expected value and a fixed input size a higher value for $p$ seems to only slightly increase the percentage of inputs with a perfect partition.
The expected value influences the percentage significantly more.
For $p=0.1, n=14$ the value decreases from 100\% at $E(X)=10$ to below 20\% at $E(X)=50000$ (figure~\ref{fig:firstBinPercentage2}).
For $p=0.9$ the percentage only drops below 50\% but still decreases by a factor of 2 (figure~\ref{fig:lastBinPercentage2}).

\begin{figure}[h]
      \centering
      \begin{minipage}[b]{0.45\textwidth}
            \caption{Percentage of Binomial inputs with perfect partitions for p = 0.1}
            \includegraphics[width=\textwidth]{figures/images/solvabilityOfInputs/solvability0_1.png}\label{fig:firstBinPercentage2}
      \end{minipage}
      \hspace{0.75cm}
      \begin{minipage}[b]{0.45\textwidth}
            \caption{Percentage of Binomial inputs with perfect partitions for p = 0.2}
            \includegraphics[width=\textwidth]{figures/images/solvabilityOfInputs/solvability0_2.png}
      \end{minipage}
\end{figure}


% \begin{figure}[h]
%       \centering
%       \begin{minipage}[b]{0.45\textwidth}
%             \caption{Percentage of Binomial inputs with perfect partitions for p = 0.3}
%             \includegraphics[width=\textwidth]{figures/images/solvabilityOfInputs/solvability0_3.png}
%       \end{minipage}
%       \hspace{0.75cm}
%       \begin{minipage}[b]{0.45\textwidth}
%             \caption{Percentage of Binomial inputs with perfect partitions for p = 0.4}
%             \includegraphics[width=\textwidth]{figures/images/solvabilityOfInputs/solvability0_4.png}
%       \end{minipage}
% \end{figure}


\begin{figure}[h]
      \centering
      \begin{minipage}[b]{0.45\textwidth}
            \caption{Percentage of Binomial inputs with perfect partitions for p = 0.5}
            \includegraphics[width=\textwidth]{figures/images/solvabilityOfInputs/solvability0_5.png}
      \end{minipage}
      \hspace{0.75cm}
      \begin{minipage}[b]{0.45\textwidth}
            \caption{Percentage of Binomial inputs with perfect partitions for p = 0.9}
            \includegraphics[width=\textwidth]{figures/images/solvabilityOfInputs/solvability0_9.png}\label{fig:lastBinPercentage2}
      \end{minipage}
\end{figure}

The last experiment showed that for $n=20$, 1000/1000 inputs had a perfect partition. This raised the question of how the amount of perfect partition changes with changing values for $m, p, n$.
Figure~\ref{fig:firstBinSolCount} to figure~\ref{fig:lastBinSolCount} show the amount of perfect partitions a binomial distribution \textasciitilde$B(m,p$) has.
For these figures 10,000 random binomial inputs with the given values for $m$ and $p$ were generated.
Each input was then tested for the number of perfect partitions it has.
The used method was again brute force to ensure correctness which was only possible due to the small input sizes.
After all runs the average values were combined in the given figures.
The value of $p$ is dependent on the picture and each value of $m \in \{10,100,1000,10000\}$ has its own graph within the figure.
The $x$-axis is the size of the input and the $y$-axis the number of perfect partitions the input has.
Notice that all graphs have a $y$-axis with a logarithmic scale.
Since the graphs are all linear the actual values rise exponentially.
The number of perfect partitions is mostly multiplied by a factor between 3 and 4 when the input size increases by 2.

\begin{figure}[h]
      \centering
      \begin{minipage}[b]{0.45\textwidth}
            \caption{Amount of perfect partitions for $p=0.1$}
            \includegraphics[width=\textwidth]{figures/images/solvabilityOfInputs/perfectPartitionCount-p0_1.png}\label{fig:firstBinSolCount}
      \end{minipage}
      \hspace{0.75cm}
      \begin{minipage}[b]{0.45\textwidth}
            \caption{Amount of perfect partitions for $p=0.9$}
            \includegraphics[width=\textwidth]{figures/images/solvabilityOfInputs/perfectPartitionCount-p0_9.png}
      \end{minipage}
\end{figure}

\begin{figure}[h]
      \centering
      \begin{minipage}[b]{0.45\textwidth}
            \caption{Amount of perfect partitions for $p=0.2$}
            \includegraphics[width=\textwidth]{figures/images/solvabilityOfInputs/perfectPartitionCount-p0_2.png}
      \end{minipage}
      \hspace{0.75cm}
      \begin{minipage}[b]{0.45\textwidth}
            \caption{Amount of perfect partitions for $p=0.5$}
            \includegraphics[width=\textwidth]{figures/images/solvabilityOfInputs/perfectPartitionCount-p0_5.png}\label{fig:lastBinSolCount}
      \end{minipage}
\end{figure}

The higher the value of $m$ the closer the curves of $p$ and $1-p$ get.
For $m=1000$ and $m=10000$ the values are almost the same for every input size.
% As in this case $m$ was not defined as $e/p$ with some values for $e$ but instead some values directly for $m$, the influence of the expected values distorts the image for the lower values of $m$.
For $p=0.1, m=10$ an input with expected values of 1 seems to much more likely to have a perfect partition than an input with expected value $10\cdot0.9=9$.
With growing $m$ this has less impact.
% Figure~\ref{fig:additionalBinSolCount} shows that for equal expected values the average amount of perfect partitions is the same for $p$ and $1-p$.
% Here this fact even holds for the smaller values of $m$.

% \begin{figure}[h]
%       \caption{Amount of perfect partitions for $p=0.1$ and $p=0.9$}
%       \centering
% \includegraphics[width=0.45\textwidth]{figures/images/solvabilityOfInputs/perfectPartitionCount-p0_1Andp0_9_2.png}\label{fig:additionalBinSolCount}
% \end{figure}

So the binomial input should be easy to solve due to the exponential number of perfect partitions.
It might be harder for the smaller values of $n$ as there are only a few perfect partitions.
Due to the small number of total possibilities it should still be easy to solve for the small values of $n$ as long as the RSH is not stuck in a local optimum.
The number of iterations might be high in terms of the big-O notation but should still be small in the absolute value.
% BEGIN RESULTS

% \section{Binomial distributed values}
This distribution has mostly small values, but occasionally it also generates bigger values.
The higher (absolute lower) the parameter the higher the values get and also the amount of big values increases.
For a parameter of $\beta=-2.75$ the distribution looks like in Figure~\ref{fig:powerDistExample1}.

\begin{figure}[h]
      \caption{Distribution of a random powerlaw input with $\beta=-2.75$}
      \centering
      \includegraphics[width=0.7\textwidth]{figures/images/numberGenerator/powerlaw_-2_75.png}\label{fig:powerDistExample1}
\end{figure}

For a value of $\beta=-1.25$ the distribution looks a bit different.
There are less small values close to one and instead also big values even over 1000.
Figure~\ref{fig:powerDistExample2} is cropped to get a more clear view for the smaller values.
The higher values mostly occurred 0 to 2 times.
The highest value 8848 occurred only once.

\begin{figure}[h]
      \caption{Distribution of a random powerlaw input with $\beta=-1.25$}
      \centering
      \includegraphics[width=0.7\textwidth]{figures/images/numberGenerator/powerlaw_-1_25.png}\label{fig:powerDistExample2}
\end{figure}
\subsection{RLS Comparison}
The following table lists the results for the RLS for inputs that are chosen from a powerlaw distribution with $\beta=-2.75$.


\makebox[\linewidth]{
\begin{tabular}{lp{3cm}p{6cm}p{6cm}}
\begin{tabular}[h]{cccccccc}
algo type&            RLS&   \RLSR[s]&   \RLSR[s]&   \RLSR[s]&   \RLSN[b]&   \RLSN[b]&   \RLSN[b]\\
algo param&             -&     s=2&     s=3&     s=4&     b=3&     b=2&     b=4\\
avg mut/change&     1.000&   1.181&   1.688&   1.865&   3.000&   1.997&   3,997\\
avg mut/step&       1.000&   1.500&   2.000&   2.500&   3.000&   2.000&   3.000\\
\hline
total avg count&   90,931& 168,311& 236,317& 307,533& 921,030& 921,030& 921,030\\
avg eval count&    90,931& 168,311& 236,317& 307,533&       -&       -&       -\\
max eval count&   156,854& 296,206& 498,474& 595,831&       -&       -&       -\\
min eval count&    64,941& 120,582& 158,304& 212,193&       -&       -&       -\\
\hline
fail ratio&         0.000&   0.000&   0.000&   0.000&   1.000&   1.000&   1.000\\
avg fail dif&           -&       -&       -&       -&      36&      53&     263\\
\end{tabular}
\end{tabular}
}


The picture for the RLS variants on this type of input is not clear.
There in no obvious tendency for neither of the variants.
The only obvious thing is the RLS being the worst of the RLS variants again.
Every variant reaches the optimal solution in every case except for the RLS which only manages for roughly 50 \% of the inputs.
The RLS-N(2) seems to be the best variant for these kinds of inputs.
The next best variants are the RLS-(R) with $k=3$ and $k=4$ which only differ by 1 \%.

\subsection{(1+1) EA Comparison}
For the (1+1) EA the best static mutation rate seems to be $3/n$. 
The probability of flipping 2 or 4 bits as n goes to infinity for mutation rate $1/n$ approaches $13/24e\approx 0.199$, for $2/n$ approaches $8/3e^2\approx 0.361$, for $3/n$ approaches $63/8e^3\approx 0.392$, for $4/n$ approaches $56/3e^4\approx 0.342$ and for $5/n$ approaches $77/2e^5\approx 0.259$. So the highest probability has $c=3$, followed by $c=4$ and $c=2$ then $c=5$ and lastly $c=1$. For higher values of $c$ the probability decreases further as the expected number of flipped bits is $c$ for mutation rate $c/n$.

\makebox[\linewidth]{
\begin{tabular}{lp{3cm}p{6cm}p{6cm}}
\begin{tabular}[h]{ccccccccc}
algo type&           EA-SM&       EA-SM&    EA-SM&    EA-SM&    EA-SM&    EA-SM&    EA-SM&    EA-SM\\
algo param&            2$/n$&        -&      3$/n$&      4$/n$&      5$/n$&     10$/n$&     50$/n$&    100$/n$\\
avg mut/change&      2.246&    1.551&    3.048&    3.936&    4.861&    9.822&   49.750&   99.707\\
avg mut/step&        2.000&    1.000&    3.000&    4.000&    5.000&   10.000&   50.000&  100.001\\
\hline
avg eval count&      3,097&    3,505&    3,518&    4,009&    4,807&    7,758&   18,457&   25,993\\
max eval count&     39,490&   60,533&   39,048&   47,881&   56,204&   91,305&  173,851&  354,479\\
min eval count&         10&        0&        6&        5&        3&        5&        9&        3\\
\hline
fail ratio&          0.000&    0.000&    0.000&    0.000&    0.000&    0.000&    0.000&    0.000\\
\end{tabular}
\end{tabular}
}


The (1+1) EA seems to perform better with a lower mutation rate.
The vales $p_m=2/n$ and $p_m=3/n$ reach an optimal solution equally fast.
From then on speed of convergence decreases with increasing mutation rate.
The only exception from this case is the (1+1) EA which performs the worst despite having the lowest mutation rate.
For the uniform distributed input all variants of the (1+1) EA reach an optimal solution within the step limit as for the previous input types.
\subsection{pmut Comparison}
The results for the $pmut$ operator are pretty similar to the results for the (1+1) EA and the RLS.
The parameter $\beta=-3.25$ which flips the least bits on average finds the solution the fastest.
The other values for $\beta$ increase the time needed for finding one of the two optimums with increasing value for $\beta$ (decreasing in the absolute value).
All variants find an optimum in every run except for $\beta=-1.25$ which has a much higher value for the number of flipped bits per steps.
The average number for the number of bits flipped in a successful mutation is much lower than for the other inputs especially for the higher (absolute lower) values for $\beta$.
For the binomial and geometric input the successful average was around 100 for $\beta=-1.25$ but for the OneMax equivalent it was only at 5.

\makebox[\linewidth]{
\scriptsize
\begin{tabular}{lp{3cm}p{6cm}p{6cm}}
\begin{tabular}[h]{m{2.5cm}m{0,40cm}m{0,40cm}m{0,40cm}m{0,40cm}m{0,40cm}m{0,40cm}m{0,40cm}m{0,40cm}m{0,40cm}m{0,40cm}m{0,40cm}m{0,40cm}m{0,40cm}m{0,40cm}m{0,40cm}m{0,40cm}m{0,40cm}m{0,40cm}}
\multicolumn{1}{c}{algo type}&\multicolumn{2}{c}{            pmut}&\multicolumn{2}{c}{     pmut}&\multicolumn{2}{c}{     pmut}&\multicolumn{2}{c}{     pmut}&\multicolumn{2}{c}{     pmut}&\multicolumn{2}{c}{     pmut}&\multicolumn{2}{c}{     pmut}&\multicolumn{2}{c}{     pmut}&\multicolumn{2}{c}{     pmut}\\
\multicolumn{1}{c}{algo param}&\multicolumn{2}{c}{           1.25}&\multicolumn{2}{c}{     1.50}&\multicolumn{2}{c}{     1.75}&\multicolumn{2}{c}{     2.00}&\multicolumn{2}{c}{     2.25}&\multicolumn{2}{c}{     2.50}&\multicolumn{2}{c}{     2.75}&\multicolumn{2}{c}{     3.00}&\multicolumn{2}{c}{     3.25}\\
\multicolumn{1}{c}{avg mut/change}&\multicolumn{2}{c}{    197.409}&\multicolumn{2}{c}{   70.534}&\multicolumn{2}{c}{   23.050}&\multicolumn{2}{c}{    8.724}&\multicolumn{2}{c}{    4.351}&\multicolumn{2}{c}{    2.777}&\multicolumn{2}{c}{    2.111}&\multicolumn{2}{c}{    1.770}&\multicolumn{2}{c}{    1.563}\\
\multicolumn{1}{c}{avg mut/step}&\multicolumn{2}{c}{      224.442}&\multicolumn{2}{c}{   70.480}&\multicolumn{2}{c}{   22.299}&\multicolumn{2}{c}{    8.470}&\multicolumn{2}{c}{    4.368}&\multicolumn{2}{c}{    2.906}&\multicolumn{2}{c}{    2.271}&\multicolumn{2}{c}{    1.934}&\multicolumn{2}{c}{    1.729}\\
\hline
\multicolumn{1}{c}{total avg count}&\multicolumn{2}{c}{        42}&\multicolumn{2}{c}{       87}&\multicolumn{2}{c}{      216}&\multicolumn{2}{c}{      503}&\multicolumn{2}{c}{    1,094}&\multicolumn{2}{c}{    4,063}&\multicolumn{2}{c}{   10,961}&\multicolumn{2}{c}{   18,727}&\multicolumn{2}{c}{   27,644}\\
\multicolumn{1}{c}{avg eval count}&\multicolumn{2}{c}{         42}&\multicolumn{2}{c}{       87}&\multicolumn{2}{c}{      216}&\multicolumn{2}{c}{      503}&\multicolumn{2}{c}{      910}&\multicolumn{2}{c}{    1,488}&\multicolumn{2}{c}{    1,676}&\multicolumn{2}{c}{    1,814}&\multicolumn{2}{c}{    1,909}\\
\multicolumn{1}{c}{max eval count}&\multicolumn{2}{c}{        303}&\multicolumn{2}{c}{      867}&\multicolumn{2}{c}{    2,843}&\multicolumn{2}{c}{    6,230}&\multicolumn{2}{c}{   10,487}&\multicolumn{2}{c}{  916,298}&\multicolumn{2}{c}{  501,346}&\multicolumn{2}{c}{  411,742}&\multicolumn{2}{c}{   12,386}\\
\multicolumn{1}{c}{min eval count}&\multicolumn{2}{c}{          0}&\multicolumn{2}{c}{        0}&\multicolumn{2}{c}{        0}&\multicolumn{2}{c}{        0}&\multicolumn{2}{c}{        0}&\multicolumn{2}{c}{        0}&\multicolumn{2}{c}{        0}&\multicolumn{2}{c}{        0}&\multicolumn{2}{c}{        0}\\
\hline
\multicolumn{1}{c}{fail ratio}&\multicolumn{2}{c}{          0.000}&\multicolumn{2}{c}{    0.000}&\multicolumn{2}{c}{    0.000}&\multicolumn{2}{c}{    0.000}&\multicolumn{2}{c}{    0.000}&\multicolumn{2}{c}{    0.003}&\multicolumn{2}{c}{    0.010}&\multicolumn{2}{c}{    0.018}&\multicolumn{2}{c}{    0.028}\\
\multicolumn{1}{c}{avg fail dif}&\multicolumn{2}{c}{            -}&\multicolumn{2}{c}{        -}&\multicolumn{2}{c}{        -}&\multicolumn{2}{c}{        -}&\multicolumn{2}{c}{      345}&\multicolumn{2}{c}{      345}&\multicolumn{2}{c}{      345}&\multicolumn{2}{c}{      345}&\multicolumn{2}{c}{      345}\\
\hline
\multicolumn{1}{c}{p-value}&&\multicolumn{2}{c}{0.0000}&\multicolumn{2}{c}{0.0000}&\multicolumn{2}{c}{0.0000}&\multicolumn{2}{c}{0.0000}&\multicolumn{2}{c}{0.0000}&\multicolumn{2}{c}{0.0000}&\multicolumn{2}{c}{0.0034}&\multicolumn{2}{c}{0.0068}\\
&&&&&&&&&&&&&&&&&&\end{tabular}
\end{tabular}
}


Here the same holds.
The higher mutation rates are less impacted than the higher rates for the RLS and the (1+1) EA.
\subsection{Comparison of the best variants}
\input{expRes/binomial/beforeBest}

\makebox[\linewidth]{
\begin{tabular}{lp{3cm}p{6cm}p{6cm}}
\begin{tabular}[h]{cccc}
algo type&        \RLSN& (1+1) EA&  pmut\\
algo param&         b=2&   3$/n$&  2.25\\
avg mut/change&   2.000& 3.092& 3.965\\
avg mut/step&     2.000& 2.999& 4.339\\
\hline
total avg count&    302&   677&   691\\
avg eval count&     302&   677&   691\\
max eval count&   1,610& 6,404& 5,205\\
min eval count&       9&    33&    17\\
\hline
fail ratio&       0.000& 0.000& 0.000\\
\end{tabular}
\end{tabular}
}


This input seems generally easy to solve as for every base algorithm are multiple variants which reach an optimal solution within 1000 steps.
The $pmut_-1.75$ variants reaches an optimal solution the fastest, but the other algorithms are almost equally fast.
All algorithms finish within $550 \pm 100$ steps on average and always in less than 2000 steps.

\input{tables/mixed/multipleN_fails.tex}

This input is only hard to solve for $n<100$.
For $n = 100$ there are only a few inputs that were not solved within the time limit and for $n\ge500$ the input is solved by each of the chosen algorithms.
This is probably caused by the many small values from the powerlaw and geometric distribution.


% \input{tables/mixed/multipleN_avg.tex}

\input{tables/mixed/multipleN_totalAvg.tex}
\section{Geometric distributed values}
This distribution has mostly small values, but occasionally it also generates bigger values.
The higher (absolute lower) the parameter the higher the values get and also the amount of big values increases.
For a parameter of $\beta=-2.75$ the distribution looks like in Figure~\ref{fig:powerDistExample1}.

\begin{figure}[h]
      \caption{Distribution of a random powerlaw input with $\beta=-2.75$}
      \centering
      \includegraphics[width=0.7\textwidth]{figures/images/numberGenerator/powerlaw_-2_75.png}\label{fig:powerDistExample1}
\end{figure}

For a value of $\beta=-1.25$ the distribution looks a bit different.
There are less small values close to one and instead also big values even over 1000.
Figure~\ref{fig:powerDistExample2} is cropped to get a more clear view for the smaller values.
The higher values mostly occurred 0 to 2 times.
The highest value 8848 occurred only once.

\begin{figure}[h]
      \caption{Distribution of a random powerlaw input with $\beta=-1.25$}
      \centering
      \includegraphics[width=0.7\textwidth]{figures/images/numberGenerator/powerlaw_-1_25.png}\label{fig:powerDistExample2}
\end{figure}
\subsection{RLS Comparison}
The following table lists the results for the RLS for inputs that are chosen from a powerlaw distribution with $\beta=-2.75$.


\makebox[\linewidth]{
\begin{tabular}{lp{3cm}p{6cm}p{6cm}}
\begin{tabular}[h]{cccccccc}
algo type&            RLS&   \RLSR[s]&   \RLSR[s]&   \RLSR[s]&   \RLSN[b]&   \RLSN[b]&   \RLSN[b]\\
algo param&             -&     s=2&     s=3&     s=4&     b=3&     b=2&     b=4\\
avg mut/change&     1.000&   1.181&   1.688&   1.865&   3.000&   1.997&   3,997\\
avg mut/step&       1.000&   1.500&   2.000&   2.500&   3.000&   2.000&   3.000\\
\hline
total avg count&   90,931& 168,311& 236,317& 307,533& 921,030& 921,030& 921,030\\
avg eval count&    90,931& 168,311& 236,317& 307,533&       -&       -&       -\\
max eval count&   156,854& 296,206& 498,474& 595,831&       -&       -&       -\\
min eval count&    64,941& 120,582& 158,304& 212,193&       -&       -&       -\\
\hline
fail ratio&         0.000&   0.000&   0.000&   0.000&   1.000&   1.000&   1.000\\
avg fail dif&           -&       -&       -&       -&      36&      53&     263\\
\end{tabular}
\end{tabular}
}


The picture for the RLS variants on this type of input is not clear.
There in no obvious tendency for neither of the variants.
The only obvious thing is the RLS being the worst of the RLS variants again.
Every variant reaches the optimal solution in every case except for the RLS which only manages for roughly 50 \% of the inputs.
The RLS-N(2) seems to be the best variant for these kinds of inputs.
The next best variants are the RLS-(R) with $k=3$ and $k=4$ which only differ by 1 \%.

\subsection{(1+1) EA Comparison}
For the (1+1) EA the best static mutation rate seems to be $3/n$. 
The probability of flipping 2 or 4 bits as n goes to infinity for mutation rate $1/n$ approaches $13/24e\approx 0.199$, for $2/n$ approaches $8/3e^2\approx 0.361$, for $3/n$ approaches $63/8e^3\approx 0.392$, for $4/n$ approaches $56/3e^4\approx 0.342$ and for $5/n$ approaches $77/2e^5\approx 0.259$. So the highest probability has $c=3$, followed by $c=4$ and $c=2$ then $c=5$ and lastly $c=1$. For higher values of $c$ the probability decreases further as the expected number of flipped bits is $c$ for mutation rate $c/n$.

\makebox[\linewidth]{
\begin{tabular}{lp{3cm}p{6cm}p{6cm}}
\begin{tabular}[h]{ccccccccc}
algo type&           EA-SM&       EA-SM&    EA-SM&    EA-SM&    EA-SM&    EA-SM&    EA-SM&    EA-SM\\
algo param&            2$/n$&        -&      3$/n$&      4$/n$&      5$/n$&     10$/n$&     50$/n$&    100$/n$\\
avg mut/change&      2.246&    1.551&    3.048&    3.936&    4.861&    9.822&   49.750&   99.707\\
avg mut/step&        2.000&    1.000&    3.000&    4.000&    5.000&   10.000&   50.000&  100.001\\
\hline
avg eval count&      3,097&    3,505&    3,518&    4,009&    4,807&    7,758&   18,457&   25,993\\
max eval count&     39,490&   60,533&   39,048&   47,881&   56,204&   91,305&  173,851&  354,479\\
min eval count&         10&        0&        6&        5&        3&        5&        9&        3\\
\hline
fail ratio&          0.000&    0.000&    0.000&    0.000&    0.000&    0.000&    0.000&    0.000\\
\end{tabular}
\end{tabular}
}


The (1+1) EA seems to perform better with a lower mutation rate.
The vales $p_m=2/n$ and $p_m=3/n$ reach an optimal solution equally fast.
From then on speed of convergence decreases with increasing mutation rate.
The only exception from this case is the (1+1) EA which performs the worst despite having the lowest mutation rate.
For the uniform distributed input all variants of the (1+1) EA reach an optimal solution within the step limit as for the previous input types.
\subsection{pmut Comparison}
The results for the $pmut$ operator are pretty similar to the results for the (1+1) EA and the RLS.
The parameter $\beta=-3.25$ which flips the least bits on average finds the solution the fastest.
The other values for $\beta$ increase the time needed for finding one of the two optimums with increasing value for $\beta$ (decreasing in the absolute value).
All variants find an optimum in every run except for $\beta=-1.25$ which has a much higher value for the number of flipped bits per steps.
The average number for the number of bits flipped in a successful mutation is much lower than for the other inputs especially for the higher (absolute lower) values for $\beta$.
For the binomial and geometric input the successful average was around 100 for $\beta=-1.25$ but for the OneMax equivalent it was only at 5.

\makebox[\linewidth]{
\scriptsize
\begin{tabular}{lp{3cm}p{6cm}p{6cm}}
\begin{tabular}[h]{m{2.5cm}m{0,40cm}m{0,40cm}m{0,40cm}m{0,40cm}m{0,40cm}m{0,40cm}m{0,40cm}m{0,40cm}m{0,40cm}m{0,40cm}m{0,40cm}m{0,40cm}m{0,40cm}m{0,40cm}m{0,40cm}m{0,40cm}m{0,40cm}m{0,40cm}}
\multicolumn{1}{c}{algo type}&\multicolumn{2}{c}{            pmut}&\multicolumn{2}{c}{     pmut}&\multicolumn{2}{c}{     pmut}&\multicolumn{2}{c}{     pmut}&\multicolumn{2}{c}{     pmut}&\multicolumn{2}{c}{     pmut}&\multicolumn{2}{c}{     pmut}&\multicolumn{2}{c}{     pmut}&\multicolumn{2}{c}{     pmut}\\
\multicolumn{1}{c}{algo param}&\multicolumn{2}{c}{           1.25}&\multicolumn{2}{c}{     1.50}&\multicolumn{2}{c}{     1.75}&\multicolumn{2}{c}{     2.00}&\multicolumn{2}{c}{     2.25}&\multicolumn{2}{c}{     2.50}&\multicolumn{2}{c}{     2.75}&\multicolumn{2}{c}{     3.00}&\multicolumn{2}{c}{     3.25}\\
\multicolumn{1}{c}{avg mut/change}&\multicolumn{2}{c}{    197.409}&\multicolumn{2}{c}{   70.534}&\multicolumn{2}{c}{   23.050}&\multicolumn{2}{c}{    8.724}&\multicolumn{2}{c}{    4.351}&\multicolumn{2}{c}{    2.777}&\multicolumn{2}{c}{    2.111}&\multicolumn{2}{c}{    1.770}&\multicolumn{2}{c}{    1.563}\\
\multicolumn{1}{c}{avg mut/step}&\multicolumn{2}{c}{      224.442}&\multicolumn{2}{c}{   70.480}&\multicolumn{2}{c}{   22.299}&\multicolumn{2}{c}{    8.470}&\multicolumn{2}{c}{    4.368}&\multicolumn{2}{c}{    2.906}&\multicolumn{2}{c}{    2.271}&\multicolumn{2}{c}{    1.934}&\multicolumn{2}{c}{    1.729}\\
\hline
\multicolumn{1}{c}{total avg count}&\multicolumn{2}{c}{        42}&\multicolumn{2}{c}{       87}&\multicolumn{2}{c}{      216}&\multicolumn{2}{c}{      503}&\multicolumn{2}{c}{    1,094}&\multicolumn{2}{c}{    4,063}&\multicolumn{2}{c}{   10,961}&\multicolumn{2}{c}{   18,727}&\multicolumn{2}{c}{   27,644}\\
\multicolumn{1}{c}{avg eval count}&\multicolumn{2}{c}{         42}&\multicolumn{2}{c}{       87}&\multicolumn{2}{c}{      216}&\multicolumn{2}{c}{      503}&\multicolumn{2}{c}{      910}&\multicolumn{2}{c}{    1,488}&\multicolumn{2}{c}{    1,676}&\multicolumn{2}{c}{    1,814}&\multicolumn{2}{c}{    1,909}\\
\multicolumn{1}{c}{max eval count}&\multicolumn{2}{c}{        303}&\multicolumn{2}{c}{      867}&\multicolumn{2}{c}{    2,843}&\multicolumn{2}{c}{    6,230}&\multicolumn{2}{c}{   10,487}&\multicolumn{2}{c}{  916,298}&\multicolumn{2}{c}{  501,346}&\multicolumn{2}{c}{  411,742}&\multicolumn{2}{c}{   12,386}\\
\multicolumn{1}{c}{min eval count}&\multicolumn{2}{c}{          0}&\multicolumn{2}{c}{        0}&\multicolumn{2}{c}{        0}&\multicolumn{2}{c}{        0}&\multicolumn{2}{c}{        0}&\multicolumn{2}{c}{        0}&\multicolumn{2}{c}{        0}&\multicolumn{2}{c}{        0}&\multicolumn{2}{c}{        0}\\
\hline
\multicolumn{1}{c}{fail ratio}&\multicolumn{2}{c}{          0.000}&\multicolumn{2}{c}{    0.000}&\multicolumn{2}{c}{    0.000}&\multicolumn{2}{c}{    0.000}&\multicolumn{2}{c}{    0.000}&\multicolumn{2}{c}{    0.003}&\multicolumn{2}{c}{    0.010}&\multicolumn{2}{c}{    0.018}&\multicolumn{2}{c}{    0.028}\\
\multicolumn{1}{c}{avg fail dif}&\multicolumn{2}{c}{            -}&\multicolumn{2}{c}{        -}&\multicolumn{2}{c}{        -}&\multicolumn{2}{c}{        -}&\multicolumn{2}{c}{      345}&\multicolumn{2}{c}{      345}&\multicolumn{2}{c}{      345}&\multicolumn{2}{c}{      345}&\multicolumn{2}{c}{      345}\\
\hline
\multicolumn{1}{c}{p-value}&&\multicolumn{2}{c}{0.0000}&\multicolumn{2}{c}{0.0000}&\multicolumn{2}{c}{0.0000}&\multicolumn{2}{c}{0.0000}&\multicolumn{2}{c}{0.0000}&\multicolumn{2}{c}{0.0000}&\multicolumn{2}{c}{0.0034}&\multicolumn{2}{c}{0.0068}\\
&&&&&&&&&&&&&&&&&&\end{tabular}
\end{tabular}
}


Here the same holds.
The higher mutation rates are less impacted than the higher rates for the RLS and the (1+1) EA.
\subsection{Comparison of the best variants}
\input{expRes/geometric/beforeBest}

\makebox[\linewidth]{
\begin{tabular}{lp{3cm}p{6cm}p{6cm}}
\begin{tabular}[h]{cccc}
algo type&        \RLSN& (1+1) EA&  pmut\\
algo param&         b=2&   3$/n$&  2.25\\
avg mut/change&   2.000& 3.092& 3.965\\
avg mut/step&     2.000& 2.999& 4.339\\
\hline
total avg count&    302&   677&   691\\
avg eval count&     302&   677&   691\\
max eval count&   1,610& 6,404& 5,205\\
min eval count&       9&    33&    17\\
\hline
fail ratio&       0.000& 0.000& 0.000\\
\end{tabular}
\end{tabular}
}


This input seems generally easy to solve as for every base algorithm are multiple variants which reach an optimal solution within 1000 steps.
The $pmut_-1.75$ variants reaches an optimal solution the fastest, but the other algorithms are almost equally fast.
All algorithms finish within $550 \pm 100$ steps on average and always in less than 2000 steps.

\input{tables/mixed/multipleN_fails.tex}

This input is only hard to solve for $n<100$.
For $n = 100$ there are only a few inputs that were not solved within the time limit and for $n\ge500$ the input is solved by each of the chosen algorithms.
This is probably caused by the many small values from the powerlaw and geometric distribution.


% \input{tables/mixed/multipleN_avg.tex}

\input{tables/mixed/multipleN_totalAvg.tex}
\section{Uniform distributed inputs}
This distribution has mostly small values, but occasionally it also generates bigger values.
The higher (absolute lower) the parameter the higher the values get and also the amount of big values increases.
For a parameter of $\beta=-2.75$ the distribution looks like in Figure~\ref{fig:powerDistExample1}.

\begin{figure}[h]
      \caption{Distribution of a random powerlaw input with $\beta=-2.75$}
      \centering
      \includegraphics[width=0.7\textwidth]{figures/images/numberGenerator/powerlaw_-2_75.png}\label{fig:powerDistExample1}
\end{figure}

For a value of $\beta=-1.25$ the distribution looks a bit different.
There are less small values close to one and instead also big values even over 1000.
Figure~\ref{fig:powerDistExample2} is cropped to get a more clear view for the smaller values.
The higher values mostly occurred 0 to 2 times.
The highest value 8848 occurred only once.

\begin{figure}[h]
      \caption{Distribution of a random powerlaw input with $\beta=-1.25$}
      \centering
      \includegraphics[width=0.7\textwidth]{figures/images/numberGenerator/powerlaw_-1_25.png}\label{fig:powerDistExample2}
\end{figure}
\subsection{RLS Comparison}
The following table lists the results for the RLS for inputs that are chosen from a powerlaw distribution with $\beta=-2.75$.


\makebox[\linewidth]{
\begin{tabular}{lp{3cm}p{6cm}p{6cm}}
\begin{tabular}[h]{cccccccc}
algo type&            RLS&   \RLSR[s]&   \RLSR[s]&   \RLSR[s]&   \RLSN[b]&   \RLSN[b]&   \RLSN[b]\\
algo param&             -&     s=2&     s=3&     s=4&     b=3&     b=2&     b=4\\
avg mut/change&     1.000&   1.181&   1.688&   1.865&   3.000&   1.997&   3,997\\
avg mut/step&       1.000&   1.500&   2.000&   2.500&   3.000&   2.000&   3.000\\
\hline
total avg count&   90,931& 168,311& 236,317& 307,533& 921,030& 921,030& 921,030\\
avg eval count&    90,931& 168,311& 236,317& 307,533&       -&       -&       -\\
max eval count&   156,854& 296,206& 498,474& 595,831&       -&       -&       -\\
min eval count&    64,941& 120,582& 158,304& 212,193&       -&       -&       -\\
\hline
fail ratio&         0.000&   0.000&   0.000&   0.000&   1.000&   1.000&   1.000\\
avg fail dif&           -&       -&       -&       -&      36&      53&     263\\
\end{tabular}
\end{tabular}
}


The picture for the RLS variants on this type of input is not clear.
There in no obvious tendency for neither of the variants.
The only obvious thing is the RLS being the worst of the RLS variants again.
Every variant reaches the optimal solution in every case except for the RLS which only manages for roughly 50 \% of the inputs.
The RLS-N(2) seems to be the best variant for these kinds of inputs.
The next best variants are the RLS-(R) with $k=3$ and $k=4$ which only differ by 1 \%.

\subsection{(1+1) EA Comparison}
For the (1+1) EA the best static mutation rate seems to be $3/n$. 
The probability of flipping 2 or 4 bits as n goes to infinity for mutation rate $1/n$ approaches $13/24e\approx 0.199$, for $2/n$ approaches $8/3e^2\approx 0.361$, for $3/n$ approaches $63/8e^3\approx 0.392$, for $4/n$ approaches $56/3e^4\approx 0.342$ and for $5/n$ approaches $77/2e^5\approx 0.259$. So the highest probability has $c=3$, followed by $c=4$ and $c=2$ then $c=5$ and lastly $c=1$. For higher values of $c$ the probability decreases further as the expected number of flipped bits is $c$ for mutation rate $c/n$.

\makebox[\linewidth]{
\begin{tabular}{lp{3cm}p{6cm}p{6cm}}
\begin{tabular}[h]{ccccccccc}
algo type&           EA-SM&       EA-SM&    EA-SM&    EA-SM&    EA-SM&    EA-SM&    EA-SM&    EA-SM\\
algo param&            2$/n$&        -&      3$/n$&      4$/n$&      5$/n$&     10$/n$&     50$/n$&    100$/n$\\
avg mut/change&      2.246&    1.551&    3.048&    3.936&    4.861&    9.822&   49.750&   99.707\\
avg mut/step&        2.000&    1.000&    3.000&    4.000&    5.000&   10.000&   50.000&  100.001\\
\hline
avg eval count&      3,097&    3,505&    3,518&    4,009&    4,807&    7,758&   18,457&   25,993\\
max eval count&     39,490&   60,533&   39,048&   47,881&   56,204&   91,305&  173,851&  354,479\\
min eval count&         10&        0&        6&        5&        3&        5&        9&        3\\
\hline
fail ratio&          0.000&    0.000&    0.000&    0.000&    0.000&    0.000&    0.000&    0.000\\
\end{tabular}
\end{tabular}
}


The (1+1) EA seems to perform better with a lower mutation rate.
The vales $p_m=2/n$ and $p_m=3/n$ reach an optimal solution equally fast.
From then on speed of convergence decreases with increasing mutation rate.
The only exception from this case is the (1+1) EA which performs the worst despite having the lowest mutation rate.
For the uniform distributed input all variants of the (1+1) EA reach an optimal solution within the step limit as for the previous input types.
\subsection{pmut Comparison}
The results for the $pmut$ operator are pretty similar to the results for the (1+1) EA and the RLS.
The parameter $\beta=-3.25$ which flips the least bits on average finds the solution the fastest.
The other values for $\beta$ increase the time needed for finding one of the two optimums with increasing value for $\beta$ (decreasing in the absolute value).
All variants find an optimum in every run except for $\beta=-1.25$ which has a much higher value for the number of flipped bits per steps.
The average number for the number of bits flipped in a successful mutation is much lower than for the other inputs especially for the higher (absolute lower) values for $\beta$.
For the binomial and geometric input the successful average was around 100 for $\beta=-1.25$ but for the OneMax equivalent it was only at 5.

\makebox[\linewidth]{
\scriptsize
\begin{tabular}{lp{3cm}p{6cm}p{6cm}}
\begin{tabular}[h]{m{2.5cm}m{0,40cm}m{0,40cm}m{0,40cm}m{0,40cm}m{0,40cm}m{0,40cm}m{0,40cm}m{0,40cm}m{0,40cm}m{0,40cm}m{0,40cm}m{0,40cm}m{0,40cm}m{0,40cm}m{0,40cm}m{0,40cm}m{0,40cm}m{0,40cm}}
\multicolumn{1}{c}{algo type}&\multicolumn{2}{c}{            pmut}&\multicolumn{2}{c}{     pmut}&\multicolumn{2}{c}{     pmut}&\multicolumn{2}{c}{     pmut}&\multicolumn{2}{c}{     pmut}&\multicolumn{2}{c}{     pmut}&\multicolumn{2}{c}{     pmut}&\multicolumn{2}{c}{     pmut}&\multicolumn{2}{c}{     pmut}\\
\multicolumn{1}{c}{algo param}&\multicolumn{2}{c}{           1.25}&\multicolumn{2}{c}{     1.50}&\multicolumn{2}{c}{     1.75}&\multicolumn{2}{c}{     2.00}&\multicolumn{2}{c}{     2.25}&\multicolumn{2}{c}{     2.50}&\multicolumn{2}{c}{     2.75}&\multicolumn{2}{c}{     3.00}&\multicolumn{2}{c}{     3.25}\\
\multicolumn{1}{c}{avg mut/change}&\multicolumn{2}{c}{    197.409}&\multicolumn{2}{c}{   70.534}&\multicolumn{2}{c}{   23.050}&\multicolumn{2}{c}{    8.724}&\multicolumn{2}{c}{    4.351}&\multicolumn{2}{c}{    2.777}&\multicolumn{2}{c}{    2.111}&\multicolumn{2}{c}{    1.770}&\multicolumn{2}{c}{    1.563}\\
\multicolumn{1}{c}{avg mut/step}&\multicolumn{2}{c}{      224.442}&\multicolumn{2}{c}{   70.480}&\multicolumn{2}{c}{   22.299}&\multicolumn{2}{c}{    8.470}&\multicolumn{2}{c}{    4.368}&\multicolumn{2}{c}{    2.906}&\multicolumn{2}{c}{    2.271}&\multicolumn{2}{c}{    1.934}&\multicolumn{2}{c}{    1.729}\\
\hline
\multicolumn{1}{c}{total avg count}&\multicolumn{2}{c}{        42}&\multicolumn{2}{c}{       87}&\multicolumn{2}{c}{      216}&\multicolumn{2}{c}{      503}&\multicolumn{2}{c}{    1,094}&\multicolumn{2}{c}{    4,063}&\multicolumn{2}{c}{   10,961}&\multicolumn{2}{c}{   18,727}&\multicolumn{2}{c}{   27,644}\\
\multicolumn{1}{c}{avg eval count}&\multicolumn{2}{c}{         42}&\multicolumn{2}{c}{       87}&\multicolumn{2}{c}{      216}&\multicolumn{2}{c}{      503}&\multicolumn{2}{c}{      910}&\multicolumn{2}{c}{    1,488}&\multicolumn{2}{c}{    1,676}&\multicolumn{2}{c}{    1,814}&\multicolumn{2}{c}{    1,909}\\
\multicolumn{1}{c}{max eval count}&\multicolumn{2}{c}{        303}&\multicolumn{2}{c}{      867}&\multicolumn{2}{c}{    2,843}&\multicolumn{2}{c}{    6,230}&\multicolumn{2}{c}{   10,487}&\multicolumn{2}{c}{  916,298}&\multicolumn{2}{c}{  501,346}&\multicolumn{2}{c}{  411,742}&\multicolumn{2}{c}{   12,386}\\
\multicolumn{1}{c}{min eval count}&\multicolumn{2}{c}{          0}&\multicolumn{2}{c}{        0}&\multicolumn{2}{c}{        0}&\multicolumn{2}{c}{        0}&\multicolumn{2}{c}{        0}&\multicolumn{2}{c}{        0}&\multicolumn{2}{c}{        0}&\multicolumn{2}{c}{        0}&\multicolumn{2}{c}{        0}\\
\hline
\multicolumn{1}{c}{fail ratio}&\multicolumn{2}{c}{          0.000}&\multicolumn{2}{c}{    0.000}&\multicolumn{2}{c}{    0.000}&\multicolumn{2}{c}{    0.000}&\multicolumn{2}{c}{    0.000}&\multicolumn{2}{c}{    0.003}&\multicolumn{2}{c}{    0.010}&\multicolumn{2}{c}{    0.018}&\multicolumn{2}{c}{    0.028}\\
\multicolumn{1}{c}{avg fail dif}&\multicolumn{2}{c}{            -}&\multicolumn{2}{c}{        -}&\multicolumn{2}{c}{        -}&\multicolumn{2}{c}{        -}&\multicolumn{2}{c}{      345}&\multicolumn{2}{c}{      345}&\multicolumn{2}{c}{      345}&\multicolumn{2}{c}{      345}&\multicolumn{2}{c}{      345}\\
\hline
\multicolumn{1}{c}{p-value}&&\multicolumn{2}{c}{0.0000}&\multicolumn{2}{c}{0.0000}&\multicolumn{2}{c}{0.0000}&\multicolumn{2}{c}{0.0000}&\multicolumn{2}{c}{0.0000}&\multicolumn{2}{c}{0.0000}&\multicolumn{2}{c}{0.0034}&\multicolumn{2}{c}{0.0068}\\
&&&&&&&&&&&&&&&&&&\end{tabular}
\end{tabular}
}


Here the same holds.
The higher mutation rates are less impacted than the higher rates for the RLS and the (1+1) EA.
\subsection{Comparison of the best variants}
\input{expRes/uniform/beforeBest}

\makebox[\linewidth]{
\begin{tabular}{lp{3cm}p{6cm}p{6cm}}
\begin{tabular}[h]{cccc}
algo type&        \RLSN& (1+1) EA&  pmut\\
algo param&         b=2&   3$/n$&  2.25\\
avg mut/change&   2.000& 3.092& 3.965\\
avg mut/step&     2.000& 2.999& 4.339\\
\hline
total avg count&    302&   677&   691\\
avg eval count&     302&   677&   691\\
max eval count&   1,610& 6,404& 5,205\\
min eval count&       9&    33&    17\\
\hline
fail ratio&       0.000& 0.000& 0.000\\
\end{tabular}
\end{tabular}
}


This input seems generally easy to solve as for every base algorithm are multiple variants which reach an optimal solution within 1000 steps.
The $pmut_-1.75$ variants reaches an optimal solution the fastest, but the other algorithms are almost equally fast.
All algorithms finish within $550 \pm 100$ steps on average and always in less than 2000 steps.

\input{tables/mixed/multipleN_fails.tex}

This input is only hard to solve for $n<100$.
For $n = 100$ there are only a few inputs that were not solved within the time limit and for $n\ge500$ the input is solved by each of the chosen algorithms.
This is probably caused by the many small values from the powerlaw and geometric distribution.


% \input{tables/mixed/multipleN_avg.tex}

\input{tables/mixed/multipleN_totalAvg.tex}
\section{powerlaw distributed inputs}
This distribution has mostly small values, but occasionally it also generates bigger values.
The higher (absolute lower) the parameter the higher the values get and also the amount of big values increases.
For a parameter of $\beta=-2.75$ the distribution looks like in Figure~\ref{fig:powerDistExample1}.

\begin{figure}[h]
      \caption{Distribution of a random powerlaw input with $\beta=-2.75$}
      \centering
      \includegraphics[width=0.7\textwidth]{figures/images/numberGenerator/powerlaw_-2_75.png}\label{fig:powerDistExample1}
\end{figure}

For a value of $\beta=-1.25$ the distribution looks a bit different.
There are less small values close to one and instead also big values even over 1000.
Figure~\ref{fig:powerDistExample2} is cropped to get a more clear view for the smaller values.
The higher values mostly occurred 0 to 2 times.
The highest value 8848 occurred only once.

\begin{figure}[h]
      \caption{Distribution of a random powerlaw input with $\beta=-1.25$}
      \centering
      \includegraphics[width=0.7\textwidth]{figures/images/numberGenerator/powerlaw_-1_25.png}\label{fig:powerDistExample2}
\end{figure}
\subsection{RLS Comparison}
The following table lists the results for the RLS for inputs that are chosen from a powerlaw distribution with $\beta=-2.75$.


\makebox[\linewidth]{
\begin{tabular}{lp{3cm}p{6cm}p{6cm}}
\begin{tabular}[h]{cccccccc}
algo type&            RLS&   \RLSR[s]&   \RLSR[s]&   \RLSR[s]&   \RLSN[b]&   \RLSN[b]&   \RLSN[b]\\
algo param&             -&     s=2&     s=3&     s=4&     b=3&     b=2&     b=4\\
avg mut/change&     1.000&   1.181&   1.688&   1.865&   3.000&   1.997&   3,997\\
avg mut/step&       1.000&   1.500&   2.000&   2.500&   3.000&   2.000&   3.000\\
\hline
total avg count&   90,931& 168,311& 236,317& 307,533& 921,030& 921,030& 921,030\\
avg eval count&    90,931& 168,311& 236,317& 307,533&       -&       -&       -\\
max eval count&   156,854& 296,206& 498,474& 595,831&       -&       -&       -\\
min eval count&    64,941& 120,582& 158,304& 212,193&       -&       -&       -\\
\hline
fail ratio&         0.000&   0.000&   0.000&   0.000&   1.000&   1.000&   1.000\\
avg fail dif&           -&       -&       -&       -&      36&      53&     263\\
\end{tabular}
\end{tabular}
}


The picture for the RLS variants on this type of input is not clear.
There in no obvious tendency for neither of the variants.
The only obvious thing is the RLS being the worst of the RLS variants again.
Every variant reaches the optimal solution in every case except for the RLS which only manages for roughly 50 \% of the inputs.
The RLS-N(2) seems to be the best variant for these kinds of inputs.
The next best variants are the RLS-(R) with $k=3$ and $k=4$ which only differ by 1 \%.

\subsection{(1+1) EA Comparison}
For the (1+1) EA the best static mutation rate seems to be $3/n$. 
The probability of flipping 2 or 4 bits as n goes to infinity for mutation rate $1/n$ approaches $13/24e\approx 0.199$, for $2/n$ approaches $8/3e^2\approx 0.361$, for $3/n$ approaches $63/8e^3\approx 0.392$, for $4/n$ approaches $56/3e^4\approx 0.342$ and for $5/n$ approaches $77/2e^5\approx 0.259$. So the highest probability has $c=3$, followed by $c=4$ and $c=2$ then $c=5$ and lastly $c=1$. For higher values of $c$ the probability decreases further as the expected number of flipped bits is $c$ for mutation rate $c/n$.

\makebox[\linewidth]{
\begin{tabular}{lp{3cm}p{6cm}p{6cm}}
\begin{tabular}[h]{ccccccccc}
algo type&           EA-SM&       EA-SM&    EA-SM&    EA-SM&    EA-SM&    EA-SM&    EA-SM&    EA-SM\\
algo param&            2$/n$&        -&      3$/n$&      4$/n$&      5$/n$&     10$/n$&     50$/n$&    100$/n$\\
avg mut/change&      2.246&    1.551&    3.048&    3.936&    4.861&    9.822&   49.750&   99.707\\
avg mut/step&        2.000&    1.000&    3.000&    4.000&    5.000&   10.000&   50.000&  100.001\\
\hline
avg eval count&      3,097&    3,505&    3,518&    4,009&    4,807&    7,758&   18,457&   25,993\\
max eval count&     39,490&   60,533&   39,048&   47,881&   56,204&   91,305&  173,851&  354,479\\
min eval count&         10&        0&        6&        5&        3&        5&        9&        3\\
\hline
fail ratio&          0.000&    0.000&    0.000&    0.000&    0.000&    0.000&    0.000&    0.000\\
\end{tabular}
\end{tabular}
}


The (1+1) EA seems to perform better with a lower mutation rate.
The vales $p_m=2/n$ and $p_m=3/n$ reach an optimal solution equally fast.
From then on speed of convergence decreases with increasing mutation rate.
The only exception from this case is the (1+1) EA which performs the worst despite having the lowest mutation rate.
For the uniform distributed input all variants of the (1+1) EA reach an optimal solution within the step limit as for the previous input types.
\subsection{pmut Comparison}
The results for the $pmut$ operator are pretty similar to the results for the (1+1) EA and the RLS.
The parameter $\beta=-3.25$ which flips the least bits on average finds the solution the fastest.
The other values for $\beta$ increase the time needed for finding one of the two optimums with increasing value for $\beta$ (decreasing in the absolute value).
All variants find an optimum in every run except for $\beta=-1.25$ which has a much higher value for the number of flipped bits per steps.
The average number for the number of bits flipped in a successful mutation is much lower than for the other inputs especially for the higher (absolute lower) values for $\beta$.
For the binomial and geometric input the successful average was around 100 for $\beta=-1.25$ but for the OneMax equivalent it was only at 5.

\makebox[\linewidth]{
\scriptsize
\begin{tabular}{lp{3cm}p{6cm}p{6cm}}
\begin{tabular}[h]{m{2.5cm}m{0,40cm}m{0,40cm}m{0,40cm}m{0,40cm}m{0,40cm}m{0,40cm}m{0,40cm}m{0,40cm}m{0,40cm}m{0,40cm}m{0,40cm}m{0,40cm}m{0,40cm}m{0,40cm}m{0,40cm}m{0,40cm}m{0,40cm}m{0,40cm}}
\multicolumn{1}{c}{algo type}&\multicolumn{2}{c}{            pmut}&\multicolumn{2}{c}{     pmut}&\multicolumn{2}{c}{     pmut}&\multicolumn{2}{c}{     pmut}&\multicolumn{2}{c}{     pmut}&\multicolumn{2}{c}{     pmut}&\multicolumn{2}{c}{     pmut}&\multicolumn{2}{c}{     pmut}&\multicolumn{2}{c}{     pmut}\\
\multicolumn{1}{c}{algo param}&\multicolumn{2}{c}{           1.25}&\multicolumn{2}{c}{     1.50}&\multicolumn{2}{c}{     1.75}&\multicolumn{2}{c}{     2.00}&\multicolumn{2}{c}{     2.25}&\multicolumn{2}{c}{     2.50}&\multicolumn{2}{c}{     2.75}&\multicolumn{2}{c}{     3.00}&\multicolumn{2}{c}{     3.25}\\
\multicolumn{1}{c}{avg mut/change}&\multicolumn{2}{c}{    197.409}&\multicolumn{2}{c}{   70.534}&\multicolumn{2}{c}{   23.050}&\multicolumn{2}{c}{    8.724}&\multicolumn{2}{c}{    4.351}&\multicolumn{2}{c}{    2.777}&\multicolumn{2}{c}{    2.111}&\multicolumn{2}{c}{    1.770}&\multicolumn{2}{c}{    1.563}\\
\multicolumn{1}{c}{avg mut/step}&\multicolumn{2}{c}{      224.442}&\multicolumn{2}{c}{   70.480}&\multicolumn{2}{c}{   22.299}&\multicolumn{2}{c}{    8.470}&\multicolumn{2}{c}{    4.368}&\multicolumn{2}{c}{    2.906}&\multicolumn{2}{c}{    2.271}&\multicolumn{2}{c}{    1.934}&\multicolumn{2}{c}{    1.729}\\
\hline
\multicolumn{1}{c}{total avg count}&\multicolumn{2}{c}{        42}&\multicolumn{2}{c}{       87}&\multicolumn{2}{c}{      216}&\multicolumn{2}{c}{      503}&\multicolumn{2}{c}{    1,094}&\multicolumn{2}{c}{    4,063}&\multicolumn{2}{c}{   10,961}&\multicolumn{2}{c}{   18,727}&\multicolumn{2}{c}{   27,644}\\
\multicolumn{1}{c}{avg eval count}&\multicolumn{2}{c}{         42}&\multicolumn{2}{c}{       87}&\multicolumn{2}{c}{      216}&\multicolumn{2}{c}{      503}&\multicolumn{2}{c}{      910}&\multicolumn{2}{c}{    1,488}&\multicolumn{2}{c}{    1,676}&\multicolumn{2}{c}{    1,814}&\multicolumn{2}{c}{    1,909}\\
\multicolumn{1}{c}{max eval count}&\multicolumn{2}{c}{        303}&\multicolumn{2}{c}{      867}&\multicolumn{2}{c}{    2,843}&\multicolumn{2}{c}{    6,230}&\multicolumn{2}{c}{   10,487}&\multicolumn{2}{c}{  916,298}&\multicolumn{2}{c}{  501,346}&\multicolumn{2}{c}{  411,742}&\multicolumn{2}{c}{   12,386}\\
\multicolumn{1}{c}{min eval count}&\multicolumn{2}{c}{          0}&\multicolumn{2}{c}{        0}&\multicolumn{2}{c}{        0}&\multicolumn{2}{c}{        0}&\multicolumn{2}{c}{        0}&\multicolumn{2}{c}{        0}&\multicolumn{2}{c}{        0}&\multicolumn{2}{c}{        0}&\multicolumn{2}{c}{        0}\\
\hline
\multicolumn{1}{c}{fail ratio}&\multicolumn{2}{c}{          0.000}&\multicolumn{2}{c}{    0.000}&\multicolumn{2}{c}{    0.000}&\multicolumn{2}{c}{    0.000}&\multicolumn{2}{c}{    0.000}&\multicolumn{2}{c}{    0.003}&\multicolumn{2}{c}{    0.010}&\multicolumn{2}{c}{    0.018}&\multicolumn{2}{c}{    0.028}\\
\multicolumn{1}{c}{avg fail dif}&\multicolumn{2}{c}{            -}&\multicolumn{2}{c}{        -}&\multicolumn{2}{c}{        -}&\multicolumn{2}{c}{        -}&\multicolumn{2}{c}{      345}&\multicolumn{2}{c}{      345}&\multicolumn{2}{c}{      345}&\multicolumn{2}{c}{      345}&\multicolumn{2}{c}{      345}\\
\hline
\multicolumn{1}{c}{p-value}&&\multicolumn{2}{c}{0.0000}&\multicolumn{2}{c}{0.0000}&\multicolumn{2}{c}{0.0000}&\multicolumn{2}{c}{0.0000}&\multicolumn{2}{c}{0.0000}&\multicolumn{2}{c}{0.0000}&\multicolumn{2}{c}{0.0034}&\multicolumn{2}{c}{0.0068}\\
&&&&&&&&&&&&&&&&&&\end{tabular}
\end{tabular}
}


Here the same holds.
The higher mutation rates are less impacted than the higher rates for the RLS and the (1+1) EA.
\subsection{Comparison of the best variants}
\input{expRes/powerlaw/beforeBest}

\makebox[\linewidth]{
\begin{tabular}{lp{3cm}p{6cm}p{6cm}}
\begin{tabular}[h]{cccc}
algo type&        \RLSN& (1+1) EA&  pmut\\
algo param&         b=2&   3$/n$&  2.25\\
avg mut/change&   2.000& 3.092& 3.965\\
avg mut/step&     2.000& 2.999& 4.339\\
\hline
total avg count&    302&   677&   691\\
avg eval count&     302&   677&   691\\
max eval count&   1,610& 6,404& 5,205\\
min eval count&       9&    33&    17\\
\hline
fail ratio&       0.000& 0.000& 0.000\\
\end{tabular}
\end{tabular}
}


This input seems generally easy to solve as for every base algorithm are multiple variants which reach an optimal solution within 1000 steps.
The $pmut_-1.75$ variants reaches an optimal solution the fastest, but the other algorithms are almost equally fast.
All algorithms finish within $550 \pm 100$ steps on average and always in less than 2000 steps.

\input{tables/mixed/multipleN_fails.tex}

This input is only hard to solve for $n<100$.
For $n = 100$ there are only a few inputs that were not solved within the time limit and for $n\ge500$ the input is solved by each of the chosen algorithms.
This is probably caused by the many small values from the powerlaw and geometric distribution.


% \input{tables/mixed/multipleN_avg.tex}

\input{tables/mixed/multipleN_totalAvg.tex}
\section{OneMax Equivalent for PARTITION}
This distribution has mostly small values, but occasionally it also generates bigger values.
The higher (absolute lower) the parameter the higher the values get and also the amount of big values increases.
For a parameter of $\beta=-2.75$ the distribution looks like in Figure~\ref{fig:powerDistExample1}.

\begin{figure}[h]
      \caption{Distribution of a random powerlaw input with $\beta=-2.75$}
      \centering
      \includegraphics[width=0.7\textwidth]{figures/images/numberGenerator/powerlaw_-2_75.png}\label{fig:powerDistExample1}
\end{figure}

For a value of $\beta=-1.25$ the distribution looks a bit different.
There are less small values close to one and instead also big values even over 1000.
Figure~\ref{fig:powerDistExample2} is cropped to get a more clear view for the smaller values.
The higher values mostly occurred 0 to 2 times.
The highest value 8848 occurred only once.

\begin{figure}[h]
      \caption{Distribution of a random powerlaw input with $\beta=-1.25$}
      \centering
      \includegraphics[width=0.7\textwidth]{figures/images/numberGenerator/powerlaw_-1_25.png}\label{fig:powerDistExample2}
\end{figure}
\subsection{RLS Comparison}
The following table lists the results for the RLS for inputs that are chosen from a powerlaw distribution with $\beta=-2.75$.


\makebox[\linewidth]{
\begin{tabular}{lp{3cm}p{6cm}p{6cm}}
\begin{tabular}[h]{cccccccc}
algo type&            RLS&   \RLSR[s]&   \RLSR[s]&   \RLSR[s]&   \RLSN[b]&   \RLSN[b]&   \RLSN[b]\\
algo param&             -&     s=2&     s=3&     s=4&     b=3&     b=2&     b=4\\
avg mut/change&     1.000&   1.181&   1.688&   1.865&   3.000&   1.997&   3,997\\
avg mut/step&       1.000&   1.500&   2.000&   2.500&   3.000&   2.000&   3.000\\
\hline
total avg count&   90,931& 168,311& 236,317& 307,533& 921,030& 921,030& 921,030\\
avg eval count&    90,931& 168,311& 236,317& 307,533&       -&       -&       -\\
max eval count&   156,854& 296,206& 498,474& 595,831&       -&       -&       -\\
min eval count&    64,941& 120,582& 158,304& 212,193&       -&       -&       -\\
\hline
fail ratio&         0.000&   0.000&   0.000&   0.000&   1.000&   1.000&   1.000\\
avg fail dif&           -&       -&       -&       -&      36&      53&     263\\
\end{tabular}
\end{tabular}
}


The picture for the RLS variants on this type of input is not clear.
There in no obvious tendency for neither of the variants.
The only obvious thing is the RLS being the worst of the RLS variants again.
Every variant reaches the optimal solution in every case except for the RLS which only manages for roughly 50 \% of the inputs.
The RLS-N(2) seems to be the best variant for these kinds of inputs.
The next best variants are the RLS-(R) with $k=3$ and $k=4$ which only differ by 1 \%.

\subsection{(1+1) EA Comparison}
For the (1+1) EA the best static mutation rate seems to be $3/n$. 
The probability of flipping 2 or 4 bits as n goes to infinity for mutation rate $1/n$ approaches $13/24e\approx 0.199$, for $2/n$ approaches $8/3e^2\approx 0.361$, for $3/n$ approaches $63/8e^3\approx 0.392$, for $4/n$ approaches $56/3e^4\approx 0.342$ and for $5/n$ approaches $77/2e^5\approx 0.259$. So the highest probability has $c=3$, followed by $c=4$ and $c=2$ then $c=5$ and lastly $c=1$. For higher values of $c$ the probability decreases further as the expected number of flipped bits is $c$ for mutation rate $c/n$.

\makebox[\linewidth]{
\begin{tabular}{lp{3cm}p{6cm}p{6cm}}
\begin{tabular}[h]{ccccccccc}
algo type&           EA-SM&       EA-SM&    EA-SM&    EA-SM&    EA-SM&    EA-SM&    EA-SM&    EA-SM\\
algo param&            2$/n$&        -&      3$/n$&      4$/n$&      5$/n$&     10$/n$&     50$/n$&    100$/n$\\
avg mut/change&      2.246&    1.551&    3.048&    3.936&    4.861&    9.822&   49.750&   99.707\\
avg mut/step&        2.000&    1.000&    3.000&    4.000&    5.000&   10.000&   50.000&  100.001\\
\hline
avg eval count&      3,097&    3,505&    3,518&    4,009&    4,807&    7,758&   18,457&   25,993\\
max eval count&     39,490&   60,533&   39,048&   47,881&   56,204&   91,305&  173,851&  354,479\\
min eval count&         10&        0&        6&        5&        3&        5&        9&        3\\
\hline
fail ratio&          0.000&    0.000&    0.000&    0.000&    0.000&    0.000&    0.000&    0.000\\
\end{tabular}
\end{tabular}
}


The (1+1) EA seems to perform better with a lower mutation rate.
The vales $p_m=2/n$ and $p_m=3/n$ reach an optimal solution equally fast.
From then on speed of convergence decreases with increasing mutation rate.
The only exception from this case is the (1+1) EA which performs the worst despite having the lowest mutation rate.
For the uniform distributed input all variants of the (1+1) EA reach an optimal solution within the step limit as for the previous input types.
\subsection{pmut Comparison}
The results for the $pmut$ operator are pretty similar to the results for the (1+1) EA and the RLS.
The parameter $\beta=-3.25$ which flips the least bits on average finds the solution the fastest.
The other values for $\beta$ increase the time needed for finding one of the two optimums with increasing value for $\beta$ (decreasing in the absolute value).
All variants find an optimum in every run except for $\beta=-1.25$ which has a much higher value for the number of flipped bits per steps.
The average number for the number of bits flipped in a successful mutation is much lower than for the other inputs especially for the higher (absolute lower) values for $\beta$.
For the binomial and geometric input the successful average was around 100 for $\beta=-1.25$ but for the OneMax equivalent it was only at 5.

\makebox[\linewidth]{
\scriptsize
\begin{tabular}{lp{3cm}p{6cm}p{6cm}}
\begin{tabular}[h]{m{2.5cm}m{0,40cm}m{0,40cm}m{0,40cm}m{0,40cm}m{0,40cm}m{0,40cm}m{0,40cm}m{0,40cm}m{0,40cm}m{0,40cm}m{0,40cm}m{0,40cm}m{0,40cm}m{0,40cm}m{0,40cm}m{0,40cm}m{0,40cm}m{0,40cm}}
\multicolumn{1}{c}{algo type}&\multicolumn{2}{c}{            pmut}&\multicolumn{2}{c}{     pmut}&\multicolumn{2}{c}{     pmut}&\multicolumn{2}{c}{     pmut}&\multicolumn{2}{c}{     pmut}&\multicolumn{2}{c}{     pmut}&\multicolumn{2}{c}{     pmut}&\multicolumn{2}{c}{     pmut}&\multicolumn{2}{c}{     pmut}\\
\multicolumn{1}{c}{algo param}&\multicolumn{2}{c}{           1.25}&\multicolumn{2}{c}{     1.50}&\multicolumn{2}{c}{     1.75}&\multicolumn{2}{c}{     2.00}&\multicolumn{2}{c}{     2.25}&\multicolumn{2}{c}{     2.50}&\multicolumn{2}{c}{     2.75}&\multicolumn{2}{c}{     3.00}&\multicolumn{2}{c}{     3.25}\\
\multicolumn{1}{c}{avg mut/change}&\multicolumn{2}{c}{    197.409}&\multicolumn{2}{c}{   70.534}&\multicolumn{2}{c}{   23.050}&\multicolumn{2}{c}{    8.724}&\multicolumn{2}{c}{    4.351}&\multicolumn{2}{c}{    2.777}&\multicolumn{2}{c}{    2.111}&\multicolumn{2}{c}{    1.770}&\multicolumn{2}{c}{    1.563}\\
\multicolumn{1}{c}{avg mut/step}&\multicolumn{2}{c}{      224.442}&\multicolumn{2}{c}{   70.480}&\multicolumn{2}{c}{   22.299}&\multicolumn{2}{c}{    8.470}&\multicolumn{2}{c}{    4.368}&\multicolumn{2}{c}{    2.906}&\multicolumn{2}{c}{    2.271}&\multicolumn{2}{c}{    1.934}&\multicolumn{2}{c}{    1.729}\\
\hline
\multicolumn{1}{c}{total avg count}&\multicolumn{2}{c}{        42}&\multicolumn{2}{c}{       87}&\multicolumn{2}{c}{      216}&\multicolumn{2}{c}{      503}&\multicolumn{2}{c}{    1,094}&\multicolumn{2}{c}{    4,063}&\multicolumn{2}{c}{   10,961}&\multicolumn{2}{c}{   18,727}&\multicolumn{2}{c}{   27,644}\\
\multicolumn{1}{c}{avg eval count}&\multicolumn{2}{c}{         42}&\multicolumn{2}{c}{       87}&\multicolumn{2}{c}{      216}&\multicolumn{2}{c}{      503}&\multicolumn{2}{c}{      910}&\multicolumn{2}{c}{    1,488}&\multicolumn{2}{c}{    1,676}&\multicolumn{2}{c}{    1,814}&\multicolumn{2}{c}{    1,909}\\
\multicolumn{1}{c}{max eval count}&\multicolumn{2}{c}{        303}&\multicolumn{2}{c}{      867}&\multicolumn{2}{c}{    2,843}&\multicolumn{2}{c}{    6,230}&\multicolumn{2}{c}{   10,487}&\multicolumn{2}{c}{  916,298}&\multicolumn{2}{c}{  501,346}&\multicolumn{2}{c}{  411,742}&\multicolumn{2}{c}{   12,386}\\
\multicolumn{1}{c}{min eval count}&\multicolumn{2}{c}{          0}&\multicolumn{2}{c}{        0}&\multicolumn{2}{c}{        0}&\multicolumn{2}{c}{        0}&\multicolumn{2}{c}{        0}&\multicolumn{2}{c}{        0}&\multicolumn{2}{c}{        0}&\multicolumn{2}{c}{        0}&\multicolumn{2}{c}{        0}\\
\hline
\multicolumn{1}{c}{fail ratio}&\multicolumn{2}{c}{          0.000}&\multicolumn{2}{c}{    0.000}&\multicolumn{2}{c}{    0.000}&\multicolumn{2}{c}{    0.000}&\multicolumn{2}{c}{    0.000}&\multicolumn{2}{c}{    0.003}&\multicolumn{2}{c}{    0.010}&\multicolumn{2}{c}{    0.018}&\multicolumn{2}{c}{    0.028}\\
\multicolumn{1}{c}{avg fail dif}&\multicolumn{2}{c}{            -}&\multicolumn{2}{c}{        -}&\multicolumn{2}{c}{        -}&\multicolumn{2}{c}{        -}&\multicolumn{2}{c}{      345}&\multicolumn{2}{c}{      345}&\multicolumn{2}{c}{      345}&\multicolumn{2}{c}{      345}&\multicolumn{2}{c}{      345}\\
\hline
\multicolumn{1}{c}{p-value}&&\multicolumn{2}{c}{0.0000}&\multicolumn{2}{c}{0.0000}&\multicolumn{2}{c}{0.0000}&\multicolumn{2}{c}{0.0000}&\multicolumn{2}{c}{0.0000}&\multicolumn{2}{c}{0.0000}&\multicolumn{2}{c}{0.0034}&\multicolumn{2}{c}{0.0068}\\
&&&&&&&&&&&&&&&&&&\end{tabular}
\end{tabular}
}


Here the same holds.
The higher mutation rates are less impacted than the higher rates for the RLS and the (1+1) EA.
\subsection{Comparison of the best variants}
\input{expRes/onemax/beforeBest}

\makebox[\linewidth]{
\begin{tabular}{lp{3cm}p{6cm}p{6cm}}
\begin{tabular}[h]{cccc}
algo type&        \RLSN& (1+1) EA&  pmut\\
algo param&         b=2&   3$/n$&  2.25\\
avg mut/change&   2.000& 3.092& 3.965\\
avg mut/step&     2.000& 2.999& 4.339\\
\hline
total avg count&    302&   677&   691\\
avg eval count&     302&   677&   691\\
max eval count&   1,610& 6,404& 5,205\\
min eval count&       9&    33&    17\\
\hline
fail ratio&       0.000& 0.000& 0.000\\
\end{tabular}
\end{tabular}
}


This input seems generally easy to solve as for every base algorithm are multiple variants which reach an optimal solution within 1000 steps.
The $pmut_-1.75$ variants reaches an optimal solution the fastest, but the other algorithms are almost equally fast.
All algorithms finish within $550 \pm 100$ steps on average and always in less than 2000 steps.

\input{tables/mixed/multipleN_fails.tex}

This input is only hard to solve for $n<100$.
For $n = 100$ there are only a few inputs that were not solved within the time limit and for $n\ge500$ the input is solved by each of the chosen algorithms.
This is probably caused by the many small values from the powerlaw and geometric distribution.


% \input{tables/mixed/multipleN_avg.tex}

\input{tables/mixed/multipleN_totalAvg.tex}
\section{Carsten Witts worst case input}
This distribution has mostly small values, but occasionally it also generates bigger values.
The higher (absolute lower) the parameter the higher the values get and also the amount of big values increases.
For a parameter of $\beta=-2.75$ the distribution looks like in Figure~\ref{fig:powerDistExample1}.

\begin{figure}[h]
      \caption{Distribution of a random powerlaw input with $\beta=-2.75$}
      \centering
      \includegraphics[width=0.7\textwidth]{figures/images/numberGenerator/powerlaw_-2_75.png}\label{fig:powerDistExample1}
\end{figure}

For a value of $\beta=-1.25$ the distribution looks a bit different.
There are less small values close to one and instead also big values even over 1000.
Figure~\ref{fig:powerDistExample2} is cropped to get a more clear view for the smaller values.
The higher values mostly occurred 0 to 2 times.
The highest value 8848 occurred only once.

\begin{figure}[h]
      \caption{Distribution of a random powerlaw input with $\beta=-1.25$}
      \centering
      \includegraphics[width=0.7\textwidth]{figures/images/numberGenerator/powerlaw_-1_25.png}\label{fig:powerDistExample2}
\end{figure}
\subsection{RLS Comparison}
The following table lists the results for the RLS for inputs that are chosen from a powerlaw distribution with $\beta=-2.75$.


\makebox[\linewidth]{
\begin{tabular}{lp{3cm}p{6cm}p{6cm}}
\begin{tabular}[h]{cccccccc}
algo type&            RLS&   \RLSR[s]&   \RLSR[s]&   \RLSR[s]&   \RLSN[b]&   \RLSN[b]&   \RLSN[b]\\
algo param&             -&     s=2&     s=3&     s=4&     b=3&     b=2&     b=4\\
avg mut/change&     1.000&   1.181&   1.688&   1.865&   3.000&   1.997&   3,997\\
avg mut/step&       1.000&   1.500&   2.000&   2.500&   3.000&   2.000&   3.000\\
\hline
total avg count&   90,931& 168,311& 236,317& 307,533& 921,030& 921,030& 921,030\\
avg eval count&    90,931& 168,311& 236,317& 307,533&       -&       -&       -\\
max eval count&   156,854& 296,206& 498,474& 595,831&       -&       -&       -\\
min eval count&    64,941& 120,582& 158,304& 212,193&       -&       -&       -\\
\hline
fail ratio&         0.000&   0.000&   0.000&   0.000&   1.000&   1.000&   1.000\\
avg fail dif&           -&       -&       -&       -&      36&      53&     263\\
\end{tabular}
\end{tabular}
}


The picture for the RLS variants on this type of input is not clear.
There in no obvious tendency for neither of the variants.
The only obvious thing is the RLS being the worst of the RLS variants again.
Every variant reaches the optimal solution in every case except for the RLS which only manages for roughly 50 \% of the inputs.
The RLS-N(2) seems to be the best variant for these kinds of inputs.
The next best variants are the RLS-(R) with $k=3$ and $k=4$ which only differ by 1 \%.

\subsection{(1+1) EA Comparison}
For the (1+1) EA the best static mutation rate seems to be $3/n$. 
The probability of flipping 2 or 4 bits as n goes to infinity for mutation rate $1/n$ approaches $13/24e\approx 0.199$, for $2/n$ approaches $8/3e^2\approx 0.361$, for $3/n$ approaches $63/8e^3\approx 0.392$, for $4/n$ approaches $56/3e^4\approx 0.342$ and for $5/n$ approaches $77/2e^5\approx 0.259$. So the highest probability has $c=3$, followed by $c=4$ and $c=2$ then $c=5$ and lastly $c=1$. For higher values of $c$ the probability decreases further as the expected number of flipped bits is $c$ for mutation rate $c/n$.

\makebox[\linewidth]{
\begin{tabular}{lp{3cm}p{6cm}p{6cm}}
\begin{tabular}[h]{ccccccccc}
algo type&           EA-SM&       EA-SM&    EA-SM&    EA-SM&    EA-SM&    EA-SM&    EA-SM&    EA-SM\\
algo param&            2$/n$&        -&      3$/n$&      4$/n$&      5$/n$&     10$/n$&     50$/n$&    100$/n$\\
avg mut/change&      2.246&    1.551&    3.048&    3.936&    4.861&    9.822&   49.750&   99.707\\
avg mut/step&        2.000&    1.000&    3.000&    4.000&    5.000&   10.000&   50.000&  100.001\\
\hline
avg eval count&      3,097&    3,505&    3,518&    4,009&    4,807&    7,758&   18,457&   25,993\\
max eval count&     39,490&   60,533&   39,048&   47,881&   56,204&   91,305&  173,851&  354,479\\
min eval count&         10&        0&        6&        5&        3&        5&        9&        3\\
\hline
fail ratio&          0.000&    0.000&    0.000&    0.000&    0.000&    0.000&    0.000&    0.000\\
\end{tabular}
\end{tabular}
}


The (1+1) EA seems to perform better with a lower mutation rate.
The vales $p_m=2/n$ and $p_m=3/n$ reach an optimal solution equally fast.
From then on speed of convergence decreases with increasing mutation rate.
The only exception from this case is the (1+1) EA which performs the worst despite having the lowest mutation rate.
For the uniform distributed input all variants of the (1+1) EA reach an optimal solution within the step limit as for the previous input types.
\subsection{pmut Comparison}
The results for the $pmut$ operator are pretty similar to the results for the (1+1) EA and the RLS.
The parameter $\beta=-3.25$ which flips the least bits on average finds the solution the fastest.
The other values for $\beta$ increase the time needed for finding one of the two optimums with increasing value for $\beta$ (decreasing in the absolute value).
All variants find an optimum in every run except for $\beta=-1.25$ which has a much higher value for the number of flipped bits per steps.
The average number for the number of bits flipped in a successful mutation is much lower than for the other inputs especially for the higher (absolute lower) values for $\beta$.
For the binomial and geometric input the successful average was around 100 for $\beta=-1.25$ but for the OneMax equivalent it was only at 5.

\makebox[\linewidth]{
\scriptsize
\begin{tabular}{lp{3cm}p{6cm}p{6cm}}
\begin{tabular}[h]{m{2.5cm}m{0,40cm}m{0,40cm}m{0,40cm}m{0,40cm}m{0,40cm}m{0,40cm}m{0,40cm}m{0,40cm}m{0,40cm}m{0,40cm}m{0,40cm}m{0,40cm}m{0,40cm}m{0,40cm}m{0,40cm}m{0,40cm}m{0,40cm}m{0,40cm}}
\multicolumn{1}{c}{algo type}&\multicolumn{2}{c}{            pmut}&\multicolumn{2}{c}{     pmut}&\multicolumn{2}{c}{     pmut}&\multicolumn{2}{c}{     pmut}&\multicolumn{2}{c}{     pmut}&\multicolumn{2}{c}{     pmut}&\multicolumn{2}{c}{     pmut}&\multicolumn{2}{c}{     pmut}&\multicolumn{2}{c}{     pmut}\\
\multicolumn{1}{c}{algo param}&\multicolumn{2}{c}{           1.25}&\multicolumn{2}{c}{     1.50}&\multicolumn{2}{c}{     1.75}&\multicolumn{2}{c}{     2.00}&\multicolumn{2}{c}{     2.25}&\multicolumn{2}{c}{     2.50}&\multicolumn{2}{c}{     2.75}&\multicolumn{2}{c}{     3.00}&\multicolumn{2}{c}{     3.25}\\
\multicolumn{1}{c}{avg mut/change}&\multicolumn{2}{c}{    197.409}&\multicolumn{2}{c}{   70.534}&\multicolumn{2}{c}{   23.050}&\multicolumn{2}{c}{    8.724}&\multicolumn{2}{c}{    4.351}&\multicolumn{2}{c}{    2.777}&\multicolumn{2}{c}{    2.111}&\multicolumn{2}{c}{    1.770}&\multicolumn{2}{c}{    1.563}\\
\multicolumn{1}{c}{avg mut/step}&\multicolumn{2}{c}{      224.442}&\multicolumn{2}{c}{   70.480}&\multicolumn{2}{c}{   22.299}&\multicolumn{2}{c}{    8.470}&\multicolumn{2}{c}{    4.368}&\multicolumn{2}{c}{    2.906}&\multicolumn{2}{c}{    2.271}&\multicolumn{2}{c}{    1.934}&\multicolumn{2}{c}{    1.729}\\
\hline
\multicolumn{1}{c}{total avg count}&\multicolumn{2}{c}{        42}&\multicolumn{2}{c}{       87}&\multicolumn{2}{c}{      216}&\multicolumn{2}{c}{      503}&\multicolumn{2}{c}{    1,094}&\multicolumn{2}{c}{    4,063}&\multicolumn{2}{c}{   10,961}&\multicolumn{2}{c}{   18,727}&\multicolumn{2}{c}{   27,644}\\
\multicolumn{1}{c}{avg eval count}&\multicolumn{2}{c}{         42}&\multicolumn{2}{c}{       87}&\multicolumn{2}{c}{      216}&\multicolumn{2}{c}{      503}&\multicolumn{2}{c}{      910}&\multicolumn{2}{c}{    1,488}&\multicolumn{2}{c}{    1,676}&\multicolumn{2}{c}{    1,814}&\multicolumn{2}{c}{    1,909}\\
\multicolumn{1}{c}{max eval count}&\multicolumn{2}{c}{        303}&\multicolumn{2}{c}{      867}&\multicolumn{2}{c}{    2,843}&\multicolumn{2}{c}{    6,230}&\multicolumn{2}{c}{   10,487}&\multicolumn{2}{c}{  916,298}&\multicolumn{2}{c}{  501,346}&\multicolumn{2}{c}{  411,742}&\multicolumn{2}{c}{   12,386}\\
\multicolumn{1}{c}{min eval count}&\multicolumn{2}{c}{          0}&\multicolumn{2}{c}{        0}&\multicolumn{2}{c}{        0}&\multicolumn{2}{c}{        0}&\multicolumn{2}{c}{        0}&\multicolumn{2}{c}{        0}&\multicolumn{2}{c}{        0}&\multicolumn{2}{c}{        0}&\multicolumn{2}{c}{        0}\\
\hline
\multicolumn{1}{c}{fail ratio}&\multicolumn{2}{c}{          0.000}&\multicolumn{2}{c}{    0.000}&\multicolumn{2}{c}{    0.000}&\multicolumn{2}{c}{    0.000}&\multicolumn{2}{c}{    0.000}&\multicolumn{2}{c}{    0.003}&\multicolumn{2}{c}{    0.010}&\multicolumn{2}{c}{    0.018}&\multicolumn{2}{c}{    0.028}\\
\multicolumn{1}{c}{avg fail dif}&\multicolumn{2}{c}{            -}&\multicolumn{2}{c}{        -}&\multicolumn{2}{c}{        -}&\multicolumn{2}{c}{        -}&\multicolumn{2}{c}{      345}&\multicolumn{2}{c}{      345}&\multicolumn{2}{c}{      345}&\multicolumn{2}{c}{      345}&\multicolumn{2}{c}{      345}\\
\hline
\multicolumn{1}{c}{p-value}&&\multicolumn{2}{c}{0.0000}&\multicolumn{2}{c}{0.0000}&\multicolumn{2}{c}{0.0000}&\multicolumn{2}{c}{0.0000}&\multicolumn{2}{c}{0.0000}&\multicolumn{2}{c}{0.0000}&\multicolumn{2}{c}{0.0034}&\multicolumn{2}{c}{0.0068}\\
&&&&&&&&&&&&&&&&&&\end{tabular}
\end{tabular}
}


Here the same holds.
The higher mutation rates are less impacted than the higher rates for the RLS and the (1+1) EA.
\subsection{Comparison of the best variants}
\input{expRes/twoThirds/beforeBest}

\makebox[\linewidth]{
\begin{tabular}{lp{3cm}p{6cm}p{6cm}}
\begin{tabular}[h]{cccc}
algo type&        \RLSN& (1+1) EA&  pmut\\
algo param&         b=2&   3$/n$&  2.25\\
avg mut/change&   2.000& 3.092& 3.965\\
avg mut/step&     2.000& 2.999& 4.339\\
\hline
total avg count&    302&   677&   691\\
avg eval count&     302&   677&   691\\
max eval count&   1,610& 6,404& 5,205\\
min eval count&       9&    33&    17\\
\hline
fail ratio&       0.000& 0.000& 0.000\\
\end{tabular}
\end{tabular}
}


This input seems generally easy to solve as for every base algorithm are multiple variants which reach an optimal solution within 1000 steps.
The $pmut_-1.75$ variants reaches an optimal solution the fastest, but the other algorithms are almost equally fast.
All algorithms finish within $550 \pm 100$ steps on average and always in less than 2000 steps.

\input{tables/mixed/multipleN_fails.tex}

This input is only hard to solve for $n<100$.
For $n = 100$ there are only a few inputs that were not solved within the time limit and for $n\ge500$ the input is solved by each of the chosen algorithms.
This is probably caused by the many small values from the powerlaw and geometric distribution.


% \input{tables/mixed/multipleN_avg.tex}

\input{tables/mixed/multipleN_totalAvg.tex}
\section{Multiple distributions mixed}
This distribution has mostly small values, but occasionally it also generates bigger values.
The higher (absolute lower) the parameter the higher the values get and also the amount of big values increases.
For a parameter of $\beta=-2.75$ the distribution looks like in Figure~\ref{fig:powerDistExample1}.

\begin{figure}[h]
      \caption{Distribution of a random powerlaw input with $\beta=-2.75$}
      \centering
      \includegraphics[width=0.7\textwidth]{figures/images/numberGenerator/powerlaw_-2_75.png}\label{fig:powerDistExample1}
\end{figure}

For a value of $\beta=-1.25$ the distribution looks a bit different.
There are less small values close to one and instead also big values even over 1000.
Figure~\ref{fig:powerDistExample2} is cropped to get a more clear view for the smaller values.
The higher values mostly occurred 0 to 2 times.
The highest value 8848 occurred only once.

\begin{figure}[h]
      \caption{Distribution of a random powerlaw input with $\beta=-1.25$}
      \centering
      \includegraphics[width=0.7\textwidth]{figures/images/numberGenerator/powerlaw_-1_25.png}\label{fig:powerDistExample2}
\end{figure}
\subsection{RLS Comparison}
The following table lists the results for the RLS for inputs that are chosen from a powerlaw distribution with $\beta=-2.75$.


\makebox[\linewidth]{
\begin{tabular}{lp{3cm}p{6cm}p{6cm}}
\begin{tabular}[h]{cccccccc}
algo type&            RLS&   \RLSR[s]&   \RLSR[s]&   \RLSR[s]&   \RLSN[b]&   \RLSN[b]&   \RLSN[b]\\
algo param&             -&     s=2&     s=3&     s=4&     b=3&     b=2&     b=4\\
avg mut/change&     1.000&   1.181&   1.688&   1.865&   3.000&   1.997&   3,997\\
avg mut/step&       1.000&   1.500&   2.000&   2.500&   3.000&   2.000&   3.000\\
\hline
total avg count&   90,931& 168,311& 236,317& 307,533& 921,030& 921,030& 921,030\\
avg eval count&    90,931& 168,311& 236,317& 307,533&       -&       -&       -\\
max eval count&   156,854& 296,206& 498,474& 595,831&       -&       -&       -\\
min eval count&    64,941& 120,582& 158,304& 212,193&       -&       -&       -\\
\hline
fail ratio&         0.000&   0.000&   0.000&   0.000&   1.000&   1.000&   1.000\\
avg fail dif&           -&       -&       -&       -&      36&      53&     263\\
\end{tabular}
\end{tabular}
}


The picture for the RLS variants on this type of input is not clear.
There in no obvious tendency for neither of the variants.
The only obvious thing is the RLS being the worst of the RLS variants again.
Every variant reaches the optimal solution in every case except for the RLS which only manages for roughly 50 \% of the inputs.
The RLS-N(2) seems to be the best variant for these kinds of inputs.
The next best variants are the RLS-(R) with $k=3$ and $k=4$ which only differ by 1 \%.

\subsection{(1+1) EA Comparison}
For the (1+1) EA the best static mutation rate seems to be $3/n$. 
The probability of flipping 2 or 4 bits as n goes to infinity for mutation rate $1/n$ approaches $13/24e\approx 0.199$, for $2/n$ approaches $8/3e^2\approx 0.361$, for $3/n$ approaches $63/8e^3\approx 0.392$, for $4/n$ approaches $56/3e^4\approx 0.342$ and for $5/n$ approaches $77/2e^5\approx 0.259$. So the highest probability has $c=3$, followed by $c=4$ and $c=2$ then $c=5$ and lastly $c=1$. For higher values of $c$ the probability decreases further as the expected number of flipped bits is $c$ for mutation rate $c/n$.

\makebox[\linewidth]{
\begin{tabular}{lp{3cm}p{6cm}p{6cm}}
\begin{tabular}[h]{ccccccccc}
algo type&           EA-SM&       EA-SM&    EA-SM&    EA-SM&    EA-SM&    EA-SM&    EA-SM&    EA-SM\\
algo param&            2$/n$&        -&      3$/n$&      4$/n$&      5$/n$&     10$/n$&     50$/n$&    100$/n$\\
avg mut/change&      2.246&    1.551&    3.048&    3.936&    4.861&    9.822&   49.750&   99.707\\
avg mut/step&        2.000&    1.000&    3.000&    4.000&    5.000&   10.000&   50.000&  100.001\\
\hline
avg eval count&      3,097&    3,505&    3,518&    4,009&    4,807&    7,758&   18,457&   25,993\\
max eval count&     39,490&   60,533&   39,048&   47,881&   56,204&   91,305&  173,851&  354,479\\
min eval count&         10&        0&        6&        5&        3&        5&        9&        3\\
\hline
fail ratio&          0.000&    0.000&    0.000&    0.000&    0.000&    0.000&    0.000&    0.000\\
\end{tabular}
\end{tabular}
}


The (1+1) EA seems to perform better with a lower mutation rate.
The vales $p_m=2/n$ and $p_m=3/n$ reach an optimal solution equally fast.
From then on speed of convergence decreases with increasing mutation rate.
The only exception from this case is the (1+1) EA which performs the worst despite having the lowest mutation rate.
For the uniform distributed input all variants of the (1+1) EA reach an optimal solution within the step limit as for the previous input types.
\subsection{pmut Comparison}
The results for the $pmut$ operator are pretty similar to the results for the (1+1) EA and the RLS.
The parameter $\beta=-3.25$ which flips the least bits on average finds the solution the fastest.
The other values for $\beta$ increase the time needed for finding one of the two optimums with increasing value for $\beta$ (decreasing in the absolute value).
All variants find an optimum in every run except for $\beta=-1.25$ which has a much higher value for the number of flipped bits per steps.
The average number for the number of bits flipped in a successful mutation is much lower than for the other inputs especially for the higher (absolute lower) values for $\beta$.
For the binomial and geometric input the successful average was around 100 for $\beta=-1.25$ but for the OneMax equivalent it was only at 5.

\makebox[\linewidth]{
\scriptsize
\begin{tabular}{lp{3cm}p{6cm}p{6cm}}
\begin{tabular}[h]{m{2.5cm}m{0,40cm}m{0,40cm}m{0,40cm}m{0,40cm}m{0,40cm}m{0,40cm}m{0,40cm}m{0,40cm}m{0,40cm}m{0,40cm}m{0,40cm}m{0,40cm}m{0,40cm}m{0,40cm}m{0,40cm}m{0,40cm}m{0,40cm}m{0,40cm}}
\multicolumn{1}{c}{algo type}&\multicolumn{2}{c}{            pmut}&\multicolumn{2}{c}{     pmut}&\multicolumn{2}{c}{     pmut}&\multicolumn{2}{c}{     pmut}&\multicolumn{2}{c}{     pmut}&\multicolumn{2}{c}{     pmut}&\multicolumn{2}{c}{     pmut}&\multicolumn{2}{c}{     pmut}&\multicolumn{2}{c}{     pmut}\\
\multicolumn{1}{c}{algo param}&\multicolumn{2}{c}{           1.25}&\multicolumn{2}{c}{     1.50}&\multicolumn{2}{c}{     1.75}&\multicolumn{2}{c}{     2.00}&\multicolumn{2}{c}{     2.25}&\multicolumn{2}{c}{     2.50}&\multicolumn{2}{c}{     2.75}&\multicolumn{2}{c}{     3.00}&\multicolumn{2}{c}{     3.25}\\
\multicolumn{1}{c}{avg mut/change}&\multicolumn{2}{c}{    197.409}&\multicolumn{2}{c}{   70.534}&\multicolumn{2}{c}{   23.050}&\multicolumn{2}{c}{    8.724}&\multicolumn{2}{c}{    4.351}&\multicolumn{2}{c}{    2.777}&\multicolumn{2}{c}{    2.111}&\multicolumn{2}{c}{    1.770}&\multicolumn{2}{c}{    1.563}\\
\multicolumn{1}{c}{avg mut/step}&\multicolumn{2}{c}{      224.442}&\multicolumn{2}{c}{   70.480}&\multicolumn{2}{c}{   22.299}&\multicolumn{2}{c}{    8.470}&\multicolumn{2}{c}{    4.368}&\multicolumn{2}{c}{    2.906}&\multicolumn{2}{c}{    2.271}&\multicolumn{2}{c}{    1.934}&\multicolumn{2}{c}{    1.729}\\
\hline
\multicolumn{1}{c}{total avg count}&\multicolumn{2}{c}{        42}&\multicolumn{2}{c}{       87}&\multicolumn{2}{c}{      216}&\multicolumn{2}{c}{      503}&\multicolumn{2}{c}{    1,094}&\multicolumn{2}{c}{    4,063}&\multicolumn{2}{c}{   10,961}&\multicolumn{2}{c}{   18,727}&\multicolumn{2}{c}{   27,644}\\
\multicolumn{1}{c}{avg eval count}&\multicolumn{2}{c}{         42}&\multicolumn{2}{c}{       87}&\multicolumn{2}{c}{      216}&\multicolumn{2}{c}{      503}&\multicolumn{2}{c}{      910}&\multicolumn{2}{c}{    1,488}&\multicolumn{2}{c}{    1,676}&\multicolumn{2}{c}{    1,814}&\multicolumn{2}{c}{    1,909}\\
\multicolumn{1}{c}{max eval count}&\multicolumn{2}{c}{        303}&\multicolumn{2}{c}{      867}&\multicolumn{2}{c}{    2,843}&\multicolumn{2}{c}{    6,230}&\multicolumn{2}{c}{   10,487}&\multicolumn{2}{c}{  916,298}&\multicolumn{2}{c}{  501,346}&\multicolumn{2}{c}{  411,742}&\multicolumn{2}{c}{   12,386}\\
\multicolumn{1}{c}{min eval count}&\multicolumn{2}{c}{          0}&\multicolumn{2}{c}{        0}&\multicolumn{2}{c}{        0}&\multicolumn{2}{c}{        0}&\multicolumn{2}{c}{        0}&\multicolumn{2}{c}{        0}&\multicolumn{2}{c}{        0}&\multicolumn{2}{c}{        0}&\multicolumn{2}{c}{        0}\\
\hline
\multicolumn{1}{c}{fail ratio}&\multicolumn{2}{c}{          0.000}&\multicolumn{2}{c}{    0.000}&\multicolumn{2}{c}{    0.000}&\multicolumn{2}{c}{    0.000}&\multicolumn{2}{c}{    0.000}&\multicolumn{2}{c}{    0.003}&\multicolumn{2}{c}{    0.010}&\multicolumn{2}{c}{    0.018}&\multicolumn{2}{c}{    0.028}\\
\multicolumn{1}{c}{avg fail dif}&\multicolumn{2}{c}{            -}&\multicolumn{2}{c}{        -}&\multicolumn{2}{c}{        -}&\multicolumn{2}{c}{        -}&\multicolumn{2}{c}{      345}&\multicolumn{2}{c}{      345}&\multicolumn{2}{c}{      345}&\multicolumn{2}{c}{      345}&\multicolumn{2}{c}{      345}\\
\hline
\multicolumn{1}{c}{p-value}&&\multicolumn{2}{c}{0.0000}&\multicolumn{2}{c}{0.0000}&\multicolumn{2}{c}{0.0000}&\multicolumn{2}{c}{0.0000}&\multicolumn{2}{c}{0.0000}&\multicolumn{2}{c}{0.0000}&\multicolumn{2}{c}{0.0034}&\multicolumn{2}{c}{0.0068}\\
&&&&&&&&&&&&&&&&&&\end{tabular}
\end{tabular}
}


Here the same holds.
The higher mutation rates are less impacted than the higher rates for the RLS and the (1+1) EA.
\subsection{Comparison of the best variants}
\input{expRes/mixed/beforeBest}

\makebox[\linewidth]{
\begin{tabular}{lp{3cm}p{6cm}p{6cm}}
\begin{tabular}[h]{cccc}
algo type&        \RLSN& (1+1) EA&  pmut\\
algo param&         b=2&   3$/n$&  2.25\\
avg mut/change&   2.000& 3.092& 3.965\\
avg mut/step&     2.000& 2.999& 4.339\\
\hline
total avg count&    302&   677&   691\\
avg eval count&     302&   677&   691\\
max eval count&   1,610& 6,404& 5,205\\
min eval count&       9&    33&    17\\
\hline
fail ratio&       0.000& 0.000& 0.000\\
\end{tabular}
\end{tabular}
}


This input seems generally easy to solve as for every base algorithm are multiple variants which reach an optimal solution within 1000 steps.
The $pmut_-1.75$ variants reaches an optimal solution the fastest, but the other algorithms are almost equally fast.
All algorithms finish within $550 \pm 100$ steps on average and always in less than 2000 steps.

\input{tables/mixed/multipleN_fails.tex}

This input is only hard to solve for $n<100$.
For $n = 100$ there are only a few inputs that were not solved within the time limit and for $n\ge500$ the input is solved by each of the chosen algorithms.
This is probably caused by the many small values from the powerlaw and geometric distribution.


% \input{tables/mixed/multipleN_avg.tex}

\input{tables/mixed/multipleN_totalAvg.tex}
\section{Multiple distributions overlapped}
This distribution has mostly small values, but occasionally it also generates bigger values.
The higher (absolute lower) the parameter the higher the values get and also the amount of big values increases.
For a parameter of $\beta=-2.75$ the distribution looks like in Figure~\ref{fig:powerDistExample1}.

\begin{figure}[h]
      \caption{Distribution of a random powerlaw input with $\beta=-2.75$}
      \centering
      \includegraphics[width=0.7\textwidth]{figures/images/numberGenerator/powerlaw_-2_75.png}\label{fig:powerDistExample1}
\end{figure}

For a value of $\beta=-1.25$ the distribution looks a bit different.
There are less small values close to one and instead also big values even over 1000.
Figure~\ref{fig:powerDistExample2} is cropped to get a more clear view for the smaller values.
The higher values mostly occurred 0 to 2 times.
The highest value 8848 occurred only once.

\begin{figure}[h]
      \caption{Distribution of a random powerlaw input with $\beta=-1.25$}
      \centering
      \includegraphics[width=0.7\textwidth]{figures/images/numberGenerator/powerlaw_-1_25.png}\label{fig:powerDistExample2}
\end{figure}
\subsection{RLS Comparison}
The following table lists the results for the RLS for inputs that are chosen from a powerlaw distribution with $\beta=-2.75$.


\makebox[\linewidth]{
\begin{tabular}{lp{3cm}p{6cm}p{6cm}}
\begin{tabular}[h]{cccccccc}
algo type&            RLS&   \RLSR[s]&   \RLSR[s]&   \RLSR[s]&   \RLSN[b]&   \RLSN[b]&   \RLSN[b]\\
algo param&             -&     s=2&     s=3&     s=4&     b=3&     b=2&     b=4\\
avg mut/change&     1.000&   1.181&   1.688&   1.865&   3.000&   1.997&   3,997\\
avg mut/step&       1.000&   1.500&   2.000&   2.500&   3.000&   2.000&   3.000\\
\hline
total avg count&   90,931& 168,311& 236,317& 307,533& 921,030& 921,030& 921,030\\
avg eval count&    90,931& 168,311& 236,317& 307,533&       -&       -&       -\\
max eval count&   156,854& 296,206& 498,474& 595,831&       -&       -&       -\\
min eval count&    64,941& 120,582& 158,304& 212,193&       -&       -&       -\\
\hline
fail ratio&         0.000&   0.000&   0.000&   0.000&   1.000&   1.000&   1.000\\
avg fail dif&           -&       -&       -&       -&      36&      53&     263\\
\end{tabular}
\end{tabular}
}


The picture for the RLS variants on this type of input is not clear.
There in no obvious tendency for neither of the variants.
The only obvious thing is the RLS being the worst of the RLS variants again.
Every variant reaches the optimal solution in every case except for the RLS which only manages for roughly 50 \% of the inputs.
The RLS-N(2) seems to be the best variant for these kinds of inputs.
The next best variants are the RLS-(R) with $k=3$ and $k=4$ which only differ by 1 \%.

\subsection{(1+1) EA Comparison}
For the (1+1) EA the best static mutation rate seems to be $3/n$. 
The probability of flipping 2 or 4 bits as n goes to infinity for mutation rate $1/n$ approaches $13/24e\approx 0.199$, for $2/n$ approaches $8/3e^2\approx 0.361$, for $3/n$ approaches $63/8e^3\approx 0.392$, for $4/n$ approaches $56/3e^4\approx 0.342$ and for $5/n$ approaches $77/2e^5\approx 0.259$. So the highest probability has $c=3$, followed by $c=4$ and $c=2$ then $c=5$ and lastly $c=1$. For higher values of $c$ the probability decreases further as the expected number of flipped bits is $c$ for mutation rate $c/n$.

\makebox[\linewidth]{
\begin{tabular}{lp{3cm}p{6cm}p{6cm}}
\begin{tabular}[h]{ccccccccc}
algo type&           EA-SM&       EA-SM&    EA-SM&    EA-SM&    EA-SM&    EA-SM&    EA-SM&    EA-SM\\
algo param&            2$/n$&        -&      3$/n$&      4$/n$&      5$/n$&     10$/n$&     50$/n$&    100$/n$\\
avg mut/change&      2.246&    1.551&    3.048&    3.936&    4.861&    9.822&   49.750&   99.707\\
avg mut/step&        2.000&    1.000&    3.000&    4.000&    5.000&   10.000&   50.000&  100.001\\
\hline
avg eval count&      3,097&    3,505&    3,518&    4,009&    4,807&    7,758&   18,457&   25,993\\
max eval count&     39,490&   60,533&   39,048&   47,881&   56,204&   91,305&  173,851&  354,479\\
min eval count&         10&        0&        6&        5&        3&        5&        9&        3\\
\hline
fail ratio&          0.000&    0.000&    0.000&    0.000&    0.000&    0.000&    0.000&    0.000\\
\end{tabular}
\end{tabular}
}


The (1+1) EA seems to perform better with a lower mutation rate.
The vales $p_m=2/n$ and $p_m=3/n$ reach an optimal solution equally fast.
From then on speed of convergence decreases with increasing mutation rate.
The only exception from this case is the (1+1) EA which performs the worst despite having the lowest mutation rate.
For the uniform distributed input all variants of the (1+1) EA reach an optimal solution within the step limit as for the previous input types.
\subsection{pmut Comparison}
The results for the $pmut$ operator are pretty similar to the results for the (1+1) EA and the RLS.
The parameter $\beta=-3.25$ which flips the least bits on average finds the solution the fastest.
The other values for $\beta$ increase the time needed for finding one of the two optimums with increasing value for $\beta$ (decreasing in the absolute value).
All variants find an optimum in every run except for $\beta=-1.25$ which has a much higher value for the number of flipped bits per steps.
The average number for the number of bits flipped in a successful mutation is much lower than for the other inputs especially for the higher (absolute lower) values for $\beta$.
For the binomial and geometric input the successful average was around 100 for $\beta=-1.25$ but for the OneMax equivalent it was only at 5.

\makebox[\linewidth]{
\scriptsize
\begin{tabular}{lp{3cm}p{6cm}p{6cm}}
\begin{tabular}[h]{m{2.5cm}m{0,40cm}m{0,40cm}m{0,40cm}m{0,40cm}m{0,40cm}m{0,40cm}m{0,40cm}m{0,40cm}m{0,40cm}m{0,40cm}m{0,40cm}m{0,40cm}m{0,40cm}m{0,40cm}m{0,40cm}m{0,40cm}m{0,40cm}m{0,40cm}}
\multicolumn{1}{c}{algo type}&\multicolumn{2}{c}{            pmut}&\multicolumn{2}{c}{     pmut}&\multicolumn{2}{c}{     pmut}&\multicolumn{2}{c}{     pmut}&\multicolumn{2}{c}{     pmut}&\multicolumn{2}{c}{     pmut}&\multicolumn{2}{c}{     pmut}&\multicolumn{2}{c}{     pmut}&\multicolumn{2}{c}{     pmut}\\
\multicolumn{1}{c}{algo param}&\multicolumn{2}{c}{           1.25}&\multicolumn{2}{c}{     1.50}&\multicolumn{2}{c}{     1.75}&\multicolumn{2}{c}{     2.00}&\multicolumn{2}{c}{     2.25}&\multicolumn{2}{c}{     2.50}&\multicolumn{2}{c}{     2.75}&\multicolumn{2}{c}{     3.00}&\multicolumn{2}{c}{     3.25}\\
\multicolumn{1}{c}{avg mut/change}&\multicolumn{2}{c}{    197.409}&\multicolumn{2}{c}{   70.534}&\multicolumn{2}{c}{   23.050}&\multicolumn{2}{c}{    8.724}&\multicolumn{2}{c}{    4.351}&\multicolumn{2}{c}{    2.777}&\multicolumn{2}{c}{    2.111}&\multicolumn{2}{c}{    1.770}&\multicolumn{2}{c}{    1.563}\\
\multicolumn{1}{c}{avg mut/step}&\multicolumn{2}{c}{      224.442}&\multicolumn{2}{c}{   70.480}&\multicolumn{2}{c}{   22.299}&\multicolumn{2}{c}{    8.470}&\multicolumn{2}{c}{    4.368}&\multicolumn{2}{c}{    2.906}&\multicolumn{2}{c}{    2.271}&\multicolumn{2}{c}{    1.934}&\multicolumn{2}{c}{    1.729}\\
\hline
\multicolumn{1}{c}{total avg count}&\multicolumn{2}{c}{        42}&\multicolumn{2}{c}{       87}&\multicolumn{2}{c}{      216}&\multicolumn{2}{c}{      503}&\multicolumn{2}{c}{    1,094}&\multicolumn{2}{c}{    4,063}&\multicolumn{2}{c}{   10,961}&\multicolumn{2}{c}{   18,727}&\multicolumn{2}{c}{   27,644}\\
\multicolumn{1}{c}{avg eval count}&\multicolumn{2}{c}{         42}&\multicolumn{2}{c}{       87}&\multicolumn{2}{c}{      216}&\multicolumn{2}{c}{      503}&\multicolumn{2}{c}{      910}&\multicolumn{2}{c}{    1,488}&\multicolumn{2}{c}{    1,676}&\multicolumn{2}{c}{    1,814}&\multicolumn{2}{c}{    1,909}\\
\multicolumn{1}{c}{max eval count}&\multicolumn{2}{c}{        303}&\multicolumn{2}{c}{      867}&\multicolumn{2}{c}{    2,843}&\multicolumn{2}{c}{    6,230}&\multicolumn{2}{c}{   10,487}&\multicolumn{2}{c}{  916,298}&\multicolumn{2}{c}{  501,346}&\multicolumn{2}{c}{  411,742}&\multicolumn{2}{c}{   12,386}\\
\multicolumn{1}{c}{min eval count}&\multicolumn{2}{c}{          0}&\multicolumn{2}{c}{        0}&\multicolumn{2}{c}{        0}&\multicolumn{2}{c}{        0}&\multicolumn{2}{c}{        0}&\multicolumn{2}{c}{        0}&\multicolumn{2}{c}{        0}&\multicolumn{2}{c}{        0}&\multicolumn{2}{c}{        0}\\
\hline
\multicolumn{1}{c}{fail ratio}&\multicolumn{2}{c}{          0.000}&\multicolumn{2}{c}{    0.000}&\multicolumn{2}{c}{    0.000}&\multicolumn{2}{c}{    0.000}&\multicolumn{2}{c}{    0.000}&\multicolumn{2}{c}{    0.003}&\multicolumn{2}{c}{    0.010}&\multicolumn{2}{c}{    0.018}&\multicolumn{2}{c}{    0.028}\\
\multicolumn{1}{c}{avg fail dif}&\multicolumn{2}{c}{            -}&\multicolumn{2}{c}{        -}&\multicolumn{2}{c}{        -}&\multicolumn{2}{c}{        -}&\multicolumn{2}{c}{      345}&\multicolumn{2}{c}{      345}&\multicolumn{2}{c}{      345}&\multicolumn{2}{c}{      345}&\multicolumn{2}{c}{      345}\\
\hline
\multicolumn{1}{c}{p-value}&&\multicolumn{2}{c}{0.0000}&\multicolumn{2}{c}{0.0000}&\multicolumn{2}{c}{0.0000}&\multicolumn{2}{c}{0.0000}&\multicolumn{2}{c}{0.0000}&\multicolumn{2}{c}{0.0000}&\multicolumn{2}{c}{0.0034}&\multicolumn{2}{c}{0.0068}\\
&&&&&&&&&&&&&&&&&&\end{tabular}
\end{tabular}
}


Here the same holds.
The higher mutation rates are less impacted than the higher rates for the RLS and the (1+1) EA.
\subsection{Comparison of the best variants}
\input{expRes/overlapped/beforeBest}

\makebox[\linewidth]{
\begin{tabular}{lp{3cm}p{6cm}p{6cm}}
\begin{tabular}[h]{cccc}
algo type&        \RLSN& (1+1) EA&  pmut\\
algo param&         b=2&   3$/n$&  2.25\\
avg mut/change&   2.000& 3.092& 3.965\\
avg mut/step&     2.000& 2.999& 4.339\\
\hline
total avg count&    302&   677&   691\\
avg eval count&     302&   677&   691\\
max eval count&   1,610& 6,404& 5,205\\
min eval count&       9&    33&    17\\
\hline
fail ratio&       0.000& 0.000& 0.000\\
\end{tabular}
\end{tabular}
}


This input seems generally easy to solve as for every base algorithm are multiple variants which reach an optimal solution within 1000 steps.
The $pmut_-1.75$ variants reaches an optimal solution the fastest, but the other algorithms are almost equally fast.
All algorithms finish within $550 \pm 100$ steps on average and always in less than 2000 steps.

\input{tables/mixed/multipleN_fails.tex}

This input is only hard to solve for $n<100$.
For $n = 100$ there are only a few inputs that were not solved within the time limit and for $n\ge500$ the input is solved by each of the chosen algorithms.
This is probably caused by the many small values from the powerlaw and geometric distribution.


% \input{tables/mixed/multipleN_avg.tex}

\input{tables/mixed/multipleN_totalAvg.tex}
\section{Multiple distributions mixed \& overlapped}
This distribution has mostly small values, but occasionally it also generates bigger values.
The higher (absolute lower) the parameter the higher the values get and also the amount of big values increases.
For a parameter of $\beta=-2.75$ the distribution looks like in Figure~\ref{fig:powerDistExample1}.

\begin{figure}[h]
      \caption{Distribution of a random powerlaw input with $\beta=-2.75$}
      \centering
      \includegraphics[width=0.7\textwidth]{figures/images/numberGenerator/powerlaw_-2_75.png}\label{fig:powerDistExample1}
\end{figure}

For a value of $\beta=-1.25$ the distribution looks a bit different.
There are less small values close to one and instead also big values even over 1000.
Figure~\ref{fig:powerDistExample2} is cropped to get a more clear view for the smaller values.
The higher values mostly occurred 0 to 2 times.
The highest value 8848 occurred only once.

\begin{figure}[h]
      \caption{Distribution of a random powerlaw input with $\beta=-1.25$}
      \centering
      \includegraphics[width=0.7\textwidth]{figures/images/numberGenerator/powerlaw_-1_25.png}\label{fig:powerDistExample2}
\end{figure}
\subsection{RLS Comparison}
The following table lists the results for the RLS for inputs that are chosen from a powerlaw distribution with $\beta=-2.75$.


\makebox[\linewidth]{
\begin{tabular}{lp{3cm}p{6cm}p{6cm}}
\begin{tabular}[h]{cccccccc}
algo type&            RLS&   \RLSR[s]&   \RLSR[s]&   \RLSR[s]&   \RLSN[b]&   \RLSN[b]&   \RLSN[b]\\
algo param&             -&     s=2&     s=3&     s=4&     b=3&     b=2&     b=4\\
avg mut/change&     1.000&   1.181&   1.688&   1.865&   3.000&   1.997&   3,997\\
avg mut/step&       1.000&   1.500&   2.000&   2.500&   3.000&   2.000&   3.000\\
\hline
total avg count&   90,931& 168,311& 236,317& 307,533& 921,030& 921,030& 921,030\\
avg eval count&    90,931& 168,311& 236,317& 307,533&       -&       -&       -\\
max eval count&   156,854& 296,206& 498,474& 595,831&       -&       -&       -\\
min eval count&    64,941& 120,582& 158,304& 212,193&       -&       -&       -\\
\hline
fail ratio&         0.000&   0.000&   0.000&   0.000&   1.000&   1.000&   1.000\\
avg fail dif&           -&       -&       -&       -&      36&      53&     263\\
\end{tabular}
\end{tabular}
}


The picture for the RLS variants on this type of input is not clear.
There in no obvious tendency for neither of the variants.
The only obvious thing is the RLS being the worst of the RLS variants again.
Every variant reaches the optimal solution in every case except for the RLS which only manages for roughly 50 \% of the inputs.
The RLS-N(2) seems to be the best variant for these kinds of inputs.
The next best variants are the RLS-(R) with $k=3$ and $k=4$ which only differ by 1 \%.

\subsection{(1+1) EA Comparison}
For the (1+1) EA the best static mutation rate seems to be $3/n$. 
The probability of flipping 2 or 4 bits as n goes to infinity for mutation rate $1/n$ approaches $13/24e\approx 0.199$, for $2/n$ approaches $8/3e^2\approx 0.361$, for $3/n$ approaches $63/8e^3\approx 0.392$, for $4/n$ approaches $56/3e^4\approx 0.342$ and for $5/n$ approaches $77/2e^5\approx 0.259$. So the highest probability has $c=3$, followed by $c=4$ and $c=2$ then $c=5$ and lastly $c=1$. For higher values of $c$ the probability decreases further as the expected number of flipped bits is $c$ for mutation rate $c/n$.

\makebox[\linewidth]{
\begin{tabular}{lp{3cm}p{6cm}p{6cm}}
\begin{tabular}[h]{ccccccccc}
algo type&           EA-SM&       EA-SM&    EA-SM&    EA-SM&    EA-SM&    EA-SM&    EA-SM&    EA-SM\\
algo param&            2$/n$&        -&      3$/n$&      4$/n$&      5$/n$&     10$/n$&     50$/n$&    100$/n$\\
avg mut/change&      2.246&    1.551&    3.048&    3.936&    4.861&    9.822&   49.750&   99.707\\
avg mut/step&        2.000&    1.000&    3.000&    4.000&    5.000&   10.000&   50.000&  100.001\\
\hline
avg eval count&      3,097&    3,505&    3,518&    4,009&    4,807&    7,758&   18,457&   25,993\\
max eval count&     39,490&   60,533&   39,048&   47,881&   56,204&   91,305&  173,851&  354,479\\
min eval count&         10&        0&        6&        5&        3&        5&        9&        3\\
\hline
fail ratio&          0.000&    0.000&    0.000&    0.000&    0.000&    0.000&    0.000&    0.000\\
\end{tabular}
\end{tabular}
}


The (1+1) EA seems to perform better with a lower mutation rate.
The vales $p_m=2/n$ and $p_m=3/n$ reach an optimal solution equally fast.
From then on speed of convergence decreases with increasing mutation rate.
The only exception from this case is the (1+1) EA which performs the worst despite having the lowest mutation rate.
For the uniform distributed input all variants of the (1+1) EA reach an optimal solution within the step limit as for the previous input types.
\subsection{pmut Comparison}
The results for the $pmut$ operator are pretty similar to the results for the (1+1) EA and the RLS.
The parameter $\beta=-3.25$ which flips the least bits on average finds the solution the fastest.
The other values for $\beta$ increase the time needed for finding one of the two optimums with increasing value for $\beta$ (decreasing in the absolute value).
All variants find an optimum in every run except for $\beta=-1.25$ which has a much higher value for the number of flipped bits per steps.
The average number for the number of bits flipped in a successful mutation is much lower than for the other inputs especially for the higher (absolute lower) values for $\beta$.
For the binomial and geometric input the successful average was around 100 for $\beta=-1.25$ but for the OneMax equivalent it was only at 5.

\makebox[\linewidth]{
\scriptsize
\begin{tabular}{lp{3cm}p{6cm}p{6cm}}
\begin{tabular}[h]{m{2.5cm}m{0,40cm}m{0,40cm}m{0,40cm}m{0,40cm}m{0,40cm}m{0,40cm}m{0,40cm}m{0,40cm}m{0,40cm}m{0,40cm}m{0,40cm}m{0,40cm}m{0,40cm}m{0,40cm}m{0,40cm}m{0,40cm}m{0,40cm}m{0,40cm}}
\multicolumn{1}{c}{algo type}&\multicolumn{2}{c}{            pmut}&\multicolumn{2}{c}{     pmut}&\multicolumn{2}{c}{     pmut}&\multicolumn{2}{c}{     pmut}&\multicolumn{2}{c}{     pmut}&\multicolumn{2}{c}{     pmut}&\multicolumn{2}{c}{     pmut}&\multicolumn{2}{c}{     pmut}&\multicolumn{2}{c}{     pmut}\\
\multicolumn{1}{c}{algo param}&\multicolumn{2}{c}{           1.25}&\multicolumn{2}{c}{     1.50}&\multicolumn{2}{c}{     1.75}&\multicolumn{2}{c}{     2.00}&\multicolumn{2}{c}{     2.25}&\multicolumn{2}{c}{     2.50}&\multicolumn{2}{c}{     2.75}&\multicolumn{2}{c}{     3.00}&\multicolumn{2}{c}{     3.25}\\
\multicolumn{1}{c}{avg mut/change}&\multicolumn{2}{c}{    197.409}&\multicolumn{2}{c}{   70.534}&\multicolumn{2}{c}{   23.050}&\multicolumn{2}{c}{    8.724}&\multicolumn{2}{c}{    4.351}&\multicolumn{2}{c}{    2.777}&\multicolumn{2}{c}{    2.111}&\multicolumn{2}{c}{    1.770}&\multicolumn{2}{c}{    1.563}\\
\multicolumn{1}{c}{avg mut/step}&\multicolumn{2}{c}{      224.442}&\multicolumn{2}{c}{   70.480}&\multicolumn{2}{c}{   22.299}&\multicolumn{2}{c}{    8.470}&\multicolumn{2}{c}{    4.368}&\multicolumn{2}{c}{    2.906}&\multicolumn{2}{c}{    2.271}&\multicolumn{2}{c}{    1.934}&\multicolumn{2}{c}{    1.729}\\
\hline
\multicolumn{1}{c}{total avg count}&\multicolumn{2}{c}{        42}&\multicolumn{2}{c}{       87}&\multicolumn{2}{c}{      216}&\multicolumn{2}{c}{      503}&\multicolumn{2}{c}{    1,094}&\multicolumn{2}{c}{    4,063}&\multicolumn{2}{c}{   10,961}&\multicolumn{2}{c}{   18,727}&\multicolumn{2}{c}{   27,644}\\
\multicolumn{1}{c}{avg eval count}&\multicolumn{2}{c}{         42}&\multicolumn{2}{c}{       87}&\multicolumn{2}{c}{      216}&\multicolumn{2}{c}{      503}&\multicolumn{2}{c}{      910}&\multicolumn{2}{c}{    1,488}&\multicolumn{2}{c}{    1,676}&\multicolumn{2}{c}{    1,814}&\multicolumn{2}{c}{    1,909}\\
\multicolumn{1}{c}{max eval count}&\multicolumn{2}{c}{        303}&\multicolumn{2}{c}{      867}&\multicolumn{2}{c}{    2,843}&\multicolumn{2}{c}{    6,230}&\multicolumn{2}{c}{   10,487}&\multicolumn{2}{c}{  916,298}&\multicolumn{2}{c}{  501,346}&\multicolumn{2}{c}{  411,742}&\multicolumn{2}{c}{   12,386}\\
\multicolumn{1}{c}{min eval count}&\multicolumn{2}{c}{          0}&\multicolumn{2}{c}{        0}&\multicolumn{2}{c}{        0}&\multicolumn{2}{c}{        0}&\multicolumn{2}{c}{        0}&\multicolumn{2}{c}{        0}&\multicolumn{2}{c}{        0}&\multicolumn{2}{c}{        0}&\multicolumn{2}{c}{        0}\\
\hline
\multicolumn{1}{c}{fail ratio}&\multicolumn{2}{c}{          0.000}&\multicolumn{2}{c}{    0.000}&\multicolumn{2}{c}{    0.000}&\multicolumn{2}{c}{    0.000}&\multicolumn{2}{c}{    0.000}&\multicolumn{2}{c}{    0.003}&\multicolumn{2}{c}{    0.010}&\multicolumn{2}{c}{    0.018}&\multicolumn{2}{c}{    0.028}\\
\multicolumn{1}{c}{avg fail dif}&\multicolumn{2}{c}{            -}&\multicolumn{2}{c}{        -}&\multicolumn{2}{c}{        -}&\multicolumn{2}{c}{        -}&\multicolumn{2}{c}{      345}&\multicolumn{2}{c}{      345}&\multicolumn{2}{c}{      345}&\multicolumn{2}{c}{      345}&\multicolumn{2}{c}{      345}\\
\hline
\multicolumn{1}{c}{p-value}&&\multicolumn{2}{c}{0.0000}&\multicolumn{2}{c}{0.0000}&\multicolumn{2}{c}{0.0000}&\multicolumn{2}{c}{0.0000}&\multicolumn{2}{c}{0.0000}&\multicolumn{2}{c}{0.0000}&\multicolumn{2}{c}{0.0034}&\multicolumn{2}{c}{0.0068}\\
&&&&&&&&&&&&&&&&&&\end{tabular}
\end{tabular}
}


Here the same holds.
The higher mutation rates are less impacted than the higher rates for the RLS and the (1+1) EA.
\subsection{Comparison of the best variants}
\input{expRes/mixedAndOverlapped/beforeBest}

\makebox[\linewidth]{
\begin{tabular}{lp{3cm}p{6cm}p{6cm}}
\begin{tabular}[h]{cccc}
algo type&        \RLSN& (1+1) EA&  pmut\\
algo param&         b=2&   3$/n$&  2.25\\
avg mut/change&   2.000& 3.092& 3.965\\
avg mut/step&     2.000& 2.999& 4.339\\
\hline
total avg count&    302&   677&   691\\
avg eval count&     302&   677&   691\\
max eval count&   1,610& 6,404& 5,205\\
min eval count&       9&    33&    17\\
\hline
fail ratio&       0.000& 0.000& 0.000\\
\end{tabular}
\end{tabular}
}


This input seems generally easy to solve as for every base algorithm are multiple variants which reach an optimal solution within 1000 steps.
The $pmut_-1.75$ variants reaches an optimal solution the fastest, but the other algorithms are almost equally fast.
All algorithms finish within $550 \pm 100$ steps on average and always in less than 2000 steps.

\input{tables/mixed/multipleN_fails.tex}

This input is only hard to solve for $n<100$.
For $n = 100$ there are only a few inputs that were not solved within the time limit and for $n\ge500$ the input is solved by each of the chosen algorithms.
This is probably caused by the many small values from the powerlaw and geometric distribution.


% \input{tables/mixed/multipleN_avg.tex}

\input{tables/mixed/multipleN_totalAvg.tex}


\section{Binomial distributed values}
In the following subsections the performance of the different algorithms is tested for different kinds of inputs.
The exact distributions of the input are explained separately in each subsection.
The procedure for each comparison is always the same. A random input is generated according to the distribution and then solved by every algorithm.
All algorithms had the same two stopping conditions.
The first was reaching a perfect partition and the second was taking more than $10 \cdot n\ln(n)$ steps or $100 \cdot n\ln(n)$ in some cases.
For the lower values of $n$ the step limit of 100,000 was used instead.
For $n=20$ giving the algorithm only 600 steps is rather small.
In some cases the smaller inputs are even more difficult to solve.
Most modern computer should be able to handle 100,000 iterations in a short amount of time anyway.
So the minimum step limit of 100,000 seemed reasonable.
If either of these conditions was met, the algorithm returned its current best solution.
This step is repeated 1000 times.
The results are presented in a table containing multiple statistics for each algorithm over all 1000 runs.
The data is explained in the table below.

\begin{tabular}{c|l}
      column name     & meaning                                                         \\ \hline
      algo type       & type of algorithm (RLS, \RLSN, \RLSR, (1+1)EA or pmut)          \\
      algo param      & parameter of the algorithm or '-' if it is the standard variant \\
      avg mut/change  & average \#bits flipped for iterations leading to an improvement \\
      avg mut/step    & average \#bits flipped for any iteration                        \\ \hline
      total avg count & average \#iterations for all runs                               \\
      avg eval count  & average \#iterations of runs returning an optimal solution      \\
      max eval count  & maximum \#iterations of runs returning an optimal solution      \\
      min eval count  & minimum \#iterations of runs returning an optimal solution      \\ \hline
      fails           & number of runs that did not find an optimal solution            \\
      fail ratio      & ratio of unsuccessful runs to all runs                          \\
      avg fail dif    & average value of $b_F-f(opt)$ for non-optimal solutions         \\
\end{tabular}

Firstly the different variants of the RLS are compared with values of $k \in\{2,3,4\}$, then the performance of the (1+1) EA with static mutation rate $c/n$ with $c \in\{1,2,3,5,10,50,100\}$ and lastly the performance of the $pmut_\beta$ mutation operator with the parameter $\beta \in \{1.25, 1.5, \dots, 2.75,3.0,3.25\}$.
Additionally the best variants of each algorithm are compared in another 1000 runs.
Afterwards there is also a comparison for multiple input sizes of the best algorithms because the best algorithm is often dependent on the size of the input.
Normally there are three tables for each input.
The first states how often the algorithms did not find an optimal solution for the different input sizes ('fails' in top left cell).
The second gives their average performance for the successful runs ('avg' in top left cell) and the last the performance for all runs ('total avg' in top left cell).
The last two tables differ in the unsuccessful runs. Often the algorithm is stuck in a local optima it won't leave in reasonable time or never for variants of the RLS.
In these cases the step limit is the deciding factor on how big the penality for this run is.
So neither of the two average values alone is enough to give a complete insight on the performance.
Sometimes a variant of the RLS is much faster than the other algorithms for a specific input but is also the only algorithm to get stuck in a local optimum.
This creates the possibility to start the RLS variant with a low step limit and switch to the (1+1) EA if the RLS variant does not return an optimal solution.
Giving both tables for the different average values might help with this decision.

The first analysed inputs are inputs following a binomial distribution \textasciitilde$B(m,p)$ as those inputs have been researched in the previous subsection.
The results showed that the expected value of a single number is the main driver for the amount of perfect partitions the input has.
The results also suggested the inputs tend to have more perfect partitions if the expected value is lower.
The more perfect partitions an input has relative to the number of all possible partitions, the more likely the different RSHs are to find one of those.
Therefore researching inputs with higher expected values seems more interesting but generating higher values takes more time with a random number generator that needs $\mathcal{O}(mp)$ time.
To keep the time for generating one set of numbers reasonable the values chosen for all tests are $m=10000, p=0.1, n=10000$ with the expected value for a single number being $mp=1000$.
Figure~\ref{fig:binDistExample} shows a random binomial distributed input of length $n=10000$.
For this input type almost every time all elements were sharply concentrated around the expected value with all values being at $1000\pm200$ (for figure~\ref{fig:binDistExample} even closer at $1000\pm121$).
So after reaching a difference between the two bins of below $(1000-200)/2=400$ the algorithm can no longer achieve an improvement by flipping a single bit.

\begin{figure}[h]
      \caption{Distribution of a random binomial input}
      \centering
      \includegraphics[width=0.7\textwidth]{figures/images/numberGenerator/binomialDistributionForN10000p0_1.png}\label{fig:binDistExample}
\end{figure}
\subsection{RLS Comparison}


\makebox[\linewidth]{
\begin{tabular}{lp{3cm}p{6cm}p{6cm}}
\begin{tabular}[h]{cccccccc}
algo type&            RLS&   \RLSR[s]&   \RLSR[s]&   \RLSR[s]&   \RLSN[b]&   \RLSN[b]&   \RLSN[b]\\
algo param&             -&     s=2&     s=3&     s=4&     b=3&     b=2&     b=4\\
avg mut/change&     1.000&   1.181&   1.688&   1.865&   3.000&   1.997&   3,997\\
avg mut/step&       1.000&   1.500&   2.000&   2.500&   3.000&   2.000&   3.000\\
\hline
total avg count&   90,931& 168,311& 236,317& 307,533& 921,030& 921,030& 921,030\\
avg eval count&    90,931& 168,311& 236,317& 307,533&       -&       -&       -\\
max eval count&   156,854& 296,206& 498,474& 595,831&       -&       -&       -\\
min eval count&    64,941& 120,582& 158,304& 212,193&       -&       -&       -\\
\hline
fail ratio&         0.000&   0.000&   0.000&   0.000&   1.000&   1.000&   1.000\\
avg fail dif&           -&       -&       -&       -&      36&      53&     263\\
\end{tabular}
\end{tabular}
}


The \RLSN[2] seems to perform the best as it mostly switches two elements which works great for binomial distributed inputs.
The same algorithm with $k=4$ performs a bit worse but still good as switching 4 elements can be beneficial as well.
The variant of \RLSN[3] on the other hand does not reach the optimal solution in 23.4\% of the inputs with an average difference of 1.
It also needs 1000 times more iterations to find an optimum on average compared to the best algorithm \RLSN[2].
The \RLSR~variants behave mostly the same with $k=2$ being the best, followed by $k=4$ and $k=3$.
In this case the variant of $k=3$ is by far not as bad as for the \RLSN[3] because the probability of flipping 2 bits is 1/3 as compared to $\mathcal{O}(n^{-1})$ for the \RLSN[3].
The \RLSR~seem all to be a good option for binomial inputs with values of $k\in\{2,3,4\}$.
The standard RLS on the other hand performs by far the worst as it only moves one element per step.
It only managed to reach the optimal solution once for 1000 different inputs.
The number of iterations for this input was only 50 so the RLS likely had a good initialisation with a few lucky steps leading directly to the optimum.
For all other cases the average difference between the bins was 254 which is close to the median of the values from 0 to $(1000-100)/2=450$.
This is likely due to the RLS being unable to improve the solution once the current solution has a difference below half of the lowest value (Corollary~\ref{cor:RLSStuck}).
\subsection{(1+1) EA Comparison}
For the (1+1) EA the best static mutation rate seems to be $3/n$. 
The probability of flipping 2 or 4 bits as $n$ goes to infinity for mutation rate $1/n$ approaches $13/24e\approx 0.199$, for $2/n$ approaches $8/3e^2\approx 0.361$, for $3/n$ approaches $63/8e^3\approx 0.392$, for $4/n$ approaches $56/3e^4\approx 0.342$ and for $5/n$ approaches $77/2e^5\approx 0.259$.
So the highest probability has $c=3$, followed by $c=4$ and $c=2$ then $c=5$ and lastly $c=1$.
For higher values of $c$ the probability decreases further as the expected number of flipped bits is $c$ for mutation rate $c/n$.

\makebox[\linewidth]{
\begin{tabular}{lp{3cm}p{6cm}p{6cm}}
\begin{tabular}[h]{ccccccccc}
algo type&           EA-SM&       EA-SM&    EA-SM&    EA-SM&    EA-SM&    EA-SM&    EA-SM&    EA-SM\\
algo param&            2$/n$&        -&      3$/n$&      4$/n$&      5$/n$&     10$/n$&     50$/n$&    100$/n$\\
avg mut/change&      2.246&    1.551&    3.048&    3.936&    4.861&    9.822&   49.750&   99.707\\
avg mut/step&        2.000&    1.000&    3.000&    4.000&    5.000&   10.000&   50.000&  100.001\\
\hline
avg eval count&      3,097&    3,505&    3,518&    4,009&    4,807&    7,758&   18,457&   25,993\\
max eval count&     39,490&   60,533&   39,048&   47,881&   56,204&   91,305&  173,851&  354,479\\
min eval count&         10&        0&        6&        5&        3&        5&        9&        3\\
\hline
fail ratio&          0.000&    0.000&    0.000&    0.000&    0.000&    0.000&    0.000&    0.000\\
\end{tabular}
\end{tabular}
}


The static mutation rate $3/n$ seems to perform the best with both $4/n$ and $2/n$ being a close second place.
The next best values are $5/n$ and the standard $1/n$ both having a clear difference between each other and the better parameters.
From then on the number of iterations rises monotonically with rising mutation rate.
The higher mutation rates perform significantly worse but the still find a solution within the limit as opposed to the standard RLS.
\subsection{pmut Comparison}


\makebox[\linewidth]{
\scriptsize
\begin{tabular}{lp{3cm}p{6cm}p{6cm}}
\begin{tabular}[h]{m{2.5cm}m{0,40cm}m{0,40cm}m{0,40cm}m{0,40cm}m{0,40cm}m{0,40cm}m{0,40cm}m{0,40cm}m{0,40cm}m{0,40cm}m{0,40cm}m{0,40cm}m{0,40cm}m{0,40cm}m{0,40cm}m{0,40cm}m{0,40cm}m{0,40cm}}
\multicolumn{1}{c}{algo type}&\multicolumn{2}{c}{            pmut}&\multicolumn{2}{c}{     pmut}&\multicolumn{2}{c}{     pmut}&\multicolumn{2}{c}{     pmut}&\multicolumn{2}{c}{     pmut}&\multicolumn{2}{c}{     pmut}&\multicolumn{2}{c}{     pmut}&\multicolumn{2}{c}{     pmut}&\multicolumn{2}{c}{     pmut}\\
\multicolumn{1}{c}{algo param}&\multicolumn{2}{c}{           1.25}&\multicolumn{2}{c}{     1.50}&\multicolumn{2}{c}{     1.75}&\multicolumn{2}{c}{     2.00}&\multicolumn{2}{c}{     2.25}&\multicolumn{2}{c}{     2.50}&\multicolumn{2}{c}{     2.75}&\multicolumn{2}{c}{     3.00}&\multicolumn{2}{c}{     3.25}\\
\multicolumn{1}{c}{avg mut/change}&\multicolumn{2}{c}{    197.409}&\multicolumn{2}{c}{   70.534}&\multicolumn{2}{c}{   23.050}&\multicolumn{2}{c}{    8.724}&\multicolumn{2}{c}{    4.351}&\multicolumn{2}{c}{    2.777}&\multicolumn{2}{c}{    2.111}&\multicolumn{2}{c}{    1.770}&\multicolumn{2}{c}{    1.563}\\
\multicolumn{1}{c}{avg mut/step}&\multicolumn{2}{c}{      224.442}&\multicolumn{2}{c}{   70.480}&\multicolumn{2}{c}{   22.299}&\multicolumn{2}{c}{    8.470}&\multicolumn{2}{c}{    4.368}&\multicolumn{2}{c}{    2.906}&\multicolumn{2}{c}{    2.271}&\multicolumn{2}{c}{    1.934}&\multicolumn{2}{c}{    1.729}\\
\hline
\multicolumn{1}{c}{total avg count}&\multicolumn{2}{c}{        42}&\multicolumn{2}{c}{       87}&\multicolumn{2}{c}{      216}&\multicolumn{2}{c}{      503}&\multicolumn{2}{c}{    1,094}&\multicolumn{2}{c}{    4,063}&\multicolumn{2}{c}{   10,961}&\multicolumn{2}{c}{   18,727}&\multicolumn{2}{c}{   27,644}\\
\multicolumn{1}{c}{avg eval count}&\multicolumn{2}{c}{         42}&\multicolumn{2}{c}{       87}&\multicolumn{2}{c}{      216}&\multicolumn{2}{c}{      503}&\multicolumn{2}{c}{      910}&\multicolumn{2}{c}{    1,488}&\multicolumn{2}{c}{    1,676}&\multicolumn{2}{c}{    1,814}&\multicolumn{2}{c}{    1,909}\\
\multicolumn{1}{c}{max eval count}&\multicolumn{2}{c}{        303}&\multicolumn{2}{c}{      867}&\multicolumn{2}{c}{    2,843}&\multicolumn{2}{c}{    6,230}&\multicolumn{2}{c}{   10,487}&\multicolumn{2}{c}{  916,298}&\multicolumn{2}{c}{  501,346}&\multicolumn{2}{c}{  411,742}&\multicolumn{2}{c}{   12,386}\\
\multicolumn{1}{c}{min eval count}&\multicolumn{2}{c}{          0}&\multicolumn{2}{c}{        0}&\multicolumn{2}{c}{        0}&\multicolumn{2}{c}{        0}&\multicolumn{2}{c}{        0}&\multicolumn{2}{c}{        0}&\multicolumn{2}{c}{        0}&\multicolumn{2}{c}{        0}&\multicolumn{2}{c}{        0}\\
\hline
\multicolumn{1}{c}{fail ratio}&\multicolumn{2}{c}{          0.000}&\multicolumn{2}{c}{    0.000}&\multicolumn{2}{c}{    0.000}&\multicolumn{2}{c}{    0.000}&\multicolumn{2}{c}{    0.000}&\multicolumn{2}{c}{    0.003}&\multicolumn{2}{c}{    0.010}&\multicolumn{2}{c}{    0.018}&\multicolumn{2}{c}{    0.028}\\
\multicolumn{1}{c}{avg fail dif}&\multicolumn{2}{c}{            -}&\multicolumn{2}{c}{        -}&\multicolumn{2}{c}{        -}&\multicolumn{2}{c}{        -}&\multicolumn{2}{c}{      345}&\multicolumn{2}{c}{      345}&\multicolumn{2}{c}{      345}&\multicolumn{2}{c}{      345}&\multicolumn{2}{c}{      345}\\
\hline
\multicolumn{1}{c}{p-value}&&\multicolumn{2}{c}{0.0000}&\multicolumn{2}{c}{0.0000}&\multicolumn{2}{c}{0.0000}&\multicolumn{2}{c}{0.0000}&\multicolumn{2}{c}{0.0000}&\multicolumn{2}{c}{0.0000}&\multicolumn{2}{c}{0.0034}&\multicolumn{2}{c}{0.0068}\\
&&&&&&&&&&&&&&&&&&\end{tabular}
\end{tabular}
}


For the $pmut_\beta$ mutation operator the choice of $\beta$ seems to be much more insignificant than for the RLS or (1+1) EA. Here all values perform comparably good with only the value of $\beta = 1.25$ having a clear performance difference compared to next best value. All values of $\beta$ reach an optimal solution in every case. The worst variant of the $pmut_\beta$ operator still performs much better than the worst value for the (1+1) EA and even better than the worst RLS variant. There is no clear winner but because $\beta=2.25$ had the best performance in this experiment, it was used for the comparison of the best variants.
\subsection{Comparison of the best variants}


\makebox[\linewidth]{
\begin{tabular}{lp{3cm}p{6cm}p{6cm}}
\begin{tabular}[h]{cccc}
algo type&        \RLSN& (1+1) EA&  pmut\\
algo param&         b=2&   3$/n$&  2.25\\
avg mut/change&   2.000& 3.092& 3.965\\
avg mut/step&     2.000& 2.999& 4.339\\
\hline
total avg count&    302&   677&   691\\
avg eval count&     302&   677&   691\\
max eval count&   1,610& 6,404& 5,205\\
min eval count&       9&    33&    17\\
\hline
fail ratio&       0.000& 0.000& 0.000\\
\end{tabular}
\end{tabular}
}


\TODO{ask what turning the tables into arrays means}
For this setting of $m=10000, p=0.1, n=10000$ the \RLSN[2] performs better than the  (1+1) EA and $pmut_\beta$ mutation for all values of $c/n$ and $\beta$ by a factor of at least 2.
This is likely from the fact that this version of the RLS flips almost only two bits which seems to be close to optimal for this kind of input.
There are many values close to the expected value which can be switched to make small adjustments to the fitness value.
The (1+1) EA with $p_m=3/n$ and $pmut_\beta$ algorithm with $\beta=2.25$ perform almost the same.
% The (1+1) EA has a slightly lower average value but also has a higher minimum value and a higher maximum value.
To further investigate which input performs best on all binomial distributed inputs now a comparison with different input lengths follows.
The parameters of the distribution were not changed.\newline
The following table shows the number of runs in which the algorithms did not find an optimal solution within 50,000 steps.
The time limit was set 50,000 because the algorithms normally reach the optimal solution within a few thousand steps.
If the solutions is not found after 50,000 steps, the algorithm is most likely stuck in a local optimum which could only be left by flipping more bits than possible for the algorithm.
%\TODO{Change multipleN fails to show percentages instead of count}
%o multipleN tables}

\begin{tabular}[h]{cccccccc}
fails&20&50&100&500&1000&5000&10000\\\hline
RLS-N (2)&998&984&973&763&541&49&1\\
RLS-N (4)&998&989&975&837&719&150&27\\
(1+1) EA (3/n)&994&990&981&862&714&194&35\\
(1+1) EA (4/n)&998&993&976&852&725&180&51\\
pmut (-2.0)&997&989&983&905&784&229&55\\
pmut (-2.25)&997&988&987&875&790&213&56\\
\end{tabular}


The (1+1) EA and $pmut_\beta$ always reach an optimal solution but the RLS does not.
The RLS variants that can only flip two bits per step perform significantly worse for small inputs.
They are probably more likely to get stuck in a local optimum where a step flipping 4 bits or more would be necessary.
So the \RLSN[2] does perform better for larger inputs but is much more likely to get stuck in a local optima.
The next table contains the average number of iterations the algorithm needed to find an optimal solution for all runs where the algorithms managed to find an optimal solution.
Here it still looks like the \RLSN[2] finds the solution with the lowest amount of steps, because the cases where the algorithm is stuck in a local optima are not contained in this table.

\begin{tabular}[h]{ccccccccc}
avg&20&50&100&500&1000&5000&10000&50000\\\hline
RLS-N (2)&172&1247&7598&38282&36214&105152&117792&125415\\
RLS-N (4)&8626&40966&43921&41924&42289&138187&201183&214170\\
(1+1) EA (3/n)&36470&42555&42439&43801&45530&145209&212281&252713\\
(1+1) EA (4/n)&37212&43144&44340&47246&46689&143697&220208&265620\\
pmut (-2.0)&40368&42601&39603&47443&46591&148390&232398&311606\\
pmut (-2.25)&39476&42113&40325&40548&43691&157816&226129&292367\\
\end{tabular}


The next table contains the overall average amount of iterations for every run.
So runs where no optimal result was found add 50,000 to the sum of all iterations.

\begin{tabular}[h]{ccccccccc}
total avg&20&50&100&500&1000&5000&10000&50000\\\hline
RLS&98050&94574&89765&60172&39985&15639&4128&1797\\
\RLSR[2]&89401&54556&18266&4218&3530&2362&2160&2229\\
(1+1) EA (1$/n$)&54971&26688&15684&8452&6567&3815&3458&3371\\
(1+1) EA (2$/n$)&26104&9724&6508&4503&4020&3171&3141&3133\\
pmut (3.25)&40934&19972&10977&5644&4406&2434&2162&2172\\
pmut (3.0)&35272&17545&10040&5222&4150&2510&2208&2213\\
\end{tabular}



Here the \RLSN[2] is only the best algorithm for values of $n \ge 500$.
Below this bound choosing the (1+1) EA with static mutation rate $3/n$ is a safer choice as the (1+1) EA reaches an optimal solution for every input (in this experiment).

\section{Geometric distributed values}
For the geometric distribution the chosen default value is $p=0.001$.
This results in an expected value of 1000 which is the same as for the binomial distribution in the last subsection.
This should make the results more comparable.
The maximum value is theoretically not limited but for the implementation in Java the maximum value was set to the maximum value of a long value $= 2^{63}-1 = 9,223,372,036,854,775,807$.
Without this maximum the value might overflow and instead be negative with high absolute value.
Figure~\ref{fig:geoDistExample} shows a random geometric distributed input.
The span of all values is way higher than for the binomial distribution, although they have same expected value.
Here the values are not in the interval $[800,1200]$ but rather between 0 and 9000.
The theoretical limitation of the values being at most $2^{63}-1$ seems to not have an influence on the results.
The geometric distribution does not only have low values close or equal to 1 but also has mostly values that are very small.
This should lead to 1-bit flips being effective as the small values can remove the small differences.
Because there are so many small values moving only one bit might be better than switching two elements.
\begin{figure}[h]
      \caption{Distribution of a random geometric input}
      \centering
      \includegraphics[width=0.7\textwidth]{figures/images/numberGenerator/geometricDistributionForp0_001.png}\label{fig:geoDistExample}
\end{figure}
\subsection{RLS Comparison}
\makebox[\linewidth]{
\begin{tabular}{lp{3cm}p{6cm}p{6cm}}
\begin{tabular}[h]{cccccccc}
algo type&            RLS&   \RLSR[s]&   \RLSR[s]&   \RLSR[s]&   \RLSN[b]&   \RLSN[b]&   \RLSN[b]\\
algo param&             -&     s=2&     s=3&     s=4&     b=3&     b=2&     b=4\\
avg mut/change&     1.000&   1.181&   1.688&   1.865&   3.000&   1.997&   3,997\\
avg mut/step&       1.000&   1.500&   2.000&   2.500&   3.000&   2.000&   3.000\\
\hline
total avg count&   90,931& 168,311& 236,317& 307,533& 921,030& 921,030& 921,030\\
avg eval count&    90,931& 168,311& 236,317& 307,533&       -&       -&       -\\
max eval count&   156,854& 296,206& 498,474& 595,831&       -&       -&       -\\
min eval count&    64,941& 120,582& 158,304& 212,193&       -&       -&       -\\
\hline
fail ratio&         0.000&   0.000&   0.000&   0.000&   1.000&   1.000&   1.000\\
avg fail dif&           -&       -&       -&       -&      36&      53&     263\\
\end{tabular}
\end{tabular}
}


For these inputs the variants of the RLS perform differently to the binomial input.
The only similarity is the RLS being the only algorithm that did not find an optimal solution for every input.
If the RLS did find an optimal solution in those 21 cases it instead might be the best RLS variant.
The other algorithms are ranked by their probability of flipping only one bit.
This means at first the three \RLSR[s] variants from 2 to 3 to 4 and then the same for the \RLSN[b] variants.
So it does seem like moving mostly one element at once is better for the geometric input in comparison to two elements for the binomial distribution.
In the 21 cases where the RLS did not find an optimal solution it was most likely stuck in a local optimum where no small value was left.

\subsection{(1+1) EA Comparison}
\makebox[\linewidth]{
\begin{tabular}{lp{3cm}p{6cm}p{6cm}}
\begin{tabular}[h]{ccccccccc}
algo type&           EA-SM&       EA-SM&    EA-SM&    EA-SM&    EA-SM&    EA-SM&    EA-SM&    EA-SM\\
algo param&            2$/n$&        -&      3$/n$&      4$/n$&      5$/n$&     10$/n$&     50$/n$&    100$/n$\\
avg mut/change&      2.246&    1.551&    3.048&    3.936&    4.861&    9.822&   49.750&   99.707\\
avg mut/step&        2.000&    1.000&    3.000&    4.000&    5.000&   10.000&   50.000&  100.001\\
\hline
avg eval count&      3,097&    3,505&    3,518&    4,009&    4,807&    7,758&   18,457&   25,993\\
max eval count&     39,490&   60,533&   39,048&   47,881&   56,204&   91,305&  173,851&  354,479\\
min eval count&         10&        0&        6&        5&        3&        5&        9&        3\\
\hline
fail ratio&          0.000&    0.000&    0.000&    0.000&    0.000&    0.000&    0.000&    0.000\\
\end{tabular}
\end{tabular}
}


The results for the (1+1) EA are similar to the results of the RLS. From mutation rate $2/n$ on the runtime increases with rising mutation rate.
The only part that does not fit into the theory of 1 bit flips being superior is the mutation rate $2/n$ performing better than the standard $1/n$.
All variants reach an optimal solution within the given limit for the number of iterations.
\subsection{pmut Comparison}
\makebox[\linewidth]{
\scriptsize
\begin{tabular}{lp{3cm}p{6cm}p{6cm}}
\begin{tabular}[h]{m{2.5cm}m{0,40cm}m{0,40cm}m{0,40cm}m{0,40cm}m{0,40cm}m{0,40cm}m{0,40cm}m{0,40cm}m{0,40cm}m{0,40cm}m{0,40cm}m{0,40cm}m{0,40cm}m{0,40cm}m{0,40cm}m{0,40cm}m{0,40cm}m{0,40cm}}
\multicolumn{1}{c}{algo type}&\multicolumn{2}{c}{            pmut}&\multicolumn{2}{c}{     pmut}&\multicolumn{2}{c}{     pmut}&\multicolumn{2}{c}{     pmut}&\multicolumn{2}{c}{     pmut}&\multicolumn{2}{c}{     pmut}&\multicolumn{2}{c}{     pmut}&\multicolumn{2}{c}{     pmut}&\multicolumn{2}{c}{     pmut}\\
\multicolumn{1}{c}{algo param}&\multicolumn{2}{c}{           1.25}&\multicolumn{2}{c}{     1.50}&\multicolumn{2}{c}{     1.75}&\multicolumn{2}{c}{     2.00}&\multicolumn{2}{c}{     2.25}&\multicolumn{2}{c}{     2.50}&\multicolumn{2}{c}{     2.75}&\multicolumn{2}{c}{     3.00}&\multicolumn{2}{c}{     3.25}\\
\multicolumn{1}{c}{avg mut/change}&\multicolumn{2}{c}{    197.409}&\multicolumn{2}{c}{   70.534}&\multicolumn{2}{c}{   23.050}&\multicolumn{2}{c}{    8.724}&\multicolumn{2}{c}{    4.351}&\multicolumn{2}{c}{    2.777}&\multicolumn{2}{c}{    2.111}&\multicolumn{2}{c}{    1.770}&\multicolumn{2}{c}{    1.563}\\
\multicolumn{1}{c}{avg mut/step}&\multicolumn{2}{c}{      224.442}&\multicolumn{2}{c}{   70.480}&\multicolumn{2}{c}{   22.299}&\multicolumn{2}{c}{    8.470}&\multicolumn{2}{c}{    4.368}&\multicolumn{2}{c}{    2.906}&\multicolumn{2}{c}{    2.271}&\multicolumn{2}{c}{    1.934}&\multicolumn{2}{c}{    1.729}\\
\hline
\multicolumn{1}{c}{total avg count}&\multicolumn{2}{c}{        42}&\multicolumn{2}{c}{       87}&\multicolumn{2}{c}{      216}&\multicolumn{2}{c}{      503}&\multicolumn{2}{c}{    1,094}&\multicolumn{2}{c}{    4,063}&\multicolumn{2}{c}{   10,961}&\multicolumn{2}{c}{   18,727}&\multicolumn{2}{c}{   27,644}\\
\multicolumn{1}{c}{avg eval count}&\multicolumn{2}{c}{         42}&\multicolumn{2}{c}{       87}&\multicolumn{2}{c}{      216}&\multicolumn{2}{c}{      503}&\multicolumn{2}{c}{      910}&\multicolumn{2}{c}{    1,488}&\multicolumn{2}{c}{    1,676}&\multicolumn{2}{c}{    1,814}&\multicolumn{2}{c}{    1,909}\\
\multicolumn{1}{c}{max eval count}&\multicolumn{2}{c}{        303}&\multicolumn{2}{c}{      867}&\multicolumn{2}{c}{    2,843}&\multicolumn{2}{c}{    6,230}&\multicolumn{2}{c}{   10,487}&\multicolumn{2}{c}{  916,298}&\multicolumn{2}{c}{  501,346}&\multicolumn{2}{c}{  411,742}&\multicolumn{2}{c}{   12,386}\\
\multicolumn{1}{c}{min eval count}&\multicolumn{2}{c}{          0}&\multicolumn{2}{c}{        0}&\multicolumn{2}{c}{        0}&\multicolumn{2}{c}{        0}&\multicolumn{2}{c}{        0}&\multicolumn{2}{c}{        0}&\multicolumn{2}{c}{        0}&\multicolumn{2}{c}{        0}&\multicolumn{2}{c}{        0}\\
\hline
\multicolumn{1}{c}{fail ratio}&\multicolumn{2}{c}{          0.000}&\multicolumn{2}{c}{    0.000}&\multicolumn{2}{c}{    0.000}&\multicolumn{2}{c}{    0.000}&\multicolumn{2}{c}{    0.000}&\multicolumn{2}{c}{    0.003}&\multicolumn{2}{c}{    0.010}&\multicolumn{2}{c}{    0.018}&\multicolumn{2}{c}{    0.028}\\
\multicolumn{1}{c}{avg fail dif}&\multicolumn{2}{c}{            -}&\multicolumn{2}{c}{        -}&\multicolumn{2}{c}{        -}&\multicolumn{2}{c}{        -}&\multicolumn{2}{c}{      345}&\multicolumn{2}{c}{      345}&\multicolumn{2}{c}{      345}&\multicolumn{2}{c}{      345}&\multicolumn{2}{c}{      345}\\
\hline
\multicolumn{1}{c}{p-value}&&\multicolumn{2}{c}{0.0000}&\multicolumn{2}{c}{0.0000}&\multicolumn{2}{c}{0.0000}&\multicolumn{2}{c}{0.0000}&\multicolumn{2}{c}{0.0000}&\multicolumn{2}{c}{0.0000}&\multicolumn{2}{c}{0.0034}&\multicolumn{2}{c}{0.0068}\\
&&&&&&&&&&&&&&&&&&\end{tabular}
\end{tabular}
}


The results for the $pmut_\beta$ operator are even more clear than for the (1+1) EA.\
With decreasing values for $\beta$ the amount of flipped bits per step increases.
The performance decreases as well with decreasing values for $\beta$ which fits into the theory of one bit flips being better for geometric distributed inputs.
The number of repetitions of the algorithm might simply be too small to make the small difference in the performance between the two values visible.
The difference in the performance for the $pmut_\beta$ operator is not as drastic as for the (1+1) EA.
Only $\beta=1.5$ and $\beta=1.25$ perform significantly worse the next best value.

\subsection{Comparison of the best variants}
% \makebox[\linewidth]{
\begin{tabular}{lp{3cm}p{6cm}p{6cm}}
\begin{tabular}[h]{cccc}
algo type&        \RLSN& (1+1) EA&  pmut\\
algo param&         b=2&   3$/n$&  2.25\\
avg mut/change&   2.000& 3.092& 3.965\\
avg mut/step&     2.000& 2.999& 4.339\\
\hline
total avg count&    302&   677&   691\\
avg eval count&     302&   677&   691\\
max eval count&   1,610& 6,404& 5,205\\
min eval count&       9&    33&    17\\
\hline
fail ratio&       0.000& 0.000& 0.000\\
\end{tabular}
\end{tabular}
}

% 
The setup for the evaluation of lower values for $n$ is mostly the same except for having a fixed time limit of 100,000 instead of using 50,000 as the limit.
The first try was executed with 50,000 but there the algorithms performed too bad for $n=20$.
Therefore in the second attempt the step limit was increased to 100,000.
The first table lists the number of runs where the different algorithms did not find the optimal solution within the time limit.

\begin{tabular}[h]{cccccccc}
fails&20&50&100&500&1000&5000&10000\\\hline
RLS-N (2)&998&984&973&763&541&49&1\\
RLS-N (4)&998&989&975&837&719&150&27\\
(1+1) EA (3/n)&994&990&981&862&714&194&35\\
(1+1) EA (4/n)&998&993&976&852&725&180&51\\
pmut (-2.0)&997&989&983&905&784&229&55\\
pmut (-2.25)&997&988&987&875&790&213&56\\
\end{tabular}


For small inputs the geometric distributed input seems to have inputs without a perfect partition because there were many iterations where neither of the algorithms found an optimal solution within the time limit.
It is still likely to have a perfect partition even for the small values in comparison to other distributions which follow afterwards.
Many algorithms especially the variants of the RLS seem to be likely to get stuck in a local optimum.
The (1+1) EA finds an optimum in most of the runs, so the geometric distributed inputs also seem to be likely to have a perfect partition for small values.
They definitely are harder to solve for smaller input sizes than the binomial inputs, but they still have a perfect partition most times.
The next table visualises the average number of iterations the algorithms needed for finding an optimal solution if the algorithm managed to do so.

\begin{tabular}[h]{ccccccccc}
avg&20&50&100&500&1000&5000&10000&50000\\\hline
RLS-N (2)&172&1247&7598&38282&36214&105152&117792&125415\\
RLS-N (4)&8626&40966&43921&41924&42289&138187&201183&214170\\
(1+1) EA (3/n)&36470&42555&42439&43801&45530&145209&212281&252713\\
(1+1) EA (4/n)&37212&43144&44340&47246&46689&143697&220208&265620\\
pmut (-2.0)&40368&42601&39603&47443&46591&148390&232398&311606\\
pmut (-2.25)&39476&42113&40325&40548&43691&157816&226129&292367\\
\end{tabular}


The variants of the (1+1) EA and of the $pmut$ algorithm seem to take about 20,000 iterations for $n=20$ if they manage to find the optimal solution.
They also perform better and better the bigger the input gets.
This is probability caused by the many additional small values that can be used for smaller adjustments to the fitness.
Also a really high value does not have as much of an effect, because there are possibly other larger values which cancel each other out, if they are in different bins.
The standard (1+1) EA does not only find a perfect partition less often, it also needs more iterations on average if it does.
So the (1+1) EA with $p_m=2/n$ performs indeed better for every input size.
The last table again lists the total average number of steps.

\begin{tabular}[h]{ccccccccc}
total avg&20&50&100&500&1000&5000&10000&50000\\\hline
RLS&98050&94574&89765&60172&39985&15639&4128&1797\\
\RLSR[2]&89401&54556&18266&4218&3530&2362&2160&2229\\
(1+1) EA (1$/n$)&54971&26688&15684&8452&6567&3815&3458&3371\\
(1+1) EA (2$/n$)&26104&9724&6508&4503&4020&3171&3141&3133\\
pmut (3.25)&40934&19972&10977&5644&4406&2434&2162&2172\\
pmut (3.0)&35272&17545&10040&5222&4150&2510&2208&2213\\
\end{tabular}


The RLS is only an option if the input is large enough ($n \ge 10,000$). For smaller input sizes especially for $n \le 100$ choosing the (1+1) EA with mutation rate $2/n$ seems like the best choice. For larger values this (1+1) EA does not find an optimal solution the fastest but is still fast enough to be a viable option. Another rather save option is $pmut_{3.25}$. This algorithm performs worse for $n \le 100$ but is still good in comparison to the other algorithms. For $n \ge 1000$ $pmut_{3.25}$ starts to outperform the best version of the (1+1) EA and almost all other researched algorithms.

\section{Uniform distributed inputs}
For the uniform distribution the default values were 1 for the lower bound and 50000 for the upper bound (exclusive).
The range was limited to 50000 to reduce the time the algorithms needs to find an optimal solution.
The higher the values are with too few values the more likely the input is to not have a perfect partition\cite{borgs2001phase}.
This will cause the algorithms to always reach the limit for the number of iterations which drastically increases the time needed for the experiment.
The length of the input was 50000.


\begin{figure}[h]
      \caption{Distribution of a random uniform input (10000 values between 1 and 100)}
      \centering
      \includegraphics[width=0.7\textwidth]{figures/images/numberGenerator/uniformDistributionMin1Max101n10000.png}\label{fig:uniDistExample}
\end{figure}
\subsection{RLS Comparison}


\makebox[\linewidth]{
\begin{tabular}{lp{3cm}p{6cm}p{6cm}}
\begin{tabular}[h]{cccccccc}
algo type&            RLS&   \RLSR[s]&   \RLSR[s]&   \RLSR[s]&   \RLSN[b]&   \RLSN[b]&   \RLSN[b]\\
algo param&             -&     s=2&     s=3&     s=4&     b=3&     b=2&     b=4\\
avg mut/change&     1.000&   1.181&   1.688&   1.865&   3.000&   1.997&   3,997\\
avg mut/step&       1.000&   1.500&   2.000&   2.500&   3.000&   2.000&   3.000\\
\hline
total avg count&   90,931& 168,311& 236,317& 307,533& 921,030& 921,030& 921,030\\
avg eval count&    90,931& 168,311& 236,317& 307,533&       -&       -&       -\\
max eval count&   156,854& 296,206& 498,474& 595,831&       -&       -&       -\\
min eval count&    64,941& 120,582& 158,304& 212,193&       -&       -&       -\\
\hline
fail ratio&         0.000&   0.000&   0.000&   0.000&   1.000&   1.000&   1.000\\
avg fail dif&           -&       -&       -&       -&      36&      53&     263\\
\end{tabular}
\end{tabular}
}


The picture for the RLS variants on this type of input is not clear.
There in no obvious tendency for neither of the variants.
The only obvious thing is the RLS being the worst of the RLS variants again.
Every variant reaches the optimal solution in every case except for the RLS which only manages for 44.7 \% of the inputs.
The RLS-N$_2$ seems to be the best variant for these kinds of inputs.
The next best variants are the RLS-R with $k=3$ and $k=4$ which only differ by 1 \%.

\subsection{(1+1) EA Comparison}


\makebox[\linewidth]{
\begin{tabular}{lp{3cm}p{6cm}p{6cm}}
\begin{tabular}[h]{ccccccccc}
algo type&           EA-SM&       EA-SM&    EA-SM&    EA-SM&    EA-SM&    EA-SM&    EA-SM&    EA-SM\\
algo param&            2$/n$&        -&      3$/n$&      4$/n$&      5$/n$&     10$/n$&     50$/n$&    100$/n$\\
avg mut/change&      2.246&    1.551&    3.048&    3.936&    4.861&    9.822&   49.750&   99.707\\
avg mut/step&        2.000&    1.000&    3.000&    4.000&    5.000&   10.000&   50.000&  100.001\\
\hline
avg eval count&      3,097&    3,505&    3,518&    4,009&    4,807&    7,758&   18,457&   25,993\\
max eval count&     39,490&   60,533&   39,048&   47,881&   56,204&   91,305&  173,851&  354,479\\
min eval count&         10&        0&        6&        5&        3&        5&        9&        3\\
\hline
fail ratio&          0.000&    0.000&    0.000&    0.000&    0.000&    0.000&    0.000&    0.000\\
\end{tabular}
\end{tabular}
}


The (1+1) EA seems to perform better with a lower mutation rate.
The vales $p_m=2/n$ and $p_m=3/n$ reach an optimal solution equally fast.
From then on the speed of convergence decreases with increasing mutation rate.
The only exception from this case is the standard (1+1) EA which performs the worst despite having the lowest mutation rate.
For the uniform distributed input all variants of the (1+1) EA reach an optimal solution within the step limit as for the previous input types.
\subsection{pmut Comparison}


\makebox[\linewidth]{
\scriptsize
\begin{tabular}{lp{3cm}p{6cm}p{6cm}}
\begin{tabular}[h]{m{2.5cm}m{0,40cm}m{0,40cm}m{0,40cm}m{0,40cm}m{0,40cm}m{0,40cm}m{0,40cm}m{0,40cm}m{0,40cm}m{0,40cm}m{0,40cm}m{0,40cm}m{0,40cm}m{0,40cm}m{0,40cm}m{0,40cm}m{0,40cm}m{0,40cm}}
\multicolumn{1}{c}{algo type}&\multicolumn{2}{c}{            pmut}&\multicolumn{2}{c}{     pmut}&\multicolumn{2}{c}{     pmut}&\multicolumn{2}{c}{     pmut}&\multicolumn{2}{c}{     pmut}&\multicolumn{2}{c}{     pmut}&\multicolumn{2}{c}{     pmut}&\multicolumn{2}{c}{     pmut}&\multicolumn{2}{c}{     pmut}\\
\multicolumn{1}{c}{algo param}&\multicolumn{2}{c}{           1.25}&\multicolumn{2}{c}{     1.50}&\multicolumn{2}{c}{     1.75}&\multicolumn{2}{c}{     2.00}&\multicolumn{2}{c}{     2.25}&\multicolumn{2}{c}{     2.50}&\multicolumn{2}{c}{     2.75}&\multicolumn{2}{c}{     3.00}&\multicolumn{2}{c}{     3.25}\\
\multicolumn{1}{c}{avg mut/change}&\multicolumn{2}{c}{    197.409}&\multicolumn{2}{c}{   70.534}&\multicolumn{2}{c}{   23.050}&\multicolumn{2}{c}{    8.724}&\multicolumn{2}{c}{    4.351}&\multicolumn{2}{c}{    2.777}&\multicolumn{2}{c}{    2.111}&\multicolumn{2}{c}{    1.770}&\multicolumn{2}{c}{    1.563}\\
\multicolumn{1}{c}{avg mut/step}&\multicolumn{2}{c}{      224.442}&\multicolumn{2}{c}{   70.480}&\multicolumn{2}{c}{   22.299}&\multicolumn{2}{c}{    8.470}&\multicolumn{2}{c}{    4.368}&\multicolumn{2}{c}{    2.906}&\multicolumn{2}{c}{    2.271}&\multicolumn{2}{c}{    1.934}&\multicolumn{2}{c}{    1.729}\\
\hline
\multicolumn{1}{c}{total avg count}&\multicolumn{2}{c}{        42}&\multicolumn{2}{c}{       87}&\multicolumn{2}{c}{      216}&\multicolumn{2}{c}{      503}&\multicolumn{2}{c}{    1,094}&\multicolumn{2}{c}{    4,063}&\multicolumn{2}{c}{   10,961}&\multicolumn{2}{c}{   18,727}&\multicolumn{2}{c}{   27,644}\\
\multicolumn{1}{c}{avg eval count}&\multicolumn{2}{c}{         42}&\multicolumn{2}{c}{       87}&\multicolumn{2}{c}{      216}&\multicolumn{2}{c}{      503}&\multicolumn{2}{c}{      910}&\multicolumn{2}{c}{    1,488}&\multicolumn{2}{c}{    1,676}&\multicolumn{2}{c}{    1,814}&\multicolumn{2}{c}{    1,909}\\
\multicolumn{1}{c}{max eval count}&\multicolumn{2}{c}{        303}&\multicolumn{2}{c}{      867}&\multicolumn{2}{c}{    2,843}&\multicolumn{2}{c}{    6,230}&\multicolumn{2}{c}{   10,487}&\multicolumn{2}{c}{  916,298}&\multicolumn{2}{c}{  501,346}&\multicolumn{2}{c}{  411,742}&\multicolumn{2}{c}{   12,386}\\
\multicolumn{1}{c}{min eval count}&\multicolumn{2}{c}{          0}&\multicolumn{2}{c}{        0}&\multicolumn{2}{c}{        0}&\multicolumn{2}{c}{        0}&\multicolumn{2}{c}{        0}&\multicolumn{2}{c}{        0}&\multicolumn{2}{c}{        0}&\multicolumn{2}{c}{        0}&\multicolumn{2}{c}{        0}\\
\hline
\multicolumn{1}{c}{fail ratio}&\multicolumn{2}{c}{          0.000}&\multicolumn{2}{c}{    0.000}&\multicolumn{2}{c}{    0.000}&\multicolumn{2}{c}{    0.000}&\multicolumn{2}{c}{    0.000}&\multicolumn{2}{c}{    0.003}&\multicolumn{2}{c}{    0.010}&\multicolumn{2}{c}{    0.018}&\multicolumn{2}{c}{    0.028}\\
\multicolumn{1}{c}{avg fail dif}&\multicolumn{2}{c}{            -}&\multicolumn{2}{c}{        -}&\multicolumn{2}{c}{        -}&\multicolumn{2}{c}{        -}&\multicolumn{2}{c}{      345}&\multicolumn{2}{c}{      345}&\multicolumn{2}{c}{      345}&\multicolumn{2}{c}{      345}&\multicolumn{2}{c}{      345}\\
\hline
\multicolumn{1}{c}{p-value}&&\multicolumn{2}{c}{0.0000}&\multicolumn{2}{c}{0.0000}&\multicolumn{2}{c}{0.0000}&\multicolumn{2}{c}{0.0000}&\multicolumn{2}{c}{0.0000}&\multicolumn{2}{c}{0.0000}&\multicolumn{2}{c}{0.0034}&\multicolumn{2}{c}{0.0068}\\
&&&&&&&&&&&&&&&&&&\end{tabular}
\end{tabular}
}


The optimal value for $\beta$ seems to be somewhere around -2.0 to -2.5.
The values next to this interval start to decrease in both directions, but -1.75 and -2.75 are still relatively close to the performance of the optimal value.
The values equally wide apart from -2.25 perform equally good.

\subsection{Comparison of the best variants}


\makebox[\linewidth]{
\begin{tabular}{lp{3cm}p{6cm}p{6cm}}
\begin{tabular}[h]{cccc}
algo type&        \RLSN& (1+1) EA&  pmut\\
algo param&         b=2&   3$/n$&  2.25\\
avg mut/change&   2.000& 3.092& 3.965\\
avg mut/step&     2.000& 2.999& 4.339\\
\hline
total avg count&    302&   677&   691\\
avg eval count&     302&   677&   691\\
max eval count&   1,610& 6,404& 5,205\\
min eval count&       9&    33&    17\\
\hline
fail ratio&       0.000& 0.000& 0.000\\
\end{tabular}
\end{tabular}
}


For the uniform distributed input the best variant of the RLS once again seems to perform the best.
But by looking at the smaller values again this does not hold in general.

\begin{tabular}[h]{cccccccc}
fails&20&50&100&500&1000&5000&10000\\\hline
RLS-N (2)&998&984&973&763&541&49&1\\
RLS-N (4)&998&989&975&837&719&150&27\\
(1+1) EA (3/n)&994&990&981&862&714&194&35\\
(1+1) EA (4/n)&998&993&976&852&725&180&51\\
pmut (-2.0)&997&989&983&905&784&229&55\\
pmut (-2.25)&997&988&987&875&790&213&56\\
\end{tabular}


The RLS variants are the most likely to get stuck in a local optimum for $n\le100$.
The (1+1) EA variants also often do not find an optimal solution, but this happens less frequently.
The more values the input has the more likely it is for any of the algorithms to find a perfect partition.
Between $n=100$ and $n=500$ the performance of the RLS-N$_2$ drastically increases and for $n\ge500$ this variant of the RLS stays the best variant for the remaining input sizes.
Uniform distributed inputs seem to be much less likely to have a perfect partition for the small input sizes which can be explained by Borgs coefficient~\cite{borgs2001phase}.

\begin{tabular}[h]{ccccccccc}
avg&20&50&100&500&1000&5000&10000&50000\\\hline
RLS-N (2)&172&1247&7598&38282&36214&105152&117792&125415\\
RLS-N (4)&8626&40966&43921&41924&42289&138187&201183&214170\\
(1+1) EA (3/n)&36470&42555&42439&43801&45530&145209&212281&252713\\
(1+1) EA (4/n)&37212&43144&44340&47246&46689&143697&220208&265620\\
pmut (-2.0)&40368&42601&39603&47443&46591&148390&232398&311606\\
pmut (-2.25)&39476&42113&40325&40548&43691&157816&226129&292367\\
\end{tabular}


The amount steps needed to find an optimal solution seems to be nearly constant for every algorithm as the number of steps does not strictly increase with $n$ but sometimes even decreases for $n\le1000$.
This is caused by the number of steps the algorithm was given.
For $n\le1000$ the step size was 100,000 and for the bigger values it was $10n\ln(n)$.
Interestingly enough the average number of steps decreases from $n=10000$ to $n=50000$ for most algorithms.

\begin{tabular}[h]{ccccccccc}
total avg&20&50&100&500&1000&5000&10000&50000\\\hline
RLS&98050&94574&89765&60172&39985&15639&4128&1797\\
\RLSR[2]&89401&54556&18266&4218&3530&2362&2160&2229\\
(1+1) EA (1$/n$)&54971&26688&15684&8452&6567&3815&3458&3371\\
(1+1) EA (2$/n$)&26104&9724&6508&4503&4020&3171&3141&3133\\
pmut (3.25)&40934&19972&10977&5644&4406&2434&2162&2172\\
pmut (3.0)&35272&17545&10040&5222&4150&2510&2208&2213\\
\end{tabular}



My general advice would be choosing the RLS-N$_2$ for $n\ge500$ and the (1+1) EA with $p_m=4/n$ otherwise.

\section{powerlaw distributed inputs}
This distribution has mostly small values, but occasionally it also generates bigger values.
The lower the parameter the higher the values get and also the amount of big values increases.
For a parameter of $\beta=2.75$ and a maximum value of 10,000 the distribution looks like in Figure~\ref{fig:powerDistExample1}.
All values are rather small and less than 100, also half of the values are one.
So this input seems rather easy for $\beta=2.75$.

\begin{figure}[h]
      \caption{Distribution of a random powerlaw input with $\beta=2.75$}
      \centering
      \includegraphics[width=0.7\textwidth]{figures/images/numberGenerator/powerlaw_-2_75.png}\label{fig:powerDistExample1}
\end{figure}

For a value of $\beta=1.25$ the distribution looks a bit different.
There are less small values close to one and instead also big values even over 1000.
Figure~\ref{fig:powerDistExample2} is cropped to get a more clear view for the smaller values.
The higher values mostly occurred 0 to 2 times.
The highest value 9948 occurred only once.
Researching inputs like this should be more interesting which is why for the experiment $\beta=1.25$ was chosen.
To give a better view on this type of input there is also a table $\beta=2.75$ at the evaluation of the (1+1) EA.\
The results for the other algorithms were mostly the same, but these are not shown here for better readability.

\begin{figure}[h]
      \caption{Distribution of a random powerlaw input with $\beta=1.25$}
      \centering
      \includegraphics[width=0.7\textwidth]{figures/images/numberGenerator/powerlaw_-1_25.png}\label{fig:powerDistExample2}
\end{figure}
\subsection{RLS Comparison}
\makebox[\linewidth]{
\begin{tabular}{lp{3cm}p{6cm}p{6cm}}
\begin{tabular}[h]{cccccccc}
algo type&            RLS&   \RLSR[s]&   \RLSR[s]&   \RLSR[s]&   \RLSN[b]&   \RLSN[b]&   \RLSN[b]\\
algo param&             -&     s=2&     s=3&     s=4&     b=3&     b=2&     b=4\\
avg mut/change&     1.000&   1.181&   1.688&   1.865&   3.000&   1.997&   3,997\\
avg mut/step&       1.000&   1.500&   2.000&   2.500&   3.000&   2.000&   3.000\\
\hline
total avg count&   90,931& 168,311& 236,317& 307,533& 921,030& 921,030& 921,030\\
avg eval count&    90,931& 168,311& 236,317& 307,533&       -&       -&       -\\
max eval count&   156,854& 296,206& 498,474& 595,831&       -&       -&       -\\
min eval count&    64,941& 120,582& 158,304& 212,193&       -&       -&       -\\
\hline
fail ratio&         0.000&   0.000&   0.000&   0.000&   1.000&   1.000&   1.000\\
avg fail dif&           -&       -&       -&       -&      36&      53&     263\\
\end{tabular}
\end{tabular}
}


The input is even easier to solve for the RLS variants than the binomial distributed inputs.
There is no clear tendency and all algorithms have a rather equal runtime and also every algorithm manages to find an optimal solution in every run.
\subsection{(1+1) EA Comparison}
The first table shows the results for parameter $\beta=2.75$

\makebox[\linewidth]{
\begin{tabular}{lp{3cm}p{6cm}p{6cm}}
\begin{tabular}[h]{m{2.5cm}m{0,45cm}m{0,45cm}m{0,45cm}m{0,45cm}m{0,45cm}m{0,45cm}m{0,45cm}m{0,45cm}m{0,45cm}m{0,45cm}m{0,45cm}m{0,45cm}m{0,45cm}m{0,45cm}m{0,45cm}m{0,45cm}}
\multicolumn{1}{c}{algo type}&\multicolumn{2}{c}{           EA-SM}&\multicolumn{2}{c}{    EA-SM}&\multicolumn{2}{c}{    EA-SM}&\multicolumn{2}{c}{    EA-SM}&\multicolumn{2}{c}{    EA-SM}&\multicolumn{2}{c}{    EA-SM}&\multicolumn{2}{c}{    EA-SM}&\multicolumn{2}{c}{       EA}\\
\multicolumn{1}{c}{algo param}&\multicolumn{2}{c}{           50$/n$}&\multicolumn{2}{c}{    100$/n$}&\multicolumn{2}{c}{     10$/n$}&\multicolumn{2}{c}{      5$/n$}&\multicolumn{2}{c}{      4$/n$}&\multicolumn{2}{c}{      3$/n$}&\multicolumn{2}{c}{      2$/n$}&\multicolumn{2}{c}{        -}\\
\multicolumn{1}{c}{avg mut/change}&\multicolumn{2}{c}{     49.922}&\multicolumn{2}{c}{   99.873}&\multicolumn{2}{c}{   10.009}&\multicolumn{2}{c}{    5.053}&\multicolumn{2}{c}{    4.105}&\multicolumn{2}{c}{    3.156}&\multicolumn{2}{c}{    2.280}&\multicolumn{2}{c}{    1.534}\\
\multicolumn{1}{c}{avg mut/step}&\multicolumn{2}{c}{       49.989}&\multicolumn{2}{c}{  100.017}&\multicolumn{2}{c}{    9.999}&\multicolumn{2}{c}{    4.999}&\multicolumn{2}{c}{    4.005}&\multicolumn{2}{c}{    3.003}&\multicolumn{2}{c}{    2.000}&\multicolumn{2}{c}{    0.999}\\
\hline
\multicolumn{1}{c}{avg eval count}&\multicolumn{2}{c}{         84}&\multicolumn{2}{c}{      103}&\multicolumn{2}{c}{      111}&\multicolumn{2}{c}{      157}&\multicolumn{2}{c}{      184}&\multicolumn{2}{c}{      208}&\multicolumn{2}{c}{      273}&\multicolumn{2}{c}{      461}\\
\multicolumn{1}{c}{max eval count}&\multicolumn{2}{c}{      1,281}&\multicolumn{2}{c}{    1,488}&\multicolumn{2}{c}{    1,946}&\multicolumn{2}{c}{    3,030}&\multicolumn{2}{c}{    3,043}&\multicolumn{2}{c}{    3,283}&\multicolumn{2}{c}{    4,744}&\multicolumn{2}{c}{    7,036}\\
\multicolumn{1}{c}{min eval count}&\multicolumn{2}{c}{          0}&\multicolumn{2}{c}{        1}&\multicolumn{2}{c}{        3}&\multicolumn{2}{c}{        1}&\multicolumn{2}{c}{        0}&\multicolumn{2}{c}{        0}&\multicolumn{2}{c}{        2}&\multicolumn{2}{c}{        0}\\
\hline
\multicolumn{1}{c}{fail ratio}&\multicolumn{2}{c}{          0.000}&\multicolumn{2}{c}{    0.000}&\multicolumn{2}{c}{    0.000}&\multicolumn{2}{c}{    0.000}&\multicolumn{2}{c}{    0.000}&\multicolumn{2}{c}{    0.000}&\multicolumn{2}{c}{    0.000}&\multicolumn{2}{c}{    0.000}\\
&&&&&&&&&&&&&&&&\end{tabular}
\end{tabular}
}


Here the same rule holds for the RLS to some extent.
Until $p_m\le50/n$ the speed of convergence increases but at $p_m=100/n$ the speed decreases again.
The optimal value seems to be somewhere around $p_m=50/n$.
The (1+1) variants are generally faster than all RLS variants when comparing the maximum number of iterations.
For mutation rates $3/n\le p_m \le 100/n$ the (1+1) EA is also faster on average.
The next table shows the results for a powerlaw distribution with $\beta=1.25$.

\makebox[\linewidth]{
\begin{tabular}{lp{3cm}p{6cm}p{6cm}}
\begin{tabular}[h]{ccccccccc}
algo type&           EA-SM&       EA-SM&    EA-SM&    EA-SM&    EA-SM&    EA-SM&    EA-SM&    EA-SM\\
algo param&            2$/n$&        -&      3$/n$&      4$/n$&      5$/n$&     10$/n$&     50$/n$&    100$/n$\\
avg mut/change&      2.246&    1.551&    3.048&    3.936&    4.861&    9.822&   49.750&   99.707\\
avg mut/step&        2.000&    1.000&    3.000&    4.000&    5.000&   10.000&   50.000&  100.001\\
\hline
avg eval count&      3,097&    3,505&    3,518&    4,009&    4,807&    7,758&   18,457&   25,993\\
max eval count&     39,490&   60,533&   39,048&   47,881&   56,204&   91,305&  173,851&  354,479\\
min eval count&         10&        0&        6&        5&        3&        5&        9&        3\\
\hline
fail ratio&          0.000&    0.000&    0.000&    0.000&    0.000&    0.000&    0.000&    0.000\\
\end{tabular}
\end{tabular}
}


With this setting the optimal value is shifted to somewhere around $p_m=4/n$.
The higher mutation rates perform drastically slower with $p_m=100/n$ being 500 times slower than the optimal value.
The speed of convergence is sometimes even to slow find an optimal solution in time $10\cdot n\ln(n)$.
\subsection{pmut Comparison}
For pmut the same holds as for the RLS.\
The more bits the algorithms flips on average the better the performance on average.
Surprisingly the performance in the worst runs behaves inverted.
The fewer bits the algorithm flips on average the more stable the search becomes.
This might be caused by the really large amount of bits flipped for the lower values.

\makebox[\linewidth]{
\scriptsize
\begin{tabular}{lp{3cm}p{6cm}p{6cm}}
\begin{tabular}[h]{m{2.5cm}m{0,40cm}m{0,40cm}m{0,40cm}m{0,40cm}m{0,40cm}m{0,40cm}m{0,40cm}m{0,40cm}m{0,40cm}m{0,40cm}m{0,40cm}m{0,40cm}m{0,40cm}m{0,40cm}m{0,40cm}m{0,40cm}m{0,40cm}m{0,40cm}}
\multicolumn{1}{c}{algo type}&\multicolumn{2}{c}{            pmut}&\multicolumn{2}{c}{     pmut}&\multicolumn{2}{c}{     pmut}&\multicolumn{2}{c}{     pmut}&\multicolumn{2}{c}{     pmut}&\multicolumn{2}{c}{     pmut}&\multicolumn{2}{c}{     pmut}&\multicolumn{2}{c}{     pmut}&\multicolumn{2}{c}{     pmut}\\
\multicolumn{1}{c}{algo param}&\multicolumn{2}{c}{           1.25}&\multicolumn{2}{c}{     1.50}&\multicolumn{2}{c}{     1.75}&\multicolumn{2}{c}{     2.00}&\multicolumn{2}{c}{     2.25}&\multicolumn{2}{c}{     2.50}&\multicolumn{2}{c}{     2.75}&\multicolumn{2}{c}{     3.00}&\multicolumn{2}{c}{     3.25}\\
\multicolumn{1}{c}{avg mut/change}&\multicolumn{2}{c}{    197.409}&\multicolumn{2}{c}{   70.534}&\multicolumn{2}{c}{   23.050}&\multicolumn{2}{c}{    8.724}&\multicolumn{2}{c}{    4.351}&\multicolumn{2}{c}{    2.777}&\multicolumn{2}{c}{    2.111}&\multicolumn{2}{c}{    1.770}&\multicolumn{2}{c}{    1.563}\\
\multicolumn{1}{c}{avg mut/step}&\multicolumn{2}{c}{      224.442}&\multicolumn{2}{c}{   70.480}&\multicolumn{2}{c}{   22.299}&\multicolumn{2}{c}{    8.470}&\multicolumn{2}{c}{    4.368}&\multicolumn{2}{c}{    2.906}&\multicolumn{2}{c}{    2.271}&\multicolumn{2}{c}{    1.934}&\multicolumn{2}{c}{    1.729}\\
\hline
\multicolumn{1}{c}{total avg count}&\multicolumn{2}{c}{        42}&\multicolumn{2}{c}{       87}&\multicolumn{2}{c}{      216}&\multicolumn{2}{c}{      503}&\multicolumn{2}{c}{    1,094}&\multicolumn{2}{c}{    4,063}&\multicolumn{2}{c}{   10,961}&\multicolumn{2}{c}{   18,727}&\multicolumn{2}{c}{   27,644}\\
\multicolumn{1}{c}{avg eval count}&\multicolumn{2}{c}{         42}&\multicolumn{2}{c}{       87}&\multicolumn{2}{c}{      216}&\multicolumn{2}{c}{      503}&\multicolumn{2}{c}{      910}&\multicolumn{2}{c}{    1,488}&\multicolumn{2}{c}{    1,676}&\multicolumn{2}{c}{    1,814}&\multicolumn{2}{c}{    1,909}\\
\multicolumn{1}{c}{max eval count}&\multicolumn{2}{c}{        303}&\multicolumn{2}{c}{      867}&\multicolumn{2}{c}{    2,843}&\multicolumn{2}{c}{    6,230}&\multicolumn{2}{c}{   10,487}&\multicolumn{2}{c}{  916,298}&\multicolumn{2}{c}{  501,346}&\multicolumn{2}{c}{  411,742}&\multicolumn{2}{c}{   12,386}\\
\multicolumn{1}{c}{min eval count}&\multicolumn{2}{c}{          0}&\multicolumn{2}{c}{        0}&\multicolumn{2}{c}{        0}&\multicolumn{2}{c}{        0}&\multicolumn{2}{c}{        0}&\multicolumn{2}{c}{        0}&\multicolumn{2}{c}{        0}&\multicolumn{2}{c}{        0}&\multicolumn{2}{c}{        0}\\
\hline
\multicolumn{1}{c}{fail ratio}&\multicolumn{2}{c}{          0.000}&\multicolumn{2}{c}{    0.000}&\multicolumn{2}{c}{    0.000}&\multicolumn{2}{c}{    0.000}&\multicolumn{2}{c}{    0.000}&\multicolumn{2}{c}{    0.003}&\multicolumn{2}{c}{    0.010}&\multicolumn{2}{c}{    0.018}&\multicolumn{2}{c}{    0.028}\\
\multicolumn{1}{c}{avg fail dif}&\multicolumn{2}{c}{            -}&\multicolumn{2}{c}{        -}&\multicolumn{2}{c}{        -}&\multicolumn{2}{c}{        -}&\multicolumn{2}{c}{      345}&\multicolumn{2}{c}{      345}&\multicolumn{2}{c}{      345}&\multicolumn{2}{c}{      345}&\multicolumn{2}{c}{      345}\\
\hline
\multicolumn{1}{c}{p-value}&&\multicolumn{2}{c}{0.0000}&\multicolumn{2}{c}{0.0000}&\multicolumn{2}{c}{0.0000}&\multicolumn{2}{c}{0.0000}&\multicolumn{2}{c}{0.0000}&\multicolumn{2}{c}{0.0000}&\multicolumn{2}{c}{0.0034}&\multicolumn{2}{c}{0.0068}\\
&&&&&&&&&&&&&&&&&&\end{tabular}
\end{tabular}
}


The optimal value here seems to be somewhere around $\beta=1.5$, so only lightly smaller in comparison to the (1+1) EA where the optimal value almost change from one side of the spectrum to the other.
Here the inverted stability of the search does not occur.
The variants that take longer on average tend to also take longer in their worst runs.

\subsection{Comparison of the best variants}
The first table again shows the results for parameter $\beta=2.75$

\makebox[\linewidth]{
\begin{tabular}{lp{3cm}p{6cm}p{6cm}}
\begin{tabular}[h]{cccc}
algo type&        \RLSN& (1+1) EA&  pmut\\
algo param&         b=2&   3$/n$&  2.25\\
avg mut/change&   2.000& 3.092& 3.965\\
avg mut/step&     2.000& 2.999& 4.339\\
\hline
total avg count&    302&   677&   691\\
avg eval count&     302&   677&   691\\
max eval count&   1,610& 6,404& 5,205\\
min eval count&       9&    33&    17\\
\hline
fail ratio&       0.000& 0.000& 0.000\\
\end{tabular}
\end{tabular}
}


The ranking follows the amount of bits the algorithms flip on average per step.
$pmut_{1.25}$ manages to find the solution in just 56 iterations on average.
The (1+1) EA with $p_m=50/n$ is slower than $pmut_{1.25}$ but instead has a lower value for the maximum number of iterations.
Both options seem fine.
Even the \RLSN[4] is still very fast for the powerlaw distributed input with $\beta = 2.75$.
For $\beta = 1.25$ the results are a bit different.

The \RLSR[4] now performs equally good as the (1+1) EA variant with $p_m=4/n$, but is still slower than $pmut_{1.5}$.
As the first inputs were less difficult to solve than the inputs with $\beta = 1.25$ the second value was chosen for the evaluation of smaller input sizes.

\begin{tabular}[h]{cccccccc}
fails&20&50&100&500&1000&5000&10000\\\hline
RLS-N (2)&998&984&973&763&541&49&1\\
RLS-N (4)&998&989&975&837&719&150&27\\
(1+1) EA (3/n)&994&990&981&862&714&194&35\\
(1+1) EA (4/n)&998&993&976&852&725&180&51\\
pmut (-2.0)&997&989&983&905&784&229&55\\
pmut (-2.25)&997&988&987&875&790&213&56\\
\end{tabular}


The RLS is once again the algorithm that is the most likely to be stuck in a local optimum.
Compared to the other algorithms it is not as drastic as for the binomial input for example.
Only for $n<500$ the algorithms do not find a global optimum in every run.
The setting of the parameter almost doesn't affect the amount of runs without an optimal result.
The main differences are between the different algorithms themselves.
This type of input is probably easy to solve if it has a perfect partition.
The two stopping conditions where a step limit and finding a perfect partition or a partition with difference of one between the two bin for uneven $n$.
So in 80\% of the runs the algorithms might have found an optimal solution, but the stopping conditions did not trigger as the solutions where not close to a perfect partition.

\begin{tabular}[h]{ccccccccc}
avg&20&50&100&500&1000&5000&10000&50000\\\hline
RLS-N (2)&172&1247&7598&38282&36214&105152&117792&125415\\
RLS-N (4)&8626&40966&43921&41924&42289&138187&201183&214170\\
(1+1) EA (3/n)&36470&42555&42439&43801&45530&145209&212281&252713\\
(1+1) EA (4/n)&37212&43144&44340&47246&46689&143697&220208&265620\\
pmut (-2.0)&40368&42601&39603&47443&46591&148390&232398&311606\\
pmut (-2.25)&39476&42113&40325&40548&43691&157816&226129&292367\\
\end{tabular}


Looking at the time the algorithms needed on average the runs that hit the step limit could have possibly been no failed runs.
The easiest are inputs with size $n=500$.
For smaller values of $n$ the algorithms sometimes fail and even in a good run they need more iterations to find an optimal solution.
Due to the increasing size of the input the algorithms need more time for the bigger values.

\begin{tabular}[h]{ccccccccc}
total avg&20&50&100&500&1000&5000&10000&50000\\\hline
RLS&98050&94574&89765&60172&39985&15639&4128&1797\\
\RLSR[2]&89401&54556&18266&4218&3530&2362&2160&2229\\
(1+1) EA (1$/n$)&54971&26688&15684&8452&6567&3815&3458&3371\\
(1+1) EA (2$/n$)&26104&9724&6508&4503&4020&3171&3141&3133\\
pmut (3.25)&40934&19972&10977&5644&4406&2434&2162&2172\\
pmut (3.0)&35272&17545&10040&5222&4150&2510&2208&2213\\
\end{tabular}



$pmut_{1.75}$ is not only the best variant for the bigger values of n but also for smaller inputs as well.
It is the least likely to be stuck in a local optimum, and it is also the fastest if it reaches a global optimum.

\section{OneMax Equivalent for PARTITION}
The input of this subsection is more or less equivalent to the OneMax problem. All values except the last are either 1 or follow any distribution. The last value is the sum of all other values. The optimal solution is therefore the 000\dots01 or
the 111\dots01 string. So the best solution is almost identical to OneMax/ZeroMax depending on the value of the last bit.\newline
For OneMax the mutation rate of 1/n is proven to be optimal for the (1+1) EA~\cite{witt2013tight}.
This should also hold for this input.
The RLS variants should also perform worse than the standard RLS.
The higher the value for $\beta$ the better the $pmut_\beta$ mutation should perform.
Flips of the first bits could decrease the runtime, depending on how often they happen.
By doing some testing with various algorithm variants of the RLS and the (1+1) EA it looked like the last bit was only flipped at most once for every input.
There was only one case where it was flipped twice, but it was never flipped more than twice per run.
The average number of flips was also mostly closer to zero than to one.\newline
The experiments where conducted with the variant where the smaller values are all one.
So for every run of each algorithm the input was $[1, 1, \dots, 1, n-1]$.
An input like this takes less time for every algorithm, but the results are mostly the same.
For some experiments not all 1000 repetitions were executed as there was a clear tendency which of the algorithms performs better.
\subsection{RLS Comparison}


\makebox[\linewidth]{
\begin{tabular}{lp{3cm}p{6cm}p{6cm}}
\begin{tabular}[h]{cccccccc}
algo type&            RLS&   \RLSR[s]&   \RLSR[s]&   \RLSR[s]&   \RLSN[b]&   \RLSN[b]&   \RLSN[b]\\
algo param&             -&     s=2&     s=3&     s=4&     b=3&     b=2&     b=4\\
avg mut/change&     1.000&   1.181&   1.688&   1.865&   3.000&   1.997&   3,997\\
avg mut/step&       1.000&   1.500&   2.000&   2.500&   3.000&   2.000&   3.000\\
\hline
total avg count&   90,931& 168,311& 236,317& 307,533& 921,030& 921,030& 921,030\\
avg eval count&    90,931& 168,311& 236,317& 307,533&       -&       -&       -\\
max eval count&   156,854& 296,206& 498,474& 595,831&       -&       -&       -\\
min eval count&    64,941& 120,582& 158,304& 212,193&       -&       -&       -\\
\hline
fail ratio&         0.000&   0.000&   0.000&   0.000&   1.000&   1.000&   1.000\\
avg fail dif&           -&       -&       -&       -&      36&      53&     263\\
\end{tabular}
\end{tabular}
}


As expected the standard RLS reaches an optimal solution the fastest.
It also reaches the optimal value for every instance.
The \RLSR~variants need more iterations to find an optimal solution.
By looking at the average values more closely it seems like the average number of steps for the \RLSR[k] is roughly $25,000 + 70,000k \pm 5,000$.
The standard RLS is equivalent to \RLSR~or \RLSN~with $k=1$.
So the value of $k=1$ seems to be optimal for the RLS variants too.
The \RLSN~variants on the other hand do not reach any of the two optimal solutions in any run.
This is most likely caused by their very low possibility of flipping only one bit in a single step.
They would eventually reach the optimal solution as well, but this would take much longer than for the RLS.\
The probability of flipping only one bit in a step is $\mathcal{O}(n^{1-k})$ which results in a single bit flip every $\mathcal{O}(n^{k-1})$ steps in expectation.
Because the fitness can only improve for OneMax making steps flipping more bits does not harm the fitness.
The bound for OneMax is $\mathcal{O}(n\log n)$ and with the previous result the expected number of steps is bounded by
$\mathcal{O}(n\cdot\mathcal{O}(n^{k-1})\cdot \log(n\cdot\mathcal{O}(n^{k-1}))) 
=\mathcal{O}(n^{k-1+1}\cdot (k-1+1)\cdot\log(n))
=\mathcal{O}(kn^{k}\cdot\log(n))$
This problem is not equivalent to OneMax, as a flip of the bit with the highest value inverts the fitness function to ZeroMax but the result might still hold as the bound for the standard RLS for this input is the same as for the RLS on OneMax.
\subsection{(1+1) EA Comparison}


\makebox[\linewidth]{
\begin{tabular}{lp{3cm}p{6cm}p{6cm}}
\begin{tabular}[h]{ccccccccc}
algo type&           EA-SM&       EA-SM&    EA-SM&    EA-SM&    EA-SM&    EA-SM&    EA-SM&    EA-SM\\
algo param&            2$/n$&        -&      3$/n$&      4$/n$&      5$/n$&     10$/n$&     50$/n$&    100$/n$\\
avg mut/change&      2.246&    1.551&    3.048&    3.936&    4.861&    9.822&   49.750&   99.707\\
avg mut/step&        2.000&    1.000&    3.000&    4.000&    5.000&   10.000&   50.000&  100.001\\
\hline
avg eval count&      3,097&    3,505&    3,518&    4,009&    4,807&    7,758&   18,457&   25,993\\
max eval count&     39,490&   60,533&   39,048&   47,881&   56,204&   91,305&  173,851&  354,479\\
min eval count&         10&        0&        6&        5&        3&        5&        9&        3\\
\hline
fail ratio&          0.000&    0.000&    0.000&    0.000&    0.000&    0.000&    0.000&    0.000\\
\end{tabular}
\end{tabular}
}


This experiment was terminated after 224 runs of the algorithms, as the results were already clear enough.
For this input the same as for OneMax holds. 
The static mutation rate $p_m=1/n$ is the optimal value and the performance of the (1+1) EA decreases with rising mutation rate.
Only for $p_m\le3/n$ the (1+1) EA managed to find one of the two optimal solutions in $10 \cdot n\ln(n)$ steps every time.
With mutation rate $p_m=4/n$ the (1+1) EA only managed to find the optimal solution in about 55 \% of the inputs.
The remaining mutation rates did not manage to find an optimal solution in any of the runs.
Another interesting fact is the average number of bits flipped in a successful step.
For the other inputs the overall average number of bits flipped in any step was mostly the same as for the average value of the successful steps. Here this is not the case.
All mutation rates flipped fewer bits in the successful steps than in the average step.
The only exception is the standard mutation rate which is caused by the steps where the algorithm would flip no bit.
Those steps decrease the number of the average case but not of the successful case as those steps were skipped.
\subsection{pmut Comparison}

\makebox[\linewidth]{
\scriptsize
\begin{tabular}{lp{3cm}p{6cm}p{6cm}}
\begin{tabular}[h]{m{2.5cm}m{0,40cm}m{0,40cm}m{0,40cm}m{0,40cm}m{0,40cm}m{0,40cm}m{0,40cm}m{0,40cm}m{0,40cm}m{0,40cm}m{0,40cm}m{0,40cm}m{0,40cm}m{0,40cm}m{0,40cm}m{0,40cm}m{0,40cm}m{0,40cm}}
\multicolumn{1}{c}{algo type}&\multicolumn{2}{c}{            pmut}&\multicolumn{2}{c}{     pmut}&\multicolumn{2}{c}{     pmut}&\multicolumn{2}{c}{     pmut}&\multicolumn{2}{c}{     pmut}&\multicolumn{2}{c}{     pmut}&\multicolumn{2}{c}{     pmut}&\multicolumn{2}{c}{     pmut}&\multicolumn{2}{c}{     pmut}\\
\multicolumn{1}{c}{algo param}&\multicolumn{2}{c}{           1.25}&\multicolumn{2}{c}{     1.50}&\multicolumn{2}{c}{     1.75}&\multicolumn{2}{c}{     2.00}&\multicolumn{2}{c}{     2.25}&\multicolumn{2}{c}{     2.50}&\multicolumn{2}{c}{     2.75}&\multicolumn{2}{c}{     3.00}&\multicolumn{2}{c}{     3.25}\\
\multicolumn{1}{c}{avg mut/change}&\multicolumn{2}{c}{    197.409}&\multicolumn{2}{c}{   70.534}&\multicolumn{2}{c}{   23.050}&\multicolumn{2}{c}{    8.724}&\multicolumn{2}{c}{    4.351}&\multicolumn{2}{c}{    2.777}&\multicolumn{2}{c}{    2.111}&\multicolumn{2}{c}{    1.770}&\multicolumn{2}{c}{    1.563}\\
\multicolumn{1}{c}{avg mut/step}&\multicolumn{2}{c}{      224.442}&\multicolumn{2}{c}{   70.480}&\multicolumn{2}{c}{   22.299}&\multicolumn{2}{c}{    8.470}&\multicolumn{2}{c}{    4.368}&\multicolumn{2}{c}{    2.906}&\multicolumn{2}{c}{    2.271}&\multicolumn{2}{c}{    1.934}&\multicolumn{2}{c}{    1.729}\\
\hline
\multicolumn{1}{c}{total avg count}&\multicolumn{2}{c}{        42}&\multicolumn{2}{c}{       87}&\multicolumn{2}{c}{      216}&\multicolumn{2}{c}{      503}&\multicolumn{2}{c}{    1,094}&\multicolumn{2}{c}{    4,063}&\multicolumn{2}{c}{   10,961}&\multicolumn{2}{c}{   18,727}&\multicolumn{2}{c}{   27,644}\\
\multicolumn{1}{c}{avg eval count}&\multicolumn{2}{c}{         42}&\multicolumn{2}{c}{       87}&\multicolumn{2}{c}{      216}&\multicolumn{2}{c}{      503}&\multicolumn{2}{c}{      910}&\multicolumn{2}{c}{    1,488}&\multicolumn{2}{c}{    1,676}&\multicolumn{2}{c}{    1,814}&\multicolumn{2}{c}{    1,909}\\
\multicolumn{1}{c}{max eval count}&\multicolumn{2}{c}{        303}&\multicolumn{2}{c}{      867}&\multicolumn{2}{c}{    2,843}&\multicolumn{2}{c}{    6,230}&\multicolumn{2}{c}{   10,487}&\multicolumn{2}{c}{  916,298}&\multicolumn{2}{c}{  501,346}&\multicolumn{2}{c}{  411,742}&\multicolumn{2}{c}{   12,386}\\
\multicolumn{1}{c}{min eval count}&\multicolumn{2}{c}{          0}&\multicolumn{2}{c}{        0}&\multicolumn{2}{c}{        0}&\multicolumn{2}{c}{        0}&\multicolumn{2}{c}{        0}&\multicolumn{2}{c}{        0}&\multicolumn{2}{c}{        0}&\multicolumn{2}{c}{        0}&\multicolumn{2}{c}{        0}\\
\hline
\multicolumn{1}{c}{fail ratio}&\multicolumn{2}{c}{          0.000}&\multicolumn{2}{c}{    0.000}&\multicolumn{2}{c}{    0.000}&\multicolumn{2}{c}{    0.000}&\multicolumn{2}{c}{    0.000}&\multicolumn{2}{c}{    0.003}&\multicolumn{2}{c}{    0.010}&\multicolumn{2}{c}{    0.018}&\multicolumn{2}{c}{    0.028}\\
\multicolumn{1}{c}{avg fail dif}&\multicolumn{2}{c}{            -}&\multicolumn{2}{c}{        -}&\multicolumn{2}{c}{        -}&\multicolumn{2}{c}{        -}&\multicolumn{2}{c}{      345}&\multicolumn{2}{c}{      345}&\multicolumn{2}{c}{      345}&\multicolumn{2}{c}{      345}&\multicolumn{2}{c}{      345}\\
\hline
\multicolumn{1}{c}{p-value}&&\multicolumn{2}{c}{0.0000}&\multicolumn{2}{c}{0.0000}&\multicolumn{2}{c}{0.0000}&\multicolumn{2}{c}{0.0000}&\multicolumn{2}{c}{0.0000}&\multicolumn{2}{c}{0.0000}&\multicolumn{2}{c}{0.0034}&\multicolumn{2}{c}{0.0068}\\
&&&&&&&&&&&&&&&&&&\end{tabular}
\end{tabular}
}


The results for the $pmut$ operator are pretty similar to the results for the (1+1) EA and the RLS.
The parameter $\beta=3.25$ which flips the least bits on average finds the solution the fastest.
The other values for $\beta$ increase the time needed for finding one of the two optimums with decreasing value for $\beta$.
All variants find an optimum in every run except for $\beta=1.25$ which has a much higher value for the number of flipped bits per steps.
The average number for the number of bits flipped in a successful mutation is much lower than for the other inputs especially for the lower values for $\beta$.
For the binomial and geometric input the successful average was around 100 for $\beta=1.25$ but for the OneMax equivalent it was only at 5.

\subsection{Comparison of the best variants}

\makebox[\linewidth]{
\begin{tabular}{lp{3cm}p{6cm}p{6cm}}
\begin{tabular}[h]{cccc}
algo type&        \RLSN& (1+1) EA&  pmut\\
algo param&         b=2&   3$/n$&  2.25\\
avg mut/change&   2.000& 3.092& 3.965\\
avg mut/step&     2.000& 2.999& 4.339\\
\hline
total avg count&    302&   677&   691\\
avg eval count&     302&   677&   691\\
max eval count&   1,610& 6,404& 5,205\\
min eval count&       9&    33&    17\\
\hline
fail ratio&       0.000& 0.000& 0.000\\
\end{tabular}
\end{tabular}
}


The results for this experiment are as expected.
All three algorithms find the optimal value within the time limit.
The RLS performs better than the (1+1) EA because it does only single bit flips.
The $pmut_{3.25}$ performs better than the standard (1+1) EA although flipping more bits on average.
This is most likely cause by the few steps where $pmut$ flips many bits which increase the average.
But $pmut$ most likely chooses to flip only one bit more often as the (1+1) EA.\newline
For this comparison neither of the algorithms failed to find one of the two optimal solutions.
The following table lists the amount of iterations the algorithms needed to find an optimal solution.

\begin{tabular}[h]{ccccccccc}
avg&20&50&100&500&1000&5000&10000&50000\\\hline
RLS-N (2)&172&1247&7598&38282&36214&105152&117792&125415\\
RLS-N (4)&8626&40966&43921&41924&42289&138187&201183&214170\\
(1+1) EA (3/n)&36470&42555&42439&43801&45530&145209&212281&252713\\
(1+1) EA (4/n)&37212&43144&44340&47246&46689&143697&220208&265620\\
pmut (-2.0)&40368&42601&39603&47443&46591&148390&232398&311606\\
pmut (-2.25)&39476&42113&40325&40548&43691&157816&226129&292367\\
\end{tabular}


The RLS performs the best closely follow by both $pmut$ variants.
The standard (1+1) EA performs a bit worse than the other three algorithms and approaches $en\ln(n)$ instead of staying close to $n\ln(n)$.

\begin{figure}[h]
      \caption{Runtime for OneMax equivalent with a $n\ln(n)$ scale}
      \centering
      \includegraphics[width=0.7\textwidth]{figures/images/oneMaxMultipleN.png}\label{fig:onemaxNlogNBound}
\end{figure}

In a previous chapter the $\mathcal{O}(n\log n)$ bound was proven for the (1+1) EA and the RLS (Theorem~\ref{theo:OneMaxResult}).
This seems to hold in practice at least for the easiest version of this input where the small values are one (see figure~\ref{fig:onemaxNlogNBound}).
Another variant of this input are uniform distributed inputs for example.
The small values in this case are chosen from \textasciitilde$U(1,49999)$ and the last value again is the sum of all other values.
This input is harder because switching multiple small values for a big value increases the fitness but increases the Hamming distance to the optimum.

\begin{figure}[h]
      \caption{Runtime for OneMax equivalent with uniform distribution on a $n\ln(n)$ scale}
      \centering
      \includegraphics[width=0.7\textwidth]{figures/images/oneMaxUniformMultipleN.png}\label{fig:onemaxUniformNlogNBound}
\end{figure}

Looking at figure~\ref{fig:onemaxUniformNlogNBound} the uniform distributed variant of the OneMax equivalent looks not much harder compared to the variant where all values are one.
The graphs are almost identical.
It was clear for the RLS because the RLS can't switch elements and there behaves exactly the same.
But even the other algorithms that are able to switch seem to not do it very often as the runtime is almost the same.
\section{Carsten Witt's worst case input}
This input is the worst case input from C. Witt in~\cite{witt2005worst} as discussed in the background section.
As all experimentally researched inputs in this paper contained only integer values this input is adjusted a bit.
To prevent the small values to be below zero they are instead normalised to 1.
The two big values are scaled by the same factor of ${((1/3+\epsilon/2)/(n-2))}^{-1}$.
The higher the value for $\epsilon$ the more likely the input is to get stuck in the local optima.
With increasing $\epsilon$ the local optima becomes less bad.
For the small values of $\epsilon$ there were only a few cases where some algorithms did not find an optimal solution.
To make this effect more visible the value of $\epsilon$ was set to $\epsilon=0.3$.\newline
For $n=10,000$ this evaluates to $w_1=w_2=5344$ and $W=9998 \cdot 1 + 2 \cdot 5344 = 20686$.
The input then looks like this: $[5344, 5344, 1, 1, \dots, 1, 1]$.
The fitness of the local optimum is $f(x) = 2 \cdot 5433 = 10688$.
To leave the local optimum the algorithm therefore has to flip at least  $5433+9998-10688 = 4654$ bits as well in the same step.
The best fitness is $f(x) = 5344 + 9998/2 = 10343$, which leads to a difference of $f(localOptimum)-f(opt) = 345$ and a approximation ratio of $f(localOptimum)/f(opt)=10688/10343=1.033$.
This is not really close to the worst case of 4/3 any more but with this setting at least many algorithms are stuck in the local optimum at least once for the 10000 runs.
\subsection{RLS Comparison}
\makebox[\linewidth]{
\begin{tabular}{lp{3cm}p{6cm}p{6cm}}
\begin{tabular}[h]{cccccccc}
algo type&            RLS&   \RLSR[s]&   \RLSR[s]&   \RLSR[s]&   \RLSN[b]&   \RLSN[b]&   \RLSN[b]\\
algo param&             -&     s=2&     s=3&     s=4&     b=3&     b=2&     b=4\\
avg mut/change&     1.000&   1.181&   1.688&   1.865&   3.000&   1.997&   3,997\\
avg mut/step&       1.000&   1.500&   2.000&   2.500&   3.000&   2.000&   3.000\\
\hline
total avg count&   90,931& 168,311& 236,317& 307,533& 921,030& 921,030& 921,030\\
avg eval count&    90,931& 168,311& 236,317& 307,533&       -&       -&       -\\
max eval count&   156,854& 296,206& 498,474& 595,831&       -&       -&       -\\
min eval count&    64,941& 120,582& 158,304& 212,193&       -&       -&       -\\
\hline
fail ratio&         0.000&   0.000&   0.000&   0.000&   1.000&   1.000&   1.000\\
avg fail dif&           -&       -&       -&       -&      36&      53&     263\\
\end{tabular}
\end{tabular}
}


The RLS is by far most likely to get stuck in the local optimum.
The general tendency is the more bits the algorithm can flip the more unlikely the local optimum becomes.
The only case where this does not hold is the \RLSN[2] being better than the \RLSR[3], although the \RLSR[3] can also flip 3 bits.
All \RLSN[s] variants had runs where they neither found the global nor one of the two local optima.
The algorithms were most likely tricked into the direction of the local optimum and did not manage to leave it.
But they were also not fast enough to reach the local optimum because of their low probability to flip only one bit.
\subsection{(1+1) EA Comparison}
\makebox[\linewidth]{
\begin{tabular}{lp{3cm}p{6cm}p{6cm}}
\begin{tabular}[h]{ccccccccc}
algo type&           EA-SM&       EA-SM&    EA-SM&    EA-SM&    EA-SM&    EA-SM&    EA-SM&    EA-SM\\
algo param&            2$/n$&        -&      3$/n$&      4$/n$&      5$/n$&     10$/n$&     50$/n$&    100$/n$\\
avg mut/change&      2.246&    1.551&    3.048&    3.936&    4.861&    9.822&   49.750&   99.707\\
avg mut/step&        2.000&    1.000&    3.000&    4.000&    5.000&   10.000&   50.000&  100.001\\
\hline
avg eval count&      3,097&    3,505&    3,518&    4,009&    4,807&    7,758&   18,457&   25,993\\
max eval count&     39,490&   60,533&   39,048&   47,881&   56,204&   91,305&  173,851&  354,479\\
min eval count&         10&        0&        6&        5&        3&        5&        9&        3\\
\hline
fail ratio&          0.000&    0.000&    0.000&    0.000&    0.000&    0.000&    0.000&    0.000\\
\end{tabular}
\end{tabular}
}


For the (1+1) EA the result is the inversion of the results for the OneMax equivalent.
The higher the mutation rate the faster the algorithm reaches a global optimum.
This holds at least up to $p_m\le100/n$.
With mutation rate $p_m\le4/n$ the algorithm reaches the worst case at least once in 10000 runs.
If the algorithm did not manage to find an optimal solution the fitness was always the same.
So there was no run where any algorithm neither found a global nor the local optimum.
\subsection{pmut Comparison}
\makebox[\linewidth]{
\scriptsize
\begin{tabular}{lp{3cm}p{6cm}p{6cm}}
\begin{tabular}[h]{m{2.5cm}m{0,40cm}m{0,40cm}m{0,40cm}m{0,40cm}m{0,40cm}m{0,40cm}m{0,40cm}m{0,40cm}m{0,40cm}m{0,40cm}m{0,40cm}m{0,40cm}m{0,40cm}m{0,40cm}m{0,40cm}m{0,40cm}m{0,40cm}m{0,40cm}}
\multicolumn{1}{c}{algo type}&\multicolumn{2}{c}{            pmut}&\multicolumn{2}{c}{     pmut}&\multicolumn{2}{c}{     pmut}&\multicolumn{2}{c}{     pmut}&\multicolumn{2}{c}{     pmut}&\multicolumn{2}{c}{     pmut}&\multicolumn{2}{c}{     pmut}&\multicolumn{2}{c}{     pmut}&\multicolumn{2}{c}{     pmut}\\
\multicolumn{1}{c}{algo param}&\multicolumn{2}{c}{           1.25}&\multicolumn{2}{c}{     1.50}&\multicolumn{2}{c}{     1.75}&\multicolumn{2}{c}{     2.00}&\multicolumn{2}{c}{     2.25}&\multicolumn{2}{c}{     2.50}&\multicolumn{2}{c}{     2.75}&\multicolumn{2}{c}{     3.00}&\multicolumn{2}{c}{     3.25}\\
\multicolumn{1}{c}{avg mut/change}&\multicolumn{2}{c}{    197.409}&\multicolumn{2}{c}{   70.534}&\multicolumn{2}{c}{   23.050}&\multicolumn{2}{c}{    8.724}&\multicolumn{2}{c}{    4.351}&\multicolumn{2}{c}{    2.777}&\multicolumn{2}{c}{    2.111}&\multicolumn{2}{c}{    1.770}&\multicolumn{2}{c}{    1.563}\\
\multicolumn{1}{c}{avg mut/step}&\multicolumn{2}{c}{      224.442}&\multicolumn{2}{c}{   70.480}&\multicolumn{2}{c}{   22.299}&\multicolumn{2}{c}{    8.470}&\multicolumn{2}{c}{    4.368}&\multicolumn{2}{c}{    2.906}&\multicolumn{2}{c}{    2.271}&\multicolumn{2}{c}{    1.934}&\multicolumn{2}{c}{    1.729}\\
\hline
\multicolumn{1}{c}{total avg count}&\multicolumn{2}{c}{        42}&\multicolumn{2}{c}{       87}&\multicolumn{2}{c}{      216}&\multicolumn{2}{c}{      503}&\multicolumn{2}{c}{    1,094}&\multicolumn{2}{c}{    4,063}&\multicolumn{2}{c}{   10,961}&\multicolumn{2}{c}{   18,727}&\multicolumn{2}{c}{   27,644}\\
\multicolumn{1}{c}{avg eval count}&\multicolumn{2}{c}{         42}&\multicolumn{2}{c}{       87}&\multicolumn{2}{c}{      216}&\multicolumn{2}{c}{      503}&\multicolumn{2}{c}{      910}&\multicolumn{2}{c}{    1,488}&\multicolumn{2}{c}{    1,676}&\multicolumn{2}{c}{    1,814}&\multicolumn{2}{c}{    1,909}\\
\multicolumn{1}{c}{max eval count}&\multicolumn{2}{c}{        303}&\multicolumn{2}{c}{      867}&\multicolumn{2}{c}{    2,843}&\multicolumn{2}{c}{    6,230}&\multicolumn{2}{c}{   10,487}&\multicolumn{2}{c}{  916,298}&\multicolumn{2}{c}{  501,346}&\multicolumn{2}{c}{  411,742}&\multicolumn{2}{c}{   12,386}\\
\multicolumn{1}{c}{min eval count}&\multicolumn{2}{c}{          0}&\multicolumn{2}{c}{        0}&\multicolumn{2}{c}{        0}&\multicolumn{2}{c}{        0}&\multicolumn{2}{c}{        0}&\multicolumn{2}{c}{        0}&\multicolumn{2}{c}{        0}&\multicolumn{2}{c}{        0}&\multicolumn{2}{c}{        0}\\
\hline
\multicolumn{1}{c}{fail ratio}&\multicolumn{2}{c}{          0.000}&\multicolumn{2}{c}{    0.000}&\multicolumn{2}{c}{    0.000}&\multicolumn{2}{c}{    0.000}&\multicolumn{2}{c}{    0.000}&\multicolumn{2}{c}{    0.003}&\multicolumn{2}{c}{    0.010}&\multicolumn{2}{c}{    0.018}&\multicolumn{2}{c}{    0.028}\\
\multicolumn{1}{c}{avg fail dif}&\multicolumn{2}{c}{            -}&\multicolumn{2}{c}{        -}&\multicolumn{2}{c}{        -}&\multicolumn{2}{c}{        -}&\multicolumn{2}{c}{      345}&\multicolumn{2}{c}{      345}&\multicolumn{2}{c}{      345}&\multicolumn{2}{c}{      345}&\multicolumn{2}{c}{      345}\\
\hline
\multicolumn{1}{c}{p-value}&&\multicolumn{2}{c}{0.0000}&\multicolumn{2}{c}{0.0000}&\multicolumn{2}{c}{0.0000}&\multicolumn{2}{c}{0.0000}&\multicolumn{2}{c}{0.0000}&\multicolumn{2}{c}{0.0000}&\multicolumn{2}{c}{0.0034}&\multicolumn{2}{c}{0.0068}\\
&&&&&&&&&&&&&&&&&&\end{tabular}
\end{tabular}
}


For $pmut$ the result is the exact same as for the (1+1) EA.\ 
The lower $\beta$ the better the performance as more bits are flipped in each step.
For the OneMax input $pmut_{1.25}$ flipped 224 bits on average per step, but the average of the successful steps was only 5.
Here the average of all steps is again 224, but the average of the successful steps is at 194.
The heavy tail here really increases the performance as most of the high values are accepted.
The algorithm is tricked into the local optima only for $\beta\le2.25$.
If the algorithm is on the path to the local optimum it is always fast enough to reach it within the time limit.
\subsection{Comparison of the best variants}
% \makebox[\linewidth]{
\begin{tabular}{lp{3cm}p{6cm}p{6cm}}
\begin{tabular}[h]{cccc}
algo type&        \RLSN& (1+1) EA&  pmut\\
algo param&         b=2&   3$/n$&  2.25\\
avg mut/change&   2.000& 3.092& 3.965\\
avg mut/step&     2.000& 2.999& 4.339\\
\hline
total avg count&    302&   677&   691\\
avg eval count&     302&   677&   691\\
max eval count&   1,610& 6,404& 5,205\\
min eval count&       9&    33&    17\\
\hline
fail ratio&       0.000& 0.000& 0.000\\
\end{tabular}
\end{tabular}
}

% 
The $pmut_{1.25}$ and the (1+1) EA with $p_m=100/n$ perform the best and always find an optimal solution within 700 iterations and even under 100 on the average case.
The \RLSN[4] performs significantly worse.
% The other algorithm flip so many bits that they are almost close to random sampling.
In the experiment with different input sizes the mutation rate of $p_m=100/n$ is $\ge1$ for $n\le100$.
If the algorithm flips every bit then it won't change its solution.
In these cases the mutation rate was then set to $p_m=1/2$.

\begin{tabular}[h]{cccccccc}
fails&20&50&100&500&1000&5000&10000\\\hline
RLS-N (2)&998&984&973&763&541&49&1\\
RLS-N (4)&998&989&975&837&719&150&27\\
(1+1) EA (3/n)&994&990&981&862&714&194&35\\
(1+1) EA (4/n)&998&993&976&852&725&180&51\\
pmut (-2.0)&997&989&983&905&784&229&55\\
pmut (-2.25)&997&988&987&875&790&213&56\\
\end{tabular}


Only the RLS variant had runs where it did not reach a global optimum.
This happened in less than 0.4 \% of the inputs for $n\le100$ and even in less than 0.1 \% for the remaining input sizes.

\begin{tabular}[h]{ccccccccc}
avg&20&50&100&500&1000&5000&10000&50000\\\hline
RLS-N (2)&172&1247&7598&38282&36214&105152&117792&125415\\
RLS-N (4)&8626&40966&43921&41924&42289&138187&201183&214170\\
(1+1) EA (3/n)&36470&42555&42439&43801&45530&145209&212281&252713\\
(1+1) EA (4/n)&37212&43144&44340&47246&46689&143697&220208&265620\\
pmut (-2.0)&40368&42601&39603&47443&46591&148390&232398&311606\\
pmut (-2.25)&39476&42113&40325&40548&43691&157816&226129&292367\\
\end{tabular}


For the lower input sizes the RLS is slower than the remaining algorithm even it manages to find a global optimum.

\begin{tabular}[h]{ccccccccc}
total avg&20&50&100&500&1000&5000&10000&50000\\\hline
RLS&98050&94574&89765&60172&39985&15639&4128&1797\\
\RLSR[2]&89401&54556&18266&4218&3530&2362&2160&2229\\
(1+1) EA (1$/n$)&54971&26688&15684&8452&6567&3815&3458&3371\\
(1+1) EA (2$/n$)&26104&9724&6508&4503&4020&3171&3141&3133\\
pmut (3.25)&40934&19972&10977&5644&4406&2434&2162&2172\\
pmut (3.0)&35272&17545&10040&5222&4150&2510&2208&2213\\
\end{tabular}



The $pmut_{1.25}$ is the best variant closely followed by the (1+1) EA.\
The RLS version is by far slower than the other to variants for the bigger input sizes.
Even for the smaller inputs it is still slower.

% \section{Multiple distributions mixed}
This input does not follow a specific distribution but rather is a mix of the previous distributions.
The value is either chosen uniform random, from a binomial distribution, from a geometric distribution  or a powerlaw distribution.
One of the distributions is chosen uniform randomly.
This process is repeated $n$ times.
The values of this distribution then follow neither of the used distributions.

\begin{figure}[h]
      \caption{Distribution of a mixed input with \textasciitilde$U(1,999)$, \textasciitilde$B(1000,0.1)$, \textasciitilde$Geo(0.01)$, $D^{-1.25}_{1000}$}
      \centering
      \includegraphics[width=0.7\textwidth]{figures/images/numberGenerator/mixed.png}\label{fig:mixedDistExample}
\end{figure}

A possible distribution is shown in Figure~\ref{fig:mixedDistExample}.
The spike in the curve is caused by the binomial distribution with an expected value of 100.
Each value occurs at least 2.5 times in expectation due to the uniform distribution.
The large spike for the lowest values is caused by both the geometric and the powerlaw distribution.
The parameter for this figure were lowered to improve the visibility of bigger values which occurs much less frequently than the small values.
With a larger span of possible values the big values would be even less visible.

The used distributions for the experiments in the next subsections were \textasciitilde$U(1,49999)$, \textasciitilde$B(10000,0.1)$, \textasciitilde$Geo(0.001)$, $D^{-1.25}_{50,000}$.
\subsection{RLS Comparison}


\makebox[\linewidth]{
\begin{tabular}{lp{3cm}p{6cm}p{6cm}}
\begin{tabular}[h]{cccccccc}
algo type&            RLS&   \RLSR[s]&   \RLSR[s]&   \RLSR[s]&   \RLSN[b]&   \RLSN[b]&   \RLSN[b]\\
algo param&             -&     s=2&     s=3&     s=4&     b=3&     b=2&     b=4\\
avg mut/change&     1.000&   1.181&   1.688&   1.865&   3.000&   1.997&   3,997\\
avg mut/step&       1.000&   1.500&   2.000&   2.500&   3.000&   2.000&   3.000\\
\hline
total avg count&   90,931& 168,311& 236,317& 307,533& 921,030& 921,030& 921,030\\
avg eval count&    90,931& 168,311& 236,317& 307,533&       -&       -&       -\\
max eval count&   156,854& 296,206& 498,474& 595,831&       -&       -&       -\\
min eval count&    64,941& 120,582& 158,304& 212,193&       -&       -&       -\\
\hline
fail ratio&         0.000&   0.000&   0.000&   0.000&   1.000&   1.000&   1.000\\
avg fail dif&           -&       -&       -&       -&      36&      53&     263\\
\end{tabular}
\end{tabular}
}


The results are mostly the same as for the geometric input.
The performance increases with the probability of flipping only one bit.
The main difference is the RLS being the best variant in this case as it reaches an optimal solution in every run.
The penalty of flipping only one bit is greater than for the geometric input but certainly not as drastic as for the OneMax equivalent.
By comparing the average number of iterations it seems like this input is easier than the geometric input.
\subsection{(1+1) EA Comparison}


\makebox[\linewidth]{
\begin{tabular}{lp{3cm}p{6cm}p{6cm}}
\begin{tabular}[h]{ccccccccc}
algo type&           EA-SM&       EA-SM&    EA-SM&    EA-SM&    EA-SM&    EA-SM&    EA-SM&    EA-SM\\
algo param&            2$/n$&        -&      3$/n$&      4$/n$&      5$/n$&     10$/n$&     50$/n$&    100$/n$\\
avg mut/change&      2.246&    1.551&    3.048&    3.936&    4.861&    9.822&   49.750&   99.707\\
avg mut/step&        2.000&    1.000&    3.000&    4.000&    5.000&   10.000&   50.000&  100.001\\
\hline
avg eval count&      3,097&    3,505&    3,518&    4,009&    4,807&    7,758&   18,457&   25,993\\
max eval count&     39,490&   60,533&   39,048&   47,881&   56,204&   91,305&  173,851&  354,479\\
min eval count&         10&        0&        6&        5&        3&        5&        9&        3\\
\hline
fail ratio&          0.000&    0.000&    0.000&    0.000&    0.000&    0.000&    0.000&    0.000\\
\end{tabular}
\end{tabular}
}


For the (1+1) EA the same holds.
It performs better with decreasing mutation rate with $p_m=2/n$ being the only exception again.
The same was true for the geometric input.
The penality for the wrong parameter is also bigger for the (1+1) EA compared to the geometric input.
So the results are rather similar to the RLS.
\subsection{pmut Comparison}


\makebox[\linewidth]{
\scriptsize
\begin{tabular}{lp{3cm}p{6cm}p{6cm}}
\begin{tabular}[h]{m{2.5cm}m{0,40cm}m{0,40cm}m{0,40cm}m{0,40cm}m{0,40cm}m{0,40cm}m{0,40cm}m{0,40cm}m{0,40cm}m{0,40cm}m{0,40cm}m{0,40cm}m{0,40cm}m{0,40cm}m{0,40cm}m{0,40cm}m{0,40cm}m{0,40cm}}
\multicolumn{1}{c}{algo type}&\multicolumn{2}{c}{            pmut}&\multicolumn{2}{c}{     pmut}&\multicolumn{2}{c}{     pmut}&\multicolumn{2}{c}{     pmut}&\multicolumn{2}{c}{     pmut}&\multicolumn{2}{c}{     pmut}&\multicolumn{2}{c}{     pmut}&\multicolumn{2}{c}{     pmut}&\multicolumn{2}{c}{     pmut}\\
\multicolumn{1}{c}{algo param}&\multicolumn{2}{c}{           1.25}&\multicolumn{2}{c}{     1.50}&\multicolumn{2}{c}{     1.75}&\multicolumn{2}{c}{     2.00}&\multicolumn{2}{c}{     2.25}&\multicolumn{2}{c}{     2.50}&\multicolumn{2}{c}{     2.75}&\multicolumn{2}{c}{     3.00}&\multicolumn{2}{c}{     3.25}\\
\multicolumn{1}{c}{avg mut/change}&\multicolumn{2}{c}{    197.409}&\multicolumn{2}{c}{   70.534}&\multicolumn{2}{c}{   23.050}&\multicolumn{2}{c}{    8.724}&\multicolumn{2}{c}{    4.351}&\multicolumn{2}{c}{    2.777}&\multicolumn{2}{c}{    2.111}&\multicolumn{2}{c}{    1.770}&\multicolumn{2}{c}{    1.563}\\
\multicolumn{1}{c}{avg mut/step}&\multicolumn{2}{c}{      224.442}&\multicolumn{2}{c}{   70.480}&\multicolumn{2}{c}{   22.299}&\multicolumn{2}{c}{    8.470}&\multicolumn{2}{c}{    4.368}&\multicolumn{2}{c}{    2.906}&\multicolumn{2}{c}{    2.271}&\multicolumn{2}{c}{    1.934}&\multicolumn{2}{c}{    1.729}\\
\hline
\multicolumn{1}{c}{total avg count}&\multicolumn{2}{c}{        42}&\multicolumn{2}{c}{       87}&\multicolumn{2}{c}{      216}&\multicolumn{2}{c}{      503}&\multicolumn{2}{c}{    1,094}&\multicolumn{2}{c}{    4,063}&\multicolumn{2}{c}{   10,961}&\multicolumn{2}{c}{   18,727}&\multicolumn{2}{c}{   27,644}\\
\multicolumn{1}{c}{avg eval count}&\multicolumn{2}{c}{         42}&\multicolumn{2}{c}{       87}&\multicolumn{2}{c}{      216}&\multicolumn{2}{c}{      503}&\multicolumn{2}{c}{      910}&\multicolumn{2}{c}{    1,488}&\multicolumn{2}{c}{    1,676}&\multicolumn{2}{c}{    1,814}&\multicolumn{2}{c}{    1,909}\\
\multicolumn{1}{c}{max eval count}&\multicolumn{2}{c}{        303}&\multicolumn{2}{c}{      867}&\multicolumn{2}{c}{    2,843}&\multicolumn{2}{c}{    6,230}&\multicolumn{2}{c}{   10,487}&\multicolumn{2}{c}{  916,298}&\multicolumn{2}{c}{  501,346}&\multicolumn{2}{c}{  411,742}&\multicolumn{2}{c}{   12,386}\\
\multicolumn{1}{c}{min eval count}&\multicolumn{2}{c}{          0}&\multicolumn{2}{c}{        0}&\multicolumn{2}{c}{        0}&\multicolumn{2}{c}{        0}&\multicolumn{2}{c}{        0}&\multicolumn{2}{c}{        0}&\multicolumn{2}{c}{        0}&\multicolumn{2}{c}{        0}&\multicolumn{2}{c}{        0}\\
\hline
\multicolumn{1}{c}{fail ratio}&\multicolumn{2}{c}{          0.000}&\multicolumn{2}{c}{    0.000}&\multicolumn{2}{c}{    0.000}&\multicolumn{2}{c}{    0.000}&\multicolumn{2}{c}{    0.000}&\multicolumn{2}{c}{    0.003}&\multicolumn{2}{c}{    0.010}&\multicolumn{2}{c}{    0.018}&\multicolumn{2}{c}{    0.028}\\
\multicolumn{1}{c}{avg fail dif}&\multicolumn{2}{c}{            -}&\multicolumn{2}{c}{        -}&\multicolumn{2}{c}{        -}&\multicolumn{2}{c}{        -}&\multicolumn{2}{c}{      345}&\multicolumn{2}{c}{      345}&\multicolumn{2}{c}{      345}&\multicolumn{2}{c}{      345}&\multicolumn{2}{c}{      345}\\
\hline
\multicolumn{1}{c}{p-value}&&\multicolumn{2}{c}{0.0000}&\multicolumn{2}{c}{0.0000}&\multicolumn{2}{c}{0.0000}&\multicolumn{2}{c}{0.0000}&\multicolumn{2}{c}{0.0000}&\multicolumn{2}{c}{0.0000}&\multicolumn{2}{c}{0.0034}&\multicolumn{2}{c}{0.0068}\\
&&&&&&&&&&&&&&&&&&\end{tabular}
\end{tabular}
}


$pmut_\beta$ also performs similar to the geometric input.
Here $\beta=-3.25$ and $\beta=-3.0$ are switched instead of $\beta = -2.5$ and $-3.75$.
Apart from that the algorithms perform strictly ranked by their probability to flip only one bit per step.
Solving the mixed input is easier for the $pmut$ operator too.
All variants manage to find an optimal solution within 1000 steps on average compared to 6,200 steps for the geometric input.
The maximum number of steps is also lower by a factor of at least 10 for every value of $\beta$.
\subsection{Comparison of the best variants}


\makebox[\linewidth]{
\begin{tabular}{lp{3cm}p{6cm}p{6cm}}
\begin{tabular}[h]{cccc}
algo type&        \RLSN& (1+1) EA&  pmut\\
algo param&         b=2&   3$/n$&  2.25\\
avg mut/change&   2.000& 3.092& 3.965\\
avg mut/step&     2.000& 2.999& 4.339\\
\hline
total avg count&    302&   677&   691\\
avg eval count&     302&   677&   691\\
max eval count&   1,610& 6,404& 5,205\\
min eval count&       9&    33&    17\\
\hline
fail ratio&       0.000& 0.000& 0.000\\
\end{tabular}
\end{tabular}
}


This input seems generally easy to solve as for every base algorithm multiple variants reach an optimal solution within 1000 steps.
The standard RLS reaches an optimal solution the fastest, but the other algorithms are almost equally fast.
All algorithms finish within 501 steps on average and always in less than 2500 steps.

\begin{tabular}[h]{cccccccc}
fails&20&50&100&500&1000&5000&10000\\\hline
RLS-N (2)&998&984&973&763&541&49&1\\
RLS-N (4)&998&989&975&837&719&150&27\\
(1+1) EA (3/n)&994&990&981&862&714&194&35\\
(1+1) EA (4/n)&998&993&976&852&725&180&51\\
pmut (-2.0)&997&989&983&905&784&229&55\\
pmut (-2.25)&997&988&987&875&790&213&56\\
\end{tabular}


This input is only hard to solve for $n<100$.
Only the RLS is stuck in a local optimum for many inputs.
The other algorithms manage to escape the local optima in most cases even for $n=100$.
For $n\ge500$ every input is solved by each of the chosen algorithms except for the standard RLS failing once for $n=500$.
This is probably caused by the many small values from the powerlaw and geometric distribution.
The longer the input the more of these small values can be used to fill the gaps.

\begin{tabular}[h]{ccccccccc}
avg&20&50&100&500&1000&5000&10000&50000\\\hline
RLS-N (2)&172&1247&7598&38282&36214&105152&117792&125415\\
RLS-N (4)&8626&40966&43921&41924&42289&138187&201183&214170\\
(1+1) EA (3/n)&36470&42555&42439&43801&45530&145209&212281&252713\\
(1+1) EA (4/n)&37212&43144&44340&47246&46689&143697&220208&265620\\
pmut (-2.0)&40368&42601&39603&47443&46591&148390&232398&311606\\
pmut (-2.25)&39476&42113&40325&40548&43691&157816&226129&292367\\
\end{tabular}


The RLS variants are only able to solve the inputs if they are lucky.
The (1+1) EAs and the $pmut$ variants find optimal solutions more often but need way more steps if they do so.
The input becomes drastically easier until $n=500$ and very slowly becomes harder with rising $n$ again.

\begin{tabular}[h]{ccccccccc}
total avg&20&50&100&500&1000&5000&10000&50000\\\hline
RLS&98050&94574&89765&60172&39985&15639&4128&1797\\
\RLSR[2]&89401&54556&18266&4218&3530&2362&2160&2229\\
(1+1) EA (1$/n$)&54971&26688&15684&8452&6567&3815&3458&3371\\
(1+1) EA (2$/n$)&26104&9724&6508&4503&4020&3171&3141&3133\\
pmut (3.25)&40934&19972&10977&5644&4406&2434&2162&2172\\
pmut (3.0)&35272&17545&10040&5222&4150&2510&2208&2213\\
\end{tabular}



For $n\le100$ the (1+1) EA with $p_m=2/n$ performs the best.
It is still good for the bigger values of $n$ but after $n\ge500$ $pmut_{-3.0}$ reaches an optimal solution faster.
The standard RLS is only the fastest for a small range of values but has the disadvantage of the poor performance for small inputs.
Choosing one of the other variants is a much safer choice.

% \section{Multiple distributions overlapped}
This input is similar to the mixed input from the last subsection.
For this distribution the values are not chosen uniform random from one of the base distributions, but instead a value from every distribution is chosen and added together.
Hence the name overlapped distribution.
For this input the step limit was increased to $100\cdot n \ln n$.
The comparison for the step limit $10\cdot n \ln n$ is contained in only for the best algorithms.

\begin{figure}[h]
      \caption{Distribution of an overlapped input with \textasciitilde$U(1,999)$, \textasciitilde$B(1000.0.1)$, \textasciitilde$Geo(0.01)$, powerlaw dist with $\beta=-1.25$}
      \centering
      \includegraphics[width=0.7\textwidth]{figures/images/numberGenerator/overlapped.png}\label{fig:overlappedDistExample}
\end{figure}

The used distributions were \textasciitilde$U(1,49999)$, \textasciitilde$B(10000.0.1)$, \textasciitilde$Geo(0.001)$, powerlaw dist with $\beta=-1.25$
\subsection{RLS Comparison}


\makebox[\linewidth]{
\begin{tabular}{lp{3cm}p{6cm}p{6cm}}
\begin{tabular}[h]{cccccccc}
algo type&            RLS&   \RLSR[s]&   \RLSR[s]&   \RLSR[s]&   \RLSN[b]&   \RLSN[b]&   \RLSN[b]\\
algo param&             -&     s=2&     s=3&     s=4&     b=3&     b=2&     b=4\\
avg mut/change&     1.000&   1.181&   1.688&   1.865&   3.000&   1.997&   3,997\\
avg mut/step&       1.000&   1.500&   2.000&   2.500&   3.000&   2.000&   3.000\\
\hline
total avg count&   90,931& 168,311& 236,317& 307,533& 921,030& 921,030& 921,030\\
avg eval count&    90,931& 168,311& 236,317& 307,533&       -&       -&       -\\
max eval count&   156,854& 296,206& 498,474& 595,831&       -&       -&       -\\
min eval count&    64,941& 120,582& 158,304& 212,193&       -&       -&       -\\
\hline
fail ratio&         0.000&   0.000&   0.000&   0.000&   1.000&   1.000&   1.000\\
avg fail dif&           -&       -&       -&       -&      36&      53&     263\\
\end{tabular}
\end{tabular}
}


The results for this distribution are rather similar to the results of the binomial distribution.
The ranking of the algorithms is completely the same.
Only the RLS almost fails to find an optimal solution for almost every input.
For the binomial distribution the $RLS-N_3$ also failed around 25\% or the inputs but here it does not.
Another big difference is the number of steps needed to find a global optimum.
For the binomial input all algorithms except the RLS were really fast and only needed below 1000 iterations on average.
All algorithm need around 500 times the number of steps on average as compared to the binomial inputs.
\subsection{(1+1) EA Comparison}


\makebox[\linewidth]{
\begin{tabular}{lp{3cm}p{6cm}p{6cm}}
\begin{tabular}[h]{ccccccccc}
algo type&           EA-SM&       EA-SM&    EA-SM&    EA-SM&    EA-SM&    EA-SM&    EA-SM&    EA-SM\\
algo param&            2$/n$&        -&      3$/n$&      4$/n$&      5$/n$&     10$/n$&     50$/n$&    100$/n$\\
avg mut/change&      2.246&    1.551&    3.048&    3.936&    4.861&    9.822&   49.750&   99.707\\
avg mut/step&        2.000&    1.000&    3.000&    4.000&    5.000&   10.000&   50.000&  100.001\\
\hline
avg eval count&      3,097&    3,505&    3,518&    4,009&    4,807&    7,758&   18,457&   25,993\\
max eval count&     39,490&   60,533&   39,048&   47,881&   56,204&   91,305&  173,851&  354,479\\
min eval count&         10&        0&        6&        5&        3&        5&        9&        3\\
\hline
fail ratio&          0.000&    0.000&    0.000&    0.000&    0.000&    0.000&    0.000&    0.000\\
\end{tabular}
\end{tabular}
}


The results for the (1+1) EA are similar but not the same.
Algorithms that had a good runtime on binomial inputs also have a good runtime in the overlapped runs, but the ranking is not the same.
For this input every the first and second place switched, the same for 3rd and 4th and also for 5th and 6th place.
This may be only caused by chance and the performance within a pair being really close but might also be caused by something else.
The (1+1) EA variants also needed about 200 times as long as for the binomial inputs.
All runs had at least 2 runs where they did not find an optimal solution.
These runs all had a remaining difference of one to the optimal value.
More time would probably be enough to reach an optimal solution in most cases.
\subsection{pmut Comparison}


\makebox[\linewidth]{
\scriptsize
\begin{tabular}{lp{3cm}p{6cm}p{6cm}}
\begin{tabular}[h]{m{2.5cm}m{0,40cm}m{0,40cm}m{0,40cm}m{0,40cm}m{0,40cm}m{0,40cm}m{0,40cm}m{0,40cm}m{0,40cm}m{0,40cm}m{0,40cm}m{0,40cm}m{0,40cm}m{0,40cm}m{0,40cm}m{0,40cm}m{0,40cm}m{0,40cm}}
\multicolumn{1}{c}{algo type}&\multicolumn{2}{c}{            pmut}&\multicolumn{2}{c}{     pmut}&\multicolumn{2}{c}{     pmut}&\multicolumn{2}{c}{     pmut}&\multicolumn{2}{c}{     pmut}&\multicolumn{2}{c}{     pmut}&\multicolumn{2}{c}{     pmut}&\multicolumn{2}{c}{     pmut}&\multicolumn{2}{c}{     pmut}\\
\multicolumn{1}{c}{algo param}&\multicolumn{2}{c}{           1.25}&\multicolumn{2}{c}{     1.50}&\multicolumn{2}{c}{     1.75}&\multicolumn{2}{c}{     2.00}&\multicolumn{2}{c}{     2.25}&\multicolumn{2}{c}{     2.50}&\multicolumn{2}{c}{     2.75}&\multicolumn{2}{c}{     3.00}&\multicolumn{2}{c}{     3.25}\\
\multicolumn{1}{c}{avg mut/change}&\multicolumn{2}{c}{    197.409}&\multicolumn{2}{c}{   70.534}&\multicolumn{2}{c}{   23.050}&\multicolumn{2}{c}{    8.724}&\multicolumn{2}{c}{    4.351}&\multicolumn{2}{c}{    2.777}&\multicolumn{2}{c}{    2.111}&\multicolumn{2}{c}{    1.770}&\multicolumn{2}{c}{    1.563}\\
\multicolumn{1}{c}{avg mut/step}&\multicolumn{2}{c}{      224.442}&\multicolumn{2}{c}{   70.480}&\multicolumn{2}{c}{   22.299}&\multicolumn{2}{c}{    8.470}&\multicolumn{2}{c}{    4.368}&\multicolumn{2}{c}{    2.906}&\multicolumn{2}{c}{    2.271}&\multicolumn{2}{c}{    1.934}&\multicolumn{2}{c}{    1.729}\\
\hline
\multicolumn{1}{c}{total avg count}&\multicolumn{2}{c}{        42}&\multicolumn{2}{c}{       87}&\multicolumn{2}{c}{      216}&\multicolumn{2}{c}{      503}&\multicolumn{2}{c}{    1,094}&\multicolumn{2}{c}{    4,063}&\multicolumn{2}{c}{   10,961}&\multicolumn{2}{c}{   18,727}&\multicolumn{2}{c}{   27,644}\\
\multicolumn{1}{c}{avg eval count}&\multicolumn{2}{c}{         42}&\multicolumn{2}{c}{       87}&\multicolumn{2}{c}{      216}&\multicolumn{2}{c}{      503}&\multicolumn{2}{c}{      910}&\multicolumn{2}{c}{    1,488}&\multicolumn{2}{c}{    1,676}&\multicolumn{2}{c}{    1,814}&\multicolumn{2}{c}{    1,909}\\
\multicolumn{1}{c}{max eval count}&\multicolumn{2}{c}{        303}&\multicolumn{2}{c}{      867}&\multicolumn{2}{c}{    2,843}&\multicolumn{2}{c}{    6,230}&\multicolumn{2}{c}{   10,487}&\multicolumn{2}{c}{  916,298}&\multicolumn{2}{c}{  501,346}&\multicolumn{2}{c}{  411,742}&\multicolumn{2}{c}{   12,386}\\
\multicolumn{1}{c}{min eval count}&\multicolumn{2}{c}{          0}&\multicolumn{2}{c}{        0}&\multicolumn{2}{c}{        0}&\multicolumn{2}{c}{        0}&\multicolumn{2}{c}{        0}&\multicolumn{2}{c}{        0}&\multicolumn{2}{c}{        0}&\multicolumn{2}{c}{        0}&\multicolumn{2}{c}{        0}\\
\hline
\multicolumn{1}{c}{fail ratio}&\multicolumn{2}{c}{          0.000}&\multicolumn{2}{c}{    0.000}&\multicolumn{2}{c}{    0.000}&\multicolumn{2}{c}{    0.000}&\multicolumn{2}{c}{    0.000}&\multicolumn{2}{c}{    0.003}&\multicolumn{2}{c}{    0.010}&\multicolumn{2}{c}{    0.018}&\multicolumn{2}{c}{    0.028}\\
\multicolumn{1}{c}{avg fail dif}&\multicolumn{2}{c}{            -}&\multicolumn{2}{c}{        -}&\multicolumn{2}{c}{        -}&\multicolumn{2}{c}{        -}&\multicolumn{2}{c}{      345}&\multicolumn{2}{c}{      345}&\multicolumn{2}{c}{      345}&\multicolumn{2}{c}{      345}&\multicolumn{2}{c}{      345}\\
\hline
\multicolumn{1}{c}{p-value}&&\multicolumn{2}{c}{0.0000}&\multicolumn{2}{c}{0.0000}&\multicolumn{2}{c}{0.0000}&\multicolumn{2}{c}{0.0000}&\multicolumn{2}{c}{0.0000}&\multicolumn{2}{c}{0.0000}&\multicolumn{2}{c}{0.0034}&\multicolumn{2}{c}{0.0068}\\
&&&&&&&&&&&&&&&&&&\end{tabular}
\end{tabular}
}


For $pmut$ the results are mostly the same as for the (1+1) EA.
The performance is about 250 times worse than for the binomial inputs, but the ranking is still pretty close.
Here the optimal parameter is around $\beta=-1.75$ instead of $\beta=-2.25$.
For $pmut$ the step limit was too small as well, but the difference to the optimal solution was only one on average.
\subsection{Comparison of the best variants}


\makebox[\linewidth]{
\begin{tabular}{lp{3cm}p{6cm}p{6cm}}
\begin{tabular}[h]{cccc}
algo type&        \RLSN& (1+1) EA&  pmut\\
algo param&         b=2&   3$/n$&  2.25\\
avg mut/change&   2.000& 3.092& 3.965\\
avg mut/step&     2.000& 2.999& 4.339\\
\hline
total avg count&    302&   677&   691\\
avg eval count&     302&   677&   691\\
max eval count&   1,610& 6,404& 5,205\\
min eval count&       9&    33&    17\\
\hline
fail ratio&       0.000& 0.000& 0.000\\
\end{tabular}
\end{tabular}
}


The results here are the same as for the binomial input. The $RLS-N_2$ performs better than the (1+1) EA and $pmut_\beta$ mutation for all values of $c/n$ and $\beta$ by a factor of 1.5 with a step limit of $10 \cdot n \ln(n)$. For the lower values of $n$ this does not hold to that extreme.

\begin{tabular}[h]{cccccccc}
fails&20&50&100&500&1000&5000&10000\\\hline
RLS-N (2)&998&984&973&763&541&49&1\\
RLS-N (4)&998&989&975&837&719&150&27\\
(1+1) EA (3/n)&994&990&981&862&714&194&35\\
(1+1) EA (4/n)&998&993&976&852&725&180&51\\
pmut (-2.0)&997&989&983&905&784&229&55\\
pmut (-2.25)&997&988&987&875&790&213&56\\
\end{tabular}


Here almost no algorithm manages to find an optimal solution in most cases as all algorithms fail for more than 80 \% of the inputs.
The RLS variants perform the worst again for the small input sizes.
Only for $n\ge500$ the $RLS-N_{2}$ is a good option due to the hug performance increase between $n=100$ and $n=500$.
Overlapped inputs stay hard to solve relatively long as only for $n\ge50,000$ all best variants of each algorithm find an optimal solution for every input.

\begin{tabular}[h]{ccccccccc}
avg&20&50&100&500&1000&5000&10000&50000\\\hline
RLS-N (2)&172&1247&7598&38282&36214&105152&117792&125415\\
RLS-N (4)&8626&40966&43921&41924&42289&138187&201183&214170\\
(1+1) EA (3/n)&36470&42555&42439&43801&45530&145209&212281&252713\\
(1+1) EA (4/n)&37212&43144&44340&47246&46689&143697&220208&265620\\
pmut (-2.0)&40368&42601&39603&47443&46591&148390&232398&311606\\
pmut (-2.25)&39476&42113&40325&40548&43691&157816&226129&292367\\
\end{tabular}


For $n\le1000$ the performance seems almost constant, but this is caused by the constant step limit of 100,000.
After $n=1000$ the value of $10 \cdot n \ln(n)$ was bigger than 100,000.

\begin{tabular}[h]{ccccccccc}
total avg&20&50&100&500&1000&5000&10000&50000\\\hline
RLS&98050&94574&89765&60172&39985&15639&4128&1797\\
\RLSR[2]&89401&54556&18266&4218&3530&2362&2160&2229\\
(1+1) EA (1$/n$)&54971&26688&15684&8452&6567&3815&3458&3371\\
(1+1) EA (2$/n$)&26104&9724&6508&4503&4020&3171&3141&3133\\
pmut (3.25)&40934&19972&10977&5644&4406&2434&2162&2172\\
pmut (3.0)&35272&17545&10040&5222&4150&2510&2208&2213\\
\end{tabular}



No algorithm is very successful for the small values of N.
Choosing the (1+1) EA for $n<500$ should result in the best runtime in most of the cases.
Only the $RLS-N_4$ is faster for $50\le n \le 100$.
For bigger input sizes the $RLS-N_2$ is clearly the best option.

\section{Multiple distributions mixed \& overlapped}
This distribution is again similar to the mixed distribution.
With probability 1/2 the value is chosen from the overlapped distribution and with remaining probability 1/2 the value is chosen from the mixed distribution.

\begin{figure}[h]
      \caption{Distribution of a mixed and overlapped input with \textasciitilde$U(1,999)$, \textasciitilde$B(1000,0.1)$, \textasciitilde$Geo(0.01)$, \textasciitilde$D^{1.25}_{1000}$}
      \centering
      \includegraphics[width=0.7\textwidth]{figures/images/numberGenerator/mixedAndOverlapped.png}\label{fig:mixAndOverlDistExample}
\end{figure}

The used distributions were \textasciitilde$U(1,49999)$, \textasciitilde$B(10000,0.1)$, \textasciitilde$Geo(0.001)$, \textasciitilde$D^{1.25}_{50000}$.
By looking at Figure~\ref{fig:mixAndOverlDistExample} it looks like the mixed and overlapped distribution is closer to the mixed distribution than to the overlapped distribution.
The results should also be closer to the mixed distribution.
\subsection{RLS Comparison}


\makebox[\linewidth]{
\begin{tabular}{lp{3cm}p{6cm}p{6cm}}
\begin{tabular}[h]{cccccccc}
algo type&            RLS&   \RLSR[s]&   \RLSR[s]&   \RLSR[s]&   \RLSN[b]&   \RLSN[b]&   \RLSN[b]\\
algo param&             -&     s=2&     s=3&     s=4&     b=3&     b=2&     b=4\\
avg mut/change&     1.000&   1.181&   1.688&   1.865&   3.000&   1.997&   3,997\\
avg mut/step&       1.000&   1.500&   2.000&   2.500&   3.000&   2.000&   3.000\\
\hline
total avg count&   90,931& 168,311& 236,317& 307,533& 921,030& 921,030& 921,030\\
avg eval count&    90,931& 168,311& 236,317& 307,533&       -&       -&       -\\
max eval count&   156,854& 296,206& 498,474& 595,831&       -&       -&       -\\
min eval count&    64,941& 120,582& 158,304& 212,193&       -&       -&       -\\
\hline
fail ratio&         0.000&   0.000&   0.000&   0.000&   1.000&   1.000&   1.000\\
avg fail dif&           -&       -&       -&       -&      36&      53&     263\\
\end{tabular}
\end{tabular}
}


As expected the results are similar to the mixed input.
There is a clear preference of algorithms with higher probability to flip only one bit per step.
Mixed and geometric distributions did not lead to very bad performance for the \RLSN[4].
The mixed and overlapped input punishes the worse mutation operators more than the other two inputs.
In that sense this type of input is closer to the OneMax equivalent, but still way less extreme.
Here still every variant reaches an optimal solution in every case.
\subsection{(1+1) EA Comparison}


\makebox[\linewidth]{
\begin{tabular}{lp{3cm}p{6cm}p{6cm}}
\begin{tabular}[h]{ccccccccc}
algo type&           EA-SM&       EA-SM&    EA-SM&    EA-SM&    EA-SM&    EA-SM&    EA-SM&    EA-SM\\
algo param&            2$/n$&        -&      3$/n$&      4$/n$&      5$/n$&     10$/n$&     50$/n$&    100$/n$\\
avg mut/change&      2.246&    1.551&    3.048&    3.936&    4.861&    9.822&   49.750&   99.707\\
avg mut/step&        2.000&    1.000&    3.000&    4.000&    5.000&   10.000&   50.000&  100.001\\
\hline
avg eval count&      3,097&    3,505&    3,518&    4,009&    4,807&    7,758&   18,457&   25,993\\
max eval count&     39,490&   60,533&   39,048&   47,881&   56,204&   91,305&  173,851&  354,479\\
min eval count&         10&        0&        6&        5&        3&        5&        9&        3\\
\hline
fail ratio&          0.000&    0.000&    0.000&    0.000&    0.000&    0.000&    0.000&    0.000\\
\end{tabular}
\end{tabular}
}


The results here are pretty similar to the results of the RLS.
The lower mutation rates perform better and higher mutation rates start to perform worse in comparison to the mixed input.
\subsection{pmut Comparison}


\makebox[\linewidth]{
\scriptsize
\begin{tabular}{lp{3cm}p{6cm}p{6cm}}
\begin{tabular}[h]{m{2.5cm}m{0,40cm}m{0,40cm}m{0,40cm}m{0,40cm}m{0,40cm}m{0,40cm}m{0,40cm}m{0,40cm}m{0,40cm}m{0,40cm}m{0,40cm}m{0,40cm}m{0,40cm}m{0,40cm}m{0,40cm}m{0,40cm}m{0,40cm}m{0,40cm}}
\multicolumn{1}{c}{algo type}&\multicolumn{2}{c}{            pmut}&\multicolumn{2}{c}{     pmut}&\multicolumn{2}{c}{     pmut}&\multicolumn{2}{c}{     pmut}&\multicolumn{2}{c}{     pmut}&\multicolumn{2}{c}{     pmut}&\multicolumn{2}{c}{     pmut}&\multicolumn{2}{c}{     pmut}&\multicolumn{2}{c}{     pmut}\\
\multicolumn{1}{c}{algo param}&\multicolumn{2}{c}{           1.25}&\multicolumn{2}{c}{     1.50}&\multicolumn{2}{c}{     1.75}&\multicolumn{2}{c}{     2.00}&\multicolumn{2}{c}{     2.25}&\multicolumn{2}{c}{     2.50}&\multicolumn{2}{c}{     2.75}&\multicolumn{2}{c}{     3.00}&\multicolumn{2}{c}{     3.25}\\
\multicolumn{1}{c}{avg mut/change}&\multicolumn{2}{c}{    197.409}&\multicolumn{2}{c}{   70.534}&\multicolumn{2}{c}{   23.050}&\multicolumn{2}{c}{    8.724}&\multicolumn{2}{c}{    4.351}&\multicolumn{2}{c}{    2.777}&\multicolumn{2}{c}{    2.111}&\multicolumn{2}{c}{    1.770}&\multicolumn{2}{c}{    1.563}\\
\multicolumn{1}{c}{avg mut/step}&\multicolumn{2}{c}{      224.442}&\multicolumn{2}{c}{   70.480}&\multicolumn{2}{c}{   22.299}&\multicolumn{2}{c}{    8.470}&\multicolumn{2}{c}{    4.368}&\multicolumn{2}{c}{    2.906}&\multicolumn{2}{c}{    2.271}&\multicolumn{2}{c}{    1.934}&\multicolumn{2}{c}{    1.729}\\
\hline
\multicolumn{1}{c}{total avg count}&\multicolumn{2}{c}{        42}&\multicolumn{2}{c}{       87}&\multicolumn{2}{c}{      216}&\multicolumn{2}{c}{      503}&\multicolumn{2}{c}{    1,094}&\multicolumn{2}{c}{    4,063}&\multicolumn{2}{c}{   10,961}&\multicolumn{2}{c}{   18,727}&\multicolumn{2}{c}{   27,644}\\
\multicolumn{1}{c}{avg eval count}&\multicolumn{2}{c}{         42}&\multicolumn{2}{c}{       87}&\multicolumn{2}{c}{      216}&\multicolumn{2}{c}{      503}&\multicolumn{2}{c}{      910}&\multicolumn{2}{c}{    1,488}&\multicolumn{2}{c}{    1,676}&\multicolumn{2}{c}{    1,814}&\multicolumn{2}{c}{    1,909}\\
\multicolumn{1}{c}{max eval count}&\multicolumn{2}{c}{        303}&\multicolumn{2}{c}{      867}&\multicolumn{2}{c}{    2,843}&\multicolumn{2}{c}{    6,230}&\multicolumn{2}{c}{   10,487}&\multicolumn{2}{c}{  916,298}&\multicolumn{2}{c}{  501,346}&\multicolumn{2}{c}{  411,742}&\multicolumn{2}{c}{   12,386}\\
\multicolumn{1}{c}{min eval count}&\multicolumn{2}{c}{          0}&\multicolumn{2}{c}{        0}&\multicolumn{2}{c}{        0}&\multicolumn{2}{c}{        0}&\multicolumn{2}{c}{        0}&\multicolumn{2}{c}{        0}&\multicolumn{2}{c}{        0}&\multicolumn{2}{c}{        0}&\multicolumn{2}{c}{        0}\\
\hline
\multicolumn{1}{c}{fail ratio}&\multicolumn{2}{c}{          0.000}&\multicolumn{2}{c}{    0.000}&\multicolumn{2}{c}{    0.000}&\multicolumn{2}{c}{    0.000}&\multicolumn{2}{c}{    0.000}&\multicolumn{2}{c}{    0.003}&\multicolumn{2}{c}{    0.010}&\multicolumn{2}{c}{    0.018}&\multicolumn{2}{c}{    0.028}\\
\multicolumn{1}{c}{avg fail dif}&\multicolumn{2}{c}{            -}&\multicolumn{2}{c}{        -}&\multicolumn{2}{c}{        -}&\multicolumn{2}{c}{        -}&\multicolumn{2}{c}{      345}&\multicolumn{2}{c}{      345}&\multicolumn{2}{c}{      345}&\multicolumn{2}{c}{      345}&\multicolumn{2}{c}{      345}\\
\hline
\multicolumn{1}{c}{p-value}&&\multicolumn{2}{c}{0.0000}&\multicolumn{2}{c}{0.0000}&\multicolumn{2}{c}{0.0000}&\multicolumn{2}{c}{0.0000}&\multicolumn{2}{c}{0.0000}&\multicolumn{2}{c}{0.0000}&\multicolumn{2}{c}{0.0034}&\multicolumn{2}{c}{0.0068}\\
&&&&&&&&&&&&&&&&&&\end{tabular}
\end{tabular}
}


Here the same holds.
The higher mutation rates are less impacted than the higher rates for the RLS and the (1+1) EA.
\subsection{Comparison of the best variants}


\makebox[\linewidth]{
\begin{tabular}{lp{3cm}p{6cm}p{6cm}}
\begin{tabular}[h]{cccc}
algo type&        \RLSN& (1+1) EA&  pmut\\
algo param&         b=2&   3$/n$&  2.25\\
avg mut/change&   2.000& 3.092& 3.965\\
avg mut/step&     2.000& 2.999& 4.339\\
\hline
total avg count&    302&   677&   691\\
avg eval count&     302&   677&   691\\
max eval count&   1,610& 6,404& 5,205\\
min eval count&       9&    33&    17\\
\hline
fail ratio&       0.000& 0.000& 0.000\\
\end{tabular}
\end{tabular}
}


The ranking of the algorithm is the same as for the other inputs with a similar preference of low mutation rates.
The RLS has the best performance closely follow by $pmut_{3.25}$ and lastly be the standard (1+1) EA.

\begin{tabular}[h]{cccccccc}
fails&20&50&100&500&1000&5000&10000\\\hline
RLS-N (2)&998&984&973&763&541&49&1\\
RLS-N (4)&998&989&975&837&719&150&27\\
(1+1) EA (3/n)&994&990&981&862&714&194&35\\
(1+1) EA (4/n)&998&993&976&852&725&180&51\\
pmut (-2.0)&997&989&983&905&784&229&55\\
pmut (-2.25)&997&988&987&875&790&213&56\\
\end{tabular}


No unexpected results for the different input sizes.
The RLS variants perform the worst for $n\le 100$.
The input is also rather hard to solve for $n\le50$ but not as hard as the overlapped distributed input for example.

\begin{tabular}[h]{ccccccccc}
avg&20&50&100&500&1000&5000&10000&50000\\\hline
RLS-N (2)&172&1247&7598&38282&36214&105152&117792&125415\\
RLS-N (4)&8626&40966&43921&41924&42289&138187&201183&214170\\
(1+1) EA (3/n)&36470&42555&42439&43801&45530&145209&212281&252713\\
(1+1) EA (4/n)&37212&43144&44340&47246&46689&143697&220208&265620\\
pmut (-2.0)&40368&42601&39603&47443&46591&148390&232398&311606\\
pmut (-2.25)&39476&42113&40325&40548&43691&157816&226129&292367\\
\end{tabular}


The input gets easier to solve up until $n=5000$ and from then on gets harder again with increasing input size.
The increase for larger input sizes is much smaller than the decrease for the small values.

\begin{tabular}[h]{ccccccccc}
total avg&20&50&100&500&1000&5000&10000&50000\\\hline
RLS&98050&94574&89765&60172&39985&15639&4128&1797\\
\RLSR[2]&89401&54556&18266&4218&3530&2362&2160&2229\\
(1+1) EA (1$/n$)&54971&26688&15684&8452&6567&3815&3458&3371\\
(1+1) EA (2$/n$)&26104&9724&6508&4503&4020&3171&3141&3133\\
pmut (3.25)&40934&19972&10977&5644&4406&2434&2162&2172\\
pmut (3.0)&35272&17545&10040&5222&4150&2510&2208&2213\\
\end{tabular}



Mixed and overlapped inputs are best solved by the (1+1) EA with $p_m=2/n$ for $n\le100$ and also relatively good for $n=500$.
After $n\ge1000$ the standard RLS becomes the best option and it seems like it stays that way for the remaining input sizes.


\section{Conclusion of empirical results}
There is no clear best algorithm for every input for PARTITION and not even a best parameter for every algorithm.
For inputs that are comparable to a linear function/OneMax for all base algorithms the parameters with the lowest mutation rate have the best runtime.
Other instances like the worst case input of C. Witt on the other hand require much higher mutation rates for the optimal performance.
Inputs generated from a powerlaw distribution showed that the optimal parameter for every algorithm is not even fixed with a specific distribution.
For inputs drawn from \textasciitilde$D^{2.75}_{50000}$ the higher mutation rates reached an optimum faster than the lower mutation rate for every algorithm variant.
If the input was drawn from \textasciitilde$D^{1.25}_{50000}$ then the fastest mutation rates for the (1+1) EA on \textasciitilde$D^{2.75}_{50000}$ distributed inputs then instead become the slowest.
So almost no general advice is possible, but a few points still hold for every input type.
The first one is the RLS being most likely to be stuck in a local optimum especially for the smaller input size.
Even if a variant of the RLS is the fastet for the bigger input sizes it is most likely to be stuck in a local optimum for $n\le100$ for most input types.
So if the input size $n\le100$ choosing the (1+1) EA or $pmut$ mutation operator is a better choice.
Another noticeable relation is that inputs that require higher mutation rates are generally easy to solve and are also solved very fast by the lower mutation rates.
A lower number of iterations also does not imply a shorter runtime in every case.
If the mutation rate $1/n$ needs only a few iterations more than $100/n$ it will still be much faster since one iteration is much shorter.
The lower mutation rates are therefore generally a better choice as they will need less time in most cases and are still rather fast if they are not the fastest.
Only if the algorithm is trapped due to its low mutation rate a higher mutation rate makes a huge difference.

Now to round this paper up there are two tables that summaries the previous results.
For each input and each algorithm the best three variants are listed in Table~\ref{table:BestAlgoVariantsTable} ordered by their average runtime.
This implies a general tendency of better algorithms but is not necessarily a complete insight as the best parameter changes depending on $n$.
Table~\ref{table:BestAlgoVariantTable} list my personal preference based on the previous results depending on the distribution and size of the input.

\begin{table}[t]
      \caption{Best algorithms variants for all researched inputs}
      \begin{tabular}{c|ccc|ccc|ccc}\label{table:BestAlgoVariantsTable}
                                                   &
            \multicolumn{3}{c|}{RLS variants}      &
            \multicolumn{3}{c|}{(1+1) EA variants} &
            \multicolumn{3}{c}{$pmut$ variants}                                                                                      \\
                                                   & 1st      & 2nd      & 3rd      & 1st     & 2nd    & 3rd    & 1st  & 2nd  & 3rd  \\\hline
            binomial                               & \RLSN[2] & \RLSN[4] & \RLSR[2] & 3$/n$   & 4$/n$  & 2$/n$  & 2.25 & 2.0  & 1.75 \\
            geometric                              & \RLSR[2] & \RLSR[3] & \RLSR[4] & 2$/n$   & 1$/n$  & 3$/n$  & 3.25 & 3.0  & 2.5  \\
            polwerlaw                              & \RLSR[4] & \RLSR[3] & \RLSN[3] & 4$/n$   & 3$/n$  & 2$/n$  & 1.5  & 1.75 & 2.0  \\
            uniform                                & \RLSN[2] & \RLSR[3] & \RLSR[4] & 3$/n$   & 2$/n$  & 4$/n$  & 2.5  & 2.0  & 2.25 \\
            linear function                        & RLS      & \RLSR[2] & \RLSR[3] & 1$/n$   & 2$/n$  & 3$/n$  & 3.5  & 3.25 & 3.0  \\
            worst case                             & \RLSN[4] & \RLSN[3] & \RLSN[2] & 100$/n$ & 50$/n$ & 10$/n$ & 1.25 & 1.5  & 1.75 \\
            combined                               & RLS      & \RLSR[2] & \RLSR[3] & 1$/n$   & 2$/n$  & 3$/n$  & 3.25 & 3.0  & 2.75 \\
      \end{tabular}
\end{table}

\begin{table}[t]
      \caption{Suggestions for the fastest algorithm on each input depending on the input size}
      \begin{tabular}{c|c|c|c|c}\label{table:BestAlgoVariantTable}
                            & $n\le100$                                         & $100<n<500$
                            & $500\le n<10,000$                                 & $10,000\le n$                            \\
            \hline
            binomial        & \multicolumn{2}{c|}{(1+1) EA $_{3/n}$}            & \multicolumn{2}{c}{\RLSN[2]}             \\
            geometric       & \multicolumn{2}{c|}{(1+1) EA $_{2/n}$}            & \multicolumn{2}{c}{$pmut_{3.25}$}        \\
            uniform         & \multicolumn{2}{c|}{(1+1) EA $_{4/n}$}            & \multicolumn{2}{c}{\RLSN[2]}             \\
            \hline
            powerlaw        & \multicolumn{4}{c}{$pmut_{1.5}$ or $pmut_{1.75}$}                                            \\
            linear function & \multicolumn{4}{c}{RLS}                                                                      \\
            worst case      & \multicolumn{4}{c}{$pmut_{1.25}$}                                                            \\
            \hline
            combined        & (1+1) EA $_{2/n}$                                 & \multicolumn{2}{c|}{$pmut_{1.25}$} & RLS \\
      \end{tabular}
\end{table}
%% conclusion.tex
%%

%% ==================
\chapter{Conclusion}
\label{ch:conclusion}
%% ==================

There is no clear best algorithm for every input for PARTITION and not even a best parameter for every algorithm.
For inputs that are comparable to a linear function/OneMax for all base algorithms the parameter with the lowest mutation rate has the best runtime.
Other instances like the worst case input of C. Witt on the other hand require much higher mutation rates for the optimal performance.
Inputs generated from a powerlaw distribution showed that the optimal parameter for every algorithm is not even fixed within a specific distribution.
For inputs drawn from \textasciitilde$D^{2.75}_{50000}$ the higher mutation rates reached an optimum faster than the lower mutation rate for every algorithm variant.
If the input was drawn from \textasciitilde$D^{1.25}_{50000}$ then the fastest mutation rates for the (1+1) EA on \textasciitilde$D^{2.75}_{50000}$ distributed inputs then instead became the slowest.\newline
So almost no general advice is possible, but a few points still hold for every input type.
The first one is the RLS being most likely to be stuck in a local optimum especially for the smaller input size.
Even if a variant of the RLS is the fastet for the bigger input sizes it is most likely to be stuck in a local optimum for $n\le100$ for most input types.
So if the input size $n\le100$ choosing the (1+1) EA or $pmut$ mutation operator is a better choice.
Another noticeable relation is that inputs that require higher mutation rates are generally easy to solve and are also solved very fast by the lower mutation rates.
A lower number of iterations also does not imply a shorter runtime in every case.
If the mutation rate $1/n$ needs only a few iterations more than $100/n$ it will still be much faster since one iteration is much shorter.
The lower mutation rates are therefore generally a better choice as they will need less time in most cases and are still rather fast if they are not the fastest.
Only if the algorithm is trapped due to its low mutation rate a higher mutation rate makes a huge difference.\newline
Another point is that the Evolutionary Algorithms perform better for larger input sizes as there are more perfect partitions.
The more perfect partition an input has, the easier it is to find one.
For the lower values of $n$ the algorithm sometimes needed 20,000 iterations on average if they managed to find a perfect partition and even longer otherwise.
A runtime of \(100,000\approx2^{14,29}\ge 2^{n-6}\ge2^{n/2}\) is exponential in the size of the input.
So for smaller values choosing other approximation algorithms or even exact algorithms will probably lead to better a better runtime.
For higher values of $n$ on easier inputs they might be efficient as well or in some cases even better.\newline
The last common relation is the less small values especially close to 1 an input has the better flipping 2 or 4 bits in a step becomes.
This was only shown for the binomial distributed inputs but on inputs from \textasciitilde$U(10^4,5\cdot 10^5)$ the results were mostly the same.
To make the thesis shorter this was not listed in the corresponding section.

\begin{table}[t]
      \caption{Best algorithms variants for all evaluated input types}
      \begin{tabular}{c|ccc|ccc|ccc}\label{table:BestAlgoVariantsTable}
                                                   &
            \multicolumn{3}{c|}{RLS variants}      &
            \multicolumn{3}{c|}{(1+1) EA variants} &
            \multicolumn{3}{c}{$pmut$ variants}                                                                                      \\
                                                   & 1st      & 2nd      & 3rd      & 1st     & 2nd    & 3rd    & 1st  & 2nd  & 3rd  \\\hline
            binomial                               & \RLSN[2] & \RLSN[4] & \RLSR[2] & 3$/n$   & 4$/n$  & 2$/n$  & 2.0  & 2.25 & 2.5  \\
            geometric                              & \RLSR[2] & \RLSR[3] & \RLSR[4] & 2$/n$   & 1$/n$  & 3$/n$  & 3.25 & 3.0  & 2.75 \\
            uniform                                & \RLSN[2] & \RLSR[3] & \RLSR[4] & 4$/n$   & 3$/n$  & 2$/n$  & 2.0  & 2.25 & 2.75 \\
            polwerlaw                              & \RLSR[4] & \RLSN[3] & \RLSR[3] & 4$/n$   & 3$/n$  & 5$/n$  & 1.5  & 1.75 & 1.25 \\
            linear function                        & RLS      & \RLSR[2] & \RLSR[3] & 1$/n$   & 2$/n$  & 3$/n$  & 3.5  & 3.25 & 3.0  \\
            worst case                             & \RLSN[4] & \RLSR[4] & \RLSN[3] & 100$/n$ & 50$/n$ & 10$/n$ & 1.25 & 1.5  & 1.75 \\
            combined                               & RLS      & \RLSR[2] & \RLSR[3] & 1$/n$   & 2$/n$  & 3$/n$  & 3.25 & 3.0  & 2.75 \\
      \end{tabular}
\end{table}

Now to round this paper up there are two tables that summarise the previous results.
For each input type and each algorithm the best three variants are listed in Table~\ref{table:BestAlgoVariantsTable} ordered by their average runtime.
This implies a general tendency of better algorithms but is not necessarily a complete insight as the best parameter and algorithm changes depending on $n$.
Table~\ref{table:BestAlgoVariantTable} list my personal preference based on the previous results depending on the distribution and size of the input.
There is no clear overall winner but if one algorithm must be chosen for any input then choosing $pmut_{2.25}$ should be a good option.
For binomial distributed inputs the \RLSN[2] and the \RLSN[4] are the fastest RLS variants but for the OneMax equivalent they need about \(\Theta(\frac{n^k}{k!})\) iterations on average and expectation.
The runtime of the other RLS variants is also too unstable.
So the RLS versions are too much dependent on the input and should not be chosen in every case, especially if the input size is small.
For the best (1+1) EA variants this does not to that extend as the mutation rate $3/n$ is under the top 3 for every input type except the worst case input.
Yet on the OneMax input and the worst case input the (1+1) EA with mutation rate $3/n$ does not reach the optimal solution in all runs in at most $10n\ln{n}$ steps.
$pmut_{2.25}$ is not always one of the best three $pmut$ variants but is also never one of the worst and for the bigger input sizes it also reaches the optimal solution for almost any input in $10n\ln{n}$ steps except for 2/10,000 runs for the Worst Case input of C. Witt.
So if only one algorithm for every input must be chosen, then $pmut_{2.25}$ should be one of the best options.

\begin{table}[h]
      \caption{Suggestions for the fastest algorithm on each input depending on the input size (the (1+1) EA is listed as EA to make the table shorter)}
      % \begin{tabular}{cccccccccc}\label{table:BestAlgoVariantTable}
      \begin{tabular}{m{2.5cm}m{1cm}m{1cm}m{0.5cm}m{0.5cm}m{0.5cm}m{0.5cm}m{0.5cm}m{0.5cm}m{1cm}m{1cm}}\label{table:BestAlgoVariantTable}
            input size $n$  & \multicolumn{2}{|c}{$100$}                          & \multicolumn{2}{c}{$500$}
                            & \multicolumn{2}{c}{$1000$}                          & \multicolumn{2}{c}{$5000$}
                            & \multicolumn{2}{c}{$50,000$}                                                                                         \\
            \hline
            geometric       & \multicolumn{5}{|c|}{EA $_{2/n}$}                   & \multicolumn{4}{c|}{$pmut_{3.25}$} & RLS                       \\
            \hline
            binomial        & \multicolumn{3}{|c|}{EA $_{3/n}$}                   &
            \multicolumn{7}{c}{\multirow{2}{*}{\RLSN[2]}}                                                                                          \\
            uniform         & \multicolumn{3}{|c|}{EA $_{4/n}$}                   & \multicolumn{7}{c}{}                                           \\
            \hline
            powerlaw        & \multicolumn{10}{|c}{$pmut_{1.5}$ or $pmut_{1.75}$}                                                                  \\
            \hline
            linear function & \multicolumn{10}{|c}{RLS}                                                                                            \\
            \hline
            worst case      & \multicolumn{10}{|c}{$pmut_{1.25}$}                                                                                  \\
            \hline
            combined        & \multicolumn{1}{|c|}{EA $_{2/n}$}                   & \multicolumn{6}{c|}{$pmut_{3.25}$} & \multicolumn{3}{c}{RLS}   \\
                            &                                                     &                                    &
                            &                                                     &                                    &
                            &                                                     &                                    &                         & \\
      \end{tabular}
\end{table}



%% --------------------
%% |   Bibliography   |
%% --------------------

\cleardoublepage
\phantomsection
\addcontentsline{toc}{chapter}{\bibname}

\iflanguage{english}
{\bibliographystyle{alpha}}
{\bibliographystyle{babalpha-fl}} % german style

\bibliography{references}


%% ----------------
%% |   Appendix   |
%% ----------------

% \cleardoublepage
% \input{chapters/appendix}


\end{document}
