\section{Improving bounds on the RLS and the (1+1) EA}
\begin{lemma}\label{OneMaxResult}
    If $w_1 \ge \frac W 2$  then the RLS and the (1+1) EA reach the optimal value in expected time $\Theta(n\log{}n)$
\end{lemma}
\begin{proof}
    The optimal solution is putting $w_1$ in one bin and all other elements in the other bin. So the problem is almost identical to OneMax/ZeroMax. A single bit flip of the first bit can only happen, if the emptier bin has a weight of at most $\frac {W-w_1}{2}$. After this flip the weight of the emptier bin is at least $\frac {W-w_1}{2}$ and therefore another single bit flip of $w_1$ can only happen before a different bit is flipped. After a different bit has been flipped, the RLS wont flip the first bit again, because it will never result in an improvement. So the run of the RLS can be divided into three phases:
    \begin{itemize}
        \item[Phase 1:] The RLS behaves exactly like OneMax/ZeroMax and flips every bit to the opposite of the first bit (except for the first bit).
        \item[Phase 2:] The RLS flips only the first bit or bits that do not result in an improvement.
        \item[Phase 3:] The RLS behaves exactly like ZeroMax/OneMax and flips every bit to the opposite of the first bit (except for the first bit).
    \end{itemize}

    The expected length of the first phase is $\mathcal{O}(n)$ because the probability of flipping the first bit is at least ${\frac{1}{n}} \cdot {(1 - \frac{1}{n})}^{n-1} \ge \frac{1}{en}$ and therefore the expected time for such a step is at most $\mathcal{O}(\frac{1}{en}^{-1}) = \mathcal{O}(ne) = \mathcal{O}(n)$.\newline
    The length of the second phase is $\mathcal{O}(n)$ because the solution is either optimal or there is at least one bit that needs to be flipped for an optimal solution. Since the expected length of Phase 1 is $\mathcal{O}(n)$ the solution produced by the RLS won't be optimal in expectation due to the bound of $\Theta(n\log{}n)$ for OneMax/ZeroMax. This again results in expected time $\mathcal{O}(n)$.\newline
    The length of the third phase is identical to a run of the RLS on OneMax/ZeroMax where flips of the first bit are ignored as if it was already correctly flipped and therefore the expected time is $\Theta(n\log{}n)$

    So the total expected time is $\mathcal{O}(n) + \mathcal{O}(n) + \Theta(n\log{}n) = \Theta(n\log{}n)$

    The (1+1) EA can do multiple bit flips in a single step so the first bit can be flipped multiple times if the combined moved weight \(y \le b_F-b_E\). \textbf{\textit{TODO: insert proof for (1+1) EA}}.
\end{proof}

\begin{lemma}\label{approximationLemmaHelp1}
    If \(b_F \le \frac{2}{3} \cdot W\) the approximation ratio is at most $\frac{4}{3}$
\end{lemma}
\begin{proof}
    \(\frac{b_F}{opt} \le \frac{(2/3) \cdot W}{opt} \le \frac{(2/3) \cdot W}{(1/2) \cdot W} = \frac{4}{3}\), since \(opt \ge \frac{W}{2}\)
\end{proof}

\begin{corollary}\label{approximationCorollaryHelp2}
    If \(w_1 \ge \frac{W}{3}\) and \(w_1\) is in the emptier bin, then the approximation ratio is at most $\frac{4}{3}$
\end{corollary}
\begin{proof}
    $w_1$ is in the emptier bin, so \( b_F \le W - w_1 \le W - \frac{W}{3} = \frac{2W}{3} \) and with Lemma~\ref{approximationLemmaHelp1} the assumption follows.
\end{proof}

\begin{lemma}\label{movingObjectsLemma}
    Any object of weight at most $v$ can be moved from $b_F$ to $b_E$ if \(b_F - b_E \ge v\)
\end{lemma}
\begin{proof}
    \(b_F - b_E \ge v \Leftrightarrow b_F \ge b_E + v\), so after moving an object with weight at most $v$ from $b_F$ to $b_E$, the new weight of $b_E$ is at most the weight of $b_F$ before moving the object, thus the RSH accepts the step.
\end{proof}

\begin{corollary}\label{movingObjectsCorollary}
    Every object \(\le \frac{W}{3}\) can be moved from $b_F$ to $b_E$ if \(b_F \ge \frac{2W}{3}\)
\end{corollary}
\begin{proof}
    \(b_F \ge \frac{2W}{3} \Rightarrow b_E \le W - \frac{2W}{3} \le \frac{W}{3} \Rightarrow b_F - b_E \ge \frac{2W}{3} - \frac{W}{3} = \frac{W}{3}\) and with Lemma~\ref{movingObjectsLemma} the assumption follows.
\end{proof}

\begin{lemma}\label{movingObjectsLemma2}
    In expected Time $\mathcal{O}(n\log{}n)$ the weight of the fuller bin can be decreased to \(\le \frac{2W}{3}\) if every object besides the biggest in the fuller bin is at most $\frac{W}{3}$ and \(w_1 \le \frac{W}{2}\).
\end{lemma}
\begin{proof}
    In expected time $\mathcal{O}(n\log{}n)$ the RSH can move every object $\le \frac{W}{3}$ to the emptier bin as long as $b_F \ge \frac{2W}{3}$ due to Corollary~\ref{movingObjectsCorollary} and results for OneMax. So in expected Time $\mathcal{O}(n\log{}n)$ the solution can be shifted to $w_1$ being in one bin and all other objects in the other bin. The RSH will only stop moving the elements if the condition $b_F \ge \frac{2W}{3}$ is no longer satisfied (Corollary~\ref{movingObjectsCorollary}). If \(w_1 \ge \frac{W}{3}\) and every object was moved to the bin without $w_1$, then \(b_F = \max\{W-w_1, w_1\} = W-w_1 \le \frac{2W}{3}\), because \(w_1 \le \frac{W}{2}\). So either the RSH moves all objects to the emptier bin or stops moving objects because $b_F < \frac{2W}{3}$ both resulting in $b_F \le \frac{2W}{3}$. If $w_1$ is not in the fuller bin, then the result follows by Corollary~\ref{approximationCorollaryHelp2}.\newline
    Now assume \(w_1 < \frac{W}{3}\). In this case the RLS will move one object per step to the emptier bin. Each object has weight $< \frac{W}{3}$ and therefore one step can not decrease the weight of the fuller bin from $> \frac{2W}{3}$ to $\le \frac{W}{3}$. If all objects except the biggest where moved the other bin, the other bin would have a weight of at least \(W-w_1 > \frac{2W}{3}\). Therefore the RLS will find a solution with $b_F < \frac{2W}{3}$ before moving all elements from the first to the second bin. \textbf{\textit{TODO: insert proof for (1+1) EA}}
\end{proof}

\begin{lemma}\label{approximationLemma}
    The RLS and the (1+1) EA reach an approximation ratio of at most $\frac{4}{3}$ in expected time $\mathcal{O}(n\log{}n)$ if $w_1 < W/2$
\end{lemma}
\begin{proof}    
    If \(w_1+w_2 > \frac{2W}{3}\) after time $\mathcal{O}(n)$ $w_1$ and $w_2$ are separated and will remain separated afterwards (Proof by C.Witt~\cite{diekert2005stacs}). From then on the following holds. If $w_1$ is in the emptier bin, then the result follows directly by Corollary~\ref{approximationCorollaryHelp2}. Otherwise all elements in the fuller bin except $w_1$ have a weight of at most $\frac{1}{3}$ and therefore the result follows by Lemma~\ref{movingObjectsLemma2} and Lemma~\ref{approximationLemmaHelp1}. If \(w_1+w_2 \le \frac{2W}{3}\) the result follows directly by Lemma~\ref{movingObjectsLemma2} and Lemma~\ref{approximationLemmaHelp1}.
\end{proof}

\begin{corollary}
    The RLS and the (1+1) EA reach an approximation ratio of at most $\frac{4}{3}$ in expected time $\mathcal{O}(n\log{}n)$
\end{corollary}
\begin{proof}
    This follows directly from Lemma~\ref{OneMaxResult} and Lemma~\ref{approximationLemma}
\end{proof}