%% conclusion.tex
%%

%% ==================
\chapter{Conclusion}
\label{ch:conclusion}
%% ==================

There is no clear best algorithm for every input for PARTITION and not even a best parameter for every algorithm.
For inputs that are comparable to a linear function/OneMax for all base algorithms the parameter with the lowest mutation rate has the best runtime.
Other instances like the worst case input of C. Witt on the other hand require much higher mutation rates for the optimal performance.
Inputs generated from a powerlaw distribution showed that the optimal parameter for every algorithm is not even fixed within a specific distribution.
For inputs drawn from \textasciitilde$D^{2.75}_{50000}$ the higher mutation rates reached an optimum faster than the lower mutation rate for every algorithm variant.
If the input was drawn from \textasciitilde$D^{1.25}_{50000}$ then the fastest mutation rates for the (1+1) EA on \textasciitilde$D^{2.75}_{50000}$ distributed inputs then instead became the slowest.\newline
So almost no general advice is possible, but a few points still hold for every input type.
The first one is the RLS being most likely to be stuck in a local optimum especially for the smaller input size.
Even if a variant of the RLS is the fastet for the bigger input sizes it is most likely to be stuck in a local optimum for $n\le100$ for most input types.
So if the input size $n\le100$ choosing the (1+1) EA or $pmut$ mutation operator is a better choice.
Another noticeable relation is that inputs that require higher mutation rates are generally easy to solve and are also solved very fast by the lower mutation rates.
A lower number of iterations also does not imply a shorter runtime in every case.
If the mutation rate $1/n$ needs only a few iterations more than $100/n$ it will still be much faster since one iteration is much shorter.
The lower mutation rates are therefore generally a better choice as they will need less time in most cases and are still rather fast if they are not the fastest.
Only if the algorithm is trapped due to its low mutation rate a higher mutation rate makes a huge difference.\newline
Another point is that the Evolutionary Algorithms perform better for larger input sizes as there are more perfect partitions.
The more perfect partition an input has, the easier it is to find one.
For the lower values of $n$ the algorithm sometimes needed 20,000 iterations on average if they managed to find a perfect partition and even longer otherwise.
A runtime of \(100,000\approx2^{14,29}\ge 2^{n-6}\ge2^{n/2}\) is exponential in the size of the input.
So for smaller values choosing other approximation algorithms or even exact algorithms will probably lead to better a better runtime.
For higher values of $n$ on easier inputs they might be efficient as well or in some cases even better.\newline
The last common relation is the less small values especially close to 1 an input has the better flipping 2 or 4 bits in a step becomes.
This was only shown for the binomial distributed inputs but on inputs from \textasciitilde$U(10^4,5\cdot 10^5)$ the results were mostly the same.
To make the thesis shorter this was not listed in the corresponding section.

\begin{table}[t]
      \caption{Best algorithms variants for all evaluated input types}
      \begin{tabular}{c|ccc|ccc|ccc}\label{table:BestAlgoVariantsTable}
                                                   &
            \multicolumn{3}{c|}{RLS variants}      &
            \multicolumn{3}{c|}{(1+1) EA variants} &
            \multicolumn{3}{c}{$pmut$ variants}                                                                                      \\
                                                   & 1st      & 2nd      & 3rd      & 1st     & 2nd    & 3rd    & 1st  & 2nd  & 3rd  \\\hline
            binomial                               & \RLSN[2] & \RLSN[4] & \RLSR[2] & 3$/n$   & 4$/n$  & 2$/n$  & 2.0  & 2.25 & 2.5  \\
            geometric                              & \RLSR[2] & \RLSR[3] & \RLSR[4] & 2$/n$   & 1$/n$  & 3$/n$  & 3.25 & 3.0  & 2.75 \\
            uniform                                & \RLSN[2] & \RLSR[3] & \RLSR[4] & 4$/n$   & 3$/n$  & 2$/n$  & 2.0  & 2.25 & 2.75 \\
            polwerlaw                              & \RLSR[4] & \RLSN[3] & \RLSR[3] & 4$/n$   & 3$/n$  & 5$/n$  & 1.5  & 1.75 & 1.25 \\
            linear function                        & RLS      & \RLSR[2] & \RLSR[3] & 1$/n$   & 2$/n$  & 3$/n$  & 3.5  & 3.25 & 3.0  \\
            worst case                             & \RLSN[4] & \RLSR[4] & \RLSN[3] & 100$/n$ & 50$/n$ & 10$/n$ & 1.25 & 1.5  & 1.75 \\
            combined                               & RLS      & \RLSR[2] & \RLSR[3] & 1$/n$   & 2$/n$  & 3$/n$  & 3.25 & 3.0  & 2.75 \\
      \end{tabular}
\end{table}

Now to round this paper up there are two tables that summarise the previous results.
For each input type and each algorithm the best three variants are listed in Table~\ref{table:BestAlgoVariantsTable} ordered by their average runtime.
This implies a general tendency of better algorithms but is not necessarily a complete insight as the best parameter and algorithm changes depending on $n$.
Table~\ref{table:BestAlgoVariantTable} list my personal preference based on the previous results depending on the distribution and size of the input.
There is no clear overall winner but if one algorithm must be chosen for any input then choosing $pmut_{2.25}$ should be a good option.
For binomial distributed inputs the \RLSN[2] and the \RLSN[4] are the fastest RLS variants but for the OneMax equivalent they need about \(\Theta(\frac{n^k}{k!})\) iterations on average and expectation.
The runtime of the other RLS variants is also too unstable.
So the RLS versions are too much dependent on the input and should not be chosen in every case, especially if the input size is small.
For the best (1+1) EA variants this does not to that extend as the mutation rate $3/n$ is under the top 3 for every input type except the worst case input.
Yet on the OneMax input and the worst case input there are runs where the (1+1) EA with mutation rate $3/n$ does not reach the optimal solution in all runs in at most $10n\ln{n}$ steps.
$pmut_{2.25}$ is not always one of the best three $pmut$ variants but is also never one of the worst and for the bigger input sizes it also reaches the optimal solution for almost any input in $10n\ln{n}$ steps except for 2/10,000 runs for the Worst Case input of C. Witt.
So if only one algorithm for every input must be chosen, then $pmut_{2.25}$ should be one of the best options.

\begin{table}[h]
      \caption{Suggestions for the fastest algorithm on each input depending on the input size (the (1+1) EA is listed as EA to make the table shorter)}
      % \begin{tabular}{cccccccccc}\label{table:BestAlgoVariantTable}
      \begin{tabular}{m{2.5cm}m{1cm}m{1cm}m{0.5cm}m{0.5cm}m{0.5cm}m{0.5cm}m{0.5cm}m{0.5cm}m{1cm}m{1cm}}\label{table:BestAlgoVariantTable}
            input size $n$  & \multicolumn{2}{|c}{$100$}                          & \multicolumn{2}{c}{$500$}
                            & \multicolumn{2}{c}{$1000$}                          & \multicolumn{2}{c}{$5000$}
                            & \multicolumn{2}{c}{$50,000$}                                                                                         \\
            \hline
            geometric       & \multicolumn{5}{|c|}{EA $_{2/n}$}                   & \multicolumn{4}{c|}{$pmut_{3.25}$} & RLS                       \\
            \hline
            binomial        & \multicolumn{3}{|c|}{EA $_{3/n}$}                   &
            \multicolumn{7}{c}{\multirow{2}{*}{\RLSN[2]}}                                                                                          \\
            uniform         & \multicolumn{3}{|c|}{EA $_{4/n}$}                   & \multicolumn{7}{c}{}                                           \\
            \hline
            powerlaw        & \multicolumn{10}{|c}{$pmut_{1.5}$ or $pmut_{1.75}$}                                                                  \\
            \hline
            linear function & \multicolumn{10}{|c}{RLS}                                                                                            \\
            \hline
            worst case      & \multicolumn{10}{|c}{$pmut_{1.25}$}                                                                                  \\
            \hline
            combined        & \multicolumn{1}{|c|}{EA $_{2/n}$}                   & \multicolumn{6}{c|}{$pmut_{3.25}$} & \multicolumn{3}{c}{RLS}   \\
                            &                                                     &                                    &
                            &                                                     &                                    &
                            &                                                     &                                    &                         & \\
      \end{tabular}
\end{table}
