The $\text{RLS-N}_2$ seems to perform the best as it mostly switches two elements which works great for binomial distributed inputs. 
The same algorithm with $k=4$ performs a bit worse but still good as switching 4 elements can be beneficial as well.
The variant of RLS-N with $k=3$ on the other hand does not reach the optimal solution in 23.4\% of the inputs with an average difference of 1.
It also needs 1000 times more iterations to find the optimal on average compared to the best algorithm $\text{RLS-N}_2$.
The RLS-R variants behave mostly the same with $k=2$ being the best, followed by $k=4$ and $k=3$.
In this case the variant of $k=3$ is by far not as bad as for the RLS-N because the probability of flipping 2 bits is 1/3 as compared to $\mathcal{O}(n^{-1})$ for the RLS-N.
The RLS-R seem all to be good option for binomial inputs with values of $k\in\{2,3,4\}$.
The RLS on the other hand performs by far the worst as it only moves one element at a time.
It only managed to reach the optimal solution once for 1000 different inputs.
The number of iterations for this input was only 50 so the RLS likely had a good initialisation with a few lucky steps leading directly to the optimum.
For all other cases the average difference between the bins was 254 which is close to the average value of the values from 0 to $(1000-100)/2=450$.
This is likely due to the RLS being unable to improve the solution once the current solution has a difference below half of the lowest value.